\chapter{Regressão Linear}
\label{Chap:RegressoLinear}

\begin{fullwidth}
{\it
Apesar de podermos verificar previsões teóricas pela simples comparação com os dados experimentais obtidos (assumindo que eles sejam confiáveis), devido à dispersão dos dados não temos um bom método para extrair informações a partir das medidas. Veremos que para isso precisamos determinar a linha de tendência dos dados. A partir das informações para essa linha, conseguiremos extrair informações acerca de parâmetros físicos do sistema.
}
\end{fullwidth}

%%%%%%%%%%%%%%%%%%%%%%%%%%%%%%%%%%%%%
\section{Linhas de Tendência}
%%%%%%%%%%%%%%%%%%%%%%%%%%%%%%%%%%%%%

Quando realizamos um experimento, procuramos relacionar uma variável dependente a uma variável independente. Para visualizarmos a relação entre as duas, é interessante fazer uma representação gráfica da variável dependente em função dos valores da variável independente. Podemos assim verificar uma tendência geral dos pontos, que pode seguir padrões retilíneos, parabólicos, etc. Retomando a citação ao texto de Johann H. Lambert (e a extendendo), temos\cite{Lambert}
\begin{quote}
Se as observações experimentais fossem completamente precisas, essas ordenadas resultariam em um número de pontos através dos quais uma curva ou uma reta deveriam ser traçadas. No entanto, esse não é o caso, a curva/reta desvia pouco ou muito dos pontos observados. Portanto, ela deve ser traçada de maneira que passe tão próxima quanto possível das posições verdadeiras e vá, como se fosse, pelo meio dos pontos em questão.
\end{quote}
%
Muitas vezes tal padrão é muito claro, pois os pontos tem uma \emph{dispersão} baixa. Outras vezes a dispersão é razoavelmente alta e fica difícil verificar o padrão seguido pelos pontos.

Mesmo em casos em que podemos verificar um padrão aparente ao fazer um gráfico, determinar a forma mais adequada para a \emph{linha de tendência} que melhor descreve os pontos experimentais ---~ou mesmo afirmar que tais pontos seguem este padrão~--- pode ser complicado. Se, por exemplo, fizermos uma série de medidas que seguem um padrão parabólico, mas com medidas que se restringem a um intervalo pequeno da variável independente, o gráfico terá a aparência de uma reta (veja as Figuras~\ref{Fig:ParabolaReta} e~\ref{Fig:ParabolaReta2}).
\begin{figure*}
\centering
\forceversofloat
\caption{Gráfico de um conjunto de pontos que aparentemente seguem uma tendência linear. Veja também a Figura~\ref{Fig:ParabolaReta2}.}
\label{Fig:ParabolaReta}
\begin{tikzpicture}[gnuplot]
%% generated with GNUPLOT 5.0p0 (Lua 5.3; terminal rev. 99, script rev. 100)
%% 2015-05-18T23:00:31 BRT
\path (0.000,0.000) rectangle (14.000,9.000);
\gpcolor{color=gp lt color border}
\gpsetlinetype{gp lt border}
\gpsetdashtype{gp dt solid}
\gpsetlinewidth{1.00}
\draw[gp path] (1.136,0.985)--(1.316,0.985);
\draw[gp path] (13.447,0.985)--(13.267,0.985);
\node[gp node right] at (0.952,0.985) {$15$};
\draw[gp path] (1.136,2.259)--(1.316,2.259);
\draw[gp path] (13.447,2.259)--(13.267,2.259);
\node[gp node right] at (0.952,2.259) {$20$};
\draw[gp path] (1.136,3.534)--(1.316,3.534);
\draw[gp path] (13.447,3.534)--(13.267,3.534);
\node[gp node right] at (0.952,3.534) {$25$};
\draw[gp path] (1.136,4.808)--(1.316,4.808);
\draw[gp path] (13.447,4.808)--(13.267,4.808);
\node[gp node right] at (0.952,4.808) {$30$};
\draw[gp path] (1.136,6.082)--(1.316,6.082);
\draw[gp path] (13.447,6.082)--(13.267,6.082);
\node[gp node right] at (0.952,6.082) {$35$};
\draw[gp path] (1.136,7.357)--(1.316,7.357);
\draw[gp path] (13.447,7.357)--(13.267,7.357);
\node[gp node right] at (0.952,7.357) {$40$};
\draw[gp path] (1.136,8.631)--(1.316,8.631);
\draw[gp path] (13.447,8.631)--(13.267,8.631);
\node[gp node right] at (0.952,8.631) {$45$};
\draw[gp path] (1.460,0.985)--(1.460,1.165);
\draw[gp path] (1.460,8.631)--(1.460,8.451);
\node[gp node center] at (1.460,0.677) {$3.8$};
\draw[gp path] (2.756,0.985)--(2.756,1.165);
\draw[gp path] (2.756,8.631)--(2.756,8.451);
\node[gp node center] at (2.756,0.677) {$4$};
\draw[gp path] (4.052,0.985)--(4.052,1.165);
\draw[gp path] (4.052,8.631)--(4.052,8.451);
\node[gp node center] at (4.052,0.677) {$4.2$};
\draw[gp path] (5.348,0.985)--(5.348,1.165);
\draw[gp path] (5.348,8.631)--(5.348,8.451);
\node[gp node center] at (5.348,0.677) {$4.4$};
\draw[gp path] (6.644,0.985)--(6.644,1.165);
\draw[gp path] (6.644,8.631)--(6.644,8.451);
\node[gp node center] at (6.644,0.677) {$4.6$};
\draw[gp path] (7.939,0.985)--(7.939,1.165);
\draw[gp path] (7.939,8.631)--(7.939,8.451);
\node[gp node center] at (7.939,0.677) {$4.8$};
\draw[gp path] (9.235,0.985)--(9.235,1.165);
\draw[gp path] (9.235,8.631)--(9.235,8.451);
\node[gp node center] at (9.235,0.677) {$5$};
\draw[gp path] (10.531,0.985)--(10.531,1.165);
\draw[gp path] (10.531,8.631)--(10.531,8.451);
\node[gp node center] at (10.531,0.677) {$5.2$};
\draw[gp path] (11.827,0.985)--(11.827,1.165);
\draw[gp path] (11.827,8.631)--(11.827,8.451);
\node[gp node center] at (11.827,0.677) {$5.4$};
\draw[gp path] (13.123,0.985)--(13.123,1.165);
\draw[gp path] (13.123,8.631)--(13.123,8.451);
\node[gp node center] at (13.123,0.677) {$5.6$};
\draw[gp path] (1.136,8.631)--(1.136,0.985)--(13.447,0.985)--(13.447,8.631)--cycle;
\node[gp node center,rotate=-270] at (0.246,4.808) {$y$};
\node[gp node center] at (7.291,0.215) {$x$};
\node[gp node left] at (2.604,8.297) {Dados experimentais};
\gpcolor{rgb color={0.000,0.000,0.000}}
\gpsetpointsize{4.00}
\gppoint{gp mark 7}{(2.108,1.395)}
\gppoint{gp mark 7}{(3.404,3.299)}
\gppoint{gp mark 7}{(4.700,3.465)}
\gppoint{gp mark 7}{(5.996,2.400)}
\gppoint{gp mark 7}{(7.292,4.362)}
\gppoint{gp mark 7}{(8.587,5.265)}
\gppoint{gp mark 7}{(9.883,5.916)}
\gppoint{gp mark 7}{(11.179,6.895)}
\gppoint{gp mark 7}{(12.475,7.645)}
\gppoint{gp mark 7}{(1.962,8.297)}
\gpcolor{color=gp lt color border}
\node[gp node left] at (2.604,7.989) {$y = \np{14.24113} \, x - \np{38.07973}$, $r^2 = \np{0.90858}$};
\gpcolor{rgb color={0.000,0.000,0.000}}
\draw[gp path] (1.504,7.989)--(2.420,7.989);
\draw[gp path] (1.788,1.433)--(1.795,1.437)--(1.801,1.440)--(1.807,1.444)--(1.813,1.447)%
  --(1.819,1.450)--(1.825,1.454)--(1.832,1.457)--(1.838,1.461)--(1.844,1.464)--(1.850,1.468)%
  --(1.856,1.471)--(1.862,1.475)--(1.869,1.478)--(1.875,1.481)--(1.881,1.485)--(1.887,1.488)%
  --(1.893,1.492)--(1.899,1.495)--(1.905,1.499)--(1.912,1.502)--(1.918,1.506)--(1.924,1.509)%
  --(1.930,1.513)--(1.936,1.516)--(1.942,1.519)--(1.949,1.523)--(1.955,1.526)--(1.961,1.530)%
  --(1.967,1.533)--(1.973,1.537)--(1.979,1.540)--(1.985,1.544)--(1.992,1.547)--(1.998,1.550)%
  --(2.004,1.554)--(2.010,1.557)--(2.016,1.561)--(2.022,1.564)--(2.029,1.568)--(2.035,1.571)%
  --(2.041,1.575)--(2.047,1.578)--(2.053,1.581)--(2.059,1.585)--(2.065,1.588)--(2.072,1.592)%
  --(2.078,1.595)--(2.084,1.599)--(2.090,1.602)--(2.096,1.606)--(2.102,1.609)--(2.109,1.613)%
  --(2.115,1.616)--(2.121,1.619)--(2.127,1.623)--(2.133,1.626)--(2.139,1.630)--(2.146,1.633)%
  --(2.152,1.637)--(2.158,1.640)--(2.164,1.644)--(2.170,1.647)--(2.176,1.650)--(2.182,1.654)%
  --(2.189,1.657)--(2.195,1.661)--(2.201,1.664)--(2.207,1.668)--(2.213,1.671)--(2.219,1.675)%
  --(2.226,1.678)--(2.232,1.681)--(2.238,1.685)--(2.244,1.688)--(2.250,1.692)--(2.256,1.695)%
  --(2.262,1.699)--(2.269,1.702)--(2.275,1.706)--(2.281,1.709)--(2.287,1.713)--(2.293,1.716)%
  --(2.299,1.719)--(2.306,1.723)--(2.312,1.726)--(2.318,1.730)--(2.324,1.733)--(2.330,1.737)%
  --(2.336,1.740)--(2.342,1.744)--(2.349,1.747)--(2.355,1.750)--(2.361,1.754)--(2.367,1.757)%
  --(2.373,1.761)--(2.379,1.764)--(2.386,1.768)--(2.392,1.771)--(2.398,1.775)--(2.404,1.778)%
  --(2.410,1.781)--(2.416,1.785)--(2.422,1.788)--(2.429,1.792)--(2.435,1.795)--(2.441,1.799)%
  --(2.447,1.802)--(2.453,1.806)--(2.459,1.809)--(2.466,1.813)--(2.472,1.816)--(2.478,1.819)%
  --(2.484,1.823)--(2.490,1.826)--(2.496,1.830)--(2.503,1.833)--(2.509,1.837)--(2.515,1.840)%
  --(2.521,1.844)--(2.527,1.847)--(2.533,1.850)--(2.539,1.854)--(2.546,1.857)--(2.552,1.861)%
  --(2.558,1.864)--(2.564,1.868)--(2.570,1.871)--(2.576,1.875)--(2.583,1.878)--(2.589,1.881)%
  --(2.595,1.885)--(2.601,1.888)--(2.607,1.892)--(2.613,1.895)--(2.619,1.899)--(2.626,1.902)%
  --(2.632,1.906)--(2.638,1.909)--(2.644,1.912)--(2.650,1.916)--(2.656,1.919)--(2.663,1.923)%
  --(2.669,1.926)--(2.675,1.930)--(2.681,1.933)--(2.687,1.937)--(2.693,1.940)--(2.699,1.944)%
  --(2.706,1.947)--(2.712,1.950)--(2.718,1.954)--(2.724,1.957)--(2.730,1.961)--(2.736,1.964)%
  --(2.743,1.968)--(2.749,1.971)--(2.755,1.975)--(2.761,1.978)--(2.767,1.981)--(2.773,1.985)%
  --(2.780,1.988)--(2.786,1.992)--(2.792,1.995)--(2.798,1.999)--(2.804,2.002)--(2.810,2.006)%
  --(2.816,2.009)--(2.823,2.012)--(2.829,2.016)--(2.835,2.019)--(2.841,2.023)--(2.847,2.026)%
  --(2.853,2.030)--(2.860,2.033)--(2.866,2.037)--(2.872,2.040)--(2.878,2.044)--(2.884,2.047)%
  --(2.890,2.050)--(2.896,2.054)--(2.903,2.057)--(2.909,2.061)--(2.915,2.064)--(2.921,2.068)%
  --(2.927,2.071)--(2.933,2.075)--(2.940,2.078)--(2.946,2.081)--(2.952,2.085)--(2.958,2.088)%
  --(2.964,2.092)--(2.970,2.095)--(2.976,2.099)--(2.983,2.102)--(2.989,2.106)--(2.995,2.109)%
  --(3.001,2.112)--(3.007,2.116)--(3.013,2.119)--(3.020,2.123)--(3.026,2.126)--(3.032,2.130)%
  --(3.038,2.133)--(3.044,2.137)--(3.050,2.140)--(3.057,2.144)--(3.063,2.147)--(3.069,2.150)%
  --(3.075,2.154)--(3.081,2.157)--(3.087,2.161)--(3.093,2.164)--(3.100,2.168)--(3.106,2.171)%
  --(3.112,2.175)--(3.118,2.178)--(3.124,2.181)--(3.130,2.185)--(3.137,2.188)--(3.143,2.192)%
  --(3.149,2.195)--(3.155,2.199)--(3.161,2.202)--(3.167,2.206)--(3.173,2.209)--(3.180,2.212)%
  --(3.186,2.216)--(3.192,2.219)--(3.198,2.223)--(3.204,2.226)--(3.210,2.230)--(3.217,2.233)%
  --(3.223,2.237)--(3.229,2.240)--(3.235,2.244)--(3.241,2.247)--(3.247,2.250)--(3.253,2.254)%
  --(3.260,2.257)--(3.266,2.261)--(3.272,2.264)--(3.278,2.268)--(3.284,2.271)--(3.290,2.275)%
  --(3.297,2.278)--(3.303,2.281)--(3.309,2.285)--(3.315,2.288)--(3.321,2.292)--(3.327,2.295)%
  --(3.334,2.299)--(3.340,2.302)--(3.346,2.306)--(3.352,2.309)--(3.358,2.312)--(3.364,2.316)%
  --(3.370,2.319)--(3.377,2.323)--(3.383,2.326)--(3.389,2.330)--(3.395,2.333)--(3.401,2.337)%
  --(3.407,2.340)--(3.414,2.344)--(3.420,2.347)--(3.426,2.350)--(3.432,2.354)--(3.438,2.357)%
  --(3.444,2.361)--(3.450,2.364)--(3.457,2.368)--(3.463,2.371)--(3.469,2.375)--(3.475,2.378)%
  --(3.481,2.381)--(3.487,2.385)--(3.494,2.388)--(3.500,2.392)--(3.506,2.395)--(3.512,2.399)%
  --(3.518,2.402)--(3.524,2.406)--(3.530,2.409)--(3.537,2.412)--(3.543,2.416)--(3.549,2.419)%
  --(3.555,2.423)--(3.561,2.426)--(3.567,2.430)--(3.574,2.433)--(3.580,2.437)--(3.586,2.440)%
  --(3.592,2.444)--(3.598,2.447)--(3.604,2.450)--(3.611,2.454)--(3.617,2.457)--(3.623,2.461)%
  --(3.629,2.464)--(3.635,2.468)--(3.641,2.471)--(3.647,2.475)--(3.654,2.478)--(3.660,2.481)%
  --(3.666,2.485)--(3.672,2.488)--(3.678,2.492)--(3.684,2.495)--(3.691,2.499)--(3.697,2.502)%
  --(3.703,2.506)--(3.709,2.509)--(3.715,2.512)--(3.721,2.516)--(3.727,2.519)--(3.734,2.523)%
  --(3.740,2.526)--(3.746,2.530)--(3.752,2.533)--(3.758,2.537)--(3.764,2.540)--(3.771,2.544)%
  --(3.777,2.547)--(3.783,2.550)--(3.789,2.554)--(3.795,2.557)--(3.801,2.561)--(3.807,2.564)%
  --(3.814,2.568)--(3.820,2.571)--(3.826,2.575)--(3.832,2.578)--(3.838,2.581)--(3.844,2.585)%
  --(3.851,2.588)--(3.857,2.592)--(3.863,2.595)--(3.869,2.599)--(3.875,2.602)--(3.881,2.606)%
  --(3.888,2.609)--(3.894,2.612)--(3.900,2.616)--(3.906,2.619)--(3.912,2.623)--(3.918,2.626)%
  --(3.924,2.630)--(3.931,2.633)--(3.937,2.637)--(3.943,2.640)--(3.949,2.643)--(3.955,2.647)%
  --(3.961,2.650)--(3.968,2.654)--(3.974,2.657)--(3.980,2.661)--(3.986,2.664)--(3.992,2.668)%
  --(3.998,2.671)--(4.004,2.675)--(4.011,2.678)--(4.017,2.681)--(4.023,2.685)--(4.029,2.688)%
  --(4.035,2.692)--(4.041,2.695)--(4.048,2.699)--(4.054,2.702)--(4.060,2.706)--(4.066,2.709)%
  --(4.072,2.712)--(4.078,2.716)--(4.084,2.719)--(4.091,2.723)--(4.097,2.726)--(4.103,2.730)%
  --(4.109,2.733)--(4.115,2.737)--(4.121,2.740)--(4.128,2.743)--(4.134,2.747)--(4.140,2.750)%
  --(4.146,2.754)--(4.152,2.757)--(4.158,2.761)--(4.165,2.764)--(4.171,2.768)--(4.177,2.771)%
  --(4.183,2.775)--(4.189,2.778)--(4.195,2.781)--(4.201,2.785)--(4.208,2.788)--(4.214,2.792)%
  --(4.220,2.795)--(4.226,2.799)--(4.232,2.802)--(4.238,2.806)--(4.245,2.809)--(4.251,2.812)%
  --(4.257,2.816)--(4.263,2.819)--(4.269,2.823)--(4.275,2.826)--(4.281,2.830)--(4.288,2.833)%
  --(4.294,2.837)--(4.300,2.840)--(4.306,2.843)--(4.312,2.847)--(4.318,2.850)--(4.325,2.854)%
  --(4.331,2.857)--(4.337,2.861)--(4.343,2.864)--(4.349,2.868)--(4.355,2.871)--(4.361,2.875)%
  --(4.368,2.878)--(4.374,2.881)--(4.380,2.885)--(4.386,2.888)--(4.392,2.892)--(4.398,2.895)%
  --(4.405,2.899)--(4.411,2.902)--(4.417,2.906)--(4.423,2.909)--(4.429,2.912)--(4.435,2.916)%
  --(4.442,2.919)--(4.448,2.923)--(4.454,2.926)--(4.460,2.930)--(4.466,2.933)--(4.472,2.937)%
  --(4.478,2.940)--(4.485,2.943)--(4.491,2.947)--(4.497,2.950)--(4.503,2.954)--(4.509,2.957)%
  --(4.515,2.961)--(4.522,2.964)--(4.528,2.968)--(4.534,2.971)--(4.540,2.975)--(4.546,2.978)%
  --(4.552,2.981)--(4.558,2.985)--(4.565,2.988)--(4.571,2.992)--(4.577,2.995)--(4.583,2.999)%
  --(4.589,3.002)--(4.595,3.006)--(4.602,3.009)--(4.608,3.012)--(4.614,3.016)--(4.620,3.019)%
  --(4.626,3.023)--(4.632,3.026)--(4.638,3.030)--(4.645,3.033)--(4.651,3.037)--(4.657,3.040)%
  --(4.663,3.043)--(4.669,3.047)--(4.675,3.050)--(4.682,3.054)--(4.688,3.057)--(4.694,3.061)%
  --(4.700,3.064)--(4.706,3.068)--(4.712,3.071)--(4.719,3.075)--(4.725,3.078)--(4.731,3.081)%
  --(4.737,3.085)--(4.743,3.088)--(4.749,3.092)--(4.755,3.095)--(4.762,3.099)--(4.768,3.102)%
  --(4.774,3.106)--(4.780,3.109)--(4.786,3.112)--(4.792,3.116)--(4.799,3.119)--(4.805,3.123)%
  --(4.811,3.126)--(4.817,3.130)--(4.823,3.133)--(4.829,3.137)--(4.835,3.140)--(4.842,3.143)%
  --(4.848,3.147)--(4.854,3.150)--(4.860,3.154)--(4.866,3.157)--(4.872,3.161)--(4.879,3.164)%
  --(4.885,3.168)--(4.891,3.171)--(4.897,3.175)--(4.903,3.178)--(4.909,3.181)--(4.915,3.185)%
  --(4.922,3.188)--(4.928,3.192)--(4.934,3.195)--(4.940,3.199)--(4.946,3.202)--(4.952,3.206)%
  --(4.959,3.209)--(4.965,3.212)--(4.971,3.216)--(4.977,3.219)--(4.983,3.223)--(4.989,3.226)%
  --(4.995,3.230)--(5.002,3.233)--(5.008,3.237)--(5.014,3.240)--(5.020,3.243)--(5.026,3.247)%
  --(5.032,3.250)--(5.039,3.254)--(5.045,3.257)--(5.051,3.261)--(5.057,3.264)--(5.063,3.268)%
  --(5.069,3.271)--(5.076,3.275)--(5.082,3.278)--(5.088,3.281)--(5.094,3.285)--(5.100,3.288)%
  --(5.106,3.292)--(5.112,3.295)--(5.119,3.299)--(5.125,3.302)--(5.131,3.306)--(5.137,3.309)%
  --(5.143,3.312)--(5.149,3.316)--(5.156,3.319)--(5.162,3.323)--(5.168,3.326)--(5.174,3.330)%
  --(5.180,3.333)--(5.186,3.337)--(5.192,3.340)--(5.199,3.343)--(5.205,3.347)--(5.211,3.350)%
  --(5.217,3.354)--(5.223,3.357)--(5.229,3.361)--(5.236,3.364)--(5.242,3.368)--(5.248,3.371)%
  --(5.254,3.374)--(5.260,3.378)--(5.266,3.381)--(5.272,3.385)--(5.279,3.388)--(5.285,3.392)%
  --(5.291,3.395)--(5.297,3.399)--(5.303,3.402)--(5.309,3.406)--(5.316,3.409)--(5.322,3.412)%
  --(5.328,3.416)--(5.334,3.419)--(5.340,3.423)--(5.346,3.426)--(5.353,3.430)--(5.359,3.433)%
  --(5.365,3.437)--(5.371,3.440)--(5.377,3.443)--(5.383,3.447)--(5.389,3.450)--(5.396,3.454)%
  --(5.402,3.457)--(5.408,3.461)--(5.414,3.464)--(5.420,3.468)--(5.426,3.471)--(5.433,3.474)%
  --(5.439,3.478)--(5.445,3.481)--(5.451,3.485)--(5.457,3.488)--(5.463,3.492)--(5.469,3.495)%
  --(5.476,3.499)--(5.482,3.502)--(5.488,3.506)--(5.494,3.509)--(5.500,3.512)--(5.506,3.516)%
  --(5.513,3.519)--(5.519,3.523)--(5.525,3.526)--(5.531,3.530)--(5.537,3.533)--(5.543,3.537)%
  --(5.549,3.540)--(5.556,3.543)--(5.562,3.547)--(5.568,3.550)--(5.574,3.554)--(5.580,3.557)%
  --(5.586,3.561)--(5.593,3.564)--(5.599,3.568)--(5.605,3.571)--(5.611,3.574)--(5.617,3.578)%
  --(5.623,3.581)--(5.630,3.585)--(5.636,3.588)--(5.642,3.592)--(5.648,3.595)--(5.654,3.599)%
  --(5.660,3.602)--(5.666,3.606)--(5.673,3.609)--(5.679,3.612)--(5.685,3.616)--(5.691,3.619)%
  --(5.697,3.623)--(5.703,3.626)--(5.710,3.630)--(5.716,3.633)--(5.722,3.637)--(5.728,3.640)%
  --(5.734,3.643)--(5.740,3.647)--(5.746,3.650)--(5.753,3.654)--(5.759,3.657)--(5.765,3.661)%
  --(5.771,3.664)--(5.777,3.668)--(5.783,3.671)--(5.790,3.674)--(5.796,3.678)--(5.802,3.681)%
  --(5.808,3.685)--(5.814,3.688)--(5.820,3.692)--(5.826,3.695)--(5.833,3.699)--(5.839,3.702)%
  --(5.845,3.706)--(5.851,3.709)--(5.857,3.712)--(5.863,3.716)--(5.870,3.719)--(5.876,3.723)%
  --(5.882,3.726)--(5.888,3.730)--(5.894,3.733)--(5.900,3.737)--(5.907,3.740)--(5.913,3.743)%
  --(5.919,3.747)--(5.925,3.750)--(5.931,3.754)--(5.937,3.757)--(5.943,3.761)--(5.950,3.764)%
  --(5.956,3.768)--(5.962,3.771)--(5.968,3.774)--(5.974,3.778)--(5.980,3.781)--(5.987,3.785)%
  --(5.993,3.788)--(5.999,3.792)--(6.005,3.795)--(6.011,3.799)--(6.017,3.802)--(6.023,3.806)%
  --(6.030,3.809)--(6.036,3.812)--(6.042,3.816)--(6.048,3.819)--(6.054,3.823)--(6.060,3.826)%
  --(6.067,3.830)--(6.073,3.833)--(6.079,3.837)--(6.085,3.840)--(6.091,3.843)--(6.097,3.847)%
  --(6.103,3.850)--(6.110,3.854)--(6.116,3.857)--(6.122,3.861)--(6.128,3.864)--(6.134,3.868)%
  --(6.140,3.871)--(6.147,3.874)--(6.153,3.878)--(6.159,3.881)--(6.165,3.885)--(6.171,3.888)%
  --(6.177,3.892)--(6.184,3.895)--(6.190,3.899)--(6.196,3.902)--(6.202,3.906)--(6.208,3.909)%
  --(6.214,3.912)--(6.220,3.916)--(6.227,3.919)--(6.233,3.923)--(6.239,3.926)--(6.245,3.930)%
  --(6.251,3.933)--(6.257,3.937)--(6.264,3.940)--(6.270,3.943)--(6.276,3.947)--(6.282,3.950)%
  --(6.288,3.954)--(6.294,3.957)--(6.300,3.961)--(6.307,3.964)--(6.313,3.968)--(6.319,3.971)%
  --(6.325,3.974)--(6.331,3.978)--(6.337,3.981)--(6.344,3.985)--(6.350,3.988)--(6.356,3.992)%
  --(6.362,3.995)--(6.368,3.999)--(6.374,4.002)--(6.380,4.005)--(6.387,4.009)--(6.393,4.012)%
  --(6.399,4.016)--(6.405,4.019)--(6.411,4.023)--(6.417,4.026)--(6.424,4.030)--(6.430,4.033)%
  --(6.436,4.037)--(6.442,4.040)--(6.448,4.043)--(6.454,4.047)--(6.461,4.050)--(6.467,4.054)%
  --(6.473,4.057)--(6.479,4.061)--(6.485,4.064)--(6.491,4.068)--(6.497,4.071)--(6.504,4.074)%
  --(6.510,4.078)--(6.516,4.081)--(6.522,4.085)--(6.528,4.088)--(6.534,4.092)--(6.541,4.095)%
  --(6.547,4.099)--(6.553,4.102)--(6.559,4.105)--(6.565,4.109)--(6.571,4.112)--(6.577,4.116)%
  --(6.584,4.119)--(6.590,4.123)--(6.596,4.126)--(6.602,4.130)--(6.608,4.133)--(6.614,4.137)%
  --(6.621,4.140)--(6.627,4.143)--(6.633,4.147)--(6.639,4.150)--(6.645,4.154)--(6.651,4.157)%
  --(6.657,4.161)--(6.664,4.164)--(6.670,4.168)--(6.676,4.171)--(6.682,4.174)--(6.688,4.178)%
  --(6.694,4.181)--(6.701,4.185)--(6.707,4.188)--(6.713,4.192)--(6.719,4.195)--(6.725,4.199)%
  --(6.731,4.202)--(6.738,4.205)--(6.744,4.209)--(6.750,4.212)--(6.756,4.216)--(6.762,4.219)%
  --(6.768,4.223)--(6.774,4.226)--(6.781,4.230)--(6.787,4.233)--(6.793,4.237)--(6.799,4.240)%
  --(6.805,4.243)--(6.811,4.247)--(6.818,4.250)--(6.824,4.254)--(6.830,4.257)--(6.836,4.261)%
  --(6.842,4.264)--(6.848,4.268)--(6.854,4.271)--(6.861,4.274)--(6.867,4.278)--(6.873,4.281)%
  --(6.879,4.285)--(6.885,4.288)--(6.891,4.292)--(6.898,4.295)--(6.904,4.299)--(6.910,4.302)%
  --(6.916,4.305)--(6.922,4.309)--(6.928,4.312)--(6.934,4.316)--(6.941,4.319)--(6.947,4.323)%
  --(6.953,4.326)--(6.959,4.330)--(6.965,4.333)--(6.971,4.337)--(6.978,4.340)--(6.984,4.343)%
  --(6.990,4.347)--(6.996,4.350)--(7.002,4.354)--(7.008,4.357)--(7.015,4.361)--(7.021,4.364)%
  --(7.027,4.368)--(7.033,4.371)--(7.039,4.374)--(7.045,4.378)--(7.051,4.381)--(7.058,4.385)%
  --(7.064,4.388)--(7.070,4.392)--(7.076,4.395)--(7.082,4.399)--(7.088,4.402)--(7.095,4.405)%
  --(7.101,4.409)--(7.107,4.412)--(7.113,4.416)--(7.119,4.419)--(7.125,4.423)--(7.131,4.426)%
  --(7.138,4.430)--(7.144,4.433)--(7.150,4.437)--(7.156,4.440)--(7.162,4.443)--(7.168,4.447)%
  --(7.175,4.450)--(7.181,4.454)--(7.187,4.457)--(7.193,4.461)--(7.199,4.464)--(7.205,4.468)%
  --(7.211,4.471)--(7.218,4.474)--(7.224,4.478)--(7.230,4.481)--(7.236,4.485)--(7.242,4.488)%
  --(7.248,4.492)--(7.255,4.495)--(7.261,4.499)--(7.267,4.502)--(7.273,4.505)--(7.279,4.509)%
  --(7.285,4.512)--(7.292,4.516)--(7.298,4.519)--(7.304,4.523)--(7.310,4.526)--(7.316,4.530)%
  --(7.322,4.533)--(7.328,4.537)--(7.335,4.540)--(7.341,4.543)--(7.347,4.547)--(7.353,4.550)%
  --(7.359,4.554)--(7.365,4.557)--(7.372,4.561)--(7.378,4.564)--(7.384,4.568)--(7.390,4.571)%
  --(7.396,4.574)--(7.402,4.578)--(7.408,4.581)--(7.415,4.585)--(7.421,4.588)--(7.427,4.592)%
  --(7.433,4.595)--(7.439,4.599)--(7.445,4.602)--(7.452,4.605)--(7.458,4.609)--(7.464,4.612)%
  --(7.470,4.616)--(7.476,4.619)--(7.482,4.623)--(7.488,4.626)--(7.495,4.630)--(7.501,4.633)%
  --(7.507,4.637)--(7.513,4.640)--(7.519,4.643)--(7.525,4.647)--(7.532,4.650)--(7.538,4.654)%
  --(7.544,4.657)--(7.550,4.661)--(7.556,4.664)--(7.562,4.668)--(7.568,4.671)--(7.575,4.674)%
  --(7.581,4.678)--(7.587,4.681)--(7.593,4.685)--(7.599,4.688)--(7.605,4.692)--(7.612,4.695)%
  --(7.618,4.699)--(7.624,4.702)--(7.630,4.705)--(7.636,4.709)--(7.642,4.712)--(7.649,4.716)%
  --(7.655,4.719)--(7.661,4.723)--(7.667,4.726)--(7.673,4.730)--(7.679,4.733)--(7.685,4.736)%
  --(7.692,4.740)--(7.698,4.743)--(7.704,4.747)--(7.710,4.750)--(7.716,4.754)--(7.722,4.757)%
  --(7.729,4.761)--(7.735,4.764)--(7.741,4.768)--(7.747,4.771)--(7.753,4.774)--(7.759,4.778)%
  --(7.765,4.781)--(7.772,4.785)--(7.778,4.788)--(7.784,4.792)--(7.790,4.795)--(7.796,4.799)%
  --(7.802,4.802)--(7.809,4.805)--(7.815,4.809)--(7.821,4.812)--(7.827,4.816)--(7.833,4.819)%
  --(7.839,4.823)--(7.845,4.826)--(7.852,4.830)--(7.858,4.833)--(7.864,4.836)--(7.870,4.840)%
  --(7.876,4.843)--(7.882,4.847)--(7.889,4.850)--(7.895,4.854)--(7.901,4.857)--(7.907,4.861)%
  --(7.913,4.864)--(7.919,4.868)--(7.926,4.871)--(7.932,4.874)--(7.938,4.878)--(7.944,4.881)%
  --(7.950,4.885)--(7.956,4.888)--(7.962,4.892)--(7.969,4.895)--(7.975,4.899)--(7.981,4.902)%
  --(7.987,4.905)--(7.993,4.909)--(7.999,4.912)--(8.006,4.916)--(8.012,4.919)--(8.018,4.923)%
  --(8.024,4.926)--(8.030,4.930)--(8.036,4.933)--(8.042,4.936)--(8.049,4.940)--(8.055,4.943)%
  --(8.061,4.947)--(8.067,4.950)--(8.073,4.954)--(8.079,4.957)--(8.086,4.961)--(8.092,4.964)%
  --(8.098,4.968)--(8.104,4.971)--(8.110,4.974)--(8.116,4.978)--(8.122,4.981)--(8.129,4.985)%
  --(8.135,4.988)--(8.141,4.992)--(8.147,4.995)--(8.153,4.999)--(8.159,5.002)--(8.166,5.005)%
  --(8.172,5.009)--(8.178,5.012)--(8.184,5.016)--(8.190,5.019)--(8.196,5.023)--(8.203,5.026)%
  --(8.209,5.030)--(8.215,5.033)--(8.221,5.036)--(8.227,5.040)--(8.233,5.043)--(8.239,5.047)%
  --(8.246,5.050)--(8.252,5.054)--(8.258,5.057)--(8.264,5.061)--(8.270,5.064)--(8.276,5.068)%
  --(8.283,5.071)--(8.289,5.074)--(8.295,5.078)--(8.301,5.081)--(8.307,5.085)--(8.313,5.088)%
  --(8.319,5.092)--(8.326,5.095)--(8.332,5.099)--(8.338,5.102)--(8.344,5.105)--(8.350,5.109)%
  --(8.356,5.112)--(8.363,5.116)--(8.369,5.119)--(8.375,5.123)--(8.381,5.126)--(8.387,5.130)%
  --(8.393,5.133)--(8.399,5.136)--(8.406,5.140)--(8.412,5.143)--(8.418,5.147)--(8.424,5.150)%
  --(8.430,5.154)--(8.436,5.157)--(8.443,5.161)--(8.449,5.164)--(8.455,5.168)--(8.461,5.171)%
  --(8.467,5.174)--(8.473,5.178)--(8.480,5.181)--(8.486,5.185)--(8.492,5.188)--(8.498,5.192)%
  --(8.504,5.195)--(8.510,5.199)--(8.516,5.202)--(8.523,5.205)--(8.529,5.209)--(8.535,5.212)%
  --(8.541,5.216)--(8.547,5.219)--(8.553,5.223)--(8.560,5.226)--(8.566,5.230)--(8.572,5.233)%
  --(8.578,5.236)--(8.584,5.240)--(8.590,5.243)--(8.596,5.247)--(8.603,5.250)--(8.609,5.254)%
  --(8.615,5.257)--(8.621,5.261)--(8.627,5.264)--(8.633,5.268)--(8.640,5.271)--(8.646,5.274)%
  --(8.652,5.278)--(8.658,5.281)--(8.664,5.285)--(8.670,5.288)--(8.676,5.292)--(8.683,5.295)%
  --(8.689,5.299)--(8.695,5.302)--(8.701,5.305)--(8.707,5.309)--(8.713,5.312)--(8.720,5.316)%
  --(8.726,5.319)--(8.732,5.323)--(8.738,5.326)--(8.744,5.330)--(8.750,5.333)--(8.757,5.336)%
  --(8.763,5.340)--(8.769,5.343)--(8.775,5.347)--(8.781,5.350)--(8.787,5.354)--(8.793,5.357)%
  --(8.800,5.361)--(8.806,5.364)--(8.812,5.368)--(8.818,5.371)--(8.824,5.374)--(8.830,5.378)%
  --(8.837,5.381)--(8.843,5.385)--(8.849,5.388)--(8.855,5.392)--(8.861,5.395)--(8.867,5.399)%
  --(8.873,5.402)--(8.880,5.405)--(8.886,5.409)--(8.892,5.412)--(8.898,5.416)--(8.904,5.419)%
  --(8.910,5.423)--(8.917,5.426)--(8.923,5.430)--(8.929,5.433)--(8.935,5.436)--(8.941,5.440)%
  --(8.947,5.443)--(8.953,5.447)--(8.960,5.450)--(8.966,5.454)--(8.972,5.457)--(8.978,5.461)%
  --(8.984,5.464)--(8.990,5.467)--(8.997,5.471)--(9.003,5.474)--(9.009,5.478)--(9.015,5.481)%
  --(9.021,5.485)--(9.027,5.488)--(9.034,5.492)--(9.040,5.495)--(9.046,5.499)--(9.052,5.502)%
  --(9.058,5.505)--(9.064,5.509)--(9.070,5.512)--(9.077,5.516)--(9.083,5.519)--(9.089,5.523)%
  --(9.095,5.526)--(9.101,5.530)--(9.107,5.533)--(9.114,5.536)--(9.120,5.540)--(9.126,5.543)%
  --(9.132,5.547)--(9.138,5.550)--(9.144,5.554)--(9.150,5.557)--(9.157,5.561)--(9.163,5.564)%
  --(9.169,5.567)--(9.175,5.571)--(9.181,5.574)--(9.187,5.578)--(9.194,5.581)--(9.200,5.585)%
  --(9.206,5.588)--(9.212,5.592)--(9.218,5.595)--(9.224,5.599)--(9.230,5.602)--(9.237,5.605)%
  --(9.243,5.609)--(9.249,5.612)--(9.255,5.616)--(9.261,5.619)--(9.267,5.623)--(9.274,5.626)%
  --(9.280,5.630)--(9.286,5.633)--(9.292,5.636)--(9.298,5.640)--(9.304,5.643)--(9.311,5.647)%
  --(9.317,5.650)--(9.323,5.654)--(9.329,5.657)--(9.335,5.661)--(9.341,5.664)--(9.347,5.667)%
  --(9.354,5.671)--(9.360,5.674)--(9.366,5.678)--(9.372,5.681)--(9.378,5.685)--(9.384,5.688)%
  --(9.391,5.692)--(9.397,5.695)--(9.403,5.699)--(9.409,5.702)--(9.415,5.705)--(9.421,5.709)%
  --(9.427,5.712)--(9.434,5.716)--(9.440,5.719)--(9.446,5.723)--(9.452,5.726)--(9.458,5.730)%
  --(9.464,5.733)--(9.471,5.736)--(9.477,5.740)--(9.483,5.743)--(9.489,5.747)--(9.495,5.750)%
  --(9.501,5.754)--(9.507,5.757)--(9.514,5.761)--(9.520,5.764)--(9.526,5.767)--(9.532,5.771)%
  --(9.538,5.774)--(9.544,5.778)--(9.551,5.781)--(9.557,5.785)--(9.563,5.788)--(9.569,5.792)%
  --(9.575,5.795)--(9.581,5.799)--(9.588,5.802)--(9.594,5.805)--(9.600,5.809)--(9.606,5.812)%
  --(9.612,5.816)--(9.618,5.819)--(9.624,5.823)--(9.631,5.826)--(9.637,5.830)--(9.643,5.833)%
  --(9.649,5.836)--(9.655,5.840)--(9.661,5.843)--(9.668,5.847)--(9.674,5.850)--(9.680,5.854)%
  --(9.686,5.857)--(9.692,5.861)--(9.698,5.864)--(9.704,5.867)--(9.711,5.871)--(9.717,5.874)%
  --(9.723,5.878)--(9.729,5.881)--(9.735,5.885)--(9.741,5.888)--(9.748,5.892)--(9.754,5.895)%
  --(9.760,5.899)--(9.766,5.902)--(9.772,5.905)--(9.778,5.909)--(9.784,5.912)--(9.791,5.916)%
  --(9.797,5.919)--(9.803,5.923)--(9.809,5.926)--(9.815,5.930)--(9.821,5.933)--(9.828,5.936)%
  --(9.834,5.940)--(9.840,5.943)--(9.846,5.947)--(9.852,5.950)--(9.858,5.954)--(9.864,5.957)%
  --(9.871,5.961)--(9.877,5.964)--(9.883,5.967)--(9.889,5.971)--(9.895,5.974)--(9.901,5.978)%
  --(9.908,5.981)--(9.914,5.985)--(9.920,5.988)--(9.926,5.992)--(9.932,5.995)--(9.938,5.999)%
  --(9.945,6.002)--(9.951,6.005)--(9.957,6.009)--(9.963,6.012)--(9.969,6.016)--(9.975,6.019)%
  --(9.981,6.023)--(9.988,6.026)--(9.994,6.030)--(10.000,6.033)--(10.006,6.036)--(10.012,6.040)%
  --(10.018,6.043)--(10.025,6.047)--(10.031,6.050)--(10.037,6.054)--(10.043,6.057)--(10.049,6.061)%
  --(10.055,6.064)--(10.061,6.067)--(10.068,6.071)--(10.074,6.074)--(10.080,6.078)--(10.086,6.081)%
  --(10.092,6.085)--(10.098,6.088)--(10.105,6.092)--(10.111,6.095)--(10.117,6.099)--(10.123,6.102)%
  --(10.129,6.105)--(10.135,6.109)--(10.141,6.112)--(10.148,6.116)--(10.154,6.119)--(10.160,6.123)%
  --(10.166,6.126)--(10.172,6.130)--(10.178,6.133)--(10.185,6.136)--(10.191,6.140)--(10.197,6.143)%
  --(10.203,6.147)--(10.209,6.150)--(10.215,6.154)--(10.222,6.157)--(10.228,6.161)--(10.234,6.164)%
  --(10.240,6.167)--(10.246,6.171)--(10.252,6.174)--(10.258,6.178)--(10.265,6.181)--(10.271,6.185)%
  --(10.277,6.188)--(10.283,6.192)--(10.289,6.195)--(10.295,6.198)--(10.302,6.202)--(10.308,6.205)%
  --(10.314,6.209)--(10.320,6.212)--(10.326,6.216)--(10.332,6.219)--(10.338,6.223)--(10.345,6.226)%
  --(10.351,6.230)--(10.357,6.233)--(10.363,6.236)--(10.369,6.240)--(10.375,6.243)--(10.382,6.247)%
  --(10.388,6.250)--(10.394,6.254)--(10.400,6.257)--(10.406,6.261)--(10.412,6.264)--(10.418,6.267)%
  --(10.425,6.271)--(10.431,6.274)--(10.437,6.278)--(10.443,6.281)--(10.449,6.285)--(10.455,6.288)%
  --(10.462,6.292)--(10.468,6.295)--(10.474,6.298)--(10.480,6.302)--(10.486,6.305)--(10.492,6.309)%
  --(10.499,6.312)--(10.505,6.316)--(10.511,6.319)--(10.517,6.323)--(10.523,6.326)--(10.529,6.330)%
  --(10.535,6.333)--(10.542,6.336)--(10.548,6.340)--(10.554,6.343)--(10.560,6.347)--(10.566,6.350)%
  --(10.572,6.354)--(10.579,6.357)--(10.585,6.361)--(10.591,6.364)--(10.597,6.367)--(10.603,6.371)%
  --(10.609,6.374)--(10.615,6.378)--(10.622,6.381)--(10.628,6.385)--(10.634,6.388)--(10.640,6.392)%
  --(10.646,6.395)--(10.652,6.398)--(10.659,6.402)--(10.665,6.405)--(10.671,6.409)--(10.677,6.412)%
  --(10.683,6.416)--(10.689,6.419)--(10.695,6.423)--(10.702,6.426)--(10.708,6.430)--(10.714,6.433)%
  --(10.720,6.436)--(10.726,6.440)--(10.732,6.443)--(10.739,6.447)--(10.745,6.450)--(10.751,6.454)%
  --(10.757,6.457)--(10.763,6.461)--(10.769,6.464)--(10.776,6.467)--(10.782,6.471)--(10.788,6.474)%
  --(10.794,6.478)--(10.800,6.481)--(10.806,6.485)--(10.812,6.488)--(10.819,6.492)--(10.825,6.495)%
  --(10.831,6.498)--(10.837,6.502)--(10.843,6.505)--(10.849,6.509)--(10.856,6.512)--(10.862,6.516)%
  --(10.868,6.519)--(10.874,6.523)--(10.880,6.526)--(10.886,6.530)--(10.892,6.533)--(10.899,6.536)%
  --(10.905,6.540)--(10.911,6.543)--(10.917,6.547)--(10.923,6.550)--(10.929,6.554)--(10.936,6.557)%
  --(10.942,6.561)--(10.948,6.564)--(10.954,6.567)--(10.960,6.571)--(10.966,6.574)--(10.972,6.578)%
  --(10.979,6.581)--(10.985,6.585)--(10.991,6.588)--(10.997,6.592)--(11.003,6.595)--(11.009,6.598)%
  --(11.016,6.602)--(11.022,6.605)--(11.028,6.609)--(11.034,6.612)--(11.040,6.616)--(11.046,6.619)%
  --(11.053,6.623)--(11.059,6.626)--(11.065,6.630)--(11.071,6.633)--(11.077,6.636)--(11.083,6.640)%
  --(11.089,6.643)--(11.096,6.647)--(11.102,6.650)--(11.108,6.654)--(11.114,6.657)--(11.120,6.661)%
  --(11.126,6.664)--(11.133,6.667)--(11.139,6.671)--(11.145,6.674)--(11.151,6.678)--(11.157,6.681)%
  --(11.163,6.685)--(11.169,6.688)--(11.176,6.692)--(11.182,6.695)--(11.188,6.698)--(11.194,6.702)%
  --(11.200,6.705)--(11.206,6.709)--(11.213,6.712)--(11.219,6.716)--(11.225,6.719)--(11.231,6.723)%
  --(11.237,6.726)--(11.243,6.730)--(11.249,6.733)--(11.256,6.736)--(11.262,6.740)--(11.268,6.743)%
  --(11.274,6.747)--(11.280,6.750)--(11.286,6.754)--(11.293,6.757)--(11.299,6.761)--(11.305,6.764)%
  --(11.311,6.767)--(11.317,6.771)--(11.323,6.774)--(11.330,6.778)--(11.336,6.781)--(11.342,6.785)%
  --(11.348,6.788)--(11.354,6.792)--(11.360,6.795)--(11.366,6.798)--(11.373,6.802)--(11.379,6.805)%
  --(11.385,6.809)--(11.391,6.812)--(11.397,6.816)--(11.403,6.819)--(11.410,6.823)--(11.416,6.826)%
  --(11.422,6.830)--(11.428,6.833)--(11.434,6.836)--(11.440,6.840)--(11.446,6.843)--(11.453,6.847)%
  --(11.459,6.850)--(11.465,6.854)--(11.471,6.857)--(11.477,6.861)--(11.483,6.864)--(11.490,6.867)%
  --(11.496,6.871)--(11.502,6.874)--(11.508,6.878)--(11.514,6.881)--(11.520,6.885)--(11.526,6.888)%
  --(11.533,6.892)--(11.539,6.895)--(11.545,6.898)--(11.551,6.902)--(11.557,6.905)--(11.563,6.909)%
  --(11.570,6.912)--(11.576,6.916)--(11.582,6.919)--(11.588,6.923)--(11.594,6.926)--(11.600,6.929)%
  --(11.607,6.933)--(11.613,6.936)--(11.619,6.940)--(11.625,6.943)--(11.631,6.947)--(11.637,6.950)%
  --(11.643,6.954)--(11.650,6.957)--(11.656,6.961)--(11.662,6.964)--(11.668,6.967)--(11.674,6.971)%
  --(11.680,6.974)--(11.687,6.978)--(11.693,6.981)--(11.699,6.985)--(11.705,6.988)--(11.711,6.992)%
  --(11.717,6.995)--(11.723,6.998)--(11.730,7.002)--(11.736,7.005)--(11.742,7.009)--(11.748,7.012)%
  --(11.754,7.016)--(11.760,7.019)--(11.767,7.023)--(11.773,7.026)--(11.779,7.029)--(11.785,7.033)%
  --(11.791,7.036)--(11.797,7.040)--(11.803,7.043)--(11.810,7.047)--(11.816,7.050)--(11.822,7.054)%
  --(11.828,7.057)--(11.834,7.061)--(11.840,7.064)--(11.847,7.067)--(11.853,7.071)--(11.859,7.074)%
  --(11.865,7.078)--(11.871,7.081)--(11.877,7.085)--(11.884,7.088)--(11.890,7.092)--(11.896,7.095)%
  --(11.902,7.098)--(11.908,7.102)--(11.914,7.105)--(11.920,7.109)--(11.927,7.112)--(11.933,7.116)%
  --(11.939,7.119)--(11.945,7.123)--(11.951,7.126)--(11.957,7.129)--(11.964,7.133)--(11.970,7.136)%
  --(11.976,7.140)--(11.982,7.143)--(11.988,7.147)--(11.994,7.150)--(12.000,7.154)--(12.007,7.157)%
  --(12.013,7.161)--(12.019,7.164)--(12.025,7.167)--(12.031,7.171)--(12.037,7.174)--(12.044,7.178)%
  --(12.050,7.181)--(12.056,7.185)--(12.062,7.188)--(12.068,7.192)--(12.074,7.195)--(12.080,7.198)%
  --(12.087,7.202)--(12.093,7.205)--(12.099,7.209)--(12.105,7.212)--(12.111,7.216)--(12.117,7.219)%
  --(12.124,7.223)--(12.130,7.226)--(12.136,7.229)--(12.142,7.233)--(12.148,7.236)--(12.154,7.240)%
  --(12.161,7.243)--(12.167,7.247)--(12.173,7.250)--(12.179,7.254)--(12.185,7.257)--(12.191,7.261)%
  --(12.197,7.264)--(12.204,7.267)--(12.210,7.271)--(12.216,7.274)--(12.222,7.278)--(12.228,7.281)%
  --(12.234,7.285)--(12.241,7.288)--(12.247,7.292)--(12.253,7.295)--(12.259,7.298)--(12.265,7.302)%
  --(12.271,7.305)--(12.277,7.309)--(12.284,7.312)--(12.290,7.316)--(12.296,7.319)--(12.302,7.323)%
  --(12.308,7.326)--(12.314,7.329)--(12.321,7.333)--(12.327,7.336)--(12.333,7.340)--(12.339,7.343)%
  --(12.345,7.347)--(12.351,7.350)--(12.357,7.354)--(12.364,7.357)--(12.370,7.361)--(12.376,7.364)%
  --(12.382,7.367)--(12.388,7.371)--(12.394,7.374)--(12.401,7.378)--(12.407,7.381)--(12.413,7.385)%
  --(12.419,7.388)--(12.425,7.392)--(12.431,7.395)--(12.437,7.398)--(12.444,7.402)--(12.450,7.405)%
  --(12.456,7.409)--(12.462,7.412)--(12.468,7.416)--(12.474,7.419)--(12.481,7.423)--(12.487,7.426)%
  --(12.493,7.429)--(12.499,7.433)--(12.505,7.436)--(12.511,7.440)--(12.518,7.443)--(12.524,7.447)%
  --(12.530,7.450)--(12.536,7.454)--(12.542,7.457)--(12.548,7.461)--(12.554,7.464)--(12.561,7.467)%
  --(12.567,7.471)--(12.573,7.474)--(12.579,7.478)--(12.585,7.481)--(12.591,7.485)--(12.598,7.488)%
  --(12.604,7.492)--(12.610,7.495)--(12.616,7.498)--(12.622,7.502)--(12.628,7.505)--(12.634,7.509)%
  --(12.641,7.512)--(12.647,7.516)--(12.653,7.519)--(12.659,7.523)--(12.665,7.526)--(12.671,7.529)%
  --(12.678,7.533)--(12.684,7.536)--(12.690,7.540)--(12.696,7.543)--(12.702,7.547)--(12.708,7.550)%
  --(12.714,7.554)--(12.721,7.557)--(12.727,7.561)--(12.733,7.564)--(12.739,7.567)--(12.745,7.571)%
  --(12.751,7.574)--(12.758,7.578)--(12.764,7.581)--(12.770,7.585)--(12.776,7.588)--(12.782,7.592)%
  --(12.788,7.595)--(12.795,7.598);
\gpcolor{color=gp lt color border}
\draw[gp path] (1.136,8.631)--(1.136,0.985)--(13.447,0.985)--(13.447,8.631)--cycle;
%% coordinates of the plot area
\gpdefrectangularnode{gp plot 1}{\pgfpoint{1.136cm}{0.985cm}}{\pgfpoint{13.447cm}{8.631cm}}
\end{tikzpicture}
%% gnuplot variables

\end{figure*}

\begin{figure*}
\centering
\forceversofloat
\caption{Gráfico do mesmo conjunto de pontos da Figura~\ref{Fig:ParabolaReta}, juntamente com diversos outros pertencentes ao mesmo conjunto de dados. Verifique que a tendência linear aparente no primeiro gráfico já não é mais razoável. De fato, os dados correspondem a uma distribuição em torno de uma parábola.}
\label{Fig:ParabolaReta2}
\begin{tikzpicture}[gnuplot]
%% generated with GNUPLOT 5.0p6 (Lua 5.3; terminal rev. 99, script rev. 100)
%% seg 30 jul 2018 13:30:47 -03
\path (0.000,0.000) rectangle (14.000,9.000);
\gpcolor{color=gp lt color border}
\gpsetlinetype{gp lt border}
\gpsetdashtype{gp dt solid}
\gpsetlinewidth{1.00}
\draw[gp path] (1.688,1.317)--(1.868,1.317);
\draw[gp path] (13.447,1.317)--(13.267,1.317);
\node[gp node right] at (1.504,1.317) {0.0};
\draw[gp path] (1.688,2.647)--(1.868,2.647);
\draw[gp path] (13.447,2.647)--(13.267,2.647);
\node[gp node right] at (1.504,2.647) {20.0};
\draw[gp path] (1.688,3.977)--(1.868,3.977);
\draw[gp path] (13.447,3.977)--(13.267,3.977);
\node[gp node right] at (1.504,3.977) {40.0};
\draw[gp path] (1.688,5.307)--(1.868,5.307);
\draw[gp path] (13.447,5.307)--(13.267,5.307);
\node[gp node right] at (1.504,5.307) {60.0};
\draw[gp path] (1.688,6.636)--(1.868,6.636);
\draw[gp path] (13.447,6.636)--(13.267,6.636);
\node[gp node right] at (1.504,6.636) {80.0};
\draw[gp path] (1.688,7.966)--(1.868,7.966);
\draw[gp path] (13.447,7.966)--(13.267,7.966);
\node[gp node right] at (1.504,7.966) {100.0};
\draw[gp path] (1.938,0.985)--(1.938,1.165);
\draw[gp path] (1.938,8.631)--(1.938,8.451);
\node[gp node center] at (1.938,0.677) {$0$};
\draw[gp path] (3.189,0.985)--(3.189,1.165);
\draw[gp path] (3.189,8.631)--(3.189,8.451);
\node[gp node center] at (3.189,0.677) {$1$};
\draw[gp path] (4.440,0.985)--(4.440,1.165);
\draw[gp path] (4.440,8.631)--(4.440,8.451);
\node[gp node center] at (4.440,0.677) {$2$};
\draw[gp path] (5.691,0.985)--(5.691,1.165);
\draw[gp path] (5.691,8.631)--(5.691,8.451);
\node[gp node center] at (5.691,0.677) {$3$};
\draw[gp path] (6.942,0.985)--(6.942,1.165);
\draw[gp path] (6.942,8.631)--(6.942,8.451);
\node[gp node center] at (6.942,0.677) {$4$};
\draw[gp path] (8.193,0.985)--(8.193,1.165);
\draw[gp path] (8.193,8.631)--(8.193,8.451);
\node[gp node center] at (8.193,0.677) {$5$};
\draw[gp path] (9.444,0.985)--(9.444,1.165);
\draw[gp path] (9.444,8.631)--(9.444,8.451);
\node[gp node center] at (9.444,0.677) {$6$};
\draw[gp path] (10.695,0.985)--(10.695,1.165);
\draw[gp path] (10.695,8.631)--(10.695,8.451);
\node[gp node center] at (10.695,0.677) {$7$};
\draw[gp path] (11.946,0.985)--(11.946,1.165);
\draw[gp path] (11.946,8.631)--(11.946,8.451);
\node[gp node center] at (11.946,0.677) {$8$};
\draw[gp path] (13.197,0.985)--(13.197,1.165);
\draw[gp path] (13.197,8.631)--(13.197,8.451);
\node[gp node center] at (13.197,0.677) {$9$};
\draw[gp path] (1.688,8.631)--(1.688,0.985)--(13.447,0.985)--(13.447,8.631)--cycle;
\node[gp node center,rotate=-270] at (0.246,4.808) {$y$};
\node[gp node center] at (7.567,0.215) {$x$};
\node[gp node left] at (3.156,8.297) {Dados experimentais};
\gpcolor{rgb color={0.000,0.000,0.000}}
\gpsetpointsize{4.00}
\gppoint{gp mark 7}{(2.063,1.359)}
\gppoint{gp mark 7}{(2.313,1.342)}
\gppoint{gp mark 7}{(2.564,1.346)}
\gppoint{gp mark 7}{(2.814,1.359)}
\gppoint{gp mark 7}{(3.064,1.479)}
\gppoint{gp mark 7}{(3.314,1.402)}
\gppoint{gp mark 7}{(3.564,1.562)}
\gppoint{gp mark 7}{(3.815,1.691)}
\gppoint{gp mark 7}{(4.065,1.657)}
\gppoint{gp mark 7}{(4.315,1.752)}
\gppoint{gp mark 7}{(4.565,1.733)}
\gppoint{gp mark 7}{(4.815,1.690)}
\gppoint{gp mark 7}{(5.066,1.870)}
\gppoint{gp mark 7}{(5.316,2.209)}
\gppoint{gp mark 7}{(5.566,2.195)}
\gppoint{gp mark 7}{(5.816,2.422)}
\gppoint{gp mark 7}{(6.066,2.422)}
\gppoint{gp mark 7}{(6.317,2.521)}
\gppoint{gp mark 7}{(6.567,2.768)}
\gppoint{gp mark 7}{(6.817,2.422)}
\gppoint{gp mark 7}{(7.067,2.918)}
\gppoint{gp mark 7}{(7.317,2.962)}
\gppoint{gp mark 7}{(7.568,2.684)}
\gppoint{gp mark 7}{(7.818,3.196)}
\gppoint{gp mark 7}{(8.068,3.431)}
\gppoint{gp mark 7}{(8.318,3.601)}
\gppoint{gp mark 7}{(8.568,3.856)}
\gppoint{gp mark 7}{(8.818,4.052)}
\gppoint{gp mark 7}{(9.069,3.730)}
\gppoint{gp mark 7}{(9.319,4.214)}
\gppoint{gp mark 7}{(9.569,4.175)}
\gppoint{gp mark 7}{(9.819,4.119)}
\gppoint{gp mark 7}{(10.069,4.998)}
\gppoint{gp mark 7}{(10.320,4.499)}
\gppoint{gp mark 7}{(10.570,4.545)}
\gppoint{gp mark 7}{(10.820,5.653)}
\gppoint{gp mark 7}{(11.070,5.345)}
\gppoint{gp mark 7}{(11.320,5.407)}
\gppoint{gp mark 7}{(11.571,5.616)}
\gppoint{gp mark 7}{(11.821,6.437)}
\gppoint{gp mark 7}{(12.071,6.060)}
\gppoint{gp mark 7}{(12.321,6.794)}
\gppoint{gp mark 7}{(12.571,6.545)}
\gppoint{gp mark 7}{(12.822,7.510)}
\gppoint{gp mark 7}{(13.072,7.773)}
\gppoint{gp mark 7}{(2.514,8.297)}
\gpcolor{color=gp lt color border}
\node[gp node left] at (3.156,7.989) {$y = \np{1.03576} \, x^2 + \np{0.93350} \, x + \np{0.60425}$};
\gpcolor{rgb color={0.000,0.000,0.000}}
\gpsetdashtype{gp dt 2}
\draw[gp path] (2.056,7.989)--(2.972,7.989);
\draw[gp path] (1.941,1.358)--(1.947,1.358)--(1.953,1.358)--(1.958,1.359)--(1.964,1.359)%
  --(1.970,1.359)--(1.976,1.360)--(1.982,1.360)--(1.988,1.360)--(1.994,1.361)--(2.000,1.361)%
  --(2.005,1.361)--(2.011,1.361)--(2.017,1.362)--(2.023,1.362)--(2.029,1.362)--(2.035,1.363)%
  --(2.041,1.363)--(2.047,1.364)--(2.053,1.364)--(2.058,1.364)--(2.064,1.365)--(2.070,1.365)%
  --(2.076,1.365)--(2.082,1.366)--(2.088,1.366)--(2.094,1.366)--(2.100,1.367)--(2.105,1.367)%
  --(2.111,1.368)--(2.117,1.368)--(2.123,1.368)--(2.129,1.369)--(2.135,1.369)--(2.141,1.369)%
  --(2.147,1.370)--(2.152,1.370)--(2.158,1.371)--(2.164,1.371)--(2.170,1.371)--(2.176,1.372)%
  --(2.182,1.372)--(2.188,1.373)--(2.194,1.373)--(2.200,1.374)--(2.205,1.374)--(2.211,1.374)%
  --(2.217,1.375)--(2.223,1.375)--(2.229,1.376)--(2.235,1.376)--(2.241,1.377)--(2.247,1.377)%
  --(2.252,1.378)--(2.258,1.378)--(2.264,1.378)--(2.270,1.379)--(2.276,1.379)--(2.282,1.380)%
  --(2.288,1.380)--(2.294,1.381)--(2.299,1.381)--(2.305,1.382)--(2.311,1.382)--(2.317,1.383)%
  --(2.323,1.383)--(2.329,1.384)--(2.335,1.384)--(2.341,1.385)--(2.347,1.385)--(2.352,1.386)%
  --(2.358,1.386)--(2.364,1.387)--(2.370,1.387)--(2.376,1.388)--(2.382,1.388)--(2.388,1.389)%
  --(2.394,1.389)--(2.399,1.390)--(2.405,1.390)--(2.411,1.391)--(2.417,1.391)--(2.423,1.392)%
  --(2.429,1.393)--(2.435,1.393)--(2.441,1.394)--(2.446,1.394)--(2.452,1.395)--(2.458,1.395)%
  --(2.464,1.396)--(2.470,1.396)--(2.476,1.397)--(2.482,1.398)--(2.488,1.398)--(2.493,1.399)%
  --(2.499,1.399)--(2.505,1.400)--(2.511,1.400)--(2.517,1.401)--(2.523,1.402)--(2.529,1.402)%
  --(2.535,1.403)--(2.541,1.403)--(2.546,1.404)--(2.552,1.405)--(2.558,1.405)--(2.564,1.406)%
  --(2.570,1.407)--(2.576,1.407)--(2.582,1.408)--(2.588,1.408)--(2.593,1.409)--(2.599,1.410)%
  --(2.605,1.410)--(2.611,1.411)--(2.617,1.412)--(2.623,1.412)--(2.629,1.413)--(2.635,1.414)%
  --(2.640,1.414)--(2.646,1.415)--(2.652,1.415)--(2.658,1.416)--(2.664,1.417)--(2.670,1.417)%
  --(2.676,1.418)--(2.682,1.419)--(2.688,1.419)--(2.693,1.420)--(2.699,1.421)--(2.705,1.422)%
  --(2.711,1.422)--(2.717,1.423)--(2.723,1.424)--(2.729,1.424)--(2.735,1.425)--(2.740,1.426)%
  --(2.746,1.426)--(2.752,1.427)--(2.758,1.428)--(2.764,1.429)--(2.770,1.429)--(2.776,1.430)%
  --(2.782,1.431)--(2.787,1.431)--(2.793,1.432)--(2.799,1.433)--(2.805,1.434)--(2.811,1.434)%
  --(2.817,1.435)--(2.823,1.436)--(2.829,1.437)--(2.835,1.437)--(2.840,1.438)--(2.846,1.439)%
  --(2.852,1.440)--(2.858,1.440)--(2.864,1.441)--(2.870,1.442)--(2.876,1.443)--(2.882,1.444)%
  --(2.887,1.444)--(2.893,1.445)--(2.899,1.446)--(2.905,1.447)--(2.911,1.448)--(2.917,1.448)%
  --(2.923,1.449)--(2.929,1.450)--(2.934,1.451)--(2.940,1.452)--(2.946,1.452)--(2.952,1.453)%
  --(2.958,1.454)--(2.964,1.455)--(2.970,1.456)--(2.976,1.456)--(2.981,1.457)--(2.987,1.458)%
  --(2.993,1.459)--(2.999,1.460)--(3.005,1.461)--(3.011,1.461)--(3.017,1.462)--(3.023,1.463)%
  --(3.029,1.464)--(3.034,1.465)--(3.040,1.466)--(3.046,1.467)--(3.052,1.467)--(3.058,1.468)%
  --(3.064,1.469)--(3.070,1.470)--(3.076,1.471)--(3.081,1.472)--(3.087,1.473)--(3.093,1.474)%
  --(3.099,1.475)--(3.105,1.475)--(3.111,1.476)--(3.117,1.477)--(3.123,1.478)--(3.128,1.479)%
  --(3.134,1.480)--(3.140,1.481)--(3.146,1.482)--(3.152,1.483)--(3.158,1.484)--(3.164,1.485)%
  --(3.170,1.485)--(3.176,1.486)--(3.181,1.487)--(3.187,1.488)--(3.193,1.489)--(3.199,1.490)%
  --(3.205,1.491)--(3.211,1.492)--(3.217,1.493)--(3.223,1.494)--(3.228,1.495)--(3.234,1.496)%
  --(3.240,1.497)--(3.246,1.498)--(3.252,1.499)--(3.258,1.500)--(3.264,1.501)--(3.270,1.502)%
  --(3.275,1.503)--(3.281,1.504)--(3.287,1.505)--(3.293,1.506)--(3.299,1.507)--(3.305,1.508)%
  --(3.311,1.509)--(3.317,1.510)--(3.323,1.511)--(3.328,1.512)--(3.334,1.513)--(3.340,1.514)%
  --(3.346,1.515)--(3.352,1.516)--(3.358,1.517)--(3.364,1.518)--(3.370,1.519)--(3.375,1.520)%
  --(3.381,1.521)--(3.387,1.522)--(3.393,1.523)--(3.399,1.524)--(3.405,1.525)--(3.411,1.526)%
  --(3.417,1.527)--(3.422,1.528)--(3.428,1.529)--(3.434,1.530)--(3.440,1.531)--(3.446,1.532)%
  --(3.452,1.534)--(3.458,1.535)--(3.464,1.536)--(3.469,1.537)--(3.475,1.538)--(3.481,1.539)%
  --(3.487,1.540)--(3.493,1.541)--(3.499,1.542)--(3.505,1.543)--(3.511,1.544)--(3.517,1.546)%
  --(3.522,1.547)--(3.528,1.548)--(3.534,1.549)--(3.540,1.550)--(3.546,1.551)--(3.552,1.552)%
  --(3.558,1.553)--(3.564,1.555)--(3.569,1.556)--(3.575,1.557)--(3.581,1.558)--(3.587,1.559)%
  --(3.593,1.560)--(3.599,1.561)--(3.605,1.563)--(3.611,1.564)--(3.616,1.565)--(3.622,1.566)%
  --(3.628,1.567)--(3.634,1.568)--(3.640,1.569)--(3.646,1.571)--(3.652,1.572)--(3.658,1.573)%
  --(3.664,1.574)--(3.669,1.575)--(3.675,1.577)--(3.681,1.578)--(3.687,1.579)--(3.693,1.580)%
  --(3.699,1.581)--(3.705,1.583)--(3.711,1.584)--(3.716,1.585)--(3.722,1.586)--(3.728,1.587)%
  --(3.734,1.589)--(3.740,1.590)--(3.746,1.591)--(3.752,1.592)--(3.758,1.594)--(3.763,1.595)%
  --(3.769,1.596)--(3.775,1.597)--(3.781,1.599)--(3.787,1.600)--(3.793,1.601)--(3.799,1.602)%
  --(3.805,1.604)--(3.810,1.605)--(3.816,1.606)--(3.822,1.607)--(3.828,1.609)--(3.834,1.610)%
  --(3.840,1.611)--(3.846,1.612)--(3.852,1.614)--(3.858,1.615)--(3.863,1.616)--(3.869,1.618)%
  --(3.875,1.619)--(3.881,1.620)--(3.887,1.621)--(3.893,1.623)--(3.899,1.624)--(3.905,1.625)%
  --(3.910,1.627)--(3.916,1.628)--(3.922,1.629)--(3.928,1.631)--(3.934,1.632)--(3.940,1.633)%
  --(3.946,1.635)--(3.952,1.636)--(3.957,1.637)--(3.963,1.639)--(3.969,1.640)--(3.975,1.641)%
  --(3.981,1.643)--(3.987,1.644)--(3.993,1.645)--(3.999,1.647)--(4.005,1.648)--(4.010,1.649)%
  --(4.016,1.651)--(4.022,1.652)--(4.028,1.653)--(4.034,1.655)--(4.040,1.656)--(4.046,1.658)%
  --(4.052,1.659)--(4.057,1.660)--(4.063,1.662)--(4.069,1.663)--(4.075,1.665)--(4.081,1.666)%
  --(4.087,1.667)--(4.093,1.669)--(4.099,1.670)--(4.104,1.672)--(4.110,1.673)--(4.116,1.674)%
  --(4.122,1.676)--(4.128,1.677)--(4.134,1.679)--(4.140,1.680)--(4.146,1.682)--(4.152,1.683)%
  --(4.157,1.684)--(4.163,1.686)--(4.169,1.687)--(4.175,1.689)--(4.181,1.690)--(4.187,1.692)%
  --(4.193,1.693)--(4.199,1.695)--(4.204,1.696)--(4.210,1.698)--(4.216,1.699)--(4.222,1.700)%
  --(4.228,1.702)--(4.234,1.703)--(4.240,1.705)--(4.246,1.706)--(4.251,1.708)--(4.257,1.709)%
  --(4.263,1.711)--(4.269,1.712)--(4.275,1.714)--(4.281,1.715)--(4.287,1.717)--(4.293,1.718)%
  --(4.298,1.720)--(4.304,1.721)--(4.310,1.723)--(4.316,1.724)--(4.322,1.726)--(4.328,1.727)%
  --(4.334,1.729)--(4.340,1.731)--(4.346,1.732)--(4.351,1.734)--(4.357,1.735)--(4.363,1.737)%
  --(4.369,1.738)--(4.375,1.740)--(4.381,1.741)--(4.387,1.743)--(4.393,1.744)--(4.398,1.746)%
  --(4.404,1.748)--(4.410,1.749)--(4.416,1.751)--(4.422,1.752)--(4.428,1.754)--(4.434,1.755)%
  --(4.440,1.757)--(4.445,1.759)--(4.451,1.760)--(4.457,1.762)--(4.463,1.763)--(4.469,1.765)%
  --(4.475,1.767)--(4.481,1.768)--(4.487,1.770)--(4.493,1.771)--(4.498,1.773)--(4.504,1.775)%
  --(4.510,1.776)--(4.516,1.778)--(4.522,1.780)--(4.528,1.781)--(4.534,1.783)--(4.540,1.784)%
  --(4.545,1.786)--(4.551,1.788)--(4.557,1.789)--(4.563,1.791)--(4.569,1.793)--(4.575,1.794)%
  --(4.581,1.796)--(4.587,1.798)--(4.592,1.799)--(4.598,1.801)--(4.604,1.803)--(4.610,1.804)%
  --(4.616,1.806)--(4.622,1.808)--(4.628,1.809)--(4.634,1.811)--(4.640,1.813)--(4.645,1.814)%
  --(4.651,1.816)--(4.657,1.818)--(4.663,1.820)--(4.669,1.821)--(4.675,1.823)--(4.681,1.825)%
  --(4.687,1.826)--(4.692,1.828)--(4.698,1.830)--(4.704,1.832)--(4.710,1.833)--(4.716,1.835)%
  --(4.722,1.837)--(4.728,1.838)--(4.734,1.840)--(4.739,1.842)--(4.745,1.844)--(4.751,1.845)%
  --(4.757,1.847)--(4.763,1.849)--(4.769,1.851)--(4.775,1.852)--(4.781,1.854)--(4.786,1.856)%
  --(4.792,1.858)--(4.798,1.859)--(4.804,1.861)--(4.810,1.863)--(4.816,1.865)--(4.822,1.867)%
  --(4.828,1.868)--(4.834,1.870)--(4.839,1.872)--(4.845,1.874)--(4.851,1.876)--(4.857,1.877)%
  --(4.863,1.879)--(4.869,1.881)--(4.875,1.883)--(4.881,1.885)--(4.886,1.886)--(4.892,1.888)%
  --(4.898,1.890)--(4.904,1.892)--(4.910,1.894)--(4.916,1.896)--(4.922,1.897)--(4.928,1.899)%
  --(4.933,1.901)--(4.939,1.903)--(4.945,1.905)--(4.951,1.907)--(4.957,1.908)--(4.963,1.910)%
  --(4.969,1.912)--(4.975,1.914)--(4.981,1.916)--(4.986,1.918)--(4.992,1.920)--(4.998,1.921)%
  --(5.004,1.923)--(5.010,1.925)--(5.016,1.927)--(5.022,1.929)--(5.028,1.931)--(5.033,1.933)%
  --(5.039,1.935)--(5.045,1.937)--(5.051,1.938)--(5.057,1.940)--(5.063,1.942)--(5.069,1.944)%
  --(5.075,1.946)--(5.080,1.948)--(5.086,1.950)--(5.092,1.952)--(5.098,1.954)--(5.104,1.956)%
  --(5.110,1.958)--(5.116,1.960)--(5.122,1.962)--(5.128,1.963)--(5.133,1.965)--(5.139,1.967)%
  --(5.145,1.969)--(5.151,1.971)--(5.157,1.973)--(5.163,1.975)--(5.169,1.977)--(5.175,1.979)%
  --(5.180,1.981)--(5.186,1.983)--(5.192,1.985)--(5.198,1.987)--(5.204,1.989)--(5.210,1.991)%
  --(5.216,1.993)--(5.222,1.995)--(5.227,1.997)--(5.233,1.999)--(5.239,2.001)--(5.245,2.003)%
  --(5.251,2.005)--(5.257,2.007)--(5.263,2.009)--(5.269,2.011)--(5.274,2.013)--(5.280,2.015)%
  --(5.286,2.017)--(5.292,2.019)--(5.298,2.021)--(5.304,2.023)--(5.310,2.025)--(5.316,2.027)%
  --(5.322,2.029)--(5.327,2.031)--(5.333,2.033)--(5.339,2.035)--(5.345,2.037)--(5.351,2.039)%
  --(5.357,2.042)--(5.363,2.044)--(5.369,2.046)--(5.374,2.048)--(5.380,2.050)--(5.386,2.052)%
  --(5.392,2.054)--(5.398,2.056)--(5.404,2.058)--(5.410,2.060)--(5.416,2.062)--(5.421,2.064)%
  --(5.427,2.066)--(5.433,2.069)--(5.439,2.071)--(5.445,2.073)--(5.451,2.075)--(5.457,2.077)%
  --(5.463,2.079)--(5.469,2.081)--(5.474,2.083)--(5.480,2.085)--(5.486,2.088)--(5.492,2.090)%
  --(5.498,2.092)--(5.504,2.094)--(5.510,2.096)--(5.516,2.098)--(5.521,2.100)--(5.527,2.103)%
  --(5.533,2.105)--(5.539,2.107)--(5.545,2.109)--(5.551,2.111)--(5.557,2.113)--(5.563,2.116)%
  --(5.568,2.118)--(5.574,2.120)--(5.580,2.122)--(5.586,2.124)--(5.592,2.126)--(5.598,2.129)%
  --(5.604,2.131)--(5.610,2.133)--(5.616,2.135)--(5.621,2.137)--(5.627,2.140)--(5.633,2.142)%
  --(5.639,2.144)--(5.645,2.146)--(5.651,2.148)--(5.657,2.151)--(5.663,2.153)--(5.668,2.155)%
  --(5.674,2.157)--(5.680,2.159)--(5.686,2.162)--(5.692,2.164)--(5.698,2.166)--(5.704,2.168)%
  --(5.710,2.171)--(5.715,2.173)--(5.721,2.175)--(5.727,2.177)--(5.733,2.180)--(5.739,2.182)%
  --(5.745,2.184)--(5.751,2.186)--(5.757,2.189)--(5.762,2.191)--(5.768,2.193)--(5.774,2.195)%
  --(5.780,2.198)--(5.786,2.200)--(5.792,2.202)--(5.798,2.205)--(5.804,2.207)--(5.810,2.209)%
  --(5.815,2.212)--(5.821,2.214)--(5.827,2.216)--(5.833,2.218)--(5.839,2.221)--(5.845,2.223)%
  --(5.851,2.225)--(5.857,2.228)--(5.862,2.230)--(5.868,2.232)--(5.874,2.235)--(5.880,2.237)%
  --(5.886,2.239)--(5.892,2.242)--(5.898,2.244)--(5.904,2.246)--(5.909,2.249)--(5.915,2.251)%
  --(5.921,2.253)--(5.927,2.256)--(5.933,2.258)--(5.939,2.260)--(5.945,2.263)--(5.951,2.265)%
  --(5.957,2.268)--(5.962,2.270)--(5.968,2.272)--(5.974,2.275)--(5.980,2.277)--(5.986,2.279)%
  --(5.992,2.282)--(5.998,2.284)--(6.004,2.287)--(6.009,2.289)--(6.015,2.291)--(6.021,2.294)%
  --(6.027,2.296)--(6.033,2.299)--(6.039,2.301)--(6.045,2.303)--(6.051,2.306)--(6.056,2.308)%
  --(6.062,2.311)--(6.068,2.313)--(6.074,2.316)--(6.080,2.318)--(6.086,2.320)--(6.092,2.323)%
  --(6.098,2.325)--(6.104,2.328)--(6.109,2.330)--(6.115,2.333)--(6.121,2.335)--(6.127,2.338)%
  --(6.133,2.340)--(6.139,2.342)--(6.145,2.345)--(6.151,2.347)--(6.156,2.350)--(6.162,2.352)%
  --(6.168,2.355)--(6.174,2.357)--(6.180,2.360)--(6.186,2.362)--(6.192,2.365)--(6.198,2.367)%
  --(6.203,2.370)--(6.209,2.372)--(6.215,2.375)--(6.221,2.377)--(6.227,2.380)--(6.233,2.382)%
  --(6.239,2.385)--(6.245,2.387)--(6.250,2.390)--(6.256,2.392)--(6.262,2.395)--(6.268,2.397)%
  --(6.274,2.400)--(6.280,2.403)--(6.286,2.405)--(6.292,2.408)--(6.298,2.410)--(6.303,2.413)%
  --(6.309,2.415)--(6.315,2.418)--(6.321,2.420)--(6.327,2.423)--(6.333,2.426)--(6.339,2.428)%
  --(6.345,2.431)--(6.350,2.433)--(6.356,2.436)--(6.362,2.438)--(6.368,2.441)--(6.374,2.444)%
  --(6.380,2.446)--(6.386,2.449)--(6.392,2.451)--(6.397,2.454)--(6.403,2.457)--(6.409,2.459)%
  --(6.415,2.462)--(6.421,2.464)--(6.427,2.467)--(6.433,2.470)--(6.439,2.472)--(6.445,2.475)%
  --(6.450,2.477)--(6.456,2.480)--(6.462,2.483)--(6.468,2.485)--(6.474,2.488)--(6.480,2.491)%
  --(6.486,2.493)--(6.492,2.496)--(6.497,2.499)--(6.503,2.501)--(6.509,2.504)--(6.515,2.507)%
  --(6.521,2.509)--(6.527,2.512)--(6.533,2.515)--(6.539,2.517)--(6.544,2.520)--(6.550,2.523)%
  --(6.556,2.525)--(6.562,2.528)--(6.568,2.531)--(6.574,2.533)--(6.580,2.536)--(6.586,2.539)%
  --(6.592,2.541)--(6.597,2.544)--(6.603,2.547)--(6.609,2.549)--(6.615,2.552)--(6.621,2.555)%
  --(6.627,2.558)--(6.633,2.560)--(6.639,2.563)--(6.644,2.566)--(6.650,2.569)--(6.656,2.571)%
  --(6.662,2.574)--(6.668,2.577)--(6.674,2.579)--(6.680,2.582)--(6.686,2.585)--(6.691,2.588)%
  --(6.697,2.590)--(6.703,2.593)--(6.709,2.596)--(6.715,2.599)--(6.721,2.601)--(6.727,2.604)%
  --(6.733,2.607)--(6.738,2.610)--(6.744,2.613)--(6.750,2.615)--(6.756,2.618)--(6.762,2.621)%
  --(6.768,2.624)--(6.774,2.627)--(6.780,2.629)--(6.786,2.632)--(6.791,2.635)--(6.797,2.638)%
  --(6.803,2.641)--(6.809,2.643)--(6.815,2.646)--(6.821,2.649)--(6.827,2.652)--(6.833,2.655)%
  --(6.838,2.657)--(6.844,2.660)--(6.850,2.663)--(6.856,2.666)--(6.862,2.669)--(6.868,2.672)%
  --(6.874,2.674)--(6.880,2.677)--(6.885,2.680)--(6.891,2.683)--(6.897,2.686)--(6.903,2.689)%
  --(6.909,2.692)--(6.915,2.694)--(6.921,2.697)--(6.927,2.700)--(6.933,2.703)--(6.938,2.706)%
  --(6.944,2.709)--(6.950,2.712)--(6.956,2.715)--(6.962,2.717)--(6.968,2.720)--(6.974,2.723)%
  --(6.980,2.726)--(6.985,2.729)--(6.991,2.732)--(6.997,2.735)--(7.003,2.738)--(7.009,2.741)%
  --(7.015,2.744)--(7.021,2.747)--(7.027,2.749)--(7.032,2.752)--(7.038,2.755)--(7.044,2.758)%
  --(7.050,2.761)--(7.056,2.764)--(7.062,2.767)--(7.068,2.770)--(7.074,2.773)--(7.080,2.776)%
  --(7.085,2.779)--(7.091,2.782)--(7.097,2.785)--(7.103,2.788)--(7.109,2.791)--(7.115,2.794)%
  --(7.121,2.797)--(7.127,2.800)--(7.132,2.803)--(7.138,2.806)--(7.144,2.809)--(7.150,2.812)%
  --(7.156,2.815)--(7.162,2.818)--(7.168,2.821)--(7.174,2.824)--(7.179,2.827)--(7.185,2.830)%
  --(7.191,2.833)--(7.197,2.836)--(7.203,2.839)--(7.209,2.842)--(7.215,2.845)--(7.221,2.848)%
  --(7.226,2.851)--(7.232,2.854)--(7.238,2.857)--(7.244,2.860)--(7.250,2.863)--(7.256,2.866)%
  --(7.262,2.869)--(7.268,2.872)--(7.274,2.875)--(7.279,2.878)--(7.285,2.881)--(7.291,2.884)%
  --(7.297,2.887)--(7.303,2.890)--(7.309,2.893)--(7.315,2.896)--(7.321,2.899)--(7.326,2.903)%
  --(7.332,2.906)--(7.338,2.909)--(7.344,2.912)--(7.350,2.915)--(7.356,2.918)--(7.362,2.921)%
  --(7.368,2.924)--(7.373,2.927)--(7.379,2.930)--(7.385,2.934)--(7.391,2.937)--(7.397,2.940)%
  --(7.403,2.943)--(7.409,2.946)--(7.415,2.949)--(7.421,2.952)--(7.426,2.955)--(7.432,2.959)%
  --(7.438,2.962)--(7.444,2.965)--(7.450,2.968)--(7.456,2.971)--(7.462,2.974)--(7.468,2.977)%
  --(7.473,2.981)--(7.479,2.984)--(7.485,2.987)--(7.491,2.990)--(7.497,2.993)--(7.503,2.996)%
  --(7.509,3.000)--(7.515,3.003)--(7.520,3.006)--(7.526,3.009)--(7.532,3.012)--(7.538,3.015)%
  --(7.544,3.019)--(7.550,3.022)--(7.556,3.025)--(7.562,3.028)--(7.568,3.031)--(7.573,3.035)%
  --(7.579,3.038)--(7.585,3.041)--(7.591,3.044)--(7.597,3.047)--(7.603,3.051)--(7.609,3.054)%
  --(7.615,3.057)--(7.620,3.060)--(7.626,3.064)--(7.632,3.067)--(7.638,3.070)--(7.644,3.073)%
  --(7.650,3.077)--(7.656,3.080)--(7.662,3.083)--(7.667,3.086)--(7.673,3.090)--(7.679,3.093)%
  --(7.685,3.096)--(7.691,3.099)--(7.697,3.103)--(7.703,3.106)--(7.709,3.109)--(7.714,3.112)%
  --(7.720,3.116)--(7.726,3.119)--(7.732,3.122)--(7.738,3.126)--(7.744,3.129)--(7.750,3.132)%
  --(7.756,3.136)--(7.762,3.139)--(7.767,3.142)--(7.773,3.145)--(7.779,3.149)--(7.785,3.152)%
  --(7.791,3.155)--(7.797,3.159)--(7.803,3.162)--(7.809,3.165)--(7.814,3.169)--(7.820,3.172)%
  --(7.826,3.175)--(7.832,3.179)--(7.838,3.182)--(7.844,3.185)--(7.850,3.189)--(7.856,3.192)%
  --(7.861,3.195)--(7.867,3.199)--(7.873,3.202)--(7.879,3.206)--(7.885,3.209)--(7.891,3.212)%
  --(7.897,3.216)--(7.903,3.219)--(7.909,3.222)--(7.914,3.226)--(7.920,3.229)--(7.926,3.233)%
  --(7.932,3.236)--(7.938,3.239)--(7.944,3.243)--(7.950,3.246)--(7.956,3.250)--(7.961,3.253)%
  --(7.967,3.256)--(7.973,3.260)--(7.979,3.263)--(7.985,3.267)--(7.991,3.270)--(7.997,3.273)%
  --(8.003,3.277)--(8.008,3.280)--(8.014,3.284)--(8.020,3.287)--(8.026,3.291)--(8.032,3.294)%
  --(8.038,3.298)--(8.044,3.301)--(8.050,3.304)--(8.055,3.308)--(8.061,3.311)--(8.067,3.315)%
  --(8.073,3.318)--(8.079,3.322)--(8.085,3.325)--(8.091,3.329)--(8.097,3.332)--(8.103,3.336)%
  --(8.108,3.339)--(8.114,3.343)--(8.120,3.346)--(8.126,3.350)--(8.132,3.353)--(8.138,3.357)%
  --(8.144,3.360)--(8.150,3.364)--(8.155,3.367)--(8.161,3.371)--(8.167,3.374)--(8.173,3.378)%
  --(8.179,3.381)--(8.185,3.385)--(8.191,3.388)--(8.197,3.392)--(8.202,3.395)--(8.208,3.399)%
  --(8.214,3.402)--(8.220,3.406)--(8.226,3.409)--(8.232,3.413)--(8.238,3.417)--(8.244,3.420)%
  --(8.250,3.424)--(8.255,3.427)--(8.261,3.431)--(8.267,3.434)--(8.273,3.438)--(8.279,3.441)%
  --(8.285,3.445)--(8.291,3.449)--(8.297,3.452)--(8.302,3.456)--(8.308,3.459)--(8.314,3.463)%
  --(8.320,3.467)--(8.326,3.470)--(8.332,3.474)--(8.338,3.477)--(8.344,3.481)--(8.349,3.485)%
  --(8.355,3.488)--(8.361,3.492)--(8.367,3.495)--(8.373,3.499)--(8.379,3.503)--(8.385,3.506)%
  --(8.391,3.510)--(8.397,3.514)--(8.402,3.517)--(8.408,3.521)--(8.414,3.524)--(8.420,3.528)%
  --(8.426,3.532)--(8.432,3.535)--(8.438,3.539)--(8.444,3.543)--(8.449,3.546)--(8.455,3.550)%
  --(8.461,3.554)--(8.467,3.557)--(8.473,3.561)--(8.479,3.565)--(8.485,3.568)--(8.491,3.572)%
  --(8.496,3.576)--(8.502,3.579)--(8.508,3.583)--(8.514,3.587)--(8.520,3.590)--(8.526,3.594)%
  --(8.532,3.598)--(8.538,3.602)--(8.543,3.605)--(8.549,3.609)--(8.555,3.613)--(8.561,3.616)%
  --(8.567,3.620)--(8.573,3.624)--(8.579,3.628)--(8.585,3.631)--(8.591,3.635)--(8.596,3.639)%
  --(8.602,3.643)--(8.608,3.646)--(8.614,3.650)--(8.620,3.654)--(8.626,3.658)--(8.632,3.661)%
  --(8.638,3.665)--(8.643,3.669)--(8.649,3.673)--(8.655,3.676)--(8.661,3.680)--(8.667,3.684)%
  --(8.673,3.688)--(8.679,3.691)--(8.685,3.695)--(8.690,3.699)--(8.696,3.703)--(8.702,3.707)%
  --(8.708,3.710)--(8.714,3.714)--(8.720,3.718)--(8.726,3.722)--(8.732,3.726)--(8.738,3.729)%
  --(8.743,3.733)--(8.749,3.737)--(8.755,3.741)--(8.761,3.745)--(8.767,3.748)--(8.773,3.752)%
  --(8.779,3.756)--(8.785,3.760)--(8.790,3.764)--(8.796,3.768)--(8.802,3.771)--(8.808,3.775)%
  --(8.814,3.779)--(8.820,3.783)--(8.826,3.787)--(8.832,3.791)--(8.837,3.795)--(8.843,3.798)%
  --(8.849,3.802)--(8.855,3.806)--(8.861,3.810)--(8.867,3.814)--(8.873,3.818)--(8.879,3.822)%
  --(8.885,3.826)--(8.890,3.829)--(8.896,3.833)--(8.902,3.837)--(8.908,3.841)--(8.914,3.845)%
  --(8.920,3.849)--(8.926,3.853)--(8.932,3.857)--(8.937,3.861)--(8.943,3.865)--(8.949,3.869)%
  --(8.955,3.872)--(8.961,3.876)--(8.967,3.880)--(8.973,3.884)--(8.979,3.888)--(8.984,3.892)%
  --(8.990,3.896)--(8.996,3.900)--(9.002,3.904)--(9.008,3.908)--(9.014,3.912)--(9.020,3.916)%
  --(9.026,3.920)--(9.031,3.924)--(9.037,3.928)--(9.043,3.932)--(9.049,3.936)--(9.055,3.940)%
  --(9.061,3.944)--(9.067,3.948)--(9.073,3.952)--(9.079,3.955)--(9.084,3.959)--(9.090,3.963)%
  --(9.096,3.967)--(9.102,3.971)--(9.108,3.975)--(9.114,3.979)--(9.120,3.983)--(9.126,3.987)%
  --(9.131,3.991)--(9.137,3.996)--(9.143,4.000)--(9.149,4.004)--(9.155,4.008)--(9.161,4.012)%
  --(9.167,4.016)--(9.173,4.020)--(9.178,4.024)--(9.184,4.028)--(9.190,4.032)--(9.196,4.036)%
  --(9.202,4.040)--(9.208,4.044)--(9.214,4.048)--(9.220,4.052)--(9.226,4.056)--(9.231,4.060)%
  --(9.237,4.064)--(9.243,4.068)--(9.249,4.072)--(9.255,4.076)--(9.261,4.081)--(9.267,4.085)%
  --(9.273,4.089)--(9.278,4.093)--(9.284,4.097)--(9.290,4.101)--(9.296,4.105)--(9.302,4.109)%
  --(9.308,4.113)--(9.314,4.117)--(9.320,4.121)--(9.325,4.126)--(9.331,4.130)--(9.337,4.134)%
  --(9.343,4.138)--(9.349,4.142)--(9.355,4.146)--(9.361,4.150)--(9.367,4.154)--(9.373,4.159)%
  --(9.378,4.163)--(9.384,4.167)--(9.390,4.171)--(9.396,4.175)--(9.402,4.179)--(9.408,4.184)%
  --(9.414,4.188)--(9.420,4.192)--(9.425,4.196)--(9.431,4.200)--(9.437,4.204)--(9.443,4.209)%
  --(9.449,4.213)--(9.455,4.217)--(9.461,4.221)--(9.467,4.225)--(9.472,4.229)--(9.478,4.234)%
  --(9.484,4.238)--(9.490,4.242)--(9.496,4.246)--(9.502,4.250)--(9.508,4.255)--(9.514,4.259)%
  --(9.519,4.263)--(9.525,4.267)--(9.531,4.271)--(9.537,4.276)--(9.543,4.280)--(9.549,4.284)%
  --(9.555,4.288)--(9.561,4.293)--(9.567,4.297)--(9.572,4.301)--(9.578,4.305)--(9.584,4.310)%
  --(9.590,4.314)--(9.596,4.318)--(9.602,4.322)--(9.608,4.327)--(9.614,4.331)--(9.619,4.335)%
  --(9.625,4.339)--(9.631,4.344)--(9.637,4.348)--(9.643,4.352)--(9.649,4.356)--(9.655,4.361)%
  --(9.661,4.365)--(9.666,4.369)--(9.672,4.374)--(9.678,4.378)--(9.684,4.382)--(9.690,4.387)%
  --(9.696,4.391)--(9.702,4.395)--(9.708,4.399)--(9.714,4.404)--(9.719,4.408)--(9.725,4.412)%
  --(9.731,4.417)--(9.737,4.421)--(9.743,4.425)--(9.749,4.430)--(9.755,4.434)--(9.761,4.438)%
  --(9.766,4.443)--(9.772,4.447)--(9.778,4.451)--(9.784,4.456)--(9.790,4.460)--(9.796,4.464)%
  --(9.802,4.469)--(9.808,4.473)--(9.813,4.478)--(9.819,4.482)--(9.825,4.486)--(9.831,4.491)%
  --(9.837,4.495)--(9.843,4.499)--(9.849,4.504)--(9.855,4.508)--(9.861,4.513)--(9.866,4.517)%
  --(9.872,4.521)--(9.878,4.526)--(9.884,4.530)--(9.890,4.535)--(9.896,4.539)--(9.902,4.543)%
  --(9.908,4.548)--(9.913,4.552)--(9.919,4.557)--(9.925,4.561)--(9.931,4.566)--(9.937,4.570)%
  --(9.943,4.574)--(9.949,4.579)--(9.955,4.583)--(9.960,4.588)--(9.966,4.592)--(9.972,4.597)%
  --(9.978,4.601)--(9.984,4.605)--(9.990,4.610)--(9.996,4.614)--(10.002,4.619)--(10.007,4.623)%
  --(10.013,4.628)--(10.019,4.632)--(10.025,4.637)--(10.031,4.641)--(10.037,4.646)--(10.043,4.650)%
  --(10.049,4.655)--(10.055,4.659)--(10.060,4.664)--(10.066,4.668)--(10.072,4.673)--(10.078,4.677)%
  --(10.084,4.682)--(10.090,4.686)--(10.096,4.691)--(10.102,4.695)--(10.107,4.700)--(10.113,4.704)%
  --(10.119,4.709)--(10.125,4.713)--(10.131,4.718)--(10.137,4.722)--(10.143,4.727)--(10.149,4.731)%
  --(10.154,4.736)--(10.160,4.741)--(10.166,4.745)--(10.172,4.750)--(10.178,4.754)--(10.184,4.759)%
  --(10.190,4.763)--(10.196,4.768)--(10.202,4.772)--(10.207,4.777)--(10.213,4.782)--(10.219,4.786)%
  --(10.225,4.791)--(10.231,4.795)--(10.237,4.800)--(10.243,4.804)--(10.249,4.809)--(10.254,4.814)%
  --(10.260,4.818)--(10.266,4.823)--(10.272,4.827)--(10.278,4.832)--(10.284,4.837)--(10.290,4.841)%
  --(10.296,4.846)--(10.301,4.851)--(10.307,4.855)--(10.313,4.860)--(10.319,4.864)--(10.325,4.869)%
  --(10.331,4.874)--(10.337,4.878)--(10.343,4.883)--(10.349,4.888)--(10.354,4.892)--(10.360,4.897)%
  --(10.366,4.902)--(10.372,4.906)--(10.378,4.911)--(10.384,4.915)--(10.390,4.920)--(10.396,4.925)%
  --(10.401,4.929)--(10.407,4.934)--(10.413,4.939)--(10.419,4.944)--(10.425,4.948)--(10.431,4.953)%
  --(10.437,4.958)--(10.443,4.962)--(10.448,4.967)--(10.454,4.972)--(10.460,4.976)--(10.466,4.981)%
  --(10.472,4.986)--(10.478,4.990)--(10.484,4.995)--(10.490,5.000)--(10.495,5.005)--(10.501,5.009)%
  --(10.507,5.014)--(10.513,5.019)--(10.519,5.024)--(10.525,5.028)--(10.531,5.033)--(10.537,5.038)%
  --(10.543,5.042)--(10.548,5.047)--(10.554,5.052)--(10.560,5.057)--(10.566,5.061)--(10.572,5.066)%
  --(10.578,5.071)--(10.584,5.076)--(10.590,5.081)--(10.595,5.085)--(10.601,5.090)--(10.607,5.095)%
  --(10.613,5.100)--(10.619,5.104)--(10.625,5.109)--(10.631,5.114)--(10.637,5.119)--(10.642,5.124)%
  --(10.648,5.128)--(10.654,5.133)--(10.660,5.138)--(10.666,5.143)--(10.672,5.148)--(10.678,5.152)%
  --(10.684,5.157)--(10.690,5.162)--(10.695,5.167)--(10.701,5.172)--(10.707,5.176)--(10.713,5.181)%
  --(10.719,5.186)--(10.725,5.191)--(10.731,5.196)--(10.737,5.201)--(10.742,5.206)--(10.748,5.210)%
  --(10.754,5.215)--(10.760,5.220)--(10.766,5.225)--(10.772,5.230)--(10.778,5.235)--(10.784,5.240)%
  --(10.789,5.244)--(10.795,5.249)--(10.801,5.254)--(10.807,5.259)--(10.813,5.264)--(10.819,5.269)%
  --(10.825,5.274)--(10.831,5.279)--(10.837,5.283)--(10.842,5.288)--(10.848,5.293)--(10.854,5.298)%
  --(10.860,5.303)--(10.866,5.308)--(10.872,5.313)--(10.878,5.318)--(10.884,5.323)--(10.889,5.328)%
  --(10.895,5.333)--(10.901,5.338)--(10.907,5.342)--(10.913,5.347)--(10.919,5.352)--(10.925,5.357)%
  --(10.931,5.362)--(10.936,5.367)--(10.942,5.372)--(10.948,5.377)--(10.954,5.382)--(10.960,5.387)%
  --(10.966,5.392)--(10.972,5.397)--(10.978,5.402)--(10.983,5.407)--(10.989,5.412)--(10.995,5.417)%
  --(11.001,5.422)--(11.007,5.427)--(11.013,5.432)--(11.019,5.437)--(11.025,5.442)--(11.031,5.447)%
  --(11.036,5.452)--(11.042,5.457)--(11.048,5.462)--(11.054,5.467)--(11.060,5.472)--(11.066,5.477)%
  --(11.072,5.482)--(11.078,5.487)--(11.083,5.492)--(11.089,5.497)--(11.095,5.502)--(11.101,5.507)%
  --(11.107,5.512)--(11.113,5.517)--(11.119,5.522)--(11.125,5.527)--(11.130,5.532)--(11.136,5.537)%
  --(11.142,5.542)--(11.148,5.547)--(11.154,5.552)--(11.160,5.557)--(11.166,5.562)--(11.172,5.568)%
  --(11.178,5.573)--(11.183,5.578)--(11.189,5.583)--(11.195,5.588)--(11.201,5.593)--(11.207,5.598)%
  --(11.213,5.603)--(11.219,5.608)--(11.225,5.613)--(11.230,5.618)--(11.236,5.623)--(11.242,5.629)%
  --(11.248,5.634)--(11.254,5.639)--(11.260,5.644)--(11.266,5.649)--(11.272,5.654)--(11.277,5.659)%
  --(11.283,5.664)--(11.289,5.670)--(11.295,5.675)--(11.301,5.680)--(11.307,5.685)--(11.313,5.690)%
  --(11.319,5.695)--(11.325,5.700)--(11.330,5.705)--(11.336,5.711)--(11.342,5.716)--(11.348,5.721)%
  --(11.354,5.726)--(11.360,5.731)--(11.366,5.736)--(11.372,5.742)--(11.377,5.747)--(11.383,5.752)%
  --(11.389,5.757)--(11.395,5.762)--(11.401,5.768)--(11.407,5.773)--(11.413,5.778)--(11.419,5.783)%
  --(11.424,5.788)--(11.430,5.794)--(11.436,5.799)--(11.442,5.804)--(11.448,5.809)--(11.454,5.814)%
  --(11.460,5.820)--(11.466,5.825)--(11.471,5.830)--(11.477,5.835)--(11.483,5.840)--(11.489,5.846)%
  --(11.495,5.851)--(11.501,5.856)--(11.507,5.861)--(11.513,5.867)--(11.519,5.872)--(11.524,5.877)%
  --(11.530,5.882)--(11.536,5.888)--(11.542,5.893)--(11.548,5.898)--(11.554,5.903)--(11.560,5.909)%
  --(11.566,5.914)--(11.571,5.919)--(11.577,5.925)--(11.583,5.930)--(11.589,5.935)--(11.595,5.940)%
  --(11.601,5.946)--(11.607,5.951)--(11.613,5.956)--(11.618,5.962)--(11.624,5.967)--(11.630,5.972)%
  --(11.636,5.977)--(11.642,5.983)--(11.648,5.988)--(11.654,5.993)--(11.660,5.999)--(11.666,6.004)%
  --(11.671,6.009)--(11.677,6.015)--(11.683,6.020)--(11.689,6.025)--(11.695,6.031)--(11.701,6.036)%
  --(11.707,6.041)--(11.713,6.047)--(11.718,6.052)--(11.724,6.057)--(11.730,6.063)--(11.736,6.068)%
  --(11.742,6.074)--(11.748,6.079)--(11.754,6.084)--(11.760,6.090)--(11.765,6.095)--(11.771,6.100)%
  --(11.777,6.106)--(11.783,6.111)--(11.789,6.117)--(11.795,6.122)--(11.801,6.127)--(11.807,6.133)%
  --(11.812,6.138)--(11.818,6.144)--(11.824,6.149)--(11.830,6.154)--(11.836,6.160)--(11.842,6.165)%
  --(11.848,6.171)--(11.854,6.176)--(11.860,6.181)--(11.865,6.187)--(11.871,6.192)--(11.877,6.198)%
  --(11.883,6.203)--(11.889,6.209)--(11.895,6.214)--(11.901,6.220)--(11.907,6.225)--(11.912,6.230)%
  --(11.918,6.236)--(11.924,6.241)--(11.930,6.247)--(11.936,6.252)--(11.942,6.258)--(11.948,6.263)%
  --(11.954,6.269)--(11.959,6.274)--(11.965,6.280)--(11.971,6.285)--(11.977,6.291)--(11.983,6.296)%
  --(11.989,6.302)--(11.995,6.307)--(12.001,6.313)--(12.007,6.318)--(12.012,6.324)--(12.018,6.329)%
  --(12.024,6.335)--(12.030,6.340)--(12.036,6.346)--(12.042,6.351)--(12.048,6.357)--(12.054,6.362)%
  --(12.059,6.368)--(12.065,6.373)--(12.071,6.379)--(12.077,6.384)--(12.083,6.390)--(12.089,6.395)%
  --(12.095,6.401)--(12.101,6.406)--(12.106,6.412)--(12.112,6.418)--(12.118,6.423)--(12.124,6.429)%
  --(12.130,6.434)--(12.136,6.440)--(12.142,6.445)--(12.148,6.451)--(12.154,6.457)--(12.159,6.462)%
  --(12.165,6.468)--(12.171,6.473)--(12.177,6.479)--(12.183,6.484)--(12.189,6.490)--(12.195,6.496)%
  --(12.201,6.501)--(12.206,6.507)--(12.212,6.512)--(12.218,6.518)--(12.224,6.524)--(12.230,6.529)%
  --(12.236,6.535)--(12.242,6.541)--(12.248,6.546)--(12.253,6.552)--(12.259,6.557)--(12.265,6.563)%
  --(12.271,6.569)--(12.277,6.574)--(12.283,6.580)--(12.289,6.586)--(12.295,6.591)--(12.300,6.597)%
  --(12.306,6.603)--(12.312,6.608)--(12.318,6.614)--(12.324,6.620)--(12.330,6.625)--(12.336,6.631)%
  --(12.342,6.637)--(12.348,6.642)--(12.353,6.648)--(12.359,6.654)--(12.365,6.659)--(12.371,6.665)%
  --(12.377,6.671)--(12.383,6.676)--(12.389,6.682)--(12.395,6.688)--(12.400,6.694)--(12.406,6.699)%
  --(12.412,6.705)--(12.418,6.711)--(12.424,6.716)--(12.430,6.722)--(12.436,6.728)--(12.442,6.734)%
  --(12.447,6.739)--(12.453,6.745)--(12.459,6.751)--(12.465,6.756)--(12.471,6.762)--(12.477,6.768)%
  --(12.483,6.774)--(12.489,6.779)--(12.495,6.785)--(12.500,6.791)--(12.506,6.797)--(12.512,6.802)%
  --(12.518,6.808)--(12.524,6.814)--(12.530,6.820)--(12.536,6.826)--(12.542,6.831)--(12.547,6.837)%
  --(12.553,6.843)--(12.559,6.849)--(12.565,6.854)--(12.571,6.860)--(12.577,6.866)--(12.583,6.872)%
  --(12.589,6.878)--(12.594,6.883)--(12.600,6.889)--(12.606,6.895)--(12.612,6.901)--(12.618,6.907)%
  --(12.624,6.913)--(12.630,6.918)--(12.636,6.924)--(12.642,6.930)--(12.647,6.936)--(12.653,6.942)%
  --(12.659,6.948)--(12.665,6.953)--(12.671,6.959)--(12.677,6.965)--(12.683,6.971)--(12.689,6.977)%
  --(12.694,6.983)--(12.700,6.988)--(12.706,6.994)--(12.712,7.000)--(12.718,7.006)--(12.724,7.012)%
  --(12.730,7.018)--(12.736,7.024)--(12.741,7.030)--(12.747,7.035)--(12.753,7.041)--(12.759,7.047)%
  --(12.765,7.053)--(12.771,7.059)--(12.777,7.065)--(12.783,7.071)--(12.788,7.077)--(12.794,7.083)%
  --(12.800,7.089)--(12.806,7.094)--(12.812,7.100)--(12.818,7.106)--(12.824,7.112)--(12.830,7.118)%
  --(12.836,7.124)--(12.841,7.130)--(12.847,7.136)--(12.853,7.142)--(12.859,7.148)--(12.865,7.154)%
  --(12.871,7.160)--(12.877,7.166)--(12.883,7.172)--(12.888,7.178)--(12.894,7.184)--(12.900,7.189)%
  --(12.906,7.195)--(12.912,7.201)--(12.918,7.207)--(12.924,7.213)--(12.930,7.219)--(12.935,7.225)%
  --(12.941,7.231)--(12.947,7.237)--(12.953,7.243)--(12.959,7.249)--(12.965,7.255)--(12.971,7.261)%
  --(12.977,7.267)--(12.983,7.273)--(12.988,7.279)--(12.994,7.285)--(13.000,7.291)--(13.006,7.297)%
  --(13.012,7.303)--(13.018,7.309)--(13.024,7.315)--(13.030,7.321)--(13.035,7.327)--(13.041,7.333)%
  --(13.047,7.340)--(13.053,7.346)--(13.059,7.352)--(13.065,7.358)--(13.071,7.364)--(13.077,7.370)%
  --(13.082,7.376)--(13.088,7.382)--(13.094,7.388)--(13.100,7.394)--(13.106,7.400)--(13.112,7.406)%
  --(13.118,7.412)--(13.124,7.418)--(13.130,7.424)--(13.135,7.430)--(13.141,7.437)--(13.147,7.443)%
  --(13.153,7.449)--(13.159,7.455)--(13.165,7.461)--(13.171,7.467)--(13.177,7.473)--(13.182,7.479)%
  --(13.188,7.485)--(13.194,7.491)--(13.200,7.498)--(13.206,7.504)--(13.212,7.510)--(13.218,7.516)%
  --(13.224,7.522)--(13.229,7.528)--(13.235,7.534)--(13.241,7.541)--(13.247,7.547)--(13.253,7.553)%
  --(13.259,7.559);
\gpcolor{color=gp lt color border}
\node[gp node left] at (3.156,7.681) {$y = \np{14.24113} \, x - \np{38.07973}$, $r^2 = \np{0.90858}$};
\gpcolor{rgb color={0.000,0.000,0.000}}
\gpsetdashtype{gp dt solid}
\gpsetlinewidth{3.00}
\draw[gp path] (2.056,7.681)--(2.972,7.681);
\draw[gp path] (6.009,1.867)--(6.015,1.872)--(6.021,1.876)--(6.027,1.881)--(6.033,1.885)%
  --(6.039,1.889)--(6.045,1.894)--(6.051,1.898)--(6.056,1.903)--(6.062,1.907)--(6.068,1.912)%
  --(6.074,1.916)--(6.080,1.921)--(6.086,1.925)--(6.092,1.929)--(6.098,1.934)--(6.104,1.938)%
  --(6.109,1.943)--(6.115,1.947)--(6.121,1.952)--(6.127,1.956)--(6.133,1.961)--(6.139,1.965)%
  --(6.145,1.970)--(6.151,1.974)--(6.156,1.978)--(6.162,1.983)--(6.168,1.987)--(6.174,1.992)%
  --(6.180,1.996)--(6.186,2.001)--(6.192,2.005)--(6.198,2.010)--(6.203,2.014)--(6.209,2.018)%
  --(6.215,2.023)--(6.221,2.027)--(6.227,2.032)--(6.233,2.036)--(6.239,2.041)--(6.245,2.045)%
  --(6.250,2.050)--(6.256,2.054)--(6.262,2.059)--(6.268,2.063)--(6.274,2.067)--(6.280,2.072)%
  --(6.286,2.076)--(6.292,2.081)--(6.298,2.085)--(6.303,2.090)--(6.309,2.094)--(6.315,2.099)%
  --(6.321,2.103)--(6.327,2.107)--(6.333,2.112)--(6.339,2.116)--(6.345,2.121)--(6.350,2.125)%
  --(6.356,2.130)--(6.362,2.134)--(6.368,2.139)--(6.374,2.143)--(6.380,2.148)--(6.386,2.152)%
  --(6.392,2.156)--(6.397,2.161)--(6.403,2.165)--(6.409,2.170)--(6.415,2.174)--(6.421,2.179)%
  --(6.427,2.183)--(6.433,2.188)--(6.439,2.192)--(6.445,2.196)--(6.450,2.201)--(6.456,2.205)%
  --(6.462,2.210)--(6.468,2.214)--(6.474,2.219)--(6.480,2.223)--(6.486,2.228)--(6.492,2.232)%
  --(6.497,2.237)--(6.503,2.241)--(6.509,2.245)--(6.515,2.250)--(6.521,2.254)--(6.527,2.259)%
  --(6.533,2.263)--(6.539,2.268)--(6.544,2.272)--(6.550,2.277)--(6.556,2.281)--(6.562,2.285)%
  --(6.568,2.290)--(6.574,2.294)--(6.580,2.299)--(6.586,2.303)--(6.592,2.308)--(6.597,2.312)%
  --(6.603,2.317)--(6.609,2.321)--(6.615,2.326)--(6.621,2.330)--(6.627,2.334)--(6.633,2.339)%
  --(6.639,2.343)--(6.644,2.348)--(6.650,2.352)--(6.656,2.357)--(6.662,2.361)--(6.668,2.366)%
  --(6.674,2.370)--(6.680,2.374)--(6.686,2.379)--(6.691,2.383)--(6.697,2.388)--(6.703,2.392)%
  --(6.709,2.397)--(6.715,2.401)--(6.721,2.406)--(6.727,2.410)--(6.733,2.415)--(6.738,2.419)%
  --(6.744,2.423)--(6.750,2.428)--(6.756,2.432)--(6.762,2.437)--(6.768,2.441)--(6.774,2.446)%
  --(6.780,2.450)--(6.786,2.455)--(6.791,2.459)--(6.797,2.463)--(6.803,2.468)--(6.809,2.472)%
  --(6.815,2.477)--(6.821,2.481)--(6.827,2.486)--(6.833,2.490)--(6.838,2.495)--(6.844,2.499)%
  --(6.850,2.504)--(6.856,2.508)--(6.862,2.512)--(6.868,2.517)--(6.874,2.521)--(6.880,2.526)%
  --(6.885,2.530)--(6.891,2.535)--(6.897,2.539)--(6.903,2.544)--(6.909,2.548)--(6.915,2.552)%
  --(6.921,2.557)--(6.927,2.561)--(6.933,2.566)--(6.938,2.570)--(6.944,2.575)--(6.950,2.579)%
  --(6.956,2.584)--(6.962,2.588)--(6.968,2.593)--(6.974,2.597)--(6.980,2.601)--(6.985,2.606)%
  --(6.991,2.610)--(6.997,2.615)--(7.003,2.619)--(7.009,2.624)--(7.015,2.628)--(7.021,2.633)%
  --(7.027,2.637)--(7.032,2.641)--(7.038,2.646)--(7.044,2.650)--(7.050,2.655)--(7.056,2.659)%
  --(7.062,2.664)--(7.068,2.668)--(7.074,2.673)--(7.080,2.677)--(7.085,2.682)--(7.091,2.686)%
  --(7.097,2.690)--(7.103,2.695)--(7.109,2.699)--(7.115,2.704)--(7.121,2.708)--(7.127,2.713)%
  --(7.132,2.717)--(7.138,2.722)--(7.144,2.726)--(7.150,2.730)--(7.156,2.735)--(7.162,2.739)%
  --(7.168,2.744)--(7.174,2.748)--(7.179,2.753)--(7.185,2.757)--(7.191,2.762)--(7.197,2.766)%
  --(7.203,2.771)--(7.209,2.775)--(7.215,2.779)--(7.221,2.784)--(7.226,2.788)--(7.232,2.793)%
  --(7.238,2.797)--(7.244,2.802)--(7.250,2.806)--(7.256,2.811)--(7.262,2.815)--(7.268,2.819)%
  --(7.274,2.824)--(7.279,2.828)--(7.285,2.833)--(7.291,2.837)--(7.297,2.842)--(7.303,2.846)%
  --(7.309,2.851)--(7.315,2.855)--(7.321,2.860)--(7.326,2.864)--(7.332,2.868)--(7.338,2.873)%
  --(7.344,2.877)--(7.350,2.882)--(7.356,2.886)--(7.362,2.891)--(7.368,2.895)--(7.373,2.900)%
  --(7.379,2.904)--(7.385,2.908)--(7.391,2.913)--(7.397,2.917)--(7.403,2.922)--(7.409,2.926)%
  --(7.415,2.931)--(7.421,2.935)--(7.426,2.940)--(7.432,2.944)--(7.438,2.949)--(7.444,2.953)%
  --(7.450,2.957)--(7.456,2.962)--(7.462,2.966)--(7.468,2.971)--(7.473,2.975)--(7.479,2.980)%
  --(7.485,2.984)--(7.491,2.989)--(7.497,2.993)--(7.503,2.998)--(7.509,3.002)--(7.515,3.006)%
  --(7.520,3.011)--(7.526,3.015)--(7.532,3.020)--(7.538,3.024)--(7.544,3.029)--(7.550,3.033)%
  --(7.556,3.038)--(7.562,3.042)--(7.568,3.046)--(7.573,3.051)--(7.579,3.055)--(7.585,3.060)%
  --(7.591,3.064)--(7.597,3.069)--(7.603,3.073)--(7.609,3.078)--(7.615,3.082)--(7.620,3.087)%
  --(7.626,3.091)--(7.632,3.095)--(7.638,3.100)--(7.644,3.104)--(7.650,3.109)--(7.656,3.113)%
  --(7.662,3.118)--(7.667,3.122)--(7.673,3.127)--(7.679,3.131)--(7.685,3.135)--(7.691,3.140)%
  --(7.697,3.144)--(7.703,3.149)--(7.709,3.153)--(7.714,3.158)--(7.720,3.162)--(7.726,3.167)%
  --(7.732,3.171)--(7.738,3.176)--(7.744,3.180)--(7.750,3.184)--(7.756,3.189)--(7.762,3.193)%
  --(7.767,3.198)--(7.773,3.202)--(7.779,3.207)--(7.785,3.211)--(7.791,3.216)--(7.797,3.220)%
  --(7.803,3.224)--(7.809,3.229)--(7.814,3.233)--(7.820,3.238)--(7.826,3.242)--(7.832,3.247)%
  --(7.838,3.251)--(7.844,3.256)--(7.850,3.260)--(7.856,3.265)--(7.861,3.269)--(7.867,3.273)%
  --(7.873,3.278)--(7.879,3.282)--(7.885,3.287)--(7.891,3.291)--(7.897,3.296)--(7.903,3.300)%
  --(7.909,3.305)--(7.914,3.309)--(7.920,3.313)--(7.926,3.318)--(7.932,3.322)--(7.938,3.327)%
  --(7.944,3.331)--(7.950,3.336)--(7.956,3.340)--(7.961,3.345)--(7.967,3.349)--(7.973,3.354)%
  --(7.979,3.358)--(7.985,3.362)--(7.991,3.367)--(7.997,3.371)--(8.003,3.376)--(8.008,3.380)%
  --(8.014,3.385)--(8.020,3.389)--(8.026,3.394)--(8.032,3.398)--(8.038,3.402)--(8.044,3.407)%
  --(8.050,3.411)--(8.055,3.416)--(8.061,3.420)--(8.067,3.425)--(8.073,3.429)--(8.079,3.434)%
  --(8.085,3.438)--(8.091,3.443)--(8.097,3.447)--(8.103,3.451)--(8.108,3.456)--(8.114,3.460)%
  --(8.120,3.465)--(8.126,3.469)--(8.132,3.474)--(8.138,3.478)--(8.144,3.483)--(8.150,3.487)%
  --(8.155,3.491)--(8.161,3.496)--(8.167,3.500)--(8.173,3.505)--(8.179,3.509)--(8.185,3.514)%
  --(8.191,3.518)--(8.197,3.523)--(8.202,3.527)--(8.208,3.532)--(8.214,3.536)--(8.220,3.540)%
  --(8.226,3.545)--(8.232,3.549)--(8.238,3.554)--(8.244,3.558)--(8.250,3.563)--(8.255,3.567)%
  --(8.261,3.572)--(8.267,3.576)--(8.273,3.580)--(8.279,3.585)--(8.285,3.589)--(8.291,3.594)%
  --(8.297,3.598)--(8.302,3.603)--(8.308,3.607)--(8.314,3.612)--(8.320,3.616)--(8.326,3.621)%
  --(8.332,3.625)--(8.338,3.629)--(8.344,3.634)--(8.349,3.638)--(8.355,3.643)--(8.361,3.647)%
  --(8.367,3.652)--(8.373,3.656)--(8.379,3.661)--(8.385,3.665)--(8.391,3.669)--(8.397,3.674)%
  --(8.402,3.678)--(8.408,3.683)--(8.414,3.687)--(8.420,3.692)--(8.426,3.696)--(8.432,3.701)%
  --(8.438,3.705)--(8.444,3.710)--(8.449,3.714)--(8.455,3.718)--(8.461,3.723)--(8.467,3.727)%
  --(8.473,3.732)--(8.479,3.736)--(8.485,3.741)--(8.491,3.745)--(8.496,3.750)--(8.502,3.754)%
  --(8.508,3.758)--(8.514,3.763)--(8.520,3.767)--(8.526,3.772)--(8.532,3.776)--(8.538,3.781)%
  --(8.543,3.785)--(8.549,3.790)--(8.555,3.794)--(8.561,3.799)--(8.567,3.803)--(8.573,3.807)%
  --(8.579,3.812)--(8.585,3.816)--(8.591,3.821)--(8.596,3.825)--(8.602,3.830)--(8.608,3.834)%
  --(8.614,3.839)--(8.620,3.843)--(8.626,3.847)--(8.632,3.852)--(8.638,3.856)--(8.643,3.861)%
  --(8.649,3.865)--(8.655,3.870)--(8.661,3.874)--(8.667,3.879)--(8.673,3.883)--(8.679,3.888)%
  --(8.685,3.892)--(8.690,3.896)--(8.696,3.901)--(8.702,3.905)--(8.708,3.910)--(8.714,3.914)%
  --(8.720,3.919)--(8.726,3.923)--(8.732,3.928)--(8.738,3.932)--(8.743,3.936)--(8.749,3.941)%
  --(8.755,3.945)--(8.761,3.950)--(8.767,3.954)--(8.773,3.959)--(8.779,3.963)--(8.785,3.968)%
  --(8.790,3.972)--(8.796,3.977)--(8.802,3.981)--(8.808,3.985)--(8.814,3.990)--(8.820,3.994)%
  --(8.826,3.999)--(8.832,4.003)--(8.837,4.008)--(8.843,4.012)--(8.849,4.017)--(8.855,4.021)%
  --(8.861,4.025)--(8.867,4.030)--(8.873,4.034)--(8.879,4.039)--(8.885,4.043)--(8.890,4.048)%
  --(8.896,4.052)--(8.902,4.057)--(8.908,4.061)--(8.914,4.066)--(8.920,4.070)--(8.926,4.074)%
  --(8.932,4.079)--(8.937,4.083)--(8.943,4.088)--(8.949,4.092)--(8.955,4.097)--(8.961,4.101)%
  --(8.967,4.106)--(8.973,4.110)--(8.979,4.114)--(8.984,4.119)--(8.990,4.123)--(8.996,4.128)%
  --(9.002,4.132)--(9.008,4.137)--(9.014,4.141)--(9.020,4.146)--(9.026,4.150)--(9.031,4.155)%
  --(9.037,4.159)--(9.043,4.163)--(9.049,4.168)--(9.055,4.172)--(9.061,4.177)--(9.067,4.181)%
  --(9.073,4.186)--(9.079,4.190)--(9.084,4.195)--(9.090,4.199)--(9.096,4.204)--(9.102,4.208)%
  --(9.108,4.212)--(9.114,4.217)--(9.120,4.221)--(9.126,4.226)--(9.131,4.230)--(9.137,4.235)%
  --(9.143,4.239)--(9.149,4.244)--(9.155,4.248)--(9.161,4.252)--(9.167,4.257)--(9.173,4.261)%
  --(9.178,4.266)--(9.184,4.270)--(9.190,4.275)--(9.196,4.279)--(9.202,4.284)--(9.208,4.288)%
  --(9.214,4.293)--(9.220,4.297)--(9.226,4.301)--(9.231,4.306)--(9.237,4.310)--(9.243,4.315)%
  --(9.249,4.319)--(9.255,4.324)--(9.261,4.328)--(9.267,4.333)--(9.273,4.337)--(9.278,4.341)%
  --(9.284,4.346)--(9.290,4.350)--(9.296,4.355)--(9.302,4.359)--(9.308,4.364)--(9.314,4.368)%
  --(9.320,4.373)--(9.325,4.377)--(9.331,4.382)--(9.337,4.386)--(9.343,4.390)--(9.349,4.395)%
  --(9.355,4.399)--(9.361,4.404)--(9.367,4.408)--(9.373,4.413)--(9.378,4.417)--(9.384,4.422)%
  --(9.390,4.426)--(9.396,4.430)--(9.402,4.435)--(9.408,4.439)--(9.414,4.444)--(9.420,4.448)%
  --(9.425,4.453)--(9.431,4.457)--(9.437,4.462)--(9.443,4.466)--(9.449,4.471)--(9.455,4.475)%
  --(9.461,4.479)--(9.467,4.484)--(9.472,4.488)--(9.478,4.493)--(9.484,4.497)--(9.490,4.502)%
  --(9.496,4.506)--(9.502,4.511)--(9.508,4.515)--(9.514,4.519)--(9.519,4.524)--(9.525,4.528)%
  --(9.531,4.533)--(9.537,4.537)--(9.543,4.542)--(9.549,4.546)--(9.555,4.551)--(9.561,4.555)%
  --(9.567,4.560)--(9.572,4.564)--(9.578,4.568)--(9.584,4.573)--(9.590,4.577)--(9.596,4.582)%
  --(9.602,4.586)--(9.608,4.591)--(9.614,4.595)--(9.619,4.600)--(9.625,4.604)--(9.631,4.608)%
  --(9.637,4.613)--(9.643,4.617)--(9.649,4.622)--(9.655,4.626)--(9.661,4.631)--(9.666,4.635)%
  --(9.672,4.640)--(9.678,4.644)--(9.684,4.649)--(9.690,4.653)--(9.696,4.657)--(9.702,4.662)%
  --(9.708,4.666)--(9.714,4.671)--(9.719,4.675)--(9.725,4.680)--(9.731,4.684)--(9.737,4.689)%
  --(9.743,4.693)--(9.749,4.697)--(9.755,4.702);
\gpcolor{color=gp lt color border}
\gpsetlinewidth{1.00}
\draw[gp path] (1.688,8.631)--(1.688,0.985)--(13.447,0.985)--(13.447,8.631)--cycle;
%% coordinates of the plot area
\gpdefrectangularnode{gp plot 1}{\pgfpoint{1.688cm}{0.985cm}}{\pgfpoint{13.447cm}{8.631cm}}
\end{tikzpicture}
%% gnuplot variables

\end{figure*}

Determinar o padrão seguido pelos pontos, é, portanto, uma tarefa que não pode ser feita a partir de um gráfico: o mais adequado é termos uma \emph{teoria acerca do fenômeno físico} que descreva qual é o padrão que os pontos devem seguir, e usar um gráfico para determinar se tal descrição é coerente. Verificaremos posteriormente como determinar se uma teoria é plausível ou não, por hora vamos nos preocupar em determinar a \emph{melhor reta} que ajusta um conjunto de dados.

Existem outros tipos de regressão, como a logarítmica ou exponencial, ou mesmo processos capazes de calcular os coeficientes para uma equação com uma forma qualquer. No entanto, nos restringiremos ao caso linear devido ao fato de que a maioria das calculadoras científicas é capaz de realizar tal processo. Além disso, uma vez conhecida a equação da reta, adicioná-la ao gráfico dos dados experimentais é uma questão de calcular dois pontos e traçar uma reta com uma régua.% \comment{se for colocar mais tipos, tem que tirar esse ``nos restringiremos''.}

%%%%%%%%%%%%%%%%%%%%%%%%%%
\section{Regressão Linear}
\label{Sec:RegressaoLinear}
%%%%%%%%%%%%%%%%%%%%%%%%%%

Sempre que tivermos um conjunto de dados, podemos calcular a melhor reta que o representa através de um processo de \emph{regressão linear}. Este processo consiste em aplicar um método matemático que tome os dados experimentais e calcule os coeficientes linear $A$ e angular $B$ para a equação da reta:
\begin{equation}
	y = A + Bx.
\end{equation}
%
A dimensão do coeficiente linear é a mesma que a da variável dependente $y$ ---~ou seja, é a mesma que do eixo $y$ em um gráfico $y \times x$~---, enquanto a dimensão do coeficiente angular é a mesma que a da razão entre a dimensão da variável dependente pela dimensão da variável independente ---~isto é, é a razão entre a dimensão do eixo $y$ e a dimensão do eixo $x$ em um gráfico $y \times x$~---. 

O método utilizado para obter tais coeficientes é o de \emph{mínimos quadrados}. Nele, são encontrados os coeficientes de forma a minimizar o quadrado da distância entre os pontos experimentais e a ``melhor reta''. É possível mostrar\cite{Taylor}, utilizando técnicas de cálculo, que para minimizarmos a soma do quadrado das distâncias entre os pontos e a melhor reta, os coeficientes linear e angular são dados por
\begin{align}
	A &= \frac{\sum x_i^2 \sum y_i - \sum x_i \sum x_iy_i}{N \sum x_i^2 - (\sum x_i)^2} \\
	B &= \frac{N\sum x_iy_i - \sum x_i \sum y_i}{N \sum x_i^2 - (\sum x_i)^2}.
\end{align}
%
onde as somas se dão sobre todas mas medidas $x_i$ e $y_i$, $N$ representa o número de medidas, e as médias das medidas para as variáveis $x$ e $y$ são representadas por $\mean{x}$ e $\mean{y}$, respectivamente. É importante lembrar que as relações acima são válidas somente para o caso em que os erros associados às medidas de $x$ e $y$ são constantes, isto é, a incerteza $\delta x$ para as medidas de $x$ são todas iguais, assim como a incerteza $\delta y$ para as medidas de $y$. Após fazer a regressão, podemos utilizar os coeficientes para traçar as \emph{retas de tendência} nos gráficos\footnote{Quando tais retas forem calculadas, adicione as equações resultantes aos gráficos.}.

Ao fazermos o processo de regressão linear, a calculadora também calculará o \emph{coeficiente de correlação linear} $r$, dado por
\begin{equation}
	r = \frac{\sum (x_i - \mean{x})(y_i-\mean{y})}{\sqrt{\sum(x_i - \mean{x})^2\sum (y_i - \mean{y})^2}}.
\end{equation}
%
Tal coeficiente pode ser interpretado como um ``índice de confiança'' e geralmente é calculado ao quadrado, pois seus valores podem variar entre $-1$ e $1$. Quanto mais próximo $r$ for de $\pm 1$, menor é a dispersão dos pontos em relação ao comportamento retilíneo, ou seja, maiores são as chances de que o fenômeno estudado e que deu origem aos dados siga uma relação linear. Este número geralmente se parece com algo como $r^2=\np{0,99998}$, $r^2=\np{0,997}$, $r^2= \np{0,990}$, etc., quando os dados são altamente lineares e relativamente abundantes. Se a dispersão em relação ao comportamento linear for grande e forem poucos os pontos, $r^2$ pode cair para \np{0,7}, ou valores menores.

%%%%%%%%%%%%%%%%%%%%%%%%%%%%%%%%%%%%%%%%%%
\paragraph{Regressão linear e linearidade}
%%%%%%%%%%%%%%%%%%%%%%%%%%%%%%%%%%%%%%%%%%

\begin{margintable}
\centering
\label{tab:TabelaDadosLin}
\begin{tabular}{cccc}
\toprule     
$x$ & $y_1$ & $y_2$ & $y_3$ \\
\midrule      
\np{0.714}  & \np{14.577}  &   \np{0.678}   &  \np{ 1.235} \\	
\np{2.693}  & \np{20.696}  &   \np{8.806}   &  \np{-0.153} \\	
\np{4.389}  & \np{25.226}  &   \np{20.988}  &  \np{-0.657} \\ 
\np{4.960}  & \np{27.449}  &   \np{28.474}  &  \np{-0.592} \\	
\np{6.245}  & \np{30.242}  &   \np{40.030}  &  \np{ 0.961} \\ 
\np{7.277}  & \np{33.378}  &   \np{55.780}  &  \np{ 1.149} \\ 
\np{7.579}  & \np{34.195}  &   \np{64.904}  &  \np{ 1.635} \\ 
\np{7.719}  & \np{35.715}  &   \np{64.394}  &  \np{ 1.476} \\ 
\np{7.912}  & \np{35.011}  &   \np{66.412}  &  \np{ 1.739} \\ 
\np{8.280}  & \np{37.529}  &   \np{74.632}  &  \np{ 1.249} \\ 
\np{9.034}  & \np{40.590}  &   \np{87.455}  &  \np{ 0.200} \\ 
\np{9.442}  & \np{39.156}  &   \np{94.785}  &  \np{ 0.146} \\ 
\np{10.306} & \np{43.238}  &   \np{113.030} &  \np{-0.231} \\	
\np{10.572} & \np{42.406}  &   \np{111.970} &  \np{-0.575} \\		
\np{11.177} & \np{44.796}  &   \np{129.481} &  \np{ 0.121} \\
\np{15.335} & \np{57.611}  &   \np{235.805} &  \np{ 0.446} \\
\np{17.023} & \np{63.832}  &   \np{294.533} &  \np{-0.780} \\
\np{18.926} & \np{68.063}  &   \np{372.048} &  \np{ 0.500} \\
\np{20.608} & \np{74.408}  &   \np{426.581} &  \np{ 1.118} \\
\np{20.876} & \np{75.083}  &   \np{456.391} &  \np{ 1.065} \\
\np{21.095} & \np{75.248}  &   \np{452.660} &  \np{ 1.085} \\
\np{22.225} & \np{77.243}  &   \np{507.765} &  \np{ 0.271} \\
\np{22.407} & \np{81.058}  &   \np{509.275} &  \np{-0.230} \\
\np{22.469} & \np{78.821}  &   \np{521.414} &  \np{ 0.394} \\
\np{23.077} & \np{80.714}  &   \np{554.175} &  \np{-0.198} \\
\np{26.421} & \np{91.433}  &   \np{700.788} &  \np{ 1.869} \\
\np{26.863} & \np{91.777}  &   \np{732.601} &  \np{ 1.724} \\
\np{27.360} & \np{93.291}  &   \np{773.226} &  \np{ 1.452} \\
\bottomrule
\end{tabular}
\vspace{1mm}
\caption{Dados de três variáveis ($y_1$, $y_2$ e $y_3$) em função de uma quarta ($x$).}
\end{margintable}

Um aspecto importante a ser considerado é se os dados realmente seguem uma tendência linear: uma distribuição qualquer de pontos, mesmo que visivelmente não linear, pode ser descrita por uma melhor reta. Caso a tendência não seja linear, o coeficiente $r^2$ resultará um valor baixo. Nas Figuras~\ref{RetasConjuntosDados1} a~\ref{RetasConjuntosDados3} mostramos três conjuntos de dados distintos  juntamente com suas respectivas melhores retas e coeficientes de dispersão $r^2$. Os dados foram gerados utilizando as expressões $y_1 = A + B(x+r_1) + r_2$, $y_2 = A(r_1+x)^2 + B + r_2$ e $y_3 = A\sen (x+r_1) + B + r_2$, onde $r_1$ e $r_2$ são números aleatórios entre zero e um, enquanto $A$ e $B$ representam constantes arbitrárias ---~veja a Tabela~\ref{tab:TabelaDadosLin}~---. Verificamos que nos três casos podemos calcular uma melhor reta, ainda que para $y_2$ e $y_3$ o comportamento não seja linear.

Dentre as três figuras, a Figura~\ref{RetasConjuntosDados2} se destaca. Nesse caso, temos um comportamento não linear, porém verificamos que o valor do coeficiente $r^2$ é relativamente alto. Ocorre que o comportamento dos dados sofre uma distorção relativamente pequena ---~apesar de visivelmente não linear~--- em relação a um comportamento linear e por isso obtemos um valor grande para o coeficiente de dispersão. Muitas vezes um comportamento verdadeiramente linear pode apresentar uma distorção pequena como esta, porém devido a flutuações aleatórias, e não devido à não-lineariedade. A distinção entre essas duas possibilidades pode ser feita realizando mais medidas: Se o comportamento for verdadeiramente linear, o valor de $r^2$ tenderá a crescer, enquanto no caso de termos um comportamento não-linear, o valor desse coeficiente deve permanecer constante, ou diminuir.

\begin{figure}[!h]\forcerectofloat
\centering
\begin{tikzpicture}[gnuplot]
%% generated with GNUPLOT 5.0p6 (Lua 5.3; terminal rev. 99, script rev. 100)
%% seg 30 jul 2018 14:02:03 -03
\path (0.000,0.000) rectangle (10.000,7.000);
\gpcolor{color=gp lt color border}
\gpsetlinetype{gp lt border}
\gpsetdashtype{gp dt solid}
\gpsetlinewidth{1.00}
\draw[gp path] (1.320,1.792)--(1.500,1.792);
\draw[gp path] (9.447,1.792)--(9.267,1.792);
\node[gp node right] at (1.136,1.792) {$20$};
\draw[gp path] (1.320,2.867)--(1.500,2.867);
\draw[gp path] (9.447,2.867)--(9.267,2.867);
\node[gp node right] at (1.136,2.867) {$40$};
\draw[gp path] (1.320,3.942)--(1.500,3.942);
\draw[gp path] (9.447,3.942)--(9.267,3.942);
\node[gp node right] at (1.136,3.942) {$60$};
\draw[gp path] (1.320,5.018)--(1.500,5.018);
\draw[gp path] (9.447,5.018)--(9.267,5.018);
\node[gp node right] at (1.136,5.018) {$80$};
\draw[gp path] (1.320,6.093)--(1.500,6.093);
\draw[gp path] (9.447,6.093)--(9.267,6.093);
\node[gp node right] at (1.136,6.093) {$100$};
\draw[gp path] (1.582,0.985)--(1.582,1.165);
\draw[gp path] (1.582,6.631)--(1.582,6.451);
\node[gp node center] at (1.582,0.677) {$0$};
\draw[gp path] (2.893,0.985)--(2.893,1.165);
\draw[gp path] (2.893,6.631)--(2.893,6.451);
\node[gp node center] at (2.893,0.677) {$5$};
\draw[gp path] (4.204,0.985)--(4.204,1.165);
\draw[gp path] (4.204,6.631)--(4.204,6.451);
\node[gp node center] at (4.204,0.677) {$10$};
\draw[gp path] (5.515,0.985)--(5.515,1.165);
\draw[gp path] (5.515,6.631)--(5.515,6.451);
\node[gp node center] at (5.515,0.677) {$15$};
\draw[gp path] (6.825,0.985)--(6.825,1.165);
\draw[gp path] (6.825,6.631)--(6.825,6.451);
\node[gp node center] at (6.825,0.677) {$20$};
\draw[gp path] (8.136,0.985)--(8.136,1.165);
\draw[gp path] (8.136,6.631)--(8.136,6.451);
\node[gp node center] at (8.136,0.677) {$25$};
\draw[gp path] (9.447,0.985)--(9.447,1.165);
\draw[gp path] (9.447,6.631)--(9.447,6.451);
\node[gp node center] at (9.447,0.677) {$30$};
\draw[gp path] (1.320,6.631)--(1.320,0.985)--(9.447,0.985)--(9.447,6.631)--cycle;
\node[gp node center,rotate=-270] at (0.246,3.808) {$y_1$};
\node[gp node center] at (5.383,0.215) {$x$};
\node[gp node left] at (2.788,6.297) {Dados experimentais};
\gpcolor{rgb color={0.000,0.000,0.000}}
\gpsetlinewidth{2.00}
\gpsetpointsize{4.00}
\gppoint{gp mark 7}{(1.769,1.500)}
\gppoint{gp mark 7}{(2.288,1.829)}
\gppoint{gp mark 7}{(2.733,2.073)}
\gppoint{gp mark 7}{(2.882,2.192)}
\gppoint{gp mark 7}{(3.219,2.342)}
\gppoint{gp mark 7}{(3.490,2.511)}
\gppoint{gp mark 7}{(3.569,2.555)}
\gppoint{gp mark 7}{(3.606,2.637)}
\gppoint{gp mark 7}{(3.656,2.599)}
\gppoint{gp mark 7}{(3.753,2.734)}
\gppoint{gp mark 7}{(3.950,2.899)}
\gppoint{gp mark 7}{(4.057,2.822)}
\gppoint{gp mark 7}{(4.284,3.041)}
\gppoint{gp mark 7}{(4.354,2.996)}
\gppoint{gp mark 7}{(4.512,3.125)}
\gppoint{gp mark 7}{(5.602,3.814)}
\gppoint{gp mark 7}{(6.045,4.148)}
\gppoint{gp mark 7}{(6.544,4.376)}
\gppoint{gp mark 7}{(6.985,4.717)}
\gppoint{gp mark 7}{(7.055,4.753)}
\gppoint{gp mark 7}{(7.112,4.762)}
\gppoint{gp mark 7}{(7.409,4.870)}
\gppoint{gp mark 7}{(7.457,5.075)}
\gppoint{gp mark 7}{(7.473,4.954)}
\gppoint{gp mark 7}{(7.632,5.056)}
\gppoint{gp mark 7}{(8.509,5.633)}
\gppoint{gp mark 7}{(8.624,5.651)}
\gppoint{gp mark 7}{(8.755,5.733)}
\gppoint{gp mark 7}{(2.146,6.297)}
\gpcolor{color=gp lt color border}
\node[gp node left] at (2.788,5.989) {Regressão linear};
\gpcolor{rgb color={0.000,0.000,0.000}}
\draw[gp path] (1.688,5.989)--(2.604,5.989);
\draw[gp path] (1.714,1.450)--(1.718,1.452)--(1.722,1.455)--(1.726,1.457)--(1.730,1.460)%
  --(1.734,1.462)--(1.739,1.465)--(1.743,1.467)--(1.747,1.470)--(1.751,1.472)--(1.755,1.475)%
  --(1.759,1.477)--(1.763,1.480)--(1.767,1.482)--(1.771,1.485)--(1.775,1.487)--(1.779,1.490)%
  --(1.783,1.492)--(1.787,1.495)--(1.791,1.497)--(1.795,1.500)--(1.799,1.502)--(1.804,1.504)%
  --(1.808,1.507)--(1.812,1.509)--(1.816,1.512)--(1.820,1.514)--(1.824,1.517)--(1.828,1.519)%
  --(1.832,1.522)--(1.836,1.524)--(1.840,1.527)--(1.844,1.529)--(1.848,1.532)--(1.852,1.534)%
  --(1.856,1.537)--(1.860,1.539)--(1.865,1.542)--(1.869,1.544)--(1.873,1.547)--(1.877,1.549)%
  --(1.881,1.552)--(1.885,1.554)--(1.889,1.557)--(1.893,1.559)--(1.897,1.562)--(1.901,1.564)%
  --(1.905,1.567)--(1.909,1.569)--(1.913,1.572)--(1.917,1.574)--(1.921,1.577)--(1.925,1.579)%
  --(1.930,1.582)--(1.934,1.584)--(1.938,1.587)--(1.942,1.589)--(1.946,1.592)--(1.950,1.594)%
  --(1.954,1.597)--(1.958,1.599)--(1.962,1.601)--(1.966,1.604)--(1.970,1.606)--(1.974,1.609)%
  --(1.978,1.611)--(1.982,1.614)--(1.986,1.616)--(1.990,1.619)--(1.995,1.621)--(1.999,1.624)%
  --(2.003,1.626)--(2.007,1.629)--(2.011,1.631)--(2.015,1.634)--(2.019,1.636)--(2.023,1.639)%
  --(2.027,1.641)--(2.031,1.644)--(2.035,1.646)--(2.039,1.649)--(2.043,1.651)--(2.047,1.654)%
  --(2.051,1.656)--(2.055,1.659)--(2.060,1.661)--(2.064,1.664)--(2.068,1.666)--(2.072,1.669)%
  --(2.076,1.671)--(2.080,1.674)--(2.084,1.676)--(2.088,1.679)--(2.092,1.681)--(2.096,1.684)%
  --(2.100,1.686)--(2.104,1.689)--(2.108,1.691)--(2.112,1.694)--(2.116,1.696)--(2.121,1.698)%
  --(2.125,1.701)--(2.129,1.703)--(2.133,1.706)--(2.137,1.708)--(2.141,1.711)--(2.145,1.713)%
  --(2.149,1.716)--(2.153,1.718)--(2.157,1.721)--(2.161,1.723)--(2.165,1.726)--(2.169,1.728)%
  --(2.173,1.731)--(2.177,1.733)--(2.181,1.736)--(2.186,1.738)--(2.190,1.741)--(2.194,1.743)%
  --(2.198,1.746)--(2.202,1.748)--(2.206,1.751)--(2.210,1.753)--(2.214,1.756)--(2.218,1.758)%
  --(2.222,1.761)--(2.226,1.763)--(2.230,1.766)--(2.234,1.768)--(2.238,1.771)--(2.242,1.773)%
  --(2.246,1.776)--(2.251,1.778)--(2.255,1.781)--(2.259,1.783)--(2.263,1.786)--(2.267,1.788)%
  --(2.271,1.791)--(2.275,1.793)--(2.279,1.796)--(2.283,1.798)--(2.287,1.800)--(2.291,1.803)%
  --(2.295,1.805)--(2.299,1.808)--(2.303,1.810)--(2.307,1.813)--(2.311,1.815)--(2.316,1.818)%
  --(2.320,1.820)--(2.324,1.823)--(2.328,1.825)--(2.332,1.828)--(2.336,1.830)--(2.340,1.833)%
  --(2.344,1.835)--(2.348,1.838)--(2.352,1.840)--(2.356,1.843)--(2.360,1.845)--(2.364,1.848)%
  --(2.368,1.850)--(2.372,1.853)--(2.377,1.855)--(2.381,1.858)--(2.385,1.860)--(2.389,1.863)%
  --(2.393,1.865)--(2.397,1.868)--(2.401,1.870)--(2.405,1.873)--(2.409,1.875)--(2.413,1.878)%
  --(2.417,1.880)--(2.421,1.883)--(2.425,1.885)--(2.429,1.888)--(2.433,1.890)--(2.437,1.893)%
  --(2.442,1.895)--(2.446,1.897)--(2.450,1.900)--(2.454,1.902)--(2.458,1.905)--(2.462,1.907)%
  --(2.466,1.910)--(2.470,1.912)--(2.474,1.915)--(2.478,1.917)--(2.482,1.920)--(2.486,1.922)%
  --(2.490,1.925)--(2.494,1.927)--(2.498,1.930)--(2.502,1.932)--(2.507,1.935)--(2.511,1.937)%
  --(2.515,1.940)--(2.519,1.942)--(2.523,1.945)--(2.527,1.947)--(2.531,1.950)--(2.535,1.952)%
  --(2.539,1.955)--(2.543,1.957)--(2.547,1.960)--(2.551,1.962)--(2.555,1.965)--(2.559,1.967)%
  --(2.563,1.970)--(2.567,1.972)--(2.572,1.975)--(2.576,1.977)--(2.580,1.980)--(2.584,1.982)%
  --(2.588,1.985)--(2.592,1.987)--(2.596,1.990)--(2.600,1.992)--(2.604,1.995)--(2.608,1.997)%
  --(2.612,1.999)--(2.616,2.002)--(2.620,2.004)--(2.624,2.007)--(2.628,2.009)--(2.633,2.012)%
  --(2.637,2.014)--(2.641,2.017)--(2.645,2.019)--(2.649,2.022)--(2.653,2.024)--(2.657,2.027)%
  --(2.661,2.029)--(2.665,2.032)--(2.669,2.034)--(2.673,2.037)--(2.677,2.039)--(2.681,2.042)%
  --(2.685,2.044)--(2.689,2.047)--(2.693,2.049)--(2.698,2.052)--(2.702,2.054)--(2.706,2.057)%
  --(2.710,2.059)--(2.714,2.062)--(2.718,2.064)--(2.722,2.067)--(2.726,2.069)--(2.730,2.072)%
  --(2.734,2.074)--(2.738,2.077)--(2.742,2.079)--(2.746,2.082)--(2.750,2.084)--(2.754,2.087)%
  --(2.758,2.089)--(2.763,2.092)--(2.767,2.094)--(2.771,2.096)--(2.775,2.099)--(2.779,2.101)%
  --(2.783,2.104)--(2.787,2.106)--(2.791,2.109)--(2.795,2.111)--(2.799,2.114)--(2.803,2.116)%
  --(2.807,2.119)--(2.811,2.121)--(2.815,2.124)--(2.819,2.126)--(2.823,2.129)--(2.828,2.131)%
  --(2.832,2.134)--(2.836,2.136)--(2.840,2.139)--(2.844,2.141)--(2.848,2.144)--(2.852,2.146)%
  --(2.856,2.149)--(2.860,2.151)--(2.864,2.154)--(2.868,2.156)--(2.872,2.159)--(2.876,2.161)%
  --(2.880,2.164)--(2.884,2.166)--(2.889,2.169)--(2.893,2.171)--(2.897,2.174)--(2.901,2.176)%
  --(2.905,2.179)--(2.909,2.181)--(2.913,2.184)--(2.917,2.186)--(2.921,2.189)--(2.925,2.191)%
  --(2.929,2.193)--(2.933,2.196)--(2.937,2.198)--(2.941,2.201)--(2.945,2.203)--(2.949,2.206)%
  --(2.954,2.208)--(2.958,2.211)--(2.962,2.213)--(2.966,2.216)--(2.970,2.218)--(2.974,2.221)%
  --(2.978,2.223)--(2.982,2.226)--(2.986,2.228)--(2.990,2.231)--(2.994,2.233)--(2.998,2.236)%
  --(3.002,2.238)--(3.006,2.241)--(3.010,2.243)--(3.014,2.246)--(3.019,2.248)--(3.023,2.251)%
  --(3.027,2.253)--(3.031,2.256)--(3.035,2.258)--(3.039,2.261)--(3.043,2.263)--(3.047,2.266)%
  --(3.051,2.268)--(3.055,2.271)--(3.059,2.273)--(3.063,2.276)--(3.067,2.278)--(3.071,2.281)%
  --(3.075,2.283)--(3.079,2.286)--(3.084,2.288)--(3.088,2.291)--(3.092,2.293)--(3.096,2.295)%
  --(3.100,2.298)--(3.104,2.300)--(3.108,2.303)--(3.112,2.305)--(3.116,2.308)--(3.120,2.310)%
  --(3.124,2.313)--(3.128,2.315)--(3.132,2.318)--(3.136,2.320)--(3.140,2.323)--(3.145,2.325)%
  --(3.149,2.328)--(3.153,2.330)--(3.157,2.333)--(3.161,2.335)--(3.165,2.338)--(3.169,2.340)%
  --(3.173,2.343)--(3.177,2.345)--(3.181,2.348)--(3.185,2.350)--(3.189,2.353)--(3.193,2.355)%
  --(3.197,2.358)--(3.201,2.360)--(3.205,2.363)--(3.210,2.365)--(3.214,2.368)--(3.218,2.370)%
  --(3.222,2.373)--(3.226,2.375)--(3.230,2.378)--(3.234,2.380)--(3.238,2.383)--(3.242,2.385)%
  --(3.246,2.388)--(3.250,2.390)--(3.254,2.392)--(3.258,2.395)--(3.262,2.397)--(3.266,2.400)%
  --(3.270,2.402)--(3.275,2.405)--(3.279,2.407)--(3.283,2.410)--(3.287,2.412)--(3.291,2.415)%
  --(3.295,2.417)--(3.299,2.420)--(3.303,2.422)--(3.307,2.425)--(3.311,2.427)--(3.315,2.430)%
  --(3.319,2.432)--(3.323,2.435)--(3.327,2.437)--(3.331,2.440)--(3.335,2.442)--(3.340,2.445)%
  --(3.344,2.447)--(3.348,2.450)--(3.352,2.452)--(3.356,2.455)--(3.360,2.457)--(3.364,2.460)%
  --(3.368,2.462)--(3.372,2.465)--(3.376,2.467)--(3.380,2.470)--(3.384,2.472)--(3.388,2.475)%
  --(3.392,2.477)--(3.396,2.480)--(3.401,2.482)--(3.405,2.485)--(3.409,2.487)--(3.413,2.490)%
  --(3.417,2.492)--(3.421,2.494)--(3.425,2.497)--(3.429,2.499)--(3.433,2.502)--(3.437,2.504)%
  --(3.441,2.507)--(3.445,2.509)--(3.449,2.512)--(3.453,2.514)--(3.457,2.517)--(3.461,2.519)%
  --(3.466,2.522)--(3.470,2.524)--(3.474,2.527)--(3.478,2.529)--(3.482,2.532)--(3.486,2.534)%
  --(3.490,2.537)--(3.494,2.539)--(3.498,2.542)--(3.502,2.544)--(3.506,2.547)--(3.510,2.549)%
  --(3.514,2.552)--(3.518,2.554)--(3.522,2.557)--(3.526,2.559)--(3.531,2.562)--(3.535,2.564)%
  --(3.539,2.567)--(3.543,2.569)--(3.547,2.572)--(3.551,2.574)--(3.555,2.577)--(3.559,2.579)%
  --(3.563,2.582)--(3.567,2.584)--(3.571,2.587)--(3.575,2.589)--(3.579,2.591)--(3.583,2.594)%
  --(3.587,2.596)--(3.591,2.599)--(3.596,2.601)--(3.600,2.604)--(3.604,2.606)--(3.608,2.609)%
  --(3.612,2.611)--(3.616,2.614)--(3.620,2.616)--(3.624,2.619)--(3.628,2.621)--(3.632,2.624)%
  --(3.636,2.626)--(3.640,2.629)--(3.644,2.631)--(3.648,2.634)--(3.652,2.636)--(3.657,2.639)%
  --(3.661,2.641)--(3.665,2.644)--(3.669,2.646)--(3.673,2.649)--(3.677,2.651)--(3.681,2.654)%
  --(3.685,2.656)--(3.689,2.659)--(3.693,2.661)--(3.697,2.664)--(3.701,2.666)--(3.705,2.669)%
  --(3.709,2.671)--(3.713,2.674)--(3.717,2.676)--(3.722,2.679)--(3.726,2.681)--(3.730,2.684)%
  --(3.734,2.686)--(3.738,2.688)--(3.742,2.691)--(3.746,2.693)--(3.750,2.696)--(3.754,2.698)%
  --(3.758,2.701)--(3.762,2.703)--(3.766,2.706)--(3.770,2.708)--(3.774,2.711)--(3.778,2.713)%
  --(3.782,2.716)--(3.787,2.718)--(3.791,2.721)--(3.795,2.723)--(3.799,2.726)--(3.803,2.728)%
  --(3.807,2.731)--(3.811,2.733)--(3.815,2.736)--(3.819,2.738)--(3.823,2.741)--(3.827,2.743)%
  --(3.831,2.746)--(3.835,2.748)--(3.839,2.751)--(3.843,2.753)--(3.847,2.756)--(3.852,2.758)%
  --(3.856,2.761)--(3.860,2.763)--(3.864,2.766)--(3.868,2.768)--(3.872,2.771)--(3.876,2.773)%
  --(3.880,2.776)--(3.884,2.778)--(3.888,2.781)--(3.892,2.783)--(3.896,2.786)--(3.900,2.788)%
  --(3.904,2.790)--(3.908,2.793)--(3.913,2.795)--(3.917,2.798)--(3.921,2.800)--(3.925,2.803)%
  --(3.929,2.805)--(3.933,2.808)--(3.937,2.810)--(3.941,2.813)--(3.945,2.815)--(3.949,2.818)%
  --(3.953,2.820)--(3.957,2.823)--(3.961,2.825)--(3.965,2.828)--(3.969,2.830)--(3.973,2.833)%
  --(3.978,2.835)--(3.982,2.838)--(3.986,2.840)--(3.990,2.843)--(3.994,2.845)--(3.998,2.848)%
  --(4.002,2.850)--(4.006,2.853)--(4.010,2.855)--(4.014,2.858)--(4.018,2.860)--(4.022,2.863)%
  --(4.026,2.865)--(4.030,2.868)--(4.034,2.870)--(4.038,2.873)--(4.043,2.875)--(4.047,2.878)%
  --(4.051,2.880)--(4.055,2.883)--(4.059,2.885)--(4.063,2.887)--(4.067,2.890)--(4.071,2.892)%
  --(4.075,2.895)--(4.079,2.897)--(4.083,2.900)--(4.087,2.902)--(4.091,2.905)--(4.095,2.907)%
  --(4.099,2.910)--(4.103,2.912)--(4.108,2.915)--(4.112,2.917)--(4.116,2.920)--(4.120,2.922)%
  --(4.124,2.925)--(4.128,2.927)--(4.132,2.930)--(4.136,2.932)--(4.140,2.935)--(4.144,2.937)%
  --(4.148,2.940)--(4.152,2.942)--(4.156,2.945)--(4.160,2.947)--(4.164,2.950)--(4.169,2.952)%
  --(4.173,2.955)--(4.177,2.957)--(4.181,2.960)--(4.185,2.962)--(4.189,2.965)--(4.193,2.967)%
  --(4.197,2.970)--(4.201,2.972)--(4.205,2.975)--(4.209,2.977)--(4.213,2.980)--(4.217,2.982)%
  --(4.221,2.985)--(4.225,2.987)--(4.229,2.989)--(4.234,2.992)--(4.238,2.994)--(4.242,2.997)%
  --(4.246,2.999)--(4.250,3.002)--(4.254,3.004)--(4.258,3.007)--(4.262,3.009)--(4.266,3.012)%
  --(4.270,3.014)--(4.274,3.017)--(4.278,3.019)--(4.282,3.022)--(4.286,3.024)--(4.290,3.027)%
  --(4.294,3.029)--(4.299,3.032)--(4.303,3.034)--(4.307,3.037)--(4.311,3.039)--(4.315,3.042)%
  --(4.319,3.044)--(4.323,3.047)--(4.327,3.049)--(4.331,3.052)--(4.335,3.054)--(4.339,3.057)%
  --(4.343,3.059)--(4.347,3.062)--(4.351,3.064)--(4.355,3.067)--(4.359,3.069)--(4.364,3.072)%
  --(4.368,3.074)--(4.372,3.077)--(4.376,3.079)--(4.380,3.082)--(4.384,3.084)--(4.388,3.086)%
  --(4.392,3.089)--(4.396,3.091)--(4.400,3.094)--(4.404,3.096)--(4.408,3.099)--(4.412,3.101)%
  --(4.416,3.104)--(4.420,3.106)--(4.425,3.109)--(4.429,3.111)--(4.433,3.114)--(4.437,3.116)%
  --(4.441,3.119)--(4.445,3.121)--(4.449,3.124)--(4.453,3.126)--(4.457,3.129)--(4.461,3.131)%
  --(4.465,3.134)--(4.469,3.136)--(4.473,3.139)--(4.477,3.141)--(4.481,3.144)--(4.485,3.146)%
  --(4.490,3.149)--(4.494,3.151)--(4.498,3.154)--(4.502,3.156)--(4.506,3.159)--(4.510,3.161)%
  --(4.514,3.164)--(4.518,3.166)--(4.522,3.169)--(4.526,3.171)--(4.530,3.174)--(4.534,3.176)%
  --(4.538,3.179)--(4.542,3.181)--(4.546,3.183)--(4.550,3.186)--(4.555,3.188)--(4.559,3.191)%
  --(4.563,3.193)--(4.567,3.196)--(4.571,3.198)--(4.575,3.201)--(4.579,3.203)--(4.583,3.206)%
  --(4.587,3.208)--(4.591,3.211)--(4.595,3.213)--(4.599,3.216)--(4.603,3.218)--(4.607,3.221)%
  --(4.611,3.223)--(4.615,3.226)--(4.620,3.228)--(4.624,3.231)--(4.628,3.233)--(4.632,3.236)%
  --(4.636,3.238)--(4.640,3.241)--(4.644,3.243)--(4.648,3.246)--(4.652,3.248)--(4.656,3.251)%
  --(4.660,3.253)--(4.664,3.256)--(4.668,3.258)--(4.672,3.261)--(4.676,3.263)--(4.681,3.266)%
  --(4.685,3.268)--(4.689,3.271)--(4.693,3.273)--(4.697,3.276)--(4.701,3.278)--(4.705,3.281)%
  --(4.709,3.283)--(4.713,3.285)--(4.717,3.288)--(4.721,3.290)--(4.725,3.293)--(4.729,3.295)%
  --(4.733,3.298)--(4.737,3.300)--(4.741,3.303)--(4.746,3.305)--(4.750,3.308)--(4.754,3.310)%
  --(4.758,3.313)--(4.762,3.315)--(4.766,3.318)--(4.770,3.320)--(4.774,3.323)--(4.778,3.325)%
  --(4.782,3.328)--(4.786,3.330)--(4.790,3.333)--(4.794,3.335)--(4.798,3.338)--(4.802,3.340)%
  --(4.806,3.343)--(4.811,3.345)--(4.815,3.348)--(4.819,3.350)--(4.823,3.353)--(4.827,3.355)%
  --(4.831,3.358)--(4.835,3.360)--(4.839,3.363)--(4.843,3.365)--(4.847,3.368)--(4.851,3.370)%
  --(4.855,3.373)--(4.859,3.375)--(4.863,3.378)--(4.867,3.380)--(4.871,3.382)--(4.876,3.385)%
  --(4.880,3.387)--(4.884,3.390)--(4.888,3.392)--(4.892,3.395)--(4.896,3.397)--(4.900,3.400)%
  --(4.904,3.402)--(4.908,3.405)--(4.912,3.407)--(4.916,3.410)--(4.920,3.412)--(4.924,3.415)%
  --(4.928,3.417)--(4.932,3.420)--(4.937,3.422)--(4.941,3.425)--(4.945,3.427)--(4.949,3.430)%
  --(4.953,3.432)--(4.957,3.435)--(4.961,3.437)--(4.965,3.440)--(4.969,3.442)--(4.973,3.445)%
  --(4.977,3.447)--(4.981,3.450)--(4.985,3.452)--(4.989,3.455)--(4.993,3.457)--(4.997,3.460)%
  --(5.002,3.462)--(5.006,3.465)--(5.010,3.467)--(5.014,3.470)--(5.018,3.472)--(5.022,3.475)%
  --(5.026,3.477)--(5.030,3.480)--(5.034,3.482)--(5.038,3.484)--(5.042,3.487)--(5.046,3.489)%
  --(5.050,3.492)--(5.054,3.494)--(5.058,3.497)--(5.062,3.499)--(5.067,3.502)--(5.071,3.504)%
  --(5.075,3.507)--(5.079,3.509)--(5.083,3.512)--(5.087,3.514)--(5.091,3.517)--(5.095,3.519)%
  --(5.099,3.522)--(5.103,3.524)--(5.107,3.527)--(5.111,3.529)--(5.115,3.532)--(5.119,3.534)%
  --(5.123,3.537)--(5.127,3.539)--(5.132,3.542)--(5.136,3.544)--(5.140,3.547)--(5.144,3.549)%
  --(5.148,3.552)--(5.152,3.554)--(5.156,3.557)--(5.160,3.559)--(5.164,3.562)--(5.168,3.564)%
  --(5.172,3.567)--(5.176,3.569)--(5.180,3.572)--(5.184,3.574)--(5.188,3.577)--(5.193,3.579)%
  --(5.197,3.581)--(5.201,3.584)--(5.205,3.586)--(5.209,3.589)--(5.213,3.591)--(5.217,3.594)%
  --(5.221,3.596)--(5.225,3.599)--(5.229,3.601)--(5.233,3.604)--(5.237,3.606)--(5.241,3.609)%
  --(5.245,3.611)--(5.249,3.614)--(5.253,3.616)--(5.258,3.619)--(5.262,3.621)--(5.266,3.624)%
  --(5.270,3.626)--(5.274,3.629)--(5.278,3.631)--(5.282,3.634)--(5.286,3.636)--(5.290,3.639)%
  --(5.294,3.641)--(5.298,3.644)--(5.302,3.646)--(5.306,3.649)--(5.310,3.651)--(5.314,3.654)%
  --(5.318,3.656)--(5.323,3.659)--(5.327,3.661)--(5.331,3.664)--(5.335,3.666)--(5.339,3.669)%
  --(5.343,3.671)--(5.347,3.674)--(5.351,3.676)--(5.355,3.678)--(5.359,3.681)--(5.363,3.683)%
  --(5.367,3.686)--(5.371,3.688)--(5.375,3.691)--(5.379,3.693)--(5.384,3.696)--(5.388,3.698)%
  --(5.392,3.701)--(5.396,3.703)--(5.400,3.706)--(5.404,3.708)--(5.408,3.711)--(5.412,3.713)%
  --(5.416,3.716)--(5.420,3.718)--(5.424,3.721)--(5.428,3.723)--(5.432,3.726)--(5.436,3.728)%
  --(5.440,3.731)--(5.444,3.733)--(5.449,3.736)--(5.453,3.738)--(5.457,3.741)--(5.461,3.743)%
  --(5.465,3.746)--(5.469,3.748)--(5.473,3.751)--(5.477,3.753)--(5.481,3.756)--(5.485,3.758)%
  --(5.489,3.761)--(5.493,3.763)--(5.497,3.766)--(5.501,3.768)--(5.505,3.771)--(5.509,3.773)%
  --(5.514,3.776)--(5.518,3.778)--(5.522,3.780)--(5.526,3.783)--(5.530,3.785)--(5.534,3.788)%
  --(5.538,3.790)--(5.542,3.793)--(5.546,3.795)--(5.550,3.798)--(5.554,3.800)--(5.558,3.803)%
  --(5.562,3.805)--(5.566,3.808)--(5.570,3.810)--(5.574,3.813)--(5.579,3.815)--(5.583,3.818)%
  --(5.587,3.820)--(5.591,3.823)--(5.595,3.825)--(5.599,3.828)--(5.603,3.830)--(5.607,3.833)%
  --(5.611,3.835)--(5.615,3.838)--(5.619,3.840)--(5.623,3.843)--(5.627,3.845)--(5.631,3.848)%
  --(5.635,3.850)--(5.640,3.853)--(5.644,3.855)--(5.648,3.858)--(5.652,3.860)--(5.656,3.863)%
  --(5.660,3.865)--(5.664,3.868)--(5.668,3.870)--(5.672,3.873)--(5.676,3.875)--(5.680,3.877)%
  --(5.684,3.880)--(5.688,3.882)--(5.692,3.885)--(5.696,3.887)--(5.700,3.890)--(5.705,3.892)%
  --(5.709,3.895)--(5.713,3.897)--(5.717,3.900)--(5.721,3.902)--(5.725,3.905)--(5.729,3.907)%
  --(5.733,3.910)--(5.737,3.912)--(5.741,3.915)--(5.745,3.917)--(5.749,3.920)--(5.753,3.922)%
  --(5.757,3.925)--(5.761,3.927)--(5.765,3.930)--(5.770,3.932)--(5.774,3.935)--(5.778,3.937)%
  --(5.782,3.940)--(5.786,3.942)--(5.790,3.945)--(5.794,3.947)--(5.798,3.950)--(5.802,3.952)%
  --(5.806,3.955)--(5.810,3.957)--(5.814,3.960)--(5.818,3.962)--(5.822,3.965)--(5.826,3.967)%
  --(5.830,3.970)--(5.835,3.972)--(5.839,3.975)--(5.843,3.977)--(5.847,3.979)--(5.851,3.982)%
  --(5.855,3.984)--(5.859,3.987)--(5.863,3.989)--(5.867,3.992)--(5.871,3.994)--(5.875,3.997)%
  --(5.879,3.999)--(5.883,4.002)--(5.887,4.004)--(5.891,4.007)--(5.896,4.009)--(5.900,4.012)%
  --(5.904,4.014)--(5.908,4.017)--(5.912,4.019)--(5.916,4.022)--(5.920,4.024)--(5.924,4.027)%
  --(5.928,4.029)--(5.932,4.032)--(5.936,4.034)--(5.940,4.037)--(5.944,4.039)--(5.948,4.042)%
  --(5.952,4.044)--(5.956,4.047)--(5.961,4.049)--(5.965,4.052)--(5.969,4.054)--(5.973,4.057)%
  --(5.977,4.059)--(5.981,4.062)--(5.985,4.064)--(5.989,4.067)--(5.993,4.069)--(5.997,4.072)%
  --(6.001,4.074)--(6.005,4.076)--(6.009,4.079)--(6.013,4.081)--(6.017,4.084)--(6.021,4.086)%
  --(6.026,4.089)--(6.030,4.091)--(6.034,4.094)--(6.038,4.096)--(6.042,4.099)--(6.046,4.101)%
  --(6.050,4.104)--(6.054,4.106)--(6.058,4.109)--(6.062,4.111)--(6.066,4.114)--(6.070,4.116)%
  --(6.074,4.119)--(6.078,4.121)--(6.082,4.124)--(6.086,4.126)--(6.091,4.129)--(6.095,4.131)%
  --(6.099,4.134)--(6.103,4.136)--(6.107,4.139)--(6.111,4.141)--(6.115,4.144)--(6.119,4.146)%
  --(6.123,4.149)--(6.127,4.151)--(6.131,4.154)--(6.135,4.156)--(6.139,4.159)--(6.143,4.161)%
  --(6.147,4.164)--(6.152,4.166)--(6.156,4.169)--(6.160,4.171)--(6.164,4.173)--(6.168,4.176)%
  --(6.172,4.178)--(6.176,4.181)--(6.180,4.183)--(6.184,4.186)--(6.188,4.188)--(6.192,4.191)%
  --(6.196,4.193)--(6.200,4.196)--(6.204,4.198)--(6.208,4.201)--(6.212,4.203)--(6.217,4.206)%
  --(6.221,4.208)--(6.225,4.211)--(6.229,4.213)--(6.233,4.216)--(6.237,4.218)--(6.241,4.221)%
  --(6.245,4.223)--(6.249,4.226)--(6.253,4.228)--(6.257,4.231)--(6.261,4.233)--(6.265,4.236)%
  --(6.269,4.238)--(6.273,4.241)--(6.277,4.243)--(6.282,4.246)--(6.286,4.248)--(6.290,4.251)%
  --(6.294,4.253)--(6.298,4.256)--(6.302,4.258)--(6.306,4.261)--(6.310,4.263)--(6.314,4.266)%
  --(6.318,4.268)--(6.322,4.271)--(6.326,4.273)--(6.330,4.275)--(6.334,4.278)--(6.338,4.280)%
  --(6.342,4.283)--(6.347,4.285)--(6.351,4.288)--(6.355,4.290)--(6.359,4.293)--(6.363,4.295)%
  --(6.367,4.298)--(6.371,4.300)--(6.375,4.303)--(6.379,4.305)--(6.383,4.308)--(6.387,4.310)%
  --(6.391,4.313)--(6.395,4.315)--(6.399,4.318)--(6.403,4.320)--(6.408,4.323)--(6.412,4.325)%
  --(6.416,4.328)--(6.420,4.330)--(6.424,4.333)--(6.428,4.335)--(6.432,4.338)--(6.436,4.340)%
  --(6.440,4.343)--(6.444,4.345)--(6.448,4.348)--(6.452,4.350)--(6.456,4.353)--(6.460,4.355)%
  --(6.464,4.358)--(6.468,4.360)--(6.473,4.363)--(6.477,4.365)--(6.481,4.368)--(6.485,4.370)%
  --(6.489,4.372)--(6.493,4.375)--(6.497,4.377)--(6.501,4.380)--(6.505,4.382)--(6.509,4.385)%
  --(6.513,4.387)--(6.517,4.390)--(6.521,4.392)--(6.525,4.395)--(6.529,4.397)--(6.533,4.400)%
  --(6.538,4.402)--(6.542,4.405)--(6.546,4.407)--(6.550,4.410)--(6.554,4.412)--(6.558,4.415)%
  --(6.562,4.417)--(6.566,4.420)--(6.570,4.422)--(6.574,4.425)--(6.578,4.427)--(6.582,4.430)%
  --(6.586,4.432)--(6.590,4.435)--(6.594,4.437)--(6.598,4.440)--(6.603,4.442)--(6.607,4.445)%
  --(6.611,4.447)--(6.615,4.450)--(6.619,4.452)--(6.623,4.455)--(6.627,4.457)--(6.631,4.460)%
  --(6.635,4.462)--(6.639,4.465)--(6.643,4.467)--(6.647,4.470)--(6.651,4.472)--(6.655,4.474)%
  --(6.659,4.477)--(6.664,4.479)--(6.668,4.482)--(6.672,4.484)--(6.676,4.487)--(6.680,4.489)%
  --(6.684,4.492)--(6.688,4.494)--(6.692,4.497)--(6.696,4.499)--(6.700,4.502)--(6.704,4.504)%
  --(6.708,4.507)--(6.712,4.509)--(6.716,4.512)--(6.720,4.514)--(6.724,4.517)--(6.729,4.519)%
  --(6.733,4.522)--(6.737,4.524)--(6.741,4.527)--(6.745,4.529)--(6.749,4.532)--(6.753,4.534)%
  --(6.757,4.537)--(6.761,4.539)--(6.765,4.542)--(6.769,4.544)--(6.773,4.547)--(6.777,4.549)%
  --(6.781,4.552)--(6.785,4.554)--(6.789,4.557)--(6.794,4.559)--(6.798,4.562)--(6.802,4.564)%
  --(6.806,4.567)--(6.810,4.569)--(6.814,4.571)--(6.818,4.574)--(6.822,4.576)--(6.826,4.579)%
  --(6.830,4.581)--(6.834,4.584)--(6.838,4.586)--(6.842,4.589)--(6.846,4.591)--(6.850,4.594)%
  --(6.854,4.596)--(6.859,4.599)--(6.863,4.601)--(6.867,4.604)--(6.871,4.606)--(6.875,4.609)%
  --(6.879,4.611)--(6.883,4.614)--(6.887,4.616)--(6.891,4.619)--(6.895,4.621)--(6.899,4.624)%
  --(6.903,4.626)--(6.907,4.629)--(6.911,4.631)--(6.915,4.634)--(6.920,4.636)--(6.924,4.639)%
  --(6.928,4.641)--(6.932,4.644)--(6.936,4.646)--(6.940,4.649)--(6.944,4.651)--(6.948,4.654)%
  --(6.952,4.656)--(6.956,4.659)--(6.960,4.661)--(6.964,4.664)--(6.968,4.666)--(6.972,4.668)%
  --(6.976,4.671)--(6.980,4.673)--(6.985,4.676)--(6.989,4.678)--(6.993,4.681)--(6.997,4.683)%
  --(7.001,4.686)--(7.005,4.688)--(7.009,4.691)--(7.013,4.693)--(7.017,4.696)--(7.021,4.698)%
  --(7.025,4.701)--(7.029,4.703)--(7.033,4.706)--(7.037,4.708)--(7.041,4.711)--(7.045,4.713)%
  --(7.050,4.716)--(7.054,4.718)--(7.058,4.721)--(7.062,4.723)--(7.066,4.726)--(7.070,4.728)%
  --(7.074,4.731)--(7.078,4.733)--(7.082,4.736)--(7.086,4.738)--(7.090,4.741)--(7.094,4.743)%
  --(7.098,4.746)--(7.102,4.748)--(7.106,4.751)--(7.110,4.753)--(7.115,4.756)--(7.119,4.758)%
  --(7.123,4.761)--(7.127,4.763)--(7.131,4.766)--(7.135,4.768)--(7.139,4.770)--(7.143,4.773)%
  --(7.147,4.775)--(7.151,4.778)--(7.155,4.780)--(7.159,4.783)--(7.163,4.785)--(7.167,4.788)%
  --(7.171,4.790)--(7.176,4.793)--(7.180,4.795)--(7.184,4.798)--(7.188,4.800)--(7.192,4.803)%
  --(7.196,4.805)--(7.200,4.808)--(7.204,4.810)--(7.208,4.813)--(7.212,4.815)--(7.216,4.818)%
  --(7.220,4.820)--(7.224,4.823)--(7.228,4.825)--(7.232,4.828)--(7.236,4.830)--(7.241,4.833)%
  --(7.245,4.835)--(7.249,4.838)--(7.253,4.840)--(7.257,4.843)--(7.261,4.845)--(7.265,4.848)%
  --(7.269,4.850)--(7.273,4.853)--(7.277,4.855)--(7.281,4.858)--(7.285,4.860)--(7.289,4.863)%
  --(7.293,4.865)--(7.297,4.867)--(7.301,4.870)--(7.306,4.872)--(7.310,4.875)--(7.314,4.877)%
  --(7.318,4.880)--(7.322,4.882)--(7.326,4.885)--(7.330,4.887)--(7.334,4.890)--(7.338,4.892)%
  --(7.342,4.895)--(7.346,4.897)--(7.350,4.900)--(7.354,4.902)--(7.358,4.905)--(7.362,4.907)%
  --(7.366,4.910)--(7.371,4.912)--(7.375,4.915)--(7.379,4.917)--(7.383,4.920)--(7.387,4.922)%
  --(7.391,4.925)--(7.395,4.927)--(7.399,4.930)--(7.403,4.932)--(7.407,4.935)--(7.411,4.937)%
  --(7.415,4.940)--(7.419,4.942)--(7.423,4.945)--(7.427,4.947)--(7.432,4.950)--(7.436,4.952)%
  --(7.440,4.955)--(7.444,4.957)--(7.448,4.960)--(7.452,4.962)--(7.456,4.965)--(7.460,4.967)%
  --(7.464,4.969)--(7.468,4.972)--(7.472,4.974)--(7.476,4.977)--(7.480,4.979)--(7.484,4.982)%
  --(7.488,4.984)--(7.492,4.987)--(7.497,4.989)--(7.501,4.992)--(7.505,4.994)--(7.509,4.997)%
  --(7.513,4.999)--(7.517,5.002)--(7.521,5.004)--(7.525,5.007)--(7.529,5.009)--(7.533,5.012)%
  --(7.537,5.014)--(7.541,5.017)--(7.545,5.019)--(7.549,5.022)--(7.553,5.024)--(7.557,5.027)%
  --(7.562,5.029)--(7.566,5.032)--(7.570,5.034)--(7.574,5.037)--(7.578,5.039)--(7.582,5.042)%
  --(7.586,5.044)--(7.590,5.047)--(7.594,5.049)--(7.598,5.052)--(7.602,5.054)--(7.606,5.057)%
  --(7.610,5.059)--(7.614,5.062)--(7.618,5.064)--(7.622,5.066)--(7.627,5.069)--(7.631,5.071)%
  --(7.635,5.074)--(7.639,5.076)--(7.643,5.079)--(7.647,5.081)--(7.651,5.084)--(7.655,5.086)%
  --(7.659,5.089)--(7.663,5.091)--(7.667,5.094)--(7.671,5.096)--(7.675,5.099)--(7.679,5.101)%
  --(7.683,5.104)--(7.688,5.106)--(7.692,5.109)--(7.696,5.111)--(7.700,5.114)--(7.704,5.116)%
  --(7.708,5.119)--(7.712,5.121)--(7.716,5.124)--(7.720,5.126)--(7.724,5.129)--(7.728,5.131)%
  --(7.732,5.134)--(7.736,5.136)--(7.740,5.139)--(7.744,5.141)--(7.748,5.144)--(7.753,5.146)%
  --(7.757,5.149)--(7.761,5.151)--(7.765,5.154)--(7.769,5.156)--(7.773,5.159)--(7.777,5.161)%
  --(7.781,5.163)--(7.785,5.166)--(7.789,5.168)--(7.793,5.171)--(7.797,5.173)--(7.801,5.176)%
  --(7.805,5.178)--(7.809,5.181)--(7.813,5.183)--(7.818,5.186)--(7.822,5.188)--(7.826,5.191)%
  --(7.830,5.193)--(7.834,5.196)--(7.838,5.198)--(7.842,5.201)--(7.846,5.203)--(7.850,5.206)%
  --(7.854,5.208)--(7.858,5.211)--(7.862,5.213)--(7.866,5.216)--(7.870,5.218)--(7.874,5.221)%
  --(7.878,5.223)--(7.883,5.226)--(7.887,5.228)--(7.891,5.231)--(7.895,5.233)--(7.899,5.236)%
  --(7.903,5.238)--(7.907,5.241)--(7.911,5.243)--(7.915,5.246)--(7.919,5.248)--(7.923,5.251)%
  --(7.927,5.253)--(7.931,5.256)--(7.935,5.258)--(7.939,5.261)--(7.944,5.263)--(7.948,5.265)%
  --(7.952,5.268)--(7.956,5.270)--(7.960,5.273)--(7.964,5.275)--(7.968,5.278)--(7.972,5.280)%
  --(7.976,5.283)--(7.980,5.285)--(7.984,5.288)--(7.988,5.290)--(7.992,5.293)--(7.996,5.295)%
  --(8.000,5.298)--(8.004,5.300)--(8.009,5.303)--(8.013,5.305)--(8.017,5.308)--(8.021,5.310)%
  --(8.025,5.313)--(8.029,5.315)--(8.033,5.318)--(8.037,5.320)--(8.041,5.323)--(8.045,5.325)%
  --(8.049,5.328)--(8.053,5.330)--(8.057,5.333)--(8.061,5.335)--(8.065,5.338)--(8.069,5.340)%
  --(8.074,5.343)--(8.078,5.345)--(8.082,5.348)--(8.086,5.350)--(8.090,5.353)--(8.094,5.355)%
  --(8.098,5.358)--(8.102,5.360)--(8.106,5.362)--(8.110,5.365)--(8.114,5.367)--(8.118,5.370)%
  --(8.122,5.372)--(8.126,5.375)--(8.130,5.377)--(8.134,5.380)--(8.139,5.382)--(8.143,5.385)%
  --(8.147,5.387)--(8.151,5.390)--(8.155,5.392)--(8.159,5.395)--(8.163,5.397)--(8.167,5.400)%
  --(8.171,5.402)--(8.175,5.405)--(8.179,5.407)--(8.183,5.410)--(8.187,5.412)--(8.191,5.415)%
  --(8.195,5.417)--(8.200,5.420)--(8.204,5.422)--(8.208,5.425)--(8.212,5.427)--(8.216,5.430)%
  --(8.220,5.432)--(8.224,5.435)--(8.228,5.437)--(8.232,5.440)--(8.236,5.442)--(8.240,5.445)%
  --(8.244,5.447)--(8.248,5.450)--(8.252,5.452)--(8.256,5.455)--(8.260,5.457)--(8.265,5.460)%
  --(8.269,5.462)--(8.273,5.464)--(8.277,5.467)--(8.281,5.469)--(8.285,5.472)--(8.289,5.474)%
  --(8.293,5.477)--(8.297,5.479)--(8.301,5.482)--(8.305,5.484)--(8.309,5.487)--(8.313,5.489)%
  --(8.317,5.492)--(8.321,5.494)--(8.325,5.497)--(8.330,5.499)--(8.334,5.502)--(8.338,5.504)%
  --(8.342,5.507)--(8.346,5.509)--(8.350,5.512)--(8.354,5.514)--(8.358,5.517)--(8.362,5.519)%
  --(8.366,5.522)--(8.370,5.524)--(8.374,5.527)--(8.378,5.529)--(8.382,5.532)--(8.386,5.534)%
  --(8.390,5.537)--(8.395,5.539)--(8.399,5.542)--(8.403,5.544)--(8.407,5.547)--(8.411,5.549)%
  --(8.415,5.552)--(8.419,5.554)--(8.423,5.557)--(8.427,5.559)--(8.431,5.561)--(8.435,5.564)%
  --(8.439,5.566)--(8.443,5.569)--(8.447,5.571)--(8.451,5.574)--(8.456,5.576)--(8.460,5.579)%
  --(8.464,5.581)--(8.468,5.584)--(8.472,5.586)--(8.476,5.589)--(8.480,5.591)--(8.484,5.594)%
  --(8.488,5.596)--(8.492,5.599)--(8.496,5.601)--(8.500,5.604)--(8.504,5.606)--(8.508,5.609)%
  --(8.512,5.611)--(8.516,5.614)--(8.521,5.616)--(8.525,5.619)--(8.529,5.621)--(8.533,5.624)%
  --(8.537,5.626)--(8.541,5.629)--(8.545,5.631)--(8.549,5.634)--(8.553,5.636)--(8.557,5.639)%
  --(8.561,5.641)--(8.565,5.644)--(8.569,5.646)--(8.573,5.649)--(8.577,5.651)--(8.581,5.654)%
  --(8.586,5.656)--(8.590,5.658)--(8.594,5.661)--(8.598,5.663)--(8.602,5.666)--(8.606,5.668)%
  --(8.610,5.671)--(8.614,5.673)--(8.618,5.676)--(8.622,5.678)--(8.626,5.681)--(8.630,5.683)%
  --(8.634,5.686)--(8.638,5.688)--(8.642,5.691)--(8.646,5.693)--(8.651,5.696)--(8.655,5.698)%
  --(8.659,5.701)--(8.663,5.703)--(8.667,5.706)--(8.671,5.708)--(8.675,5.711)--(8.679,5.713)%
  --(8.683,5.716)--(8.687,5.718)--(8.691,5.721)--(8.695,5.723)--(8.699,5.726)--(8.703,5.728)%
  --(8.707,5.731)--(8.712,5.733)--(8.716,5.736)--(8.720,5.738)--(8.724,5.741)--(8.728,5.743)%
  --(8.732,5.746)--(8.736,5.748)--(8.740,5.751)--(8.744,5.753)--(8.748,5.756)--(8.752,5.758)%
  --(8.756,5.760)--(8.760,5.763)--(8.764,5.765)--(8.768,5.768)--(8.772,5.770)--(8.777,5.773)%
  --(8.781,5.775)--(8.785,5.778)--(8.789,5.780)--(8.793,5.783)--(8.797,5.785)--(8.801,5.788)%
  --(8.805,5.790)--(8.809,5.793)--(8.813,5.795)--(8.817,5.798)--(8.821,5.800)--(8.825,5.803)%
  --(8.829,5.805)--(8.833,5.808)--(8.837,5.810)--(8.842,5.813)--(8.846,5.815)--(8.850,5.818)%
  --(8.854,5.820)--(8.858,5.823)--(8.862,5.825)--(8.866,5.828)--(8.870,5.830)--(8.874,5.833)%
  --(8.878,5.835)--(8.882,5.838)--(8.886,5.840)--(8.890,5.843)--(8.894,5.845)--(8.898,5.848)%
  --(8.902,5.850)--(8.907,5.853)--(8.911,5.855)--(8.915,5.857)--(8.919,5.860);
\gpcolor{color=gp lt color border}
\gpsetlinewidth{1.00}
\draw[gp path] (1.320,6.631)--(1.320,0.985)--(9.447,0.985)--(9.447,6.631)--cycle;
%% coordinates of the plot area
\gpdefrectangularnode{gp plot 1}{\pgfpoint{1.320cm}{0.985cm}}{\pgfpoint{9.447cm}{6.631cm}}
\end{tikzpicture}
%% gnuplot variables

\caption{Conjunto de dados 1. A reta da regressão linear é dada por $y(x)=\np{2.984480401}x + \np{12.14046264}$, $r^2 = \np{0.998901256}$.}
\label{RetasConjuntosDados1}
\end{figure}

\begin{figure}[!h]\forcerectofloat
\centering
\begin{tikzpicture}[gnuplot]
%% generated with GNUPLOT 5.0p0 (Lua 5.3; terminal rev. 99, script rev. 100)
%% 2015-05-18T22:58:50 BRT
\path (0.000,0.000) rectangle (14.000,9.000);
\gpcolor{color=gp lt color border}
\gpsetlinetype{gp lt border}
\gpsetdashtype{gp dt solid}
\gpsetlinewidth{1.00}
\draw[gp path] (1.320,2.028)--(1.500,2.028);
\draw[gp path] (13.447,2.028)--(13.267,2.028);
\node[gp node right] at (1.136,2.028) {$0$};
\draw[gp path] (1.320,3.418)--(1.500,3.418);
\draw[gp path] (13.447,3.418)--(13.267,3.418);
\node[gp node right] at (1.136,3.418) {$200$};
\draw[gp path] (1.320,4.808)--(1.500,4.808);
\draw[gp path] (13.447,4.808)--(13.267,4.808);
\node[gp node right] at (1.136,4.808) {$400$};
\draw[gp path] (1.320,6.198)--(1.500,6.198);
\draw[gp path] (13.447,6.198)--(13.267,6.198);
\node[gp node right] at (1.136,6.198) {$600$};
\draw[gp path] (1.320,7.588)--(1.500,7.588);
\draw[gp path] (13.447,7.588)--(13.267,7.588);
\node[gp node right] at (1.136,7.588) {$800$};
\draw[gp path] (1.711,0.985)--(1.711,1.165);
\draw[gp path] (1.711,8.631)--(1.711,8.451);
\node[gp node center] at (1.711,0.677) {$0$};
\draw[gp path] (3.667,0.985)--(3.667,1.165);
\draw[gp path] (3.667,8.631)--(3.667,8.451);
\node[gp node center] at (3.667,0.677) {$5$};
\draw[gp path] (5.623,0.985)--(5.623,1.165);
\draw[gp path] (5.623,8.631)--(5.623,8.451);
\node[gp node center] at (5.623,0.677) {$10$};
\draw[gp path] (7.579,0.985)--(7.579,1.165);
\draw[gp path] (7.579,8.631)--(7.579,8.451);
\node[gp node center] at (7.579,0.677) {$15$};
\draw[gp path] (9.535,0.985)--(9.535,1.165);
\draw[gp path] (9.535,8.631)--(9.535,8.451);
\node[gp node center] at (9.535,0.677) {$20$};
\draw[gp path] (11.491,0.985)--(11.491,1.165);
\draw[gp path] (11.491,8.631)--(11.491,8.451);
\node[gp node center] at (11.491,0.677) {$25$};
\draw[gp path] (13.447,0.985)--(13.447,1.165);
\draw[gp path] (13.447,8.631)--(13.447,8.451);
\node[gp node center] at (13.447,0.677) {$30$};
\draw[gp path] (1.320,8.631)--(1.320,0.985)--(13.447,0.985)--(13.447,8.631)--cycle;
\node[gp node center,rotate=-270] at (0.246,4.808) {$y_2$};
\node[gp node center] at (7.383,0.215) {$x$};
\node[gp node left] at (2.788,8.297) {Dados experimentais};
\gpcolor{rgb color={0.000,0.000,0.000}}
\gpsetlinewidth{2.00}
\gpsetpointsize{4.00}
\gppoint{gp mark 7}{(1.990,2.032)}
\gppoint{gp mark 7}{(2.765,2.089)}
\gppoint{gp mark 7}{(3.428,2.174)}
\gppoint{gp mark 7}{(3.652,2.226)}
\gppoint{gp mark 7}{(4.154,2.306)}
\gppoint{gp mark 7}{(4.558,2.415)}
\gppoint{gp mark 7}{(4.676,2.479)}
\gppoint{gp mark 7}{(4.731,2.475)}
\gppoint{gp mark 7}{(4.806,2.489)}
\gppoint{gp mark 7}{(4.950,2.546)}
\gppoint{gp mark 7}{(5.245,2.636)}
\gppoint{gp mark 7}{(5.405,2.686)}
\gppoint{gp mark 7}{(5.743,2.813)}
\gppoint{gp mark 7}{(5.847,2.806)}
\gppoint{gp mark 7}{(6.084,2.928)}
\gppoint{gp mark 7}{(7.710,3.667)}
\gppoint{gp mark 7}{(8.370,4.075)}
\gppoint{gp mark 7}{(9.115,4.614)}
\gppoint{gp mark 7}{(9.773,4.993)}
\gppoint{gp mark 7}{(9.878,5.200)}
\gppoint{gp mark 7}{(9.963,5.174)}
\gppoint{gp mark 7}{(10.406,5.557)}
\gppoint{gp mark 7}{(10.477,5.568)}
\gppoint{gp mark 7}{(10.501,5.652)}
\gppoint{gp mark 7}{(10.739,5.880)}
\gppoint{gp mark 7}{(12.047,6.899)}
\gppoint{gp mark 7}{(12.220,7.120)}
\gppoint{gp mark 7}{(12.414,7.402)}
\gppoint{gp mark 7}{(2.146,8.297)}
\gpcolor{color=gp lt color border}
\node[gp node left] at (2.788,7.989) {$y(x)=\np{29.9779011494} x - \np{152.917852680}$, $r^2=\np{0.9565748438}$};
\gpcolor{rgb color={0.000,0.000,0.000}}
\draw[gp path] (1.688,7.989)--(2.604,7.989);
\draw[gp path] (1.908,1.070)--(1.914,1.073)--(1.920,1.076)--(1.926,1.079)--(1.932,1.083)%
  --(1.938,1.086)--(1.945,1.089)--(1.951,1.092)--(1.957,1.095)--(1.963,1.099)--(1.969,1.102)%
  --(1.975,1.105)--(1.981,1.108)--(1.987,1.112)--(1.993,1.115)--(1.999,1.118)--(2.005,1.121)%
  --(2.011,1.125)--(2.017,1.128)--(2.023,1.131)--(2.029,1.134)--(2.035,1.137)--(2.042,1.141)%
  --(2.048,1.144)--(2.054,1.147)--(2.060,1.150)--(2.066,1.154)--(2.072,1.157)--(2.078,1.160)%
  --(2.084,1.163)--(2.090,1.167)--(2.096,1.170)--(2.102,1.173)--(2.108,1.176)--(2.114,1.179)%
  --(2.120,1.183)--(2.126,1.186)--(2.133,1.189)--(2.139,1.192)--(2.145,1.196)--(2.151,1.199)%
  --(2.157,1.202)--(2.163,1.205)--(2.169,1.209)--(2.175,1.212)--(2.181,1.215)--(2.187,1.218)%
  --(2.193,1.221)--(2.199,1.225)--(2.205,1.228)--(2.211,1.231)--(2.217,1.234)--(2.223,1.238)%
  --(2.230,1.241)--(2.236,1.244)--(2.242,1.247)--(2.248,1.251)--(2.254,1.254)--(2.260,1.257)%
  --(2.266,1.260)--(2.272,1.263)--(2.278,1.267)--(2.284,1.270)--(2.290,1.273)--(2.296,1.276)%
  --(2.302,1.280)--(2.308,1.283)--(2.314,1.286)--(2.320,1.289)--(2.327,1.292)--(2.333,1.296)%
  --(2.339,1.299)--(2.345,1.302)--(2.351,1.305)--(2.357,1.309)--(2.363,1.312)--(2.369,1.315)%
  --(2.375,1.318)--(2.381,1.322)--(2.387,1.325)--(2.393,1.328)--(2.399,1.331)--(2.405,1.334)%
  --(2.411,1.338)--(2.417,1.341)--(2.424,1.344)--(2.430,1.347)--(2.436,1.351)--(2.442,1.354)%
  --(2.448,1.357)--(2.454,1.360)--(2.460,1.364)--(2.466,1.367)--(2.472,1.370)--(2.478,1.373)%
  --(2.484,1.376)--(2.490,1.380)--(2.496,1.383)--(2.502,1.386)--(2.508,1.389)--(2.515,1.393)%
  --(2.521,1.396)--(2.527,1.399)--(2.533,1.402)--(2.539,1.406)--(2.545,1.409)--(2.551,1.412)%
  --(2.557,1.415)--(2.563,1.418)--(2.569,1.422)--(2.575,1.425)--(2.581,1.428)--(2.587,1.431)%
  --(2.593,1.435)--(2.599,1.438)--(2.605,1.441)--(2.612,1.444)--(2.618,1.448)--(2.624,1.451)%
  --(2.630,1.454)--(2.636,1.457)--(2.642,1.460)--(2.648,1.464)--(2.654,1.467)--(2.660,1.470)%
  --(2.666,1.473)--(2.672,1.477)--(2.678,1.480)--(2.684,1.483)--(2.690,1.486)--(2.696,1.490)%
  --(2.702,1.493)--(2.709,1.496)--(2.715,1.499)--(2.721,1.502)--(2.727,1.506)--(2.733,1.509)%
  --(2.739,1.512)--(2.745,1.515)--(2.751,1.519)--(2.757,1.522)--(2.763,1.525)--(2.769,1.528)%
  --(2.775,1.531)--(2.781,1.535)--(2.787,1.538)--(2.793,1.541)--(2.799,1.544)--(2.806,1.548)%
  --(2.812,1.551)--(2.818,1.554)--(2.824,1.557)--(2.830,1.561)--(2.836,1.564)--(2.842,1.567)%
  --(2.848,1.570)--(2.854,1.573)--(2.860,1.577)--(2.866,1.580)--(2.872,1.583)--(2.878,1.586)%
  --(2.884,1.590)--(2.890,1.593)--(2.897,1.596)--(2.903,1.599)--(2.909,1.603)--(2.915,1.606)%
  --(2.921,1.609)--(2.927,1.612)--(2.933,1.615)--(2.939,1.619)--(2.945,1.622)--(2.951,1.625)%
  --(2.957,1.628)--(2.963,1.632)--(2.969,1.635)--(2.975,1.638)--(2.981,1.641)--(2.987,1.645)%
  --(2.994,1.648)--(3.000,1.651)--(3.006,1.654)--(3.012,1.657)--(3.018,1.661)--(3.024,1.664)%
  --(3.030,1.667)--(3.036,1.670)--(3.042,1.674)--(3.048,1.677)--(3.054,1.680)--(3.060,1.683)%
  --(3.066,1.687)--(3.072,1.690)--(3.078,1.693)--(3.084,1.696)--(3.091,1.699)--(3.097,1.703)%
  --(3.103,1.706)--(3.109,1.709)--(3.115,1.712)--(3.121,1.716)--(3.127,1.719)--(3.133,1.722)%
  --(3.139,1.725)--(3.145,1.729)--(3.151,1.732)--(3.157,1.735)--(3.163,1.738)--(3.169,1.741)%
  --(3.175,1.745)--(3.181,1.748)--(3.188,1.751)--(3.194,1.754)--(3.200,1.758)--(3.206,1.761)%
  --(3.212,1.764)--(3.218,1.767)--(3.224,1.770)--(3.230,1.774)--(3.236,1.777)--(3.242,1.780)%
  --(3.248,1.783)--(3.254,1.787)--(3.260,1.790)--(3.266,1.793)--(3.272,1.796)--(3.279,1.800)%
  --(3.285,1.803)--(3.291,1.806)--(3.297,1.809)--(3.303,1.812)--(3.309,1.816)--(3.315,1.819)%
  --(3.321,1.822)--(3.327,1.825)--(3.333,1.829)--(3.339,1.832)--(3.345,1.835)--(3.351,1.838)%
  --(3.357,1.842)--(3.363,1.845)--(3.369,1.848)--(3.376,1.851)--(3.382,1.854)--(3.388,1.858)%
  --(3.394,1.861)--(3.400,1.864)--(3.406,1.867)--(3.412,1.871)--(3.418,1.874)--(3.424,1.877)%
  --(3.430,1.880)--(3.436,1.884)--(3.442,1.887)--(3.448,1.890)--(3.454,1.893)--(3.460,1.896)%
  --(3.466,1.900)--(3.473,1.903)--(3.479,1.906)--(3.485,1.909)--(3.491,1.913)--(3.497,1.916)%
  --(3.503,1.919)--(3.509,1.922)--(3.515,1.926)--(3.521,1.929)--(3.527,1.932)--(3.533,1.935)%
  --(3.539,1.938)--(3.545,1.942)--(3.551,1.945)--(3.557,1.948)--(3.563,1.951)--(3.570,1.955)%
  --(3.576,1.958)--(3.582,1.961)--(3.588,1.964)--(3.594,1.968)--(3.600,1.971)--(3.606,1.974)%
  --(3.612,1.977)--(3.618,1.980)--(3.624,1.984)--(3.630,1.987)--(3.636,1.990)--(3.642,1.993)%
  --(3.648,1.997)--(3.654,2.000)--(3.661,2.003)--(3.667,2.006)--(3.673,2.010)--(3.679,2.013)%
  --(3.685,2.016)--(3.691,2.019)--(3.697,2.022)--(3.703,2.026)--(3.709,2.029)--(3.715,2.032)%
  --(3.721,2.035)--(3.727,2.039)--(3.733,2.042)--(3.739,2.045)--(3.745,2.048)--(3.751,2.051)%
  --(3.758,2.055)--(3.764,2.058)--(3.770,2.061)--(3.776,2.064)--(3.782,2.068)--(3.788,2.071)%
  --(3.794,2.074)--(3.800,2.077)--(3.806,2.081)--(3.812,2.084)--(3.818,2.087)--(3.824,2.090)%
  --(3.830,2.093)--(3.836,2.097)--(3.842,2.100)--(3.848,2.103)--(3.855,2.106)--(3.861,2.110)%
  --(3.867,2.113)--(3.873,2.116)--(3.879,2.119)--(3.885,2.123)--(3.891,2.126)--(3.897,2.129)%
  --(3.903,2.132)--(3.909,2.135)--(3.915,2.139)--(3.921,2.142)--(3.927,2.145)--(3.933,2.148)%
  --(3.939,2.152)--(3.945,2.155)--(3.952,2.158)--(3.958,2.161)--(3.964,2.165)--(3.970,2.168)%
  --(3.976,2.171)--(3.982,2.174)--(3.988,2.177)--(3.994,2.181)--(4.000,2.184)--(4.006,2.187)%
  --(4.012,2.190)--(4.018,2.194)--(4.024,2.197)--(4.030,2.200)--(4.036,2.203)--(4.043,2.207)%
  --(4.049,2.210)--(4.055,2.213)--(4.061,2.216)--(4.067,2.219)--(4.073,2.223)--(4.079,2.226)%
  --(4.085,2.229)--(4.091,2.232)--(4.097,2.236)--(4.103,2.239)--(4.109,2.242)--(4.115,2.245)%
  --(4.121,2.249)--(4.127,2.252)--(4.133,2.255)--(4.140,2.258)--(4.146,2.261)--(4.152,2.265)%
  --(4.158,2.268)--(4.164,2.271)--(4.170,2.274)--(4.176,2.278)--(4.182,2.281)--(4.188,2.284)%
  --(4.194,2.287)--(4.200,2.290)--(4.206,2.294)--(4.212,2.297)--(4.218,2.300)--(4.224,2.303)%
  --(4.230,2.307)--(4.237,2.310)--(4.243,2.313)--(4.249,2.316)--(4.255,2.320)--(4.261,2.323)%
  --(4.267,2.326)--(4.273,2.329)--(4.279,2.332)--(4.285,2.336)--(4.291,2.339)--(4.297,2.342)%
  --(4.303,2.345)--(4.309,2.349)--(4.315,2.352)--(4.321,2.355)--(4.327,2.358)--(4.334,2.362)%
  --(4.340,2.365)--(4.346,2.368)--(4.352,2.371)--(4.358,2.374)--(4.364,2.378)--(4.370,2.381)%
  --(4.376,2.384)--(4.382,2.387)--(4.388,2.391)--(4.394,2.394)--(4.400,2.397)--(4.406,2.400)%
  --(4.412,2.404)--(4.418,2.407)--(4.425,2.410)--(4.431,2.413)--(4.437,2.416)--(4.443,2.420)%
  --(4.449,2.423)--(4.455,2.426)--(4.461,2.429)--(4.467,2.433)--(4.473,2.436)--(4.479,2.439)%
  --(4.485,2.442)--(4.491,2.446)--(4.497,2.449)--(4.503,2.452)--(4.509,2.455)--(4.515,2.458)%
  --(4.522,2.462)--(4.528,2.465)--(4.534,2.468)--(4.540,2.471)--(4.546,2.475)--(4.552,2.478)%
  --(4.558,2.481)--(4.564,2.484)--(4.570,2.488)--(4.576,2.491)--(4.582,2.494)--(4.588,2.497)%
  --(4.594,2.500)--(4.600,2.504)--(4.606,2.507)--(4.612,2.510)--(4.619,2.513)--(4.625,2.517)%
  --(4.631,2.520)--(4.637,2.523)--(4.643,2.526)--(4.649,2.530)--(4.655,2.533)--(4.661,2.536)%
  --(4.667,2.539)--(4.673,2.542)--(4.679,2.546)--(4.685,2.549)--(4.691,2.552)--(4.697,2.555)%
  --(4.703,2.559)--(4.709,2.562)--(4.716,2.565)--(4.722,2.568)--(4.728,2.571)--(4.734,2.575)%
  --(4.740,2.578)--(4.746,2.581)--(4.752,2.584)--(4.758,2.588)--(4.764,2.591)--(4.770,2.594)%
  --(4.776,2.597)--(4.782,2.601)--(4.788,2.604)--(4.794,2.607)--(4.800,2.610)--(4.807,2.613)%
  --(4.813,2.617)--(4.819,2.620)--(4.825,2.623)--(4.831,2.626)--(4.837,2.630)--(4.843,2.633)%
  --(4.849,2.636)--(4.855,2.639)--(4.861,2.643)--(4.867,2.646)--(4.873,2.649)--(4.879,2.652)%
  --(4.885,2.655)--(4.891,2.659)--(4.897,2.662)--(4.904,2.665)--(4.910,2.668)--(4.916,2.672)%
  --(4.922,2.675)--(4.928,2.678)--(4.934,2.681)--(4.940,2.685)--(4.946,2.688)--(4.952,2.691)%
  --(4.958,2.694)--(4.964,2.697)--(4.970,2.701)--(4.976,2.704)--(4.982,2.707)--(4.988,2.710)%
  --(4.994,2.714)--(5.001,2.717)--(5.007,2.720)--(5.013,2.723)--(5.019,2.727)--(5.025,2.730)%
  --(5.031,2.733)--(5.037,2.736)--(5.043,2.739)--(5.049,2.743)--(5.055,2.746)--(5.061,2.749)%
  --(5.067,2.752)--(5.073,2.756)--(5.079,2.759)--(5.085,2.762)--(5.091,2.765)--(5.098,2.769)%
  --(5.104,2.772)--(5.110,2.775)--(5.116,2.778)--(5.122,2.781)--(5.128,2.785)--(5.134,2.788)%
  --(5.140,2.791)--(5.146,2.794)--(5.152,2.798)--(5.158,2.801)--(5.164,2.804)--(5.170,2.807)%
  --(5.176,2.810)--(5.182,2.814)--(5.189,2.817)--(5.195,2.820)--(5.201,2.823)--(5.207,2.827)%
  --(5.213,2.830)--(5.219,2.833)--(5.225,2.836)--(5.231,2.840)--(5.237,2.843)--(5.243,2.846)%
  --(5.249,2.849)--(5.255,2.852)--(5.261,2.856)--(5.267,2.859)--(5.273,2.862)--(5.279,2.865)%
  --(5.286,2.869)--(5.292,2.872)--(5.298,2.875)--(5.304,2.878)--(5.310,2.882)--(5.316,2.885)%
  --(5.322,2.888)--(5.328,2.891)--(5.334,2.894)--(5.340,2.898)--(5.346,2.901)--(5.352,2.904)%
  --(5.358,2.907)--(5.364,2.911)--(5.370,2.914)--(5.376,2.917)--(5.383,2.920)--(5.389,2.924)%
  --(5.395,2.927)--(5.401,2.930)--(5.407,2.933)--(5.413,2.936)--(5.419,2.940)--(5.425,2.943)%
  --(5.431,2.946)--(5.437,2.949)--(5.443,2.953)--(5.449,2.956)--(5.455,2.959)--(5.461,2.962)%
  --(5.467,2.966)--(5.473,2.969)--(5.480,2.972)--(5.486,2.975)--(5.492,2.978)--(5.498,2.982)%
  --(5.504,2.985)--(5.510,2.988)--(5.516,2.991)--(5.522,2.995)--(5.528,2.998)--(5.534,3.001)%
  --(5.540,3.004)--(5.546,3.008)--(5.552,3.011)--(5.558,3.014)--(5.564,3.017)--(5.571,3.020)%
  --(5.577,3.024)--(5.583,3.027)--(5.589,3.030)--(5.595,3.033)--(5.601,3.037)--(5.607,3.040)%
  --(5.613,3.043)--(5.619,3.046)--(5.625,3.049)--(5.631,3.053)--(5.637,3.056)--(5.643,3.059)%
  --(5.649,3.062)--(5.655,3.066)--(5.661,3.069)--(5.668,3.072)--(5.674,3.075)--(5.680,3.079)%
  --(5.686,3.082)--(5.692,3.085)--(5.698,3.088)--(5.704,3.091)--(5.710,3.095)--(5.716,3.098)%
  --(5.722,3.101)--(5.728,3.104)--(5.734,3.108)--(5.740,3.111)--(5.746,3.114)--(5.752,3.117)%
  --(5.758,3.121)--(5.765,3.124)--(5.771,3.127)--(5.777,3.130)--(5.783,3.133)--(5.789,3.137)%
  --(5.795,3.140)--(5.801,3.143)--(5.807,3.146)--(5.813,3.150)--(5.819,3.153)--(5.825,3.156)%
  --(5.831,3.159)--(5.837,3.163)--(5.843,3.166)--(5.849,3.169)--(5.855,3.172)--(5.862,3.175)%
  --(5.868,3.179)--(5.874,3.182)--(5.880,3.185)--(5.886,3.188)--(5.892,3.192)--(5.898,3.195)%
  --(5.904,3.198)--(5.910,3.201)--(5.916,3.205)--(5.922,3.208)--(5.928,3.211)--(5.934,3.214)%
  --(5.940,3.217)--(5.946,3.221)--(5.953,3.224)--(5.959,3.227)--(5.965,3.230)--(5.971,3.234)%
  --(5.977,3.237)--(5.983,3.240)--(5.989,3.243)--(5.995,3.247)--(6.001,3.250)--(6.007,3.253)%
  --(6.013,3.256)--(6.019,3.259)--(6.025,3.263)--(6.031,3.266)--(6.037,3.269)--(6.043,3.272)%
  --(6.050,3.276)--(6.056,3.279)--(6.062,3.282)--(6.068,3.285)--(6.074,3.289)--(6.080,3.292)%
  --(6.086,3.295)--(6.092,3.298)--(6.098,3.301)--(6.104,3.305)--(6.110,3.308)--(6.116,3.311)%
  --(6.122,3.314)--(6.128,3.318)--(6.134,3.321)--(6.140,3.324)--(6.147,3.327)--(6.153,3.330)%
  --(6.159,3.334)--(6.165,3.337)--(6.171,3.340)--(6.177,3.343)--(6.183,3.347)--(6.189,3.350)%
  --(6.195,3.353)--(6.201,3.356)--(6.207,3.360)--(6.213,3.363)--(6.219,3.366)--(6.225,3.369)%
  --(6.231,3.372)--(6.237,3.376)--(6.244,3.379)--(6.250,3.382)--(6.256,3.385)--(6.262,3.389)%
  --(6.268,3.392)--(6.274,3.395)--(6.280,3.398)--(6.286,3.402)--(6.292,3.405)--(6.298,3.408)%
  --(6.304,3.411)--(6.310,3.414)--(6.316,3.418)--(6.322,3.421)--(6.328,3.424)--(6.335,3.427)%
  --(6.341,3.431)--(6.347,3.434)--(6.353,3.437)--(6.359,3.440)--(6.365,3.444)--(6.371,3.447)%
  --(6.377,3.450)--(6.383,3.453)--(6.389,3.456)--(6.395,3.460)--(6.401,3.463)--(6.407,3.466)%
  --(6.413,3.469)--(6.419,3.473)--(6.425,3.476)--(6.432,3.479)--(6.438,3.482)--(6.444,3.486)%
  --(6.450,3.489)--(6.456,3.492)--(6.462,3.495)--(6.468,3.498)--(6.474,3.502)--(6.480,3.505)%
  --(6.486,3.508)--(6.492,3.511)--(6.498,3.515)--(6.504,3.518)--(6.510,3.521)--(6.516,3.524)%
  --(6.522,3.528)--(6.529,3.531)--(6.535,3.534)--(6.541,3.537)--(6.547,3.540)--(6.553,3.544)%
  --(6.559,3.547)--(6.565,3.550)--(6.571,3.553)--(6.577,3.557)--(6.583,3.560)--(6.589,3.563)%
  --(6.595,3.566)--(6.601,3.569)--(6.607,3.573)--(6.613,3.576)--(6.619,3.579)--(6.626,3.582)%
  --(6.632,3.586)--(6.638,3.589)--(6.644,3.592)--(6.650,3.595)--(6.656,3.599)--(6.662,3.602)%
  --(6.668,3.605)--(6.674,3.608)--(6.680,3.611)--(6.686,3.615)--(6.692,3.618)--(6.698,3.621)%
  --(6.704,3.624)--(6.710,3.628)--(6.717,3.631)--(6.723,3.634)--(6.729,3.637)--(6.735,3.641)%
  --(6.741,3.644)--(6.747,3.647)--(6.753,3.650)--(6.759,3.653)--(6.765,3.657)--(6.771,3.660)%
  --(6.777,3.663)--(6.783,3.666)--(6.789,3.670)--(6.795,3.673)--(6.801,3.676)--(6.807,3.679)%
  --(6.814,3.683)--(6.820,3.686)--(6.826,3.689)--(6.832,3.692)--(6.838,3.695)--(6.844,3.699)%
  --(6.850,3.702)--(6.856,3.705)--(6.862,3.708)--(6.868,3.712)--(6.874,3.715)--(6.880,3.718)%
  --(6.886,3.721)--(6.892,3.725)--(6.898,3.728)--(6.904,3.731)--(6.911,3.734)--(6.917,3.737)%
  --(6.923,3.741)--(6.929,3.744)--(6.935,3.747)--(6.941,3.750)--(6.947,3.754)--(6.953,3.757)%
  --(6.959,3.760)--(6.965,3.763)--(6.971,3.767)--(6.977,3.770)--(6.983,3.773)--(6.989,3.776)%
  --(6.995,3.779)--(7.001,3.783)--(7.008,3.786)--(7.014,3.789)--(7.020,3.792)--(7.026,3.796)%
  --(7.032,3.799)--(7.038,3.802)--(7.044,3.805)--(7.050,3.808)--(7.056,3.812)--(7.062,3.815)%
  --(7.068,3.818)--(7.074,3.821)--(7.080,3.825)--(7.086,3.828)--(7.092,3.831)--(7.099,3.834)%
  --(7.105,3.838)--(7.111,3.841)--(7.117,3.844)--(7.123,3.847)--(7.129,3.850)--(7.135,3.854)%
  --(7.141,3.857)--(7.147,3.860)--(7.153,3.863)--(7.159,3.867)--(7.165,3.870)--(7.171,3.873)%
  --(7.177,3.876)--(7.183,3.880)--(7.189,3.883)--(7.196,3.886)--(7.202,3.889)--(7.208,3.892)%
  --(7.214,3.896)--(7.220,3.899)--(7.226,3.902)--(7.232,3.905)--(7.238,3.909)--(7.244,3.912)%
  --(7.250,3.915)--(7.256,3.918)--(7.262,3.922)--(7.268,3.925)--(7.274,3.928)--(7.280,3.931)%
  --(7.286,3.934)--(7.293,3.938)--(7.299,3.941)--(7.305,3.944)--(7.311,3.947)--(7.317,3.951)%
  --(7.323,3.954)--(7.329,3.957)--(7.335,3.960)--(7.341,3.964)--(7.347,3.967)--(7.353,3.970)%
  --(7.359,3.973)--(7.365,3.976)--(7.371,3.980)--(7.377,3.983)--(7.384,3.986)--(7.390,3.989)%
  --(7.396,3.993)--(7.402,3.996)--(7.408,3.999)--(7.414,4.002)--(7.420,4.006)--(7.426,4.009)%
  --(7.432,4.012)--(7.438,4.015)--(7.444,4.018)--(7.450,4.022)--(7.456,4.025)--(7.462,4.028)%
  --(7.468,4.031)--(7.474,4.035)--(7.481,4.038)--(7.487,4.041)--(7.493,4.044)--(7.499,4.048)%
  --(7.505,4.051)--(7.511,4.054)--(7.517,4.057)--(7.523,4.060)--(7.529,4.064)--(7.535,4.067)%
  --(7.541,4.070)--(7.547,4.073)--(7.553,4.077)--(7.559,4.080)--(7.565,4.083)--(7.571,4.086)%
  --(7.578,4.089)--(7.584,4.093)--(7.590,4.096)--(7.596,4.099)--(7.602,4.102)--(7.608,4.106)%
  --(7.614,4.109)--(7.620,4.112)--(7.626,4.115)--(7.632,4.119)--(7.638,4.122)--(7.644,4.125)%
  --(7.650,4.128)--(7.656,4.131)--(7.662,4.135)--(7.668,4.138)--(7.675,4.141)--(7.681,4.144)%
  --(7.687,4.148)--(7.693,4.151)--(7.699,4.154)--(7.705,4.157)--(7.711,4.161)--(7.717,4.164)%
  --(7.723,4.167)--(7.729,4.170)--(7.735,4.173)--(7.741,4.177)--(7.747,4.180)--(7.753,4.183)%
  --(7.759,4.186)--(7.766,4.190)--(7.772,4.193)--(7.778,4.196)--(7.784,4.199)--(7.790,4.203)%
  --(7.796,4.206)--(7.802,4.209)--(7.808,4.212)--(7.814,4.215)--(7.820,4.219)--(7.826,4.222)%
  --(7.832,4.225)--(7.838,4.228)--(7.844,4.232)--(7.850,4.235)--(7.856,4.238)--(7.863,4.241)%
  --(7.869,4.245)--(7.875,4.248)--(7.881,4.251)--(7.887,4.254)--(7.893,4.257)--(7.899,4.261)%
  --(7.905,4.264)--(7.911,4.267)--(7.917,4.270)--(7.923,4.274)--(7.929,4.277)--(7.935,4.280)%
  --(7.941,4.283)--(7.947,4.287)--(7.953,4.290)--(7.960,4.293)--(7.966,4.296)--(7.972,4.299)%
  --(7.978,4.303)--(7.984,4.306)--(7.990,4.309)--(7.996,4.312)--(8.002,4.316)--(8.008,4.319)%
  --(8.014,4.322)--(8.020,4.325)--(8.026,4.328)--(8.032,4.332)--(8.038,4.335)--(8.044,4.338)%
  --(8.050,4.341)--(8.057,4.345)--(8.063,4.348)--(8.069,4.351)--(8.075,4.354)--(8.081,4.358)%
  --(8.087,4.361)--(8.093,4.364)--(8.099,4.367)--(8.105,4.370)--(8.111,4.374)--(8.117,4.377)%
  --(8.123,4.380)--(8.129,4.383)--(8.135,4.387)--(8.141,4.390)--(8.148,4.393)--(8.154,4.396)%
  --(8.160,4.400)--(8.166,4.403)--(8.172,4.406)--(8.178,4.409)--(8.184,4.412)--(8.190,4.416)%
  --(8.196,4.419)--(8.202,4.422)--(8.208,4.425)--(8.214,4.429)--(8.220,4.432)--(8.226,4.435)%
  --(8.232,4.438)--(8.238,4.442)--(8.245,4.445)--(8.251,4.448)--(8.257,4.451)--(8.263,4.454)%
  --(8.269,4.458)--(8.275,4.461)--(8.281,4.464)--(8.287,4.467)--(8.293,4.471)--(8.299,4.474)%
  --(8.305,4.477)--(8.311,4.480)--(8.317,4.484)--(8.323,4.487)--(8.329,4.490)--(8.335,4.493)%
  --(8.342,4.496)--(8.348,4.500)--(8.354,4.503)--(8.360,4.506)--(8.366,4.509)--(8.372,4.513)%
  --(8.378,4.516)--(8.384,4.519)--(8.390,4.522)--(8.396,4.526)--(8.402,4.529)--(8.408,4.532)%
  --(8.414,4.535)--(8.420,4.538)--(8.426,4.542)--(8.432,4.545)--(8.439,4.548)--(8.445,4.551)%
  --(8.451,4.555)--(8.457,4.558)--(8.463,4.561)--(8.469,4.564)--(8.475,4.567)--(8.481,4.571)%
  --(8.487,4.574)--(8.493,4.577)--(8.499,4.580)--(8.505,4.584)--(8.511,4.587)--(8.517,4.590)%
  --(8.523,4.593)--(8.530,4.597)--(8.536,4.600)--(8.542,4.603)--(8.548,4.606)--(8.554,4.609)%
  --(8.560,4.613)--(8.566,4.616)--(8.572,4.619)--(8.578,4.622)--(8.584,4.626)--(8.590,4.629)%
  --(8.596,4.632)--(8.602,4.635)--(8.608,4.639)--(8.614,4.642)--(8.620,4.645)--(8.627,4.648)%
  --(8.633,4.651)--(8.639,4.655)--(8.645,4.658)--(8.651,4.661)--(8.657,4.664)--(8.663,4.668)%
  --(8.669,4.671)--(8.675,4.674)--(8.681,4.677)--(8.687,4.681)--(8.693,4.684)--(8.699,4.687)%
  --(8.705,4.690)--(8.711,4.693)--(8.717,4.697)--(8.724,4.700)--(8.730,4.703)--(8.736,4.706)%
  --(8.742,4.710)--(8.748,4.713)--(8.754,4.716)--(8.760,4.719)--(8.766,4.723)--(8.772,4.726)%
  --(8.778,4.729)--(8.784,4.732)--(8.790,4.735)--(8.796,4.739)--(8.802,4.742)--(8.808,4.745)%
  --(8.814,4.748)--(8.821,4.752)--(8.827,4.755)--(8.833,4.758)--(8.839,4.761)--(8.845,4.765)%
  --(8.851,4.768)--(8.857,4.771)--(8.863,4.774)--(8.869,4.777)--(8.875,4.781)--(8.881,4.784)%
  --(8.887,4.787)--(8.893,4.790)--(8.899,4.794)--(8.905,4.797)--(8.912,4.800)--(8.918,4.803)%
  --(8.924,4.807)--(8.930,4.810)--(8.936,4.813)--(8.942,4.816)--(8.948,4.819)--(8.954,4.823)%
  --(8.960,4.826)--(8.966,4.829)--(8.972,4.832)--(8.978,4.836)--(8.984,4.839)--(8.990,4.842)%
  --(8.996,4.845)--(9.002,4.848)--(9.009,4.852)--(9.015,4.855)--(9.021,4.858)--(9.027,4.861)%
  --(9.033,4.865)--(9.039,4.868)--(9.045,4.871)--(9.051,4.874)--(9.057,4.878)--(9.063,4.881)%
  --(9.069,4.884)--(9.075,4.887)--(9.081,4.890)--(9.087,4.894)--(9.093,4.897)--(9.099,4.900)%
  --(9.106,4.903)--(9.112,4.907)--(9.118,4.910)--(9.124,4.913)--(9.130,4.916)--(9.136,4.920)%
  --(9.142,4.923)--(9.148,4.926)--(9.154,4.929)--(9.160,4.932)--(9.166,4.936)--(9.172,4.939)%
  --(9.178,4.942)--(9.184,4.945)--(9.190,4.949)--(9.196,4.952)--(9.203,4.955)--(9.209,4.958)%
  --(9.215,4.962)--(9.221,4.965)--(9.227,4.968)--(9.233,4.971)--(9.239,4.974)--(9.245,4.978)%
  --(9.251,4.981)--(9.257,4.984)--(9.263,4.987)--(9.269,4.991)--(9.275,4.994)--(9.281,4.997)%
  --(9.287,5.000)--(9.294,5.004)--(9.300,5.007)--(9.306,5.010)--(9.312,5.013)--(9.318,5.016)%
  --(9.324,5.020)--(9.330,5.023)--(9.336,5.026)--(9.342,5.029)--(9.348,5.033)--(9.354,5.036)%
  --(9.360,5.039)--(9.366,5.042)--(9.372,5.046)--(9.378,5.049)--(9.384,5.052)--(9.391,5.055)%
  --(9.397,5.058)--(9.403,5.062)--(9.409,5.065)--(9.415,5.068)--(9.421,5.071)--(9.427,5.075)%
  --(9.433,5.078)--(9.439,5.081)--(9.445,5.084)--(9.451,5.087)--(9.457,5.091)--(9.463,5.094)%
  --(9.469,5.097)--(9.475,5.100)--(9.481,5.104)--(9.488,5.107)--(9.494,5.110)--(9.500,5.113)%
  --(9.506,5.117)--(9.512,5.120)--(9.518,5.123)--(9.524,5.126)--(9.530,5.129)--(9.536,5.133)%
  --(9.542,5.136)--(9.548,5.139)--(9.554,5.142)--(9.560,5.146)--(9.566,5.149)--(9.572,5.152)%
  --(9.578,5.155)--(9.585,5.159)--(9.591,5.162)--(9.597,5.165)--(9.603,5.168)--(9.609,5.171)%
  --(9.615,5.175)--(9.621,5.178)--(9.627,5.181)--(9.633,5.184)--(9.639,5.188)--(9.645,5.191)%
  --(9.651,5.194)--(9.657,5.197)--(9.663,5.201)--(9.669,5.204)--(9.676,5.207)--(9.682,5.210)%
  --(9.688,5.213)--(9.694,5.217)--(9.700,5.220)--(9.706,5.223)--(9.712,5.226)--(9.718,5.230)%
  --(9.724,5.233)--(9.730,5.236)--(9.736,5.239)--(9.742,5.243)--(9.748,5.246)--(9.754,5.249)%
  --(9.760,5.252)--(9.766,5.255)--(9.773,5.259)--(9.779,5.262)--(9.785,5.265)--(9.791,5.268)%
  --(9.797,5.272)--(9.803,5.275)--(9.809,5.278)--(9.815,5.281)--(9.821,5.285)--(9.827,5.288)%
  --(9.833,5.291)--(9.839,5.294)--(9.845,5.297)--(9.851,5.301)--(9.857,5.304)--(9.863,5.307)%
  --(9.870,5.310)--(9.876,5.314)--(9.882,5.317)--(9.888,5.320)--(9.894,5.323)--(9.900,5.327)%
  --(9.906,5.330)--(9.912,5.333)--(9.918,5.336)--(9.924,5.339)--(9.930,5.343)--(9.936,5.346)%
  --(9.942,5.349)--(9.948,5.352)--(9.954,5.356)--(9.960,5.359)--(9.967,5.362)--(9.973,5.365)%
  --(9.979,5.368)--(9.985,5.372)--(9.991,5.375)--(9.997,5.378)--(10.003,5.381)--(10.009,5.385)%
  --(10.015,5.388)--(10.021,5.391)--(10.027,5.394)--(10.033,5.398)--(10.039,5.401)--(10.045,5.404)%
  --(10.051,5.407)--(10.058,5.410)--(10.064,5.414)--(10.070,5.417)--(10.076,5.420)--(10.082,5.423)%
  --(10.088,5.427)--(10.094,5.430)--(10.100,5.433)--(10.106,5.436)--(10.112,5.440)--(10.118,5.443)%
  --(10.124,5.446)--(10.130,5.449)--(10.136,5.452)--(10.142,5.456)--(10.148,5.459)--(10.155,5.462)%
  --(10.161,5.465)--(10.167,5.469)--(10.173,5.472)--(10.179,5.475)--(10.185,5.478)--(10.191,5.482)%
  --(10.197,5.485)--(10.203,5.488)--(10.209,5.491)--(10.215,5.494)--(10.221,5.498)--(10.227,5.501)%
  --(10.233,5.504)--(10.239,5.507)--(10.245,5.511)--(10.252,5.514)--(10.258,5.517)--(10.264,5.520)%
  --(10.270,5.524)--(10.276,5.527)--(10.282,5.530)--(10.288,5.533)--(10.294,5.536)--(10.300,5.540)%
  --(10.306,5.543)--(10.312,5.546)--(10.318,5.549)--(10.324,5.553)--(10.330,5.556)--(10.336,5.559)%
  --(10.342,5.562)--(10.349,5.566)--(10.355,5.569)--(10.361,5.572)--(10.367,5.575)--(10.373,5.578)%
  --(10.379,5.582)--(10.385,5.585)--(10.391,5.588)--(10.397,5.591)--(10.403,5.595)--(10.409,5.598)%
  --(10.415,5.601)--(10.421,5.604)--(10.427,5.607)--(10.433,5.611)--(10.440,5.614)--(10.446,5.617)%
  --(10.452,5.620)--(10.458,5.624)--(10.464,5.627)--(10.470,5.630)--(10.476,5.633)--(10.482,5.637)%
  --(10.488,5.640)--(10.494,5.643)--(10.500,5.646)--(10.506,5.649)--(10.512,5.653)--(10.518,5.656)%
  --(10.524,5.659)--(10.530,5.662)--(10.537,5.666)--(10.543,5.669)--(10.549,5.672)--(10.555,5.675)%
  --(10.561,5.679)--(10.567,5.682)--(10.573,5.685)--(10.579,5.688)--(10.585,5.691)--(10.591,5.695)%
  --(10.597,5.698)--(10.603,5.701)--(10.609,5.704)--(10.615,5.708)--(10.621,5.711)--(10.627,5.714)%
  --(10.634,5.717)--(10.640,5.721)--(10.646,5.724)--(10.652,5.727)--(10.658,5.730)--(10.664,5.733)%
  --(10.670,5.737)--(10.676,5.740)--(10.682,5.743)--(10.688,5.746)--(10.694,5.750)--(10.700,5.753)%
  --(10.706,5.756)--(10.712,5.759)--(10.718,5.763)--(10.724,5.766)--(10.731,5.769)--(10.737,5.772)%
  --(10.743,5.775)--(10.749,5.779)--(10.755,5.782)--(10.761,5.785)--(10.767,5.788)--(10.773,5.792)%
  --(10.779,5.795)--(10.785,5.798)--(10.791,5.801)--(10.797,5.805)--(10.803,5.808)--(10.809,5.811)%
  --(10.815,5.814)--(10.822,5.817)--(10.828,5.821)--(10.834,5.824)--(10.840,5.827)--(10.846,5.830)%
  --(10.852,5.834)--(10.858,5.837)--(10.864,5.840)--(10.870,5.843)--(10.876,5.846)--(10.882,5.850)%
  --(10.888,5.853)--(10.894,5.856)--(10.900,5.859)--(10.906,5.863)--(10.912,5.866)--(10.919,5.869)%
  --(10.925,5.872)--(10.931,5.876)--(10.937,5.879)--(10.943,5.882)--(10.949,5.885)--(10.955,5.888)%
  --(10.961,5.892)--(10.967,5.895)--(10.973,5.898)--(10.979,5.901)--(10.985,5.905)--(10.991,5.908)%
  --(10.997,5.911)--(11.003,5.914)--(11.009,5.918)--(11.016,5.921)--(11.022,5.924)--(11.028,5.927)%
  --(11.034,5.930)--(11.040,5.934)--(11.046,5.937)--(11.052,5.940)--(11.058,5.943)--(11.064,5.947)%
  --(11.070,5.950)--(11.076,5.953)--(11.082,5.956)--(11.088,5.960)--(11.094,5.963)--(11.100,5.966)%
  --(11.106,5.969)--(11.113,5.972)--(11.119,5.976)--(11.125,5.979)--(11.131,5.982)--(11.137,5.985)%
  --(11.143,5.989)--(11.149,5.992)--(11.155,5.995)--(11.161,5.998)--(11.167,6.002)--(11.173,6.005)%
  --(11.179,6.008)--(11.185,6.011)--(11.191,6.014)--(11.197,6.018)--(11.204,6.021)--(11.210,6.024)%
  --(11.216,6.027)--(11.222,6.031)--(11.228,6.034)--(11.234,6.037)--(11.240,6.040)--(11.246,6.044)%
  --(11.252,6.047)--(11.258,6.050)--(11.264,6.053)--(11.270,6.056)--(11.276,6.060)--(11.282,6.063)%
  --(11.288,6.066)--(11.294,6.069)--(11.301,6.073)--(11.307,6.076)--(11.313,6.079)--(11.319,6.082)%
  --(11.325,6.086)--(11.331,6.089)--(11.337,6.092)--(11.343,6.095)--(11.349,6.098)--(11.355,6.102)%
  --(11.361,6.105)--(11.367,6.108)--(11.373,6.111)--(11.379,6.115)--(11.385,6.118)--(11.391,6.121)%
  --(11.398,6.124)--(11.404,6.127)--(11.410,6.131)--(11.416,6.134)--(11.422,6.137)--(11.428,6.140)%
  --(11.434,6.144)--(11.440,6.147)--(11.446,6.150)--(11.452,6.153)--(11.458,6.157)--(11.464,6.160)%
  --(11.470,6.163)--(11.476,6.166)--(11.482,6.169)--(11.488,6.173)--(11.495,6.176)--(11.501,6.179)%
  --(11.507,6.182)--(11.513,6.186)--(11.519,6.189)--(11.525,6.192)--(11.531,6.195)--(11.537,6.199)%
  --(11.543,6.202)--(11.549,6.205)--(11.555,6.208)--(11.561,6.211)--(11.567,6.215)--(11.573,6.218)%
  --(11.579,6.221)--(11.586,6.224)--(11.592,6.228)--(11.598,6.231)--(11.604,6.234)--(11.610,6.237)%
  --(11.616,6.241)--(11.622,6.244)--(11.628,6.247)--(11.634,6.250)--(11.640,6.253)--(11.646,6.257)%
  --(11.652,6.260)--(11.658,6.263)--(11.664,6.266)--(11.670,6.270)--(11.676,6.273)--(11.683,6.276)%
  --(11.689,6.279)--(11.695,6.283)--(11.701,6.286)--(11.707,6.289)--(11.713,6.292)--(11.719,6.295)%
  --(11.725,6.299)--(11.731,6.302)--(11.737,6.305)--(11.743,6.308)--(11.749,6.312)--(11.755,6.315)%
  --(11.761,6.318)--(11.767,6.321)--(11.773,6.325)--(11.780,6.328)--(11.786,6.331)--(11.792,6.334)%
  --(11.798,6.337)--(11.804,6.341)--(11.810,6.344)--(11.816,6.347)--(11.822,6.350)--(11.828,6.354)%
  --(11.834,6.357)--(11.840,6.360)--(11.846,6.363)--(11.852,6.366)--(11.858,6.370)--(11.864,6.373)%
  --(11.870,6.376)--(11.877,6.379)--(11.883,6.383)--(11.889,6.386)--(11.895,6.389)--(11.901,6.392)%
  --(11.907,6.396)--(11.913,6.399)--(11.919,6.402)--(11.925,6.405)--(11.931,6.408)--(11.937,6.412)%
  --(11.943,6.415)--(11.949,6.418)--(11.955,6.421)--(11.961,6.425)--(11.968,6.428)--(11.974,6.431)%
  --(11.980,6.434)--(11.986,6.438)--(11.992,6.441)--(11.998,6.444)--(12.004,6.447)--(12.010,6.450)%
  --(12.016,6.454)--(12.022,6.457)--(12.028,6.460)--(12.034,6.463)--(12.040,6.467)--(12.046,6.470)%
  --(12.052,6.473)--(12.058,6.476)--(12.065,6.480)--(12.071,6.483)--(12.077,6.486)--(12.083,6.489)%
  --(12.089,6.492)--(12.095,6.496)--(12.101,6.499)--(12.107,6.502)--(12.113,6.505)--(12.119,6.509)%
  --(12.125,6.512)--(12.131,6.515)--(12.137,6.518)--(12.143,6.522)--(12.149,6.525)--(12.155,6.528)%
  --(12.162,6.531)--(12.168,6.534)--(12.174,6.538)--(12.180,6.541)--(12.186,6.544)--(12.192,6.547)%
  --(12.198,6.551)--(12.204,6.554)--(12.210,6.557)--(12.216,6.560)--(12.222,6.564)--(12.228,6.567)%
  --(12.234,6.570)--(12.240,6.573)--(12.246,6.576)--(12.252,6.580)--(12.259,6.583)--(12.265,6.586)%
  --(12.271,6.589)--(12.277,6.593)--(12.283,6.596)--(12.289,6.599)--(12.295,6.602)--(12.301,6.605)%
  --(12.307,6.609)--(12.313,6.612)--(12.319,6.615)--(12.325,6.618)--(12.331,6.622)--(12.337,6.625)%
  --(12.343,6.628)--(12.350,6.631)--(12.356,6.635)--(12.362,6.638)--(12.368,6.641)--(12.374,6.644)%
  --(12.380,6.647)--(12.386,6.651)--(12.392,6.654)--(12.398,6.657)--(12.404,6.660)--(12.410,6.664)%
  --(12.416,6.667)--(12.422,6.670)--(12.428,6.673)--(12.434,6.677)--(12.440,6.680)--(12.447,6.683)%
  --(12.453,6.686)--(12.459,6.689)--(12.465,6.693)--(12.471,6.696)--(12.477,6.699)--(12.483,6.702)%
  --(12.489,6.706)--(12.495,6.709)--(12.501,6.712)--(12.507,6.715)--(12.513,6.719)--(12.519,6.722)%
  --(12.525,6.725)--(12.531,6.728)--(12.537,6.731)--(12.544,6.735)--(12.550,6.738)--(12.556,6.741)%
  --(12.562,6.744)--(12.568,6.748)--(12.574,6.751)--(12.580,6.754)--(12.586,6.757)--(12.592,6.761)%
  --(12.598,6.764)--(12.604,6.767)--(12.610,6.770)--(12.616,6.773)--(12.622,6.777)--(12.628,6.780)%
  --(12.634,6.783)--(12.641,6.786)--(12.647,6.790)--(12.653,6.793)--(12.659,6.796);
\gpcolor{color=gp lt color border}
\gpsetlinewidth{1.00}
\draw[gp path] (1.320,8.631)--(1.320,0.985)--(13.447,0.985)--(13.447,8.631)--cycle;
%% coordinates of the plot area
\gpdefrectangularnode{gp plot 1}{\pgfpoint{1.320cm}{0.985cm}}{\pgfpoint{13.447cm}{8.631cm}}
\end{tikzpicture}
%% gnuplot variables

\caption{Conjunto de dados 2. A reta da regressão linear é dada por $y(x)=\np{29.97823068} x - \np{152.924429}$, $r^2=\np{0.956577278}$.}
\label{RetasConjuntosDados2}
\end{figure}

\begin{figure}[!h]\forcerectofloat
\centering
\begin{tikzpicture}[gnuplot]
%% generated with GNUPLOT 5.0p0 (Lua 5.3; terminal rev. 99, script rev. 100)
%% 2015-05-18T22:58:50 BRT
\path (0.000,0.000) rectangle (14.000,9.000);
\gpcolor{color=gp lt color border}
\gpsetlinetype{gp lt border}
\gpsetdashtype{gp dt solid}
\gpsetlinewidth{1.00}
\draw[gp path] (1.504,0.985)--(1.684,0.985);
\draw[gp path] (13.447,0.985)--(13.267,0.985);
\node[gp node right] at (1.320,0.985) {-2.0};
\draw[gp path] (1.504,2.375)--(1.684,2.375);
\draw[gp path] (13.447,2.375)--(13.267,2.375);
\node[gp node right] at (1.320,2.375) {-1.0};
\draw[gp path] (1.504,3.765)--(1.684,3.765);
\draw[gp path] (13.447,3.765)--(13.267,3.765);
\node[gp node right] at (1.320,3.765) {0.0};
\draw[gp path] (1.504,5.156)--(1.684,5.156);
\draw[gp path] (13.447,5.156)--(13.267,5.156);
\node[gp node right] at (1.320,5.156) {1.0};
\draw[gp path] (1.504,6.546)--(1.684,6.546);
\draw[gp path] (13.447,6.546)--(13.267,6.546);
\node[gp node right] at (1.320,6.546) {2.0};
\draw[gp path] (1.504,7.936)--(1.684,7.936);
\draw[gp path] (13.447,7.936)--(13.267,7.936);
\node[gp node right] at (1.320,7.936) {3.0};
\draw[gp path] (1.889,0.985)--(1.889,1.165);
\draw[gp path] (1.889,8.631)--(1.889,8.451);
\node[gp node center] at (1.889,0.677) {$0$};
\draw[gp path] (3.816,0.985)--(3.816,1.165);
\draw[gp path] (3.816,8.631)--(3.816,8.451);
\node[gp node center] at (3.816,0.677) {$5$};
\draw[gp path] (5.742,0.985)--(5.742,1.165);
\draw[gp path] (5.742,8.631)--(5.742,8.451);
\node[gp node center] at (5.742,0.677) {$10$};
\draw[gp path] (7.668,0.985)--(7.668,1.165);
\draw[gp path] (7.668,8.631)--(7.668,8.451);
\node[gp node center] at (7.668,0.677) {$15$};
\draw[gp path] (9.594,0.985)--(9.594,1.165);
\draw[gp path] (9.594,8.631)--(9.594,8.451);
\node[gp node center] at (9.594,0.677) {$20$};
\draw[gp path] (11.521,0.985)--(11.521,1.165);
\draw[gp path] (11.521,8.631)--(11.521,8.451);
\node[gp node center] at (11.521,0.677) {$25$};
\draw[gp path] (13.447,0.985)--(13.447,1.165);
\draw[gp path] (13.447,8.631)--(13.447,8.451);
\node[gp node center] at (13.447,0.677) {$30$};
\draw[gp path] (1.504,8.631)--(1.504,0.985)--(13.447,0.985)--(13.447,8.631)--cycle;
\node[gp node center,rotate=-270] at (0.246,4.808) {$y_3$};
\node[gp node center] at (7.475,0.215) {$x$};
\node[gp node left] at (2.972,8.297) {Dados experimentais};
\gpcolor{rgb color={0.000,0.000,0.000}}
\gpsetlinewidth{2.00}
\gpsetpointsize{4.00}
\gppoint{gp mark 7}{(2.164,5.483)}
\gppoint{gp mark 7}{(2.927,3.553)}
\gppoint{gp mark 7}{(3.580,2.852)}
\gppoint{gp mark 7}{(3.800,2.943)}
\gppoint{gp mark 7}{(4.295,5.102)}
\gppoint{gp mark 7}{(4.693,5.363)}
\gppoint{gp mark 7}{(4.809,6.038)}
\gppoint{gp mark 7}{(4.863,5.818)}
\gppoint{gp mark 7}{(4.937,6.183)}
\gppoint{gp mark 7}{(5.079,5.502)}
\gppoint{gp mark 7}{(5.370,4.044)}
\gppoint{gp mark 7}{(5.527,3.968)}
\gppoint{gp mark 7}{(5.860,3.444)}
\gppoint{gp mark 7}{(5.962,2.966)}
\gppoint{gp mark 7}{(6.195,3.934)}
\gppoint{gp mark 7}{(7.797,4.385)}
\gppoint{gp mark 7}{(8.447,2.681)}
\gppoint{gp mark 7}{(9.181,4.460)}
\gppoint{gp mark 7}{(9.829,5.320)}
\gppoint{gp mark 7}{(9.932,5.245)}
\gppoint{gp mark 7}{(10.016,5.273)}
\gppoint{gp mark 7}{(10.452,4.142)}
\gppoint{gp mark 7}{(10.522,3.446)}
\gppoint{gp mark 7}{(10.546,4.313)}
\gppoint{gp mark 7}{(10.780,3.490)}
\gppoint{gp mark 7}{(12.068,6.364)}
\gppoint{gp mark 7}{(12.238,6.162)}
\gppoint{gp mark 7}{(12.430,5.784)}
\gppoint{gp mark 7}{(2.330,8.297)}
\gpcolor{color=gp lt color border}
\node[gp node left] at (2.972,7.989) {$y(x)=\np{0.0192473251} x + \np{0.3162986129}$, $r^2=\np{0,0359267778}$};
\gpcolor{rgb color={0.000,0.000,0.000}}
\draw[gp path] (1.872,7.989)--(2.788,7.989);
\draw[gp path] (2.083,4.219)--(2.089,4.219)--(2.095,4.219)--(2.101,4.220)--(2.107,4.220)%
  --(2.113,4.221)--(2.119,4.221)--(2.125,4.221)--(2.131,4.222)--(2.137,4.222)--(2.143,4.223)%
  --(2.149,4.223)--(2.155,4.224)--(2.161,4.224)--(2.167,4.224)--(2.173,4.225)--(2.179,4.225)%
  --(2.185,4.226)--(2.191,4.226)--(2.197,4.226)--(2.203,4.227)--(2.209,4.227)--(2.215,4.228)%
  --(2.221,4.228)--(2.227,4.229)--(2.233,4.229)--(2.238,4.229)--(2.244,4.230)--(2.250,4.230)%
  --(2.256,4.231)--(2.262,4.231)--(2.268,4.231)--(2.274,4.232)--(2.280,4.232)--(2.286,4.233)%
  --(2.292,4.233)--(2.298,4.233)--(2.304,4.234)--(2.310,4.234)--(2.316,4.235)--(2.322,4.235)%
  --(2.328,4.236)--(2.334,4.236)--(2.340,4.236)--(2.346,4.237)--(2.352,4.237)--(2.358,4.238)%
  --(2.364,4.238)--(2.370,4.238)--(2.376,4.239)--(2.382,4.239)--(2.388,4.240)--(2.394,4.240)%
  --(2.400,4.241)--(2.406,4.241)--(2.412,4.241)--(2.418,4.242)--(2.424,4.242)--(2.430,4.243)%
  --(2.436,4.243)--(2.442,4.243)--(2.447,4.244)--(2.453,4.244)--(2.459,4.245)--(2.465,4.245)%
  --(2.471,4.246)--(2.477,4.246)--(2.483,4.246)--(2.489,4.247)--(2.495,4.247)--(2.501,4.248)%
  --(2.507,4.248)--(2.513,4.248)--(2.519,4.249)--(2.525,4.249)--(2.531,4.250)--(2.537,4.250)%
  --(2.543,4.250)--(2.549,4.251)--(2.555,4.251)--(2.561,4.252)--(2.567,4.252)--(2.573,4.253)%
  --(2.579,4.253)--(2.585,4.253)--(2.591,4.254)--(2.597,4.254)--(2.603,4.255)--(2.609,4.255)%
  --(2.615,4.255)--(2.621,4.256)--(2.627,4.256)--(2.633,4.257)--(2.639,4.257)--(2.645,4.258)%
  --(2.651,4.258)--(2.656,4.258)--(2.662,4.259)--(2.668,4.259)--(2.674,4.260)--(2.680,4.260)%
  --(2.686,4.260)--(2.692,4.261)--(2.698,4.261)--(2.704,4.262)--(2.710,4.262)--(2.716,4.263)%
  --(2.722,4.263)--(2.728,4.263)--(2.734,4.264)--(2.740,4.264)--(2.746,4.265)--(2.752,4.265)%
  --(2.758,4.265)--(2.764,4.266)--(2.770,4.266)--(2.776,4.267)--(2.782,4.267)--(2.788,4.267)%
  --(2.794,4.268)--(2.800,4.268)--(2.806,4.269)--(2.812,4.269)--(2.818,4.270)--(2.824,4.270)%
  --(2.830,4.270)--(2.836,4.271)--(2.842,4.271)--(2.848,4.272)--(2.854,4.272)--(2.860,4.272)%
  --(2.866,4.273)--(2.871,4.273)--(2.877,4.274)--(2.883,4.274)--(2.889,4.275)--(2.895,4.275)%
  --(2.901,4.275)--(2.907,4.276)--(2.913,4.276)--(2.919,4.277)--(2.925,4.277)--(2.931,4.277)%
  --(2.937,4.278)--(2.943,4.278)--(2.949,4.279)--(2.955,4.279)--(2.961,4.280)--(2.967,4.280)%
  --(2.973,4.280)--(2.979,4.281)--(2.985,4.281)--(2.991,4.282)--(2.997,4.282)--(3.003,4.282)%
  --(3.009,4.283)--(3.015,4.283)--(3.021,4.284)--(3.027,4.284)--(3.033,4.284)--(3.039,4.285)%
  --(3.045,4.285)--(3.051,4.286)--(3.057,4.286)--(3.063,4.287)--(3.069,4.287)--(3.075,4.287)%
  --(3.080,4.288)--(3.086,4.288)--(3.092,4.289)--(3.098,4.289)--(3.104,4.289)--(3.110,4.290)%
  --(3.116,4.290)--(3.122,4.291)--(3.128,4.291)--(3.134,4.292)--(3.140,4.292)--(3.146,4.292)%
  --(3.152,4.293)--(3.158,4.293)--(3.164,4.294)--(3.170,4.294)--(3.176,4.294)--(3.182,4.295)%
  --(3.188,4.295)--(3.194,4.296)--(3.200,4.296)--(3.206,4.297)--(3.212,4.297)--(3.218,4.297)%
  --(3.224,4.298)--(3.230,4.298)--(3.236,4.299)--(3.242,4.299)--(3.248,4.299)--(3.254,4.300)%
  --(3.260,4.300)--(3.266,4.301)--(3.272,4.301)--(3.278,4.301)--(3.284,4.302)--(3.289,4.302)%
  --(3.295,4.303)--(3.301,4.303)--(3.307,4.304)--(3.313,4.304)--(3.319,4.304)--(3.325,4.305)%
  --(3.331,4.305)--(3.337,4.306)--(3.343,4.306)--(3.349,4.306)--(3.355,4.307)--(3.361,4.307)%
  --(3.367,4.308)--(3.373,4.308)--(3.379,4.309)--(3.385,4.309)--(3.391,4.309)--(3.397,4.310)%
  --(3.403,4.310)--(3.409,4.311)--(3.415,4.311)--(3.421,4.311)--(3.427,4.312)--(3.433,4.312)%
  --(3.439,4.313)--(3.445,4.313)--(3.451,4.314)--(3.457,4.314)--(3.463,4.314)--(3.469,4.315)%
  --(3.475,4.315)--(3.481,4.316)--(3.487,4.316)--(3.493,4.316)--(3.498,4.317)--(3.504,4.317)%
  --(3.510,4.318)--(3.516,4.318)--(3.522,4.319)--(3.528,4.319)--(3.534,4.319)--(3.540,4.320)%
  --(3.546,4.320)--(3.552,4.321)--(3.558,4.321)--(3.564,4.321)--(3.570,4.322)--(3.576,4.322)%
  --(3.582,4.323)--(3.588,4.323)--(3.594,4.323)--(3.600,4.324)--(3.606,4.324)--(3.612,4.325)%
  --(3.618,4.325)--(3.624,4.326)--(3.630,4.326)--(3.636,4.326)--(3.642,4.327)--(3.648,4.327)%
  --(3.654,4.328)--(3.660,4.328)--(3.666,4.328)--(3.672,4.329)--(3.678,4.329)--(3.684,4.330)%
  --(3.690,4.330)--(3.696,4.331)--(3.702,4.331)--(3.707,4.331)--(3.713,4.332)--(3.719,4.332)%
  --(3.725,4.333)--(3.731,4.333)--(3.737,4.333)--(3.743,4.334)--(3.749,4.334)--(3.755,4.335)%
  --(3.761,4.335)--(3.767,4.336)--(3.773,4.336)--(3.779,4.336)--(3.785,4.337)--(3.791,4.337)%
  --(3.797,4.338)--(3.803,4.338)--(3.809,4.338)--(3.815,4.339)--(3.821,4.339)--(3.827,4.340)%
  --(3.833,4.340)--(3.839,4.340)--(3.845,4.341)--(3.851,4.341)--(3.857,4.342)--(3.863,4.342)%
  --(3.869,4.343)--(3.875,4.343)--(3.881,4.343)--(3.887,4.344)--(3.893,4.344)--(3.899,4.345)%
  --(3.905,4.345)--(3.911,4.345)--(3.916,4.346)--(3.922,4.346)--(3.928,4.347)--(3.934,4.347)%
  --(3.940,4.348)--(3.946,4.348)--(3.952,4.348)--(3.958,4.349)--(3.964,4.349)--(3.970,4.350)%
  --(3.976,4.350)--(3.982,4.350)--(3.988,4.351)--(3.994,4.351)--(4.000,4.352)--(4.006,4.352)%
  --(4.012,4.353)--(4.018,4.353)--(4.024,4.353)--(4.030,4.354)--(4.036,4.354)--(4.042,4.355)%
  --(4.048,4.355)--(4.054,4.355)--(4.060,4.356)--(4.066,4.356)--(4.072,4.357)--(4.078,4.357)%
  --(4.084,4.357)--(4.090,4.358)--(4.096,4.358)--(4.102,4.359)--(4.108,4.359)--(4.114,4.360)%
  --(4.120,4.360)--(4.125,4.360)--(4.131,4.361)--(4.137,4.361)--(4.143,4.362)--(4.149,4.362)%
  --(4.155,4.362)--(4.161,4.363)--(4.167,4.363)--(4.173,4.364)--(4.179,4.364)--(4.185,4.365)%
  --(4.191,4.365)--(4.197,4.365)--(4.203,4.366)--(4.209,4.366)--(4.215,4.367)--(4.221,4.367)%
  --(4.227,4.367)--(4.233,4.368)--(4.239,4.368)--(4.245,4.369)--(4.251,4.369)--(4.257,4.370)%
  --(4.263,4.370)--(4.269,4.370)--(4.275,4.371)--(4.281,4.371)--(4.287,4.372)--(4.293,4.372)%
  --(4.299,4.372)--(4.305,4.373)--(4.311,4.373)--(4.317,4.374)--(4.323,4.374)--(4.329,4.374)%
  --(4.334,4.375)--(4.340,4.375)--(4.346,4.376)--(4.352,4.376)--(4.358,4.377)--(4.364,4.377)%
  --(4.370,4.377)--(4.376,4.378)--(4.382,4.378)--(4.388,4.379)--(4.394,4.379)--(4.400,4.379)%
  --(4.406,4.380)--(4.412,4.380)--(4.418,4.381)--(4.424,4.381)--(4.430,4.382)--(4.436,4.382)%
  --(4.442,4.382)--(4.448,4.383)--(4.454,4.383)--(4.460,4.384)--(4.466,4.384)--(4.472,4.384)%
  --(4.478,4.385)--(4.484,4.385)--(4.490,4.386)--(4.496,4.386)--(4.502,4.387)--(4.508,4.387)%
  --(4.514,4.387)--(4.520,4.388)--(4.526,4.388)--(4.532,4.389)--(4.538,4.389)--(4.543,4.389)%
  --(4.549,4.390)--(4.555,4.390)--(4.561,4.391)--(4.567,4.391)--(4.573,4.391)--(4.579,4.392)%
  --(4.585,4.392)--(4.591,4.393)--(4.597,4.393)--(4.603,4.394)--(4.609,4.394)--(4.615,4.394)%
  --(4.621,4.395)--(4.627,4.395)--(4.633,4.396)--(4.639,4.396)--(4.645,4.396)--(4.651,4.397)%
  --(4.657,4.397)--(4.663,4.398)--(4.669,4.398)--(4.675,4.399)--(4.681,4.399)--(4.687,4.399)%
  --(4.693,4.400)--(4.699,4.400)--(4.705,4.401)--(4.711,4.401)--(4.717,4.401)--(4.723,4.402)%
  --(4.729,4.402)--(4.735,4.403)--(4.741,4.403)--(4.747,4.404)--(4.752,4.404)--(4.758,4.404)%
  --(4.764,4.405)--(4.770,4.405)--(4.776,4.406)--(4.782,4.406)--(4.788,4.406)--(4.794,4.407)%
  --(4.800,4.407)--(4.806,4.408)--(4.812,4.408)--(4.818,4.408)--(4.824,4.409)--(4.830,4.409)%
  --(4.836,4.410)--(4.842,4.410)--(4.848,4.411)--(4.854,4.411)--(4.860,4.411)--(4.866,4.412)%
  --(4.872,4.412)--(4.878,4.413)--(4.884,4.413)--(4.890,4.413)--(4.896,4.414)--(4.902,4.414)%
  --(4.908,4.415)--(4.914,4.415)--(4.920,4.416)--(4.926,4.416)--(4.932,4.416)--(4.938,4.417)%
  --(4.944,4.417)--(4.950,4.418)--(4.956,4.418)--(4.961,4.418)--(4.967,4.419)--(4.973,4.419)%
  --(4.979,4.420)--(4.985,4.420)--(4.991,4.421)--(4.997,4.421)--(5.003,4.421)--(5.009,4.422)%
  --(5.015,4.422)--(5.021,4.423)--(5.027,4.423)--(5.033,4.423)--(5.039,4.424)--(5.045,4.424)%
  --(5.051,4.425)--(5.057,4.425)--(5.063,4.426)--(5.069,4.426)--(5.075,4.426)--(5.081,4.427)%
  --(5.087,4.427)--(5.093,4.428)--(5.099,4.428)--(5.105,4.428)--(5.111,4.429)--(5.117,4.429)%
  --(5.123,4.430)--(5.129,4.430)--(5.135,4.430)--(5.141,4.431)--(5.147,4.431)--(5.153,4.432)%
  --(5.159,4.432)--(5.165,4.433)--(5.171,4.433)--(5.176,4.433)--(5.182,4.434)--(5.188,4.434)%
  --(5.194,4.435)--(5.200,4.435)--(5.206,4.435)--(5.212,4.436)--(5.218,4.436)--(5.224,4.437)%
  --(5.230,4.437)--(5.236,4.438)--(5.242,4.438)--(5.248,4.438)--(5.254,4.439)--(5.260,4.439)%
  --(5.266,4.440)--(5.272,4.440)--(5.278,4.440)--(5.284,4.441)--(5.290,4.441)--(5.296,4.442)%
  --(5.302,4.442)--(5.308,4.443)--(5.314,4.443)--(5.320,4.443)--(5.326,4.444)--(5.332,4.444)%
  --(5.338,4.445)--(5.344,4.445)--(5.350,4.445)--(5.356,4.446)--(5.362,4.446)--(5.368,4.447)%
  --(5.374,4.447)--(5.380,4.447)--(5.385,4.448)--(5.391,4.448)--(5.397,4.449)--(5.403,4.449)%
  --(5.409,4.450)--(5.415,4.450)--(5.421,4.450)--(5.427,4.451)--(5.433,4.451)--(5.439,4.452)%
  --(5.445,4.452)--(5.451,4.452)--(5.457,4.453)--(5.463,4.453)--(5.469,4.454)--(5.475,4.454)%
  --(5.481,4.455)--(5.487,4.455)--(5.493,4.455)--(5.499,4.456)--(5.505,4.456)--(5.511,4.457)%
  --(5.517,4.457)--(5.523,4.457)--(5.529,4.458)--(5.535,4.458)--(5.541,4.459)--(5.547,4.459)%
  --(5.553,4.460)--(5.559,4.460)--(5.565,4.460)--(5.571,4.461)--(5.577,4.461)--(5.583,4.462)%
  --(5.589,4.462)--(5.594,4.462)--(5.600,4.463)--(5.606,4.463)--(5.612,4.464)--(5.618,4.464)%
  --(5.624,4.464)--(5.630,4.465)--(5.636,4.465)--(5.642,4.466)--(5.648,4.466)--(5.654,4.467)%
  --(5.660,4.467)--(5.666,4.467)--(5.672,4.468)--(5.678,4.468)--(5.684,4.469)--(5.690,4.469)%
  --(5.696,4.469)--(5.702,4.470)--(5.708,4.470)--(5.714,4.471)--(5.720,4.471)--(5.726,4.472)%
  --(5.732,4.472)--(5.738,4.472)--(5.744,4.473)--(5.750,4.473)--(5.756,4.474)--(5.762,4.474)%
  --(5.768,4.474)--(5.774,4.475)--(5.780,4.475)--(5.786,4.476)--(5.792,4.476)--(5.798,4.477)%
  --(5.803,4.477)--(5.809,4.477)--(5.815,4.478)--(5.821,4.478)--(5.827,4.479)--(5.833,4.479)%
  --(5.839,4.479)--(5.845,4.480)--(5.851,4.480)--(5.857,4.481)--(5.863,4.481)--(5.869,4.481)%
  --(5.875,4.482)--(5.881,4.482)--(5.887,4.483)--(5.893,4.483)--(5.899,4.484)--(5.905,4.484)%
  --(5.911,4.484)--(5.917,4.485)--(5.923,4.485)--(5.929,4.486)--(5.935,4.486)--(5.941,4.486)%
  --(5.947,4.487)--(5.953,4.487)--(5.959,4.488)--(5.965,4.488)--(5.971,4.489)--(5.977,4.489)%
  --(5.983,4.489)--(5.989,4.490)--(5.995,4.490)--(6.001,4.491)--(6.007,4.491)--(6.012,4.491)%
  --(6.018,4.492)--(6.024,4.492)--(6.030,4.493)--(6.036,4.493)--(6.042,4.494)--(6.048,4.494)%
  --(6.054,4.494)--(6.060,4.495)--(6.066,4.495)--(6.072,4.496)--(6.078,4.496)--(6.084,4.496)%
  --(6.090,4.497)--(6.096,4.497)--(6.102,4.498)--(6.108,4.498)--(6.114,4.498)--(6.120,4.499)%
  --(6.126,4.499)--(6.132,4.500)--(6.138,4.500)--(6.144,4.501)--(6.150,4.501)--(6.156,4.501)%
  --(6.162,4.502)--(6.168,4.502)--(6.174,4.503)--(6.180,4.503)--(6.186,4.503)--(6.192,4.504)%
  --(6.198,4.504)--(6.204,4.505)--(6.210,4.505)--(6.216,4.506)--(6.221,4.506)--(6.227,4.506)%
  --(6.233,4.507)--(6.239,4.507)--(6.245,4.508)--(6.251,4.508)--(6.257,4.508)--(6.263,4.509)%
  --(6.269,4.509)--(6.275,4.510)--(6.281,4.510)--(6.287,4.511)--(6.293,4.511)--(6.299,4.511)%
  --(6.305,4.512)--(6.311,4.512)--(6.317,4.513)--(6.323,4.513)--(6.329,4.513)--(6.335,4.514)%
  --(6.341,4.514)--(6.347,4.515)--(6.353,4.515)--(6.359,4.516)--(6.365,4.516)--(6.371,4.516)%
  --(6.377,4.517)--(6.383,4.517)--(6.389,4.518)--(6.395,4.518)--(6.401,4.518)--(6.407,4.519)%
  --(6.413,4.519)--(6.419,4.520)--(6.425,4.520)--(6.430,4.520)--(6.436,4.521)--(6.442,4.521)%
  --(6.448,4.522)--(6.454,4.522)--(6.460,4.523)--(6.466,4.523)--(6.472,4.523)--(6.478,4.524)%
  --(6.484,4.524)--(6.490,4.525)--(6.496,4.525)--(6.502,4.525)--(6.508,4.526)--(6.514,4.526)%
  --(6.520,4.527)--(6.526,4.527)--(6.532,4.528)--(6.538,4.528)--(6.544,4.528)--(6.550,4.529)%
  --(6.556,4.529)--(6.562,4.530)--(6.568,4.530)--(6.574,4.530)--(6.580,4.531)--(6.586,4.531)%
  --(6.592,4.532)--(6.598,4.532)--(6.604,4.533)--(6.610,4.533)--(6.616,4.533)--(6.622,4.534)%
  --(6.628,4.534)--(6.634,4.535)--(6.639,4.535)--(6.645,4.535)--(6.651,4.536)--(6.657,4.536)%
  --(6.663,4.537)--(6.669,4.537)--(6.675,4.537)--(6.681,4.538)--(6.687,4.538)--(6.693,4.539)%
  --(6.699,4.539)--(6.705,4.540)--(6.711,4.540)--(6.717,4.540)--(6.723,4.541)--(6.729,4.541)%
  --(6.735,4.542)--(6.741,4.542)--(6.747,4.542)--(6.753,4.543)--(6.759,4.543)--(6.765,4.544)%
  --(6.771,4.544)--(6.777,4.545)--(6.783,4.545)--(6.789,4.545)--(6.795,4.546)--(6.801,4.546)%
  --(6.807,4.547)--(6.813,4.547)--(6.819,4.547)--(6.825,4.548)--(6.831,4.548)--(6.837,4.549)%
  --(6.843,4.549)--(6.848,4.550)--(6.854,4.550)--(6.860,4.550)--(6.866,4.551)--(6.872,4.551)%
  --(6.878,4.552)--(6.884,4.552)--(6.890,4.552)--(6.896,4.553)--(6.902,4.553)--(6.908,4.554)%
  --(6.914,4.554)--(6.920,4.554)--(6.926,4.555)--(6.932,4.555)--(6.938,4.556)--(6.944,4.556)%
  --(6.950,4.557)--(6.956,4.557)--(6.962,4.557)--(6.968,4.558)--(6.974,4.558)--(6.980,4.559)%
  --(6.986,4.559)--(6.992,4.559)--(6.998,4.560)--(7.004,4.560)--(7.010,4.561)--(7.016,4.561)%
  --(7.022,4.562)--(7.028,4.562)--(7.034,4.562)--(7.040,4.563)--(7.046,4.563)--(7.052,4.564)%
  --(7.057,4.564)--(7.063,4.564)--(7.069,4.565)--(7.075,4.565)--(7.081,4.566)--(7.087,4.566)%
  --(7.093,4.567)--(7.099,4.567)--(7.105,4.567)--(7.111,4.568)--(7.117,4.568)--(7.123,4.569)%
  --(7.129,4.569)--(7.135,4.569)--(7.141,4.570)--(7.147,4.570)--(7.153,4.571)--(7.159,4.571)%
  --(7.165,4.571)--(7.171,4.572)--(7.177,4.572)--(7.183,4.573)--(7.189,4.573)--(7.195,4.574)%
  --(7.201,4.574)--(7.207,4.574)--(7.213,4.575)--(7.219,4.575)--(7.225,4.576)--(7.231,4.576)%
  --(7.237,4.576)--(7.243,4.577)--(7.249,4.577)--(7.255,4.578)--(7.261,4.578)--(7.266,4.579)%
  --(7.272,4.579)--(7.278,4.579)--(7.284,4.580)--(7.290,4.580)--(7.296,4.581)--(7.302,4.581)%
  --(7.308,4.581)--(7.314,4.582)--(7.320,4.582)--(7.326,4.583)--(7.332,4.583)--(7.338,4.584)%
  --(7.344,4.584)--(7.350,4.584)--(7.356,4.585)--(7.362,4.585)--(7.368,4.586)--(7.374,4.586)%
  --(7.380,4.586)--(7.386,4.587)--(7.392,4.587)--(7.398,4.588)--(7.404,4.588)--(7.410,4.588)%
  --(7.416,4.589)--(7.422,4.589)--(7.428,4.590)--(7.434,4.590)--(7.440,4.591)--(7.446,4.591)%
  --(7.452,4.591)--(7.458,4.592)--(7.464,4.592)--(7.470,4.593)--(7.476,4.593)--(7.481,4.593)%
  --(7.487,4.594)--(7.493,4.594)--(7.499,4.595)--(7.505,4.595)--(7.511,4.596)--(7.517,4.596)%
  --(7.523,4.596)--(7.529,4.597)--(7.535,4.597)--(7.541,4.598)--(7.547,4.598)--(7.553,4.598)%
  --(7.559,4.599)--(7.565,4.599)--(7.571,4.600)--(7.577,4.600)--(7.583,4.601)--(7.589,4.601)%
  --(7.595,4.601)--(7.601,4.602)--(7.607,4.602)--(7.613,4.603)--(7.619,4.603)--(7.625,4.603)%
  --(7.631,4.604)--(7.637,4.604)--(7.643,4.605)--(7.649,4.605)--(7.655,4.605)--(7.661,4.606)%
  --(7.667,4.606)--(7.673,4.607)--(7.679,4.607)--(7.685,4.608)--(7.690,4.608)--(7.696,4.608)%
  --(7.702,4.609)--(7.708,4.609)--(7.714,4.610)--(7.720,4.610)--(7.726,4.610)--(7.732,4.611)%
  --(7.738,4.611)--(7.744,4.612)--(7.750,4.612)--(7.756,4.613)--(7.762,4.613)--(7.768,4.613)%
  --(7.774,4.614)--(7.780,4.614)--(7.786,4.615)--(7.792,4.615)--(7.798,4.615)--(7.804,4.616)%
  --(7.810,4.616)--(7.816,4.617)--(7.822,4.617)--(7.828,4.618)--(7.834,4.618)--(7.840,4.618)%
  --(7.846,4.619)--(7.852,4.619)--(7.858,4.620)--(7.864,4.620)--(7.870,4.620)--(7.876,4.621)%
  --(7.882,4.621)--(7.888,4.622)--(7.894,4.622)--(7.899,4.623)--(7.905,4.623)--(7.911,4.623)%
  --(7.917,4.624)--(7.923,4.624)--(7.929,4.625)--(7.935,4.625)--(7.941,4.625)--(7.947,4.626)%
  --(7.953,4.626)--(7.959,4.627)--(7.965,4.627)--(7.971,4.627)--(7.977,4.628)--(7.983,4.628)%
  --(7.989,4.629)--(7.995,4.629)--(8.001,4.630)--(8.007,4.630)--(8.013,4.630)--(8.019,4.631)%
  --(8.025,4.631)--(8.031,4.632)--(8.037,4.632)--(8.043,4.632)--(8.049,4.633)--(8.055,4.633)%
  --(8.061,4.634)--(8.067,4.634)--(8.073,4.635)--(8.079,4.635)--(8.085,4.635)--(8.091,4.636)%
  --(8.097,4.636)--(8.103,4.637)--(8.108,4.637)--(8.114,4.637)--(8.120,4.638)--(8.126,4.638)%
  --(8.132,4.639)--(8.138,4.639)--(8.144,4.640)--(8.150,4.640)--(8.156,4.640)--(8.162,4.641)%
  --(8.168,4.641)--(8.174,4.642)--(8.180,4.642)--(8.186,4.642)--(8.192,4.643)--(8.198,4.643)%
  --(8.204,4.644)--(8.210,4.644)--(8.216,4.644)--(8.222,4.645)--(8.228,4.645)--(8.234,4.646)%
  --(8.240,4.646)--(8.246,4.647)--(8.252,4.647)--(8.258,4.647)--(8.264,4.648)--(8.270,4.648)%
  --(8.276,4.649)--(8.282,4.649)--(8.288,4.649)--(8.294,4.650)--(8.300,4.650)--(8.306,4.651)%
  --(8.312,4.651)--(8.317,4.652)--(8.323,4.652)--(8.329,4.652)--(8.335,4.653)--(8.341,4.653)%
  --(8.347,4.654)--(8.353,4.654)--(8.359,4.654)--(8.365,4.655)--(8.371,4.655)--(8.377,4.656)%
  --(8.383,4.656)--(8.389,4.657)--(8.395,4.657)--(8.401,4.657)--(8.407,4.658)--(8.413,4.658)%
  --(8.419,4.659)--(8.425,4.659)--(8.431,4.659)--(8.437,4.660)--(8.443,4.660)--(8.449,4.661)%
  --(8.455,4.661)--(8.461,4.661)--(8.467,4.662)--(8.473,4.662)--(8.479,4.663)--(8.485,4.663)%
  --(8.491,4.664)--(8.497,4.664)--(8.503,4.664)--(8.509,4.665)--(8.515,4.665)--(8.521,4.666)%
  --(8.526,4.666)--(8.532,4.666)--(8.538,4.667)--(8.544,4.667)--(8.550,4.668)--(8.556,4.668)%
  --(8.562,4.669)--(8.568,4.669)--(8.574,4.669)--(8.580,4.670)--(8.586,4.670)--(8.592,4.671)%
  --(8.598,4.671)--(8.604,4.671)--(8.610,4.672)--(8.616,4.672)--(8.622,4.673)--(8.628,4.673)%
  --(8.634,4.674)--(8.640,4.674)--(8.646,4.674)--(8.652,4.675)--(8.658,4.675)--(8.664,4.676)%
  --(8.670,4.676)--(8.676,4.676)--(8.682,4.677)--(8.688,4.677)--(8.694,4.678)--(8.700,4.678)%
  --(8.706,4.678)--(8.712,4.679)--(8.718,4.679)--(8.724,4.680)--(8.730,4.680)--(8.735,4.681)%
  --(8.741,4.681)--(8.747,4.681)--(8.753,4.682)--(8.759,4.682)--(8.765,4.683)--(8.771,4.683)%
  --(8.777,4.683)--(8.783,4.684)--(8.789,4.684)--(8.795,4.685)--(8.801,4.685)--(8.807,4.686)%
  --(8.813,4.686)--(8.819,4.686)--(8.825,4.687)--(8.831,4.687)--(8.837,4.688)--(8.843,4.688)%
  --(8.849,4.688)--(8.855,4.689)--(8.861,4.689)--(8.867,4.690)--(8.873,4.690)--(8.879,4.691)%
  --(8.885,4.691)--(8.891,4.691)--(8.897,4.692)--(8.903,4.692)--(8.909,4.693)--(8.915,4.693)%
  --(8.921,4.693)--(8.927,4.694)--(8.933,4.694)--(8.939,4.695)--(8.944,4.695)--(8.950,4.695)%
  --(8.956,4.696)--(8.962,4.696)--(8.968,4.697)--(8.974,4.697)--(8.980,4.698)--(8.986,4.698)%
  --(8.992,4.698)--(8.998,4.699)--(9.004,4.699)--(9.010,4.700)--(9.016,4.700)--(9.022,4.700)%
  --(9.028,4.701)--(9.034,4.701)--(9.040,4.702)--(9.046,4.702)--(9.052,4.703)--(9.058,4.703)%
  --(9.064,4.703)--(9.070,4.704)--(9.076,4.704)--(9.082,4.705)--(9.088,4.705)--(9.094,4.705)%
  --(9.100,4.706)--(9.106,4.706)--(9.112,4.707)--(9.118,4.707)--(9.124,4.708)--(9.130,4.708)%
  --(9.136,4.708)--(9.142,4.709)--(9.148,4.709)--(9.153,4.710)--(9.159,4.710)--(9.165,4.710)%
  --(9.171,4.711)--(9.177,4.711)--(9.183,4.712)--(9.189,4.712)--(9.195,4.713)--(9.201,4.713)%
  --(9.207,4.713)--(9.213,4.714)--(9.219,4.714)--(9.225,4.715)--(9.231,4.715)--(9.237,4.715)%
  --(9.243,4.716)--(9.249,4.716)--(9.255,4.717)--(9.261,4.717)--(9.267,4.717)--(9.273,4.718)%
  --(9.279,4.718)--(9.285,4.719)--(9.291,4.719)--(9.297,4.720)--(9.303,4.720)--(9.309,4.720)%
  --(9.315,4.721)--(9.321,4.721)--(9.327,4.722)--(9.333,4.722)--(9.339,4.722)--(9.345,4.723)%
  --(9.351,4.723)--(9.357,4.724)--(9.362,4.724)--(9.368,4.725)--(9.374,4.725)--(9.380,4.725)%
  --(9.386,4.726)--(9.392,4.726)--(9.398,4.727)--(9.404,4.727)--(9.410,4.727)--(9.416,4.728)%
  --(9.422,4.728)--(9.428,4.729)--(9.434,4.729)--(9.440,4.730)--(9.446,4.730)--(9.452,4.730)%
  --(9.458,4.731)--(9.464,4.731)--(9.470,4.732)--(9.476,4.732)--(9.482,4.732)--(9.488,4.733)%
  --(9.494,4.733)--(9.500,4.734)--(9.506,4.734)--(9.512,4.734)--(9.518,4.735)--(9.524,4.735)%
  --(9.530,4.736)--(9.536,4.736)--(9.542,4.737)--(9.548,4.737)--(9.554,4.737)--(9.560,4.738)%
  --(9.566,4.738)--(9.571,4.739)--(9.577,4.739)--(9.583,4.739)--(9.589,4.740)--(9.595,4.740)%
  --(9.601,4.741)--(9.607,4.741)--(9.613,4.742)--(9.619,4.742)--(9.625,4.742)--(9.631,4.743)%
  --(9.637,4.743)--(9.643,4.744)--(9.649,4.744)--(9.655,4.744)--(9.661,4.745)--(9.667,4.745)%
  --(9.673,4.746)--(9.679,4.746)--(9.685,4.747)--(9.691,4.747)--(9.697,4.747)--(9.703,4.748)%
  --(9.709,4.748)--(9.715,4.749)--(9.721,4.749)--(9.727,4.749)--(9.733,4.750)--(9.739,4.750)%
  --(9.745,4.751)--(9.751,4.751)--(9.757,4.751)--(9.763,4.752)--(9.769,4.752)--(9.775,4.753)%
  --(9.780,4.753)--(9.786,4.754)--(9.792,4.754)--(9.798,4.754)--(9.804,4.755)--(9.810,4.755)%
  --(9.816,4.756)--(9.822,4.756)--(9.828,4.756)--(9.834,4.757)--(9.840,4.757)--(9.846,4.758)%
  --(9.852,4.758)--(9.858,4.759)--(9.864,4.759)--(9.870,4.759)--(9.876,4.760)--(9.882,4.760)%
  --(9.888,4.761)--(9.894,4.761)--(9.900,4.761)--(9.906,4.762)--(9.912,4.762)--(9.918,4.763)%
  --(9.924,4.763)--(9.930,4.764)--(9.936,4.764)--(9.942,4.764)--(9.948,4.765)--(9.954,4.765)%
  --(9.960,4.766)--(9.966,4.766)--(9.972,4.766)--(9.978,4.767)--(9.984,4.767)--(9.990,4.768)%
  --(9.995,4.768)--(10.001,4.768)--(10.007,4.769)--(10.013,4.769)--(10.019,4.770)--(10.025,4.770)%
  --(10.031,4.771)--(10.037,4.771)--(10.043,4.771)--(10.049,4.772)--(10.055,4.772)--(10.061,4.773)%
  --(10.067,4.773)--(10.073,4.773)--(10.079,4.774)--(10.085,4.774)--(10.091,4.775)--(10.097,4.775)%
  --(10.103,4.776)--(10.109,4.776)--(10.115,4.776)--(10.121,4.777)--(10.127,4.777)--(10.133,4.778)%
  --(10.139,4.778)--(10.145,4.778)--(10.151,4.779)--(10.157,4.779)--(10.163,4.780)--(10.169,4.780)%
  --(10.175,4.781)--(10.181,4.781)--(10.187,4.781)--(10.193,4.782)--(10.199,4.782)--(10.204,4.783)%
  --(10.210,4.783)--(10.216,4.783)--(10.222,4.784)--(10.228,4.784)--(10.234,4.785)--(10.240,4.785)%
  --(10.246,4.785)--(10.252,4.786)--(10.258,4.786)--(10.264,4.787)--(10.270,4.787)--(10.276,4.788)%
  --(10.282,4.788)--(10.288,4.788)--(10.294,4.789)--(10.300,4.789)--(10.306,4.790)--(10.312,4.790)%
  --(10.318,4.790)--(10.324,4.791)--(10.330,4.791)--(10.336,4.792)--(10.342,4.792)--(10.348,4.793)%
  --(10.354,4.793)--(10.360,4.793)--(10.366,4.794)--(10.372,4.794)--(10.378,4.795)--(10.384,4.795)%
  --(10.390,4.795)--(10.396,4.796)--(10.402,4.796)--(10.408,4.797)--(10.413,4.797)--(10.419,4.798)%
  --(10.425,4.798)--(10.431,4.798)--(10.437,4.799)--(10.443,4.799)--(10.449,4.800)--(10.455,4.800)%
  --(10.461,4.800)--(10.467,4.801)--(10.473,4.801)--(10.479,4.802)--(10.485,4.802)--(10.491,4.802)%
  --(10.497,4.803)--(10.503,4.803)--(10.509,4.804)--(10.515,4.804)--(10.521,4.805)--(10.527,4.805)%
  --(10.533,4.805)--(10.539,4.806)--(10.545,4.806)--(10.551,4.807)--(10.557,4.807)--(10.563,4.807)%
  --(10.569,4.808)--(10.575,4.808)--(10.581,4.809)--(10.587,4.809)--(10.593,4.810)--(10.599,4.810)%
  --(10.605,4.810)--(10.611,4.811)--(10.617,4.811)--(10.622,4.812)--(10.628,4.812)--(10.634,4.812)%
  --(10.640,4.813)--(10.646,4.813)--(10.652,4.814)--(10.658,4.814)--(10.664,4.815)--(10.670,4.815)%
  --(10.676,4.815)--(10.682,4.816)--(10.688,4.816)--(10.694,4.817)--(10.700,4.817)--(10.706,4.817)%
  --(10.712,4.818)--(10.718,4.818)--(10.724,4.819)--(10.730,4.819)--(10.736,4.820)--(10.742,4.820)%
  --(10.748,4.820)--(10.754,4.821)--(10.760,4.821)--(10.766,4.822)--(10.772,4.822)--(10.778,4.822)%
  --(10.784,4.823)--(10.790,4.823)--(10.796,4.824)--(10.802,4.824)--(10.808,4.824)--(10.814,4.825)%
  --(10.820,4.825)--(10.826,4.826)--(10.831,4.826)--(10.837,4.827)--(10.843,4.827)--(10.849,4.827)%
  --(10.855,4.828)--(10.861,4.828)--(10.867,4.829)--(10.873,4.829)--(10.879,4.829)--(10.885,4.830)%
  --(10.891,4.830)--(10.897,4.831)--(10.903,4.831)--(10.909,4.832)--(10.915,4.832)--(10.921,4.832)%
  --(10.927,4.833)--(10.933,4.833)--(10.939,4.834)--(10.945,4.834)--(10.951,4.834)--(10.957,4.835)%
  --(10.963,4.835)--(10.969,4.836)--(10.975,4.836)--(10.981,4.837)--(10.987,4.837)--(10.993,4.837)%
  --(10.999,4.838)--(11.005,4.838)--(11.011,4.839)--(11.017,4.839)--(11.023,4.839)--(11.029,4.840)%
  --(11.035,4.840)--(11.040,4.841)--(11.046,4.841)--(11.052,4.841)--(11.058,4.842)--(11.064,4.842)%
  --(11.070,4.843)--(11.076,4.843)--(11.082,4.844)--(11.088,4.844)--(11.094,4.844)--(11.100,4.845)%
  --(11.106,4.845)--(11.112,4.846)--(11.118,4.846)--(11.124,4.846)--(11.130,4.847)--(11.136,4.847)%
  --(11.142,4.848)--(11.148,4.848)--(11.154,4.849)--(11.160,4.849)--(11.166,4.849)--(11.172,4.850)%
  --(11.178,4.850)--(11.184,4.851)--(11.190,4.851)--(11.196,4.851)--(11.202,4.852)--(11.208,4.852)%
  --(11.214,4.853)--(11.220,4.853)--(11.226,4.854)--(11.232,4.854)--(11.238,4.854)--(11.244,4.855)%
  --(11.249,4.855)--(11.255,4.856)--(11.261,4.856)--(11.267,4.856)--(11.273,4.857)--(11.279,4.857)%
  --(11.285,4.858)--(11.291,4.858)--(11.297,4.858)--(11.303,4.859)--(11.309,4.859)--(11.315,4.860)%
  --(11.321,4.860)--(11.327,4.861)--(11.333,4.861)--(11.339,4.861)--(11.345,4.862)--(11.351,4.862)%
  --(11.357,4.863)--(11.363,4.863)--(11.369,4.863)--(11.375,4.864)--(11.381,4.864)--(11.387,4.865)%
  --(11.393,4.865)--(11.399,4.866)--(11.405,4.866)--(11.411,4.866)--(11.417,4.867)--(11.423,4.867)%
  --(11.429,4.868)--(11.435,4.868)--(11.441,4.868)--(11.447,4.869)--(11.453,4.869)--(11.458,4.870)%
  --(11.464,4.870)--(11.470,4.871)--(11.476,4.871)--(11.482,4.871)--(11.488,4.872)--(11.494,4.872)%
  --(11.500,4.873)--(11.506,4.873)--(11.512,4.873)--(11.518,4.874)--(11.524,4.874)--(11.530,4.875)%
  --(11.536,4.875)--(11.542,4.875)--(11.548,4.876)--(11.554,4.876)--(11.560,4.877)--(11.566,4.877)%
  --(11.572,4.878)--(11.578,4.878)--(11.584,4.878)--(11.590,4.879)--(11.596,4.879)--(11.602,4.880)%
  --(11.608,4.880)--(11.614,4.880)--(11.620,4.881)--(11.626,4.881)--(11.632,4.882)--(11.638,4.882)%
  --(11.644,4.883)--(11.650,4.883)--(11.656,4.883)--(11.662,4.884)--(11.667,4.884)--(11.673,4.885)%
  --(11.679,4.885)--(11.685,4.885)--(11.691,4.886)--(11.697,4.886)--(11.703,4.887)--(11.709,4.887)%
  --(11.715,4.888)--(11.721,4.888)--(11.727,4.888)--(11.733,4.889)--(11.739,4.889)--(11.745,4.890)%
  --(11.751,4.890)--(11.757,4.890)--(11.763,4.891)--(11.769,4.891)--(11.775,4.892)--(11.781,4.892)%
  --(11.787,4.892)--(11.793,4.893)--(11.799,4.893)--(11.805,4.894)--(11.811,4.894)--(11.817,4.895)%
  --(11.823,4.895)--(11.829,4.895)--(11.835,4.896)--(11.841,4.896)--(11.847,4.897)--(11.853,4.897)%
  --(11.859,4.897)--(11.865,4.898)--(11.871,4.898)--(11.876,4.899)--(11.882,4.899)--(11.888,4.900)%
  --(11.894,4.900)--(11.900,4.900)--(11.906,4.901)--(11.912,4.901)--(11.918,4.902)--(11.924,4.902)%
  --(11.930,4.902)--(11.936,4.903)--(11.942,4.903)--(11.948,4.904)--(11.954,4.904)--(11.960,4.905)%
  --(11.966,4.905)--(11.972,4.905)--(11.978,4.906)--(11.984,4.906)--(11.990,4.907)--(11.996,4.907)%
  --(12.002,4.907)--(12.008,4.908)--(12.014,4.908)--(12.020,4.909)--(12.026,4.909)--(12.032,4.910)%
  --(12.038,4.910)--(12.044,4.910)--(12.050,4.911)--(12.056,4.911)--(12.062,4.912)--(12.068,4.912)%
  --(12.074,4.912)--(12.080,4.913)--(12.085,4.913)--(12.091,4.914)--(12.097,4.914)--(12.103,4.914)%
  --(12.109,4.915)--(12.115,4.915)--(12.121,4.916)--(12.127,4.916)--(12.133,4.917)--(12.139,4.917)%
  --(12.145,4.917)--(12.151,4.918)--(12.157,4.918)--(12.163,4.919)--(12.169,4.919)--(12.175,4.919)%
  --(12.181,4.920)--(12.187,4.920)--(12.193,4.921)--(12.199,4.921)--(12.205,4.922)--(12.211,4.922)%
  --(12.217,4.922)--(12.223,4.923)--(12.229,4.923)--(12.235,4.924)--(12.241,4.924)--(12.247,4.924)%
  --(12.253,4.925)--(12.259,4.925)--(12.265,4.926)--(12.271,4.926)--(12.277,4.927)--(12.283,4.927)%
  --(12.289,4.927)--(12.295,4.928)--(12.300,4.928)--(12.306,4.929)--(12.312,4.929)--(12.318,4.929)%
  --(12.324,4.930)--(12.330,4.930)--(12.336,4.931)--(12.342,4.931)--(12.348,4.931)--(12.354,4.932)%
  --(12.360,4.932)--(12.366,4.933)--(12.372,4.933)--(12.378,4.934)--(12.384,4.934)--(12.390,4.934)%
  --(12.396,4.935)--(12.402,4.935)--(12.408,4.936)--(12.414,4.936)--(12.420,4.936)--(12.426,4.937)%
  --(12.432,4.937)--(12.438,4.938)--(12.444,4.938)--(12.450,4.939)--(12.456,4.939)--(12.462,4.939)%
  --(12.468,4.940)--(12.474,4.940)--(12.480,4.941)--(12.486,4.941)--(12.492,4.941)--(12.498,4.942)%
  --(12.504,4.942)--(12.509,4.943)--(12.515,4.943)--(12.521,4.944)--(12.527,4.944)--(12.533,4.944)%
  --(12.539,4.945)--(12.545,4.945)--(12.551,4.946)--(12.557,4.946)--(12.563,4.946)--(12.569,4.947)%
  --(12.575,4.947)--(12.581,4.948)--(12.587,4.948)--(12.593,4.948)--(12.599,4.949)--(12.605,4.949)%
  --(12.611,4.950)--(12.617,4.950)--(12.623,4.951)--(12.629,4.951)--(12.635,4.951)--(12.641,4.952)%
  --(12.647,4.952)--(12.653,4.953)--(12.659,4.953)--(12.665,4.953)--(12.671,4.954);
\gpcolor{color=gp lt color border}
\gpsetlinewidth{1.00}
\draw[gp path] (1.504,8.631)--(1.504,0.985)--(13.447,0.985)--(13.447,8.631)--cycle;
%% coordinates of the plot area
\gpdefrectangularnode{gp plot 1}{\pgfpoint{1.504cm}{0.985cm}}{\pgfpoint{13.447cm}{8.631cm}}
\end{tikzpicture}
%% gnuplot variables

\caption{Conjunto de dados 3. A reta da regressão linear é dada por $y(x)=\np{0.019262681} x + \np{0.316038383}$, $r^2=\np{0,035985674}$.}
\label{RetasConjuntosDados3}
\end{figure}

\vfill
\pagebreak
%%%%%%%%%%%%%%%%%%%%%%%%%%%%%%%%%%%%%%%%%%%%%%%%%%%%%%%%%%%%%%%%%
\paragraph{Exemplo: Cálculo dos coeficientes da regressão linear}
%%%%%%%%%%%%%%%%%%%%%%%%%%%%%%%%%%%%%%%%%%%%%%%%%%%%%%%%%%%%%%%%%

Para determinarmos os valores das constantes $A$, $B$, e do coeficiente $r^2$ é necessário calcularmos uma série de valores intermediários. Na Tabela~\ref{Tab:TabelaExemploCalculoRegressaoLinear} (página~\pageref{Tab:TabelaExemploCalculoRegressaoLinear}) apresentamos tais valores intermediários para o caso do cálculo dos coeficientes da regressão linear para a variável $y_1$ discutida acima. Através desses valores, obtemos para o coeficiente linear
\begin{align}
    A &= \frac{\sum x_i^2 \sum y_i - \sum x_i \sum x_iy_i}{N \sum x_i^2 - (\sum x_i)^2} \\
    &= \frac{\np{7295.757} \cdot \np{1512.768} - \np{392.984} \cdot \np{26545.051}}{\np{28}\cdot\np{7295.757} - \np{392.984}^2} \\
    &= \np{12.14046264}.
\end{align}
%
Para o coeficiente angular, obtemos
\begin{align}
    B &= \frac{N\sum x_iy_i - \sum x_i \sum y_i}{N \sum x_i^2 - (\sum x_i)^2} \\
    &= \frac{\np{28} \cdot \np{26545.051} - \np{392.984}\cdot\np{1512.786}}{\np{28}\cdot\np{7295.757} - \np{392.984}^2} \\
    &= \np{2.984480401}.
\end{align}
%
Finalmente, para o coeficiente $r$ obtemos
\begin{align}
    r &= \frac{\sum (x_i - \mean{x})(y_i-\mean{y})}{\sqrt{\sum(x_i - \mean{x})^2\sum (y_i - \mean{y})^2}} \\
    &= \frac{\np{5312.883}}{\sqrt{\np{1780.170}\cdot\sum{15873.637}}} \\
    &= \np{0.999450477},
\end{align}
%
o que resulta em um $r^2$ dado por
\begin{equation}
    r^2 = \np{0.998901256}.
\end{equation}

Note que o cálculo é bastante trabalhoso se for realizado manualmente. Felizmente, esse tipo de análise ---~a regressão linear~--- é algo que é implementado na maioria dos programas de computador que permitem a análise de dados. Mesmo em planilhas de cálculo existem funções que permitem calcular as constantes $A$ e $B$, e o coeficiente $r^2$ facilmente. Funções para a determinação de tais valores também estão disponíveis em calculadoras científicas. Na seção seguinte verificaremos como proceder para inserir os dados experimentais e obter os coeficientes da regressão linear para alguns modelos comuns de calculadoras. 

\begin{table*}[tp]
\caption[][1cm]{Tabela de cálculos para a determinação das constantes $A$ e $B$, e do coeficiente $r^2$. As linhas marcadas com $\sum$ e $\mean{~}$ denotam as somas e as médias dos valores apresentados para cada variável, respectivamente. Os valores apresentados na tabela foram limitados a três casas após a vírgula, porém os cálculos foram feitos com toda a precisão disponível. \label{Tab:TabelaExemploCalculoRegressaoLinear}}
\small
\begin{tabular}{ccccccccccc}
\toprule
$i$ & $x_i$ & $y_{1,i}$ & $x_i^2$ & $x_i y_i$ & $x_i - \mean{x}$ & $y_i - \mean{y}$ & $(x_i - \mean{x})^2$ & $(y_i - \mean{y})^2$ & $(x_i - \mean{x})(y_i - \mean{y})$ \\
\midrule
1	& \np{0.714} 	& \np{14.577}	& \np{0.510}	& \np{10.408}	& \np{-13.321}	& \np{-39.451}	& \np{177.453}	& \np{1556.387}	& \np{525.533} \\
2	& \np{2.693}	& \np{20.696}	& \np{7.252}	& \np{55.734}	& \np{-11.342}	& \np{-33.332}	& \np{128.644}	& \np{1111.027}	& \np{378.057} \\
3	& \np{4.389}	& \np{25.226}	& \np{19.263}	& \np{110.717}	& \np{-9.646}	& \np{-28.802}	& \np{93.048}	& \np{829.559}	& \np{277.828} \\
4	& \np{4.960}	& \np{27.449}	& \np{24.602}	& \np{136.147}	& \np{-9.075}	& \np{-26.579}	& \np{82.358}	& \np{706.447}	& \np{241.208} \\
5	& \np{6.245}	& \np{30.242}	& \np{39.000}	& \np{188.861}	& \np{-7.790}	& \np{-23.786}	& \np{60.686}	& \np{565.777}	& \np{185.296} \\
6	& \np{7.277}	& \np{33.378}	& \np{52.955}	& \np{242.892}	& \np{-6.758}	& \np{-20.650}	& \np{45.672}	& \np{426.425}	& \np{139.556} \\
7	& \np{7.579}	& \np{34.195}	& \np{57.441}	& \np{259.164}	& \np{-6.456}	& \np{-19.833}	& \np{41.682}	& \np{393.351}	& \np{128.045} \\
8	& \np{7.719}	& \np{35.715}	& \np{59.583}	& \np{275.684}	& \np{-6.316}	& \np{-18.313}	& \np{39.894}	& \np{335.369}	& \np{115.667} \\
9	& \np{7.912}	& \np{35.011}	& \np{62.600}	& \np{277.007}	& \np{-6.123}	& \np{-19.017}	& \np{37.493}	& \np{361.649}	& \np{116.444} \\
10	& \np{8.280}	& \np{37.529}	& \np{68.558}	& \np{310.740}	& \np{-5.755}	& \np{-16.499}	& \np{33.122}	& \np{272.219}	& \np{94.955} \\
11	& \np{9.034}	& \np{40.590}	& \np{81.613}	& \np{366.690}	& \np{-5.001}	& \np{-13.438}	& \np{25.0114}	& \np{180.582}	& \np{67.206} \\
12	& \np{9.442}	& \np{39.156}	& \np{89.151}	& \np{369.711}	& \np{-4.593}	& \np{-14.872}	& \np{21.097}	& \np{221.179}	& \np{68.310} \\
13	& \np{10.306}	& \np{43.238}	& \np{106.214}	& \np{445.611}	& \np{-3.729}	& \np{-10.790}	& \np{13.907}	& \np{116.426}	& \np{40.238} \\
14	& \np{10.572}	& \np{42.406}	& \np{111.767}	& \np{448.316}	& \np{-3.463}	& \np{-11.622}	& \np{11.993}	& \np{135.073}	& \np{40.249} \\
15	& \np{11.177}	& \np{44.796}	& \np{124.925}	& \np{500.685}	& \np{-2.858}	& \np{-9.232}	& \np{8.169}	& \np{85.231}	& \np{26.387} \\
16	& \np{15.335}	& \np{57.611}	& \np{235.162}	& \np{883.465}	& \np{1.300}	& \np{3.583}	& \np{1.690}	& \np{12.837}	& \np{4.657} \\
17	& \np{17.023}	& \np{63.832}	& \np{289.783}	& \np{1086.612}	& \np{2.988}	& \np{9.804}	& \np{8.927}	& \np{96.117}	& \np{29.293} \\
18	& \np{18.926}	& \np{68.063}	& \np{358.193}	& \np{1288.160}	& \np{4.891}	& \np{14.035}	& \np{23.920}	& \np{196.979}	& \np{68.643} \\
19	& \np{20.608}	& \np{74.408}	& \np{424.690}	& \np{1533.400}	& \np{6.573}	& \np{20.380}	& \np{43.202}	& \np{415.341}	& \np{133.954} \\
20	& \np{20.876}	& \np{75.083}	& \np{435.807}	& \np{1567.433}	& \np{6.841}	& \np{21.055}	& \np{46.797}	& \np{443.310}	& \np{144.033} \\
21	& \np{21.095}	& \np{75.248}	& \np{444.999}	& \np{1587.357}	& \np{7.060}	& \np{21.220}	& \np{49.842}	& \np{450.285}	& \np{149.809} \\
22	& \np{22.225}	& \np{77.243}	& \np{493.951}	& \np{1716.726}	& \np{8.190}	& \np{23.215}	& \np{67.074}	& \np{538.933}	& \np{190.126} \\
23	& \np{22.407}	& \np{81.058}	& \np{502.074}	& \np{1816.267}	& \np{8.372}	& \np{27.030}	& \np{70.088}	& \np{730.617}	& \np{226.290} \\
24	& \np{22.469}	& \np{78.821}	& \np{504.856}	& \np{1771.029}	& \np{8.434}	& \np{24.793}	& \np{71.130}	& \np{614.689}	& \np{209.100} \\
25	& \np{23.077}	& \np{80.714}	& \np{532.548}	& \np{1862.637}	& \np{9.042}	& \np{26.686}	& \np{81.755}	& \np{712.139}	& \np{241.290} \\
26	& \np{26.421}	& \np{91.433}	& \np{698.069}	& \np{2415.751}	& \np{12.386}	& \np{37.405}	& \np{153.409}  & \np{1399.129}	& \np{463.292} \\
27	& \np{26.863}	& \np{91.777}	& \np{721.621}	& \np{2465.406}	& \np{12.828}	& \np{37.749}	& \np{164.554}  & \np{1424.982}	& \np{484.237} \\
28	& \np{27.360}	& \np{93.291}	& \np{748.570}	& \np{2552.442}	& \np{13.325}	& \np{39.263}	& \np{177.552}  & \np{1541.578}	& \np{523.172} \\
\midrule
$\sum$	& \np{392.984}	& \np{1512.786}	& \np{7295.757}	& \np{26545.051}	& 0	& 0	& \np{1780.170}	& \np{15873.637}	& \np{5312.883} \\
$\mean{~}$ & \np{14.035}	& \np{54.028} \\				
\bottomrule
\end{tabular}
\end{table*}

\FloatBarrier

%%%%%%%%%%%%%%%%%%%%%%%%%%%%%%%%%%%%%%%%%%%%%%%%%%%%%%%%%%%%%%%%%%%%%%
\subsection{Regressão linear utilizando uma calculadora}
%%%%%%%%%%%%%%%%%%%%%%%%%%%%%%%%%%%%%%%%%%%%%%%%%%%%%%%%%%%%%%%%%%%%%%

Geralmente calculadoras científicas são capazes de realizar regressões lineares e diversas outras, facilitando a obtenção da melhor reta. Apresentaremos abaixo como realizar tal cálculo em alguns modelos.

%%%%%%%%%%%%%%%%%%%%%%%%%%%
\paragraph{CASIO$^{\circledR}$ $fx$-\textit{82TL}}
%%%%%%%%%%%%%%%%%%%%%%%%%%%

\begin{itemize}
\item Pressione o botão \keystroke{~MODE~}. As informações abaixo aparecerão no visor:
\begin{center}
\begin{tabular}{p{20mm}p{20mm}p{20mm}}
COMP & SD & REG\\
1 & 2 & 3
\end{tabular}
\end{center}

\item Pressione o botão \keystroke{~3~}. Teremos no visor
\begin{center}
\begin{tabular}{p{20mm}p{20mm}p{20mm}}
Lin & Log & Exp \phantom{xxx}\ding{225} \\
1 & 2 & 3
\end{tabular}
\end{center}
Ao pressionarmos o botão \keystroke{~1~}, a calculadora estará no modo de regressão linear, indicado por \texttt{REG} no visor.

\item Podemos agora digitar o valor da variável independente $x_1$, correspondente ao primeiro ponto, seguido do botão \keystroke{~,~}. Digitamos após a vírgula o valor da variável dependente $y_1$ correspondente ao primeiro ponto. Após isso, basta pressionar \keystroke{~M+~} para inserir o par de valores na memória da calculadora. Repetiremos esse processo para cada par $x_i$, $y_i$.

\item Quando todos os valores tiverem sido inseridos, podemos recuperar os valores de $A$, $B$ e $r$ pressionando \keystroke{~SHIFT~} seguido de 
\begin{itemize}
	\item tecla \keystroke{7} e então \keystroke{=} para recuperar o valor de $A$;
	\item tecla \keystroke{8} e então \keystroke{=} para recuperar o valor de $B$;
	\item tecla \keystroke{(} e então \keystroke{=} para recuperar o valor de $r$.
\end{itemize}

Para recuperar outro valor, não é necessário inserir os pontos novamente, basta pressionar mais uma vez \keystroke{~SHIFT~} seguido da tecla correspondente à variável que desejamos.

\item Para realizar uma nova regressão, devemos antes apagar os dados da regressão anterior. Fazemos isso pressionando \keystroke{~SHIFT~} e então \keystroke{~AC/ON~} (função \texttt{Scl}). Na tela aparecerá \texttt{Scl}. Pressione \keystroke{=} para confirmar a exclusão dos dados.
\end{itemize}

%%%%%%%%%%%%%%%%%%%%%%%%%%%
\paragraph{CASIO$^{\circledR}$ $fx$-\textit{82MS}}
%%%%%%%%%%%%%%%%%%%%%%%%%%%

\begin{itemize}
\item Pressione o botão \keystroke{~MODE~}. As informações abaixo aparecerão no visor:
\begin{center}
\begin{tabular}{p{20mm}p{20mm}p{20mm}}
COMP & SD & REG\\
1 & 2 & 3
\end{tabular}
\end{center}

\item Pressione o botão \keystroke{~3~}. Teremos no visor
\begin{center}
\begin{tabular}{p{20mm}p{20mm}p{20mm}}
Lin & Log & Exp \phantom{xxx}\ding{225} \\
1 & 2 & 3
\end{tabular}
\end{center}
Ao pressionarmos o botão \keystroke{~1~}, a calculadora estará no modo de regressão linear, indicado por \texttt{REG} no visor.

\item Podemos agora digitar o valor da variável independente $x_1$, correspondente ao primeiro ponto, seguido do botão \keystroke{~,~}. Digitamos após a vírgula o valor da variável dependente $y_1$ correspondente ao primeiro ponto. Após isso, basta pressionar \keystroke{~M+~} para inserir o par de valores na memória da calculadora. Repetiremos esse processo para cada par $x_i$, $y_i$.

\item Quando todos os valores tiverem sido inseridos, podemos recuperar os valores de $A$, $B$ e $r$ pressionando \keystroke{~SHIFT~} seguido do botão \keystroke{~2~} (função \texttt{S-var}). Neste momento, aparecerá na tela
\begin{center}
\begin{tabular}{p{20mm}p{20mm}p{20mm}}
$\overline{x}$ & $x\sigma n$ & $x\sigma n-1$ \phantom{xx}\ding{225} \\
1 & 2 & 3
\end{tabular}
\end{center}
%
Se pressionarmos para a direita no botão direcional duas vezes, na tela teremos
\begin{center}
\begin{tabular}{p{20mm}p{20mm}p{20mm}}
A & B & r \\
1 & 2 & 3
\end{tabular}
\end{center}
%
Basta agora escolher qual variável desejamos, pressionar o botão correspondente ---~\keystroke{~1~}, \keystroke{~2~} ou \keystroke{~3~}~--- e pressionar \keystroke{~=~}. Para recuperar outro valor, não é necessário inserir os pontos novamente, basta pressionar mais uma vez \keystroke{~SHIFT~} seguido de \keystroke{~2~} (função \texttt{S-var}) e escolher outra variável.

\item Para realizar uma nova regressão, devemos antes apagar os dados da regressão anterior. Fazemos isso pressionando \keystroke{~SHIFT~} e então \keystroke{~MODE~} (função \texttt{CLR}). Na tela aparecerá
\begin{center}
\begin{tabular}{p{2cm}p{2cm}p{2cm}}
Scl & Mode & All \\
1 & 2 & 3
\end{tabular}
\end{center}
%
Selecione \texttt{Scl} pressionando o botão \keystroke{~1~} e pressione \keystroke{~=~}.
\end{itemize}

%%%%%%%%%%%%%%%%%%%%%%%%%%%
\paragraph{CASIO$^{\circledR}$ $fx$-\textit{570ES} e $fx$-\textit{991ES Plus}}
%%%%%%%%%%%%%%%%%%%%%%%%%%%

\begin{itemize}
	\item Ao pressionar a tecla \keystroke{~MODE~}, aparecerão diversas opções, entre elas a opção `\texttt{3:~STAT}'. Pressione a tecla \keystroke{~3~} para selecioná-la. Uma nova tela aparecerá com opções.
	\item Selecione a opção `\texttt{2:~A+Bx}' pressionando a tecla \keystroke{~2~}. Ao selecionar esta opção, uma tabela surgirá para a entrada de dados.
	\item Coloque o valor da abscissa $x_1$ da primeira medida e pressione \keystroke{~=~}. A calculadora passará ao próximo valor de abscissa automaticamente, o que torna mais fácil a entrada de todos os valores para essa variável. Ao finalizar, navegue usando as teclas direcionais até o local de inserção da primeira ordenada. Insira o valor e tecle \keystroke{~=~}. A calculadora passará automaticamente ao próximo campo, permitindo a inserção do próximo valor de ordenada. Como a calculadora tem o comportamento automático de passar ao próximo valor da mesma variável, é mais fácil segui-lo, porém podemos inserir os pares $x,y$ se utilizarmos as teclas direcionais após pressionar \keystroke{~=~}. Qualquer erro pode ser corrigido navegando novamente ao campo e reinserindo os valores.
	\item Após inserir todos os valores, pressione a tecla \keystroke{~AC~}.
	\item Para recuperar os valores de $A$, $B$, e $r$, devemos pressionar a tecla \keystroke{~SHIFT~}, seguida da tecla \keystroke{~1~} (função \texttt{STAT}). Uma nova tela de opções surgirá.
	\item No modelo $fx$-\textit{570ES} devemos selecionar a opção `\texttt{7:~Reg}' pressionando a tecla \keystroke{~7~}. No modelo $fx$-\textit{991ES Plus}, devemos pressionar a tecla \keystroke{~5~} (opção `\texttt{5:~Reg}`). Uma tela surgirá com as opções correspondentes às variáveis $A$, $B$ e $r$. Selecione a variável que deseja recuperar através das teclas numéricas correspondentes e então pressione \keystroke{~=~}. Para recuperar outra variável, basta repetir os passos desse ítem, não sendo necessário digitar novamente os dados experimentais.
	\item Para realizar uma nova regressão, basta repetirmos os passos desde o início. Quando a tabela de inserção de dados surgir, ela estará vazia.
\end{itemize}

%%%%%%%%%%%%%%%%%%%%%%%%%%%%%%%%%%%%%%%%%%%
\subsection{Interpretação dos coeficientes}
%%%%%%%%%%%%%%%%%%%%%%%%%%%%%%%%%%%%%%%%%%%

A ideia por trás do cálculo da melhor reta é estabelecer quais seriam os coeficientes mais adequados para uma relação linear que descreve o fenômeno estudado. \emph{Esses coeficientes são importantes pois estão, em geral, ligados a constantes físicas cujo valor estamos interessados em medir}. Além disso, esse processo é mais preciso do que simplesmente calcular o valor dos coeficientes da reta associados aos pontos medidos e depois fazer uma média.

\begin{margintable}\centering
\begin{tabular}{cc}
\toprule
$(t \pm \np{0,01})~\textrm{s}$ & $(v \pm \np{0,01})~\textrm{m}/\textrm{s}$ \\
\midrule
\np{0,10} & \np{24,04} \\
\np{0,20} & \np{25,14} \\
\np{0,30} & \np{25,79} \\
\np{0,40} & \np{27,08} \\
\np{0,50} & \np{27,33} \\
\np{0,60} & \np{28,79} \\
\np{0,70} & \np{29,98} \\
\np{0,80} & \np{30,61} \\
\np{0,90} & \np{31,15} \\
\np{1,00} & \np{32,94} \\
\np{1,10} & \np{34,04} \\
\np{1,20} & \np{34,78} \\
\np{1,30} & \np{35,22} \\
\np{1,40} & \np{36,10} \\
\np{1,50} & \np{37,91} \\
\bottomrule
\end{tabular}
\vspace{1mm}
\caption{Dados medidos para a velocidade em função do tempo para um experimento hipotético.}
\label{DadosVelocidade}
\end{margintable}

Vamos considerar, por exemplo, o conjunto de dados para a velocidade em função do tempo dados na Tabela~\ref{DadosVelocidade}, representados na Figura~\ref{fig:VelocidadeTempo}. Verificamos nos dados da tabela que a velocidade se altera com o tempo. Supondo que tenhamos um movimento com aceleração constante, podemos descrever os dados em função do tempo como
\begin{equation}
	v = v_0 + at.
\end{equation}
%
Comparando esta equação com a equação da reta $y = A + Bx$, verificamos as relações
\begin{align}
	y &= v \\
	A &= v_0 \\
	B &= a \\
	x &= t.
\end{align}
%
Para determinar a relação entre as variáveis das duas equações, devemos verificar qual variável dos dados foi varrida arbitrariamente ---~e que então deve corresponder à variável independente $x$~--- e qual foi lida em resposta à primeira ---~correspondendo à variável dependente $y$~---. Vemos na Tabela~\ref{DadosVelocidade} que os valores de tempo são mais condizentes com uma variação arbitrária (valores ``redondos'' ou semi-inteiros, isto é, variados com um passo regular) do que no caso dos valores da velocidade. No caso de realizarmos um experimento, não teremos problemas em determinar quem foi a variável independente, pois realizaremos essa escolha ao idealizá-lo.

Dessa forma, se tomarmos os dados e realizarmos uma regressão linear, vamos obter os valores de $v_0$ e de $a$ ---~que são desconhecidos~--- através das constantes $A$ e $B$. Para os valores da tabela obtemos $v_0 = \np{22,734 285 714 3}$ e $a = \np{9,907 142 857 1}$. Verificaremos adiante que nem todos os digitos obtidos para as constantes $A$ e $B$ são relevantes e por isso deveremos descartar alguns deles. Por ora podemos expressá-las com o mesmo número de casas após a vírgula que a variável ($x$ ou $y$) que tem menos casas após a vírgula:
\begin{align}
	A = v_0 &= \np[m/s]{22,73} \\
	B = a &= \np[m/s^2]{9,91}.
\end{align}

\begin{figure*}[!htbt]
\centering
\caption{Gráfico dos dados da Tabela~\ref{DadosVelocidade}.} 
\label{fig:VelocidadeTempo}
\begin{tikzpicture}[gnuplot]
%% generated with GNUPLOT 5.0p0 (Lua 5.3; terminal rev. 99, script rev. 100)
%% 2015-05-18T23:01:05 BRT
\path (0.000,0.000) rectangle (14.000,9.000);
\gpcolor{color=gp lt color border}
\gpsetlinetype{gp lt border}
\gpsetdashtype{gp dt solid}
\gpsetlinewidth{1.00}
\draw[gp path] (1.136,0.985)--(1.316,0.985);
\draw[gp path] (13.447,0.985)--(13.267,0.985);
\node[gp node right] at (0.952,0.985) {$20$};
\draw[gp path] (1.136,2.723)--(1.316,2.723);
\draw[gp path] (13.447,2.723)--(13.267,2.723);
\node[gp node right] at (0.952,2.723) {$25$};
\draw[gp path] (1.136,4.460)--(1.316,4.460);
\draw[gp path] (13.447,4.460)--(13.267,4.460);
\node[gp node right] at (0.952,4.460) {$30$};
\draw[gp path] (1.136,6.198)--(1.316,6.198);
\draw[gp path] (13.447,6.198)--(13.267,6.198);
\node[gp node right] at (0.952,6.198) {$35$};
\draw[gp path] (1.136,7.936)--(1.316,7.936);
\draw[gp path] (13.447,7.936)--(13.267,7.936);
\node[gp node right] at (0.952,7.936) {$40$};
\draw[gp path] (1.136,0.985)--(1.136,1.165);
\draw[gp path] (1.136,8.631)--(1.136,8.451);
\node[gp node center] at (1.136,0.677) {$0$};
\draw[gp path] (2.675,0.985)--(2.675,1.165);
\draw[gp path] (2.675,8.631)--(2.675,8.451);
\node[gp node center] at (2.675,0.677) {$0.2$};
\draw[gp path] (4.214,0.985)--(4.214,1.165);
\draw[gp path] (4.214,8.631)--(4.214,8.451);
\node[gp node center] at (4.214,0.677) {$0.4$};
\draw[gp path] (5.753,0.985)--(5.753,1.165);
\draw[gp path] (5.753,8.631)--(5.753,8.451);
\node[gp node center] at (5.753,0.677) {$0.6$};
\draw[gp path] (7.292,0.985)--(7.292,1.165);
\draw[gp path] (7.292,8.631)--(7.292,8.451);
\node[gp node center] at (7.292,0.677) {$0.8$};
\draw[gp path] (8.830,0.985)--(8.830,1.165);
\draw[gp path] (8.830,8.631)--(8.830,8.451);
\node[gp node center] at (8.830,0.677) {$1$};
\draw[gp path] (10.369,0.985)--(10.369,1.165);
\draw[gp path] (10.369,8.631)--(10.369,8.451);
\node[gp node center] at (10.369,0.677) {$1.2$};
\draw[gp path] (11.908,0.985)--(11.908,1.165);
\draw[gp path] (11.908,8.631)--(11.908,8.451);
\node[gp node center] at (11.908,0.677) {$1.4$};
\draw[gp path] (13.447,0.985)--(13.447,1.165);
\draw[gp path] (13.447,8.631)--(13.447,8.451);
\node[gp node center] at (13.447,0.677) {$1.6$};
\draw[gp path] (1.136,8.631)--(1.136,0.985)--(13.447,0.985)--(13.447,8.631)--cycle;
\node[gp node center,rotate=-270] at (0.246,4.808) {$v$~(m/s)};
\node[gp node center] at (7.291,0.215) {$t$~(s)};
\node[gp node left] at (2.604,8.297) {Dados experimentais};
\gpcolor{rgb color={0.000,0.000,0.000}}
\gpsetlinewidth{2.00}
\gpsetpointsize{4.00}
\gppoint{gp mark 7}{(1.905,2.042)}
\gppoint{gp mark 7}{(2.675,2.771)}
\gppoint{gp mark 7}{(3.444,2.997)}
\gppoint{gp mark 7}{(4.214,3.446)}
\gppoint{gp mark 7}{(4.983,3.533)}
\gppoint{gp mark 7}{(5.753,4.040)}
\gppoint{gp mark 7}{(6.522,4.454)}
\gppoint{gp mark 7}{(7.292,4.672)}
\gppoint{gp mark 7}{(8.061,4.860)}
\gppoint{gp mark 7}{(8.830,5.482)}
\gppoint{gp mark 7}{(9.600,5.865)}
\gppoint{gp mark 7}{(10.369,6.122)}
\gppoint{gp mark 7}{(11.139,6.275)}
\gppoint{gp mark 7}{(11.908,6.580)}
\gppoint{gp mark 7}{(12.678,7.210)}
\gppoint{gp mark 7}{(1.962,8.297)}
\gpcolor{color=gp lt color border}
\node[gp node left] at (2.604,7.989) {$y(x)=\np{9.9071428571}\;x + \np{22.7342857143}$, $r^2 = \np{0.9924824357}$};
\gpcolor{rgb color={0.000,0.000,0.000}}
\draw[gp path] (1.504,7.989)--(2.420,7.989);
\draw[gp path] (1.905,2.280)--(1.911,2.282)--(1.916,2.284)--(1.922,2.287)--(1.927,2.289)%
  --(1.932,2.292)--(1.938,2.294)--(1.943,2.296)--(1.949,2.299)--(1.954,2.301)--(1.959,2.304)%
  --(1.965,2.306)--(1.970,2.309)--(1.975,2.311)--(1.981,2.313)--(1.986,2.316)--(1.992,2.318)%
  --(1.997,2.321)--(2.002,2.323)--(2.008,2.325)--(2.013,2.328)--(2.019,2.330)--(2.024,2.333)%
  --(2.029,2.335)--(2.035,2.337)--(2.040,2.340)--(2.045,2.342)--(2.051,2.345)--(2.056,2.347)%
  --(2.062,2.350)--(2.067,2.352)--(2.072,2.354)--(2.078,2.357)--(2.083,2.359)--(2.089,2.362)%
  --(2.094,2.364)--(2.099,2.366)--(2.105,2.369)--(2.110,2.371)--(2.115,2.374)--(2.121,2.376)%
  --(2.126,2.378)--(2.132,2.381)--(2.137,2.383)--(2.142,2.386)--(2.148,2.388)--(2.153,2.390)%
  --(2.159,2.393)--(2.164,2.395)--(2.169,2.398)--(2.175,2.400)--(2.180,2.403)--(2.186,2.405)%
  --(2.191,2.407)--(2.196,2.410)--(2.202,2.412)--(2.207,2.415)--(2.212,2.417)--(2.218,2.419)%
  --(2.223,2.422)--(2.229,2.424)--(2.234,2.427)--(2.239,2.429)--(2.245,2.431)--(2.250,2.434)%
  --(2.256,2.436)--(2.261,2.439)--(2.266,2.441)--(2.272,2.444)--(2.277,2.446)--(2.282,2.448)%
  --(2.288,2.451)--(2.293,2.453)--(2.299,2.456)--(2.304,2.458)--(2.309,2.460)--(2.315,2.463)%
  --(2.320,2.465)--(2.326,2.468)--(2.331,2.470)--(2.336,2.472)--(2.342,2.475)--(2.347,2.477)%
  --(2.352,2.480)--(2.358,2.482)--(2.363,2.484)--(2.369,2.487)--(2.374,2.489)--(2.379,2.492)%
  --(2.385,2.494)--(2.390,2.497)--(2.396,2.499)--(2.401,2.501)--(2.406,2.504)--(2.412,2.506)%
  --(2.417,2.509)--(2.422,2.511)--(2.428,2.513)--(2.433,2.516)--(2.439,2.518)--(2.444,2.521)%
  --(2.449,2.523)--(2.455,2.525)--(2.460,2.528)--(2.466,2.530)--(2.471,2.533)--(2.476,2.535)%
  --(2.482,2.538)--(2.487,2.540)--(2.493,2.542)--(2.498,2.545)--(2.503,2.547)--(2.509,2.550)%
  --(2.514,2.552)--(2.519,2.554)--(2.525,2.557)--(2.530,2.559)--(2.536,2.562)--(2.541,2.564)%
  --(2.546,2.566)--(2.552,2.569)--(2.557,2.571)--(2.563,2.574)--(2.568,2.576)--(2.573,2.578)%
  --(2.579,2.581)--(2.584,2.583)--(2.589,2.586)--(2.595,2.588)--(2.600,2.591)--(2.606,2.593)%
  --(2.611,2.595)--(2.616,2.598)--(2.622,2.600)--(2.627,2.603)--(2.633,2.605)--(2.638,2.607)%
  --(2.643,2.610)--(2.649,2.612)--(2.654,2.615)--(2.659,2.617)--(2.665,2.619)--(2.670,2.622)%
  --(2.676,2.624)--(2.681,2.627)--(2.686,2.629)--(2.692,2.632)--(2.697,2.634)--(2.703,2.636)%
  --(2.708,2.639)--(2.713,2.641)--(2.719,2.644)--(2.724,2.646)--(2.730,2.648)--(2.735,2.651)%
  --(2.740,2.653)--(2.746,2.656)--(2.751,2.658)--(2.756,2.660)--(2.762,2.663)--(2.767,2.665)%
  --(2.773,2.668)--(2.778,2.670)--(2.783,2.672)--(2.789,2.675)--(2.794,2.677)--(2.800,2.680)%
  --(2.805,2.682)--(2.810,2.685)--(2.816,2.687)--(2.821,2.689)--(2.826,2.692)--(2.832,2.694)%
  --(2.837,2.697)--(2.843,2.699)--(2.848,2.701)--(2.853,2.704)--(2.859,2.706)--(2.864,2.709)%
  --(2.870,2.711)--(2.875,2.713)--(2.880,2.716)--(2.886,2.718)--(2.891,2.721)--(2.896,2.723)%
  --(2.902,2.725)--(2.907,2.728)--(2.913,2.730)--(2.918,2.733)--(2.923,2.735)--(2.929,2.738)%
  --(2.934,2.740)--(2.940,2.742)--(2.945,2.745)--(2.950,2.747)--(2.956,2.750)--(2.961,2.752)%
  --(2.966,2.754)--(2.972,2.757)--(2.977,2.759)--(2.983,2.762)--(2.988,2.764)--(2.993,2.766)%
  --(2.999,2.769)--(3.004,2.771)--(3.010,2.774)--(3.015,2.776)--(3.020,2.779)--(3.026,2.781)%
  --(3.031,2.783)--(3.037,2.786)--(3.042,2.788)--(3.047,2.791)--(3.053,2.793)--(3.058,2.795)%
  --(3.063,2.798)--(3.069,2.800)--(3.074,2.803)--(3.080,2.805)--(3.085,2.807)--(3.090,2.810)%
  --(3.096,2.812)--(3.101,2.815)--(3.107,2.817)--(3.112,2.819)--(3.117,2.822)--(3.123,2.824)%
  --(3.128,2.827)--(3.133,2.829)--(3.139,2.832)--(3.144,2.834)--(3.150,2.836)--(3.155,2.839)%
  --(3.160,2.841)--(3.166,2.844)--(3.171,2.846)--(3.177,2.848)--(3.182,2.851)--(3.187,2.853)%
  --(3.193,2.856)--(3.198,2.858)--(3.203,2.860)--(3.209,2.863)--(3.214,2.865)--(3.220,2.868)%
  --(3.225,2.870)--(3.230,2.873)--(3.236,2.875)--(3.241,2.877)--(3.247,2.880)--(3.252,2.882)%
  --(3.257,2.885)--(3.263,2.887)--(3.268,2.889)--(3.273,2.892)--(3.279,2.894)--(3.284,2.897)%
  --(3.290,2.899)--(3.295,2.901)--(3.300,2.904)--(3.306,2.906)--(3.311,2.909)--(3.317,2.911)%
  --(3.322,2.913)--(3.327,2.916)--(3.333,2.918)--(3.338,2.921)--(3.344,2.923)--(3.349,2.926)%
  --(3.354,2.928)--(3.360,2.930)--(3.365,2.933)--(3.370,2.935)--(3.376,2.938)--(3.381,2.940)%
  --(3.387,2.942)--(3.392,2.945)--(3.397,2.947)--(3.403,2.950)--(3.408,2.952)--(3.414,2.954)%
  --(3.419,2.957)--(3.424,2.959)--(3.430,2.962)--(3.435,2.964)--(3.440,2.967)--(3.446,2.969)%
  --(3.451,2.971)--(3.457,2.974)--(3.462,2.976)--(3.467,2.979)--(3.473,2.981)--(3.478,2.983)%
  --(3.484,2.986)--(3.489,2.988)--(3.494,2.991)--(3.500,2.993)--(3.505,2.995)--(3.510,2.998)%
  --(3.516,3.000)--(3.521,3.003)--(3.527,3.005)--(3.532,3.007)--(3.537,3.010)--(3.543,3.012)%
  --(3.548,3.015)--(3.554,3.017)--(3.559,3.020)--(3.564,3.022)--(3.570,3.024)--(3.575,3.027)%
  --(3.581,3.029)--(3.586,3.032)--(3.591,3.034)--(3.597,3.036)--(3.602,3.039)--(3.607,3.041)%
  --(3.613,3.044)--(3.618,3.046)--(3.624,3.048)--(3.629,3.051)--(3.634,3.053)--(3.640,3.056)%
  --(3.645,3.058)--(3.651,3.061)--(3.656,3.063)--(3.661,3.065)--(3.667,3.068)--(3.672,3.070)%
  --(3.677,3.073)--(3.683,3.075)--(3.688,3.077)--(3.694,3.080)--(3.699,3.082)--(3.704,3.085)%
  --(3.710,3.087)--(3.715,3.089)--(3.721,3.092)--(3.726,3.094)--(3.731,3.097)--(3.737,3.099)%
  --(3.742,3.101)--(3.747,3.104)--(3.753,3.106)--(3.758,3.109)--(3.764,3.111)--(3.769,3.114)%
  --(3.774,3.116)--(3.780,3.118)--(3.785,3.121)--(3.791,3.123)--(3.796,3.126)--(3.801,3.128)%
  --(3.807,3.130)--(3.812,3.133)--(3.817,3.135)--(3.823,3.138)--(3.828,3.140)--(3.834,3.142)%
  --(3.839,3.145)--(3.844,3.147)--(3.850,3.150)--(3.855,3.152)--(3.861,3.155)--(3.866,3.157)%
  --(3.871,3.159)--(3.877,3.162)--(3.882,3.164)--(3.888,3.167)--(3.893,3.169)--(3.898,3.171)%
  --(3.904,3.174)--(3.909,3.176)--(3.914,3.179)--(3.920,3.181)--(3.925,3.183)--(3.931,3.186)%
  --(3.936,3.188)--(3.941,3.191)--(3.947,3.193)--(3.952,3.195)--(3.958,3.198)--(3.963,3.200)%
  --(3.968,3.203)--(3.974,3.205)--(3.979,3.208)--(3.984,3.210)--(3.990,3.212)--(3.995,3.215)%
  --(4.001,3.217)--(4.006,3.220)--(4.011,3.222)--(4.017,3.224)--(4.022,3.227)--(4.028,3.229)%
  --(4.033,3.232)--(4.038,3.234)--(4.044,3.236)--(4.049,3.239)--(4.054,3.241)--(4.060,3.244)%
  --(4.065,3.246)--(4.071,3.249)--(4.076,3.251)--(4.081,3.253)--(4.087,3.256)--(4.092,3.258)%
  --(4.098,3.261)--(4.103,3.263)--(4.108,3.265)--(4.114,3.268)--(4.119,3.270)--(4.124,3.273)%
  --(4.130,3.275)--(4.135,3.277)--(4.141,3.280)--(4.146,3.282)--(4.151,3.285)--(4.157,3.287)%
  --(4.162,3.289)--(4.168,3.292)--(4.173,3.294)--(4.178,3.297)--(4.184,3.299)--(4.189,3.302)%
  --(4.195,3.304)--(4.200,3.306)--(4.205,3.309)--(4.211,3.311)--(4.216,3.314)--(4.221,3.316)%
  --(4.227,3.318)--(4.232,3.321)--(4.238,3.323)--(4.243,3.326)--(4.248,3.328)--(4.254,3.330)%
  --(4.259,3.333)--(4.265,3.335)--(4.270,3.338)--(4.275,3.340)--(4.281,3.343)--(4.286,3.345)%
  --(4.291,3.347)--(4.297,3.350)--(4.302,3.352)--(4.308,3.355)--(4.313,3.357)--(4.318,3.359)%
  --(4.324,3.362)--(4.329,3.364)--(4.335,3.367)--(4.340,3.369)--(4.345,3.371)--(4.351,3.374)%
  --(4.356,3.376)--(4.361,3.379)--(4.367,3.381)--(4.372,3.383)--(4.378,3.386)--(4.383,3.388)%
  --(4.388,3.391)--(4.394,3.393)--(4.399,3.396)--(4.405,3.398)--(4.410,3.400)--(4.415,3.403)%
  --(4.421,3.405)--(4.426,3.408)--(4.432,3.410)--(4.437,3.412)--(4.442,3.415)--(4.448,3.417)%
  --(4.453,3.420)--(4.458,3.422)--(4.464,3.424)--(4.469,3.427)--(4.475,3.429)--(4.480,3.432)%
  --(4.485,3.434)--(4.491,3.437)--(4.496,3.439)--(4.502,3.441)--(4.507,3.444)--(4.512,3.446)%
  --(4.518,3.449)--(4.523,3.451)--(4.528,3.453)--(4.534,3.456)--(4.539,3.458)--(4.545,3.461)%
  --(4.550,3.463)--(4.555,3.465)--(4.561,3.468)--(4.566,3.470)--(4.572,3.473)--(4.577,3.475)%
  --(4.582,3.477)--(4.588,3.480)--(4.593,3.482)--(4.598,3.485)--(4.604,3.487)--(4.609,3.490)%
  --(4.615,3.492)--(4.620,3.494)--(4.625,3.497)--(4.631,3.499)--(4.636,3.502)--(4.642,3.504)%
  --(4.647,3.506)--(4.652,3.509)--(4.658,3.511)--(4.663,3.514)--(4.668,3.516)--(4.674,3.518)%
  --(4.679,3.521)--(4.685,3.523)--(4.690,3.526)--(4.695,3.528)--(4.701,3.531)--(4.706,3.533)%
  --(4.712,3.535)--(4.717,3.538)--(4.722,3.540)--(4.728,3.543)--(4.733,3.545)--(4.739,3.547)%
  --(4.744,3.550)--(4.749,3.552)--(4.755,3.555)--(4.760,3.557)--(4.765,3.559)--(4.771,3.562)%
  --(4.776,3.564)--(4.782,3.567)--(4.787,3.569)--(4.792,3.571)--(4.798,3.574)--(4.803,3.576)%
  --(4.809,3.579)--(4.814,3.581)--(4.819,3.584)--(4.825,3.586)--(4.830,3.588)--(4.835,3.591)%
  --(4.841,3.593)--(4.846,3.596)--(4.852,3.598)--(4.857,3.600)--(4.862,3.603)--(4.868,3.605)%
  --(4.873,3.608)--(4.879,3.610)--(4.884,3.612)--(4.889,3.615)--(4.895,3.617)--(4.900,3.620)%
  --(4.905,3.622)--(4.911,3.625)--(4.916,3.627)--(4.922,3.629)--(4.927,3.632)--(4.932,3.634)%
  --(4.938,3.637)--(4.943,3.639)--(4.949,3.641)--(4.954,3.644)--(4.959,3.646)--(4.965,3.649)%
  --(4.970,3.651)--(4.975,3.653)--(4.981,3.656)--(4.986,3.658)--(4.992,3.661)--(4.997,3.663)%
  --(5.002,3.665)--(5.008,3.668)--(5.013,3.670)--(5.019,3.673)--(5.024,3.675)--(5.029,3.678)%
  --(5.035,3.680)--(5.040,3.682)--(5.046,3.685)--(5.051,3.687)--(5.056,3.690)--(5.062,3.692)%
  --(5.067,3.694)--(5.072,3.697)--(5.078,3.699)--(5.083,3.702)--(5.089,3.704)--(5.094,3.706)%
  --(5.099,3.709)--(5.105,3.711)--(5.110,3.714)--(5.116,3.716)--(5.121,3.719)--(5.126,3.721)%
  --(5.132,3.723)--(5.137,3.726)--(5.142,3.728)--(5.148,3.731)--(5.153,3.733)--(5.159,3.735)%
  --(5.164,3.738)--(5.169,3.740)--(5.175,3.743)--(5.180,3.745)--(5.186,3.747)--(5.191,3.750)%
  --(5.196,3.752)--(5.202,3.755)--(5.207,3.757)--(5.212,3.759)--(5.218,3.762)--(5.223,3.764)%
  --(5.229,3.767)--(5.234,3.769)--(5.239,3.772)--(5.245,3.774)--(5.250,3.776)--(5.256,3.779)%
  --(5.261,3.781)--(5.266,3.784)--(5.272,3.786)--(5.277,3.788)--(5.282,3.791)--(5.288,3.793)%
  --(5.293,3.796)--(5.299,3.798)--(5.304,3.800)--(5.309,3.803)--(5.315,3.805)--(5.320,3.808)%
  --(5.326,3.810)--(5.331,3.813)--(5.336,3.815)--(5.342,3.817)--(5.347,3.820)--(5.353,3.822)%
  --(5.358,3.825)--(5.363,3.827)--(5.369,3.829)--(5.374,3.832)--(5.379,3.834)--(5.385,3.837)%
  --(5.390,3.839)--(5.396,3.841)--(5.401,3.844)--(5.406,3.846)--(5.412,3.849)--(5.417,3.851)%
  --(5.423,3.853)--(5.428,3.856)--(5.433,3.858)--(5.439,3.861)--(5.444,3.863)--(5.449,3.866)%
  --(5.455,3.868)--(5.460,3.870)--(5.466,3.873)--(5.471,3.875)--(5.476,3.878)--(5.482,3.880)%
  --(5.487,3.882)--(5.493,3.885)--(5.498,3.887)--(5.503,3.890)--(5.509,3.892)--(5.514,3.894)%
  --(5.519,3.897)--(5.525,3.899)--(5.530,3.902)--(5.536,3.904)--(5.541,3.907)--(5.546,3.909)%
  --(5.552,3.911)--(5.557,3.914)--(5.563,3.916)--(5.568,3.919)--(5.573,3.921)--(5.579,3.923)%
  --(5.584,3.926)--(5.590,3.928)--(5.595,3.931)--(5.600,3.933)--(5.606,3.935)--(5.611,3.938)%
  --(5.616,3.940)--(5.622,3.943)--(5.627,3.945)--(5.633,3.947)--(5.638,3.950)--(5.643,3.952)%
  --(5.649,3.955)--(5.654,3.957)--(5.660,3.960)--(5.665,3.962)--(5.670,3.964)--(5.676,3.967)%
  --(5.681,3.969)--(5.686,3.972)--(5.692,3.974)--(5.697,3.976)--(5.703,3.979)--(5.708,3.981)%
  --(5.713,3.984)--(5.719,3.986)--(5.724,3.988)--(5.730,3.991)--(5.735,3.993)--(5.740,3.996)%
  --(5.746,3.998)--(5.751,4.001)--(5.756,4.003)--(5.762,4.005)--(5.767,4.008)--(5.773,4.010)%
  --(5.778,4.013)--(5.783,4.015)--(5.789,4.017)--(5.794,4.020)--(5.800,4.022)--(5.805,4.025)%
  --(5.810,4.027)--(5.816,4.029)--(5.821,4.032)--(5.826,4.034)--(5.832,4.037)--(5.837,4.039)%
  --(5.843,4.041)--(5.848,4.044)--(5.853,4.046)--(5.859,4.049)--(5.864,4.051)--(5.870,4.054)%
  --(5.875,4.056)--(5.880,4.058)--(5.886,4.061)--(5.891,4.063)--(5.897,4.066)--(5.902,4.068)%
  --(5.907,4.070)--(5.913,4.073)--(5.918,4.075)--(5.923,4.078)--(5.929,4.080)--(5.934,4.082)%
  --(5.940,4.085)--(5.945,4.087)--(5.950,4.090)--(5.956,4.092)--(5.961,4.095)--(5.967,4.097)%
  --(5.972,4.099)--(5.977,4.102)--(5.983,4.104)--(5.988,4.107)--(5.993,4.109)--(5.999,4.111)%
  --(6.004,4.114)--(6.010,4.116)--(6.015,4.119)--(6.020,4.121)--(6.026,4.123)--(6.031,4.126)%
  --(6.037,4.128)--(6.042,4.131)--(6.047,4.133)--(6.053,4.135)--(6.058,4.138)--(6.063,4.140)%
  --(6.069,4.143)--(6.074,4.145)--(6.080,4.148)--(6.085,4.150)--(6.090,4.152)--(6.096,4.155)%
  --(6.101,4.157)--(6.107,4.160)--(6.112,4.162)--(6.117,4.164)--(6.123,4.167)--(6.128,4.169)%
  --(6.133,4.172)--(6.139,4.174)--(6.144,4.176)--(6.150,4.179)--(6.155,4.181)--(6.160,4.184)%
  --(6.166,4.186)--(6.171,4.189)--(6.177,4.191)--(6.182,4.193)--(6.187,4.196)--(6.193,4.198)%
  --(6.198,4.201)--(6.204,4.203)--(6.209,4.205)--(6.214,4.208)--(6.220,4.210)--(6.225,4.213)%
  --(6.230,4.215)--(6.236,4.217)--(6.241,4.220)--(6.247,4.222)--(6.252,4.225)--(6.257,4.227)%
  --(6.263,4.229)--(6.268,4.232)--(6.274,4.234)--(6.279,4.237)--(6.284,4.239)--(6.290,4.242)%
  --(6.295,4.244)--(6.300,4.246)--(6.306,4.249)--(6.311,4.251)--(6.317,4.254)--(6.322,4.256)%
  --(6.327,4.258)--(6.333,4.261)--(6.338,4.263)--(6.344,4.266)--(6.349,4.268)--(6.354,4.270)%
  --(6.360,4.273)--(6.365,4.275)--(6.370,4.278)--(6.376,4.280)--(6.381,4.283)--(6.387,4.285)%
  --(6.392,4.287)--(6.397,4.290)--(6.403,4.292)--(6.408,4.295)--(6.414,4.297)--(6.419,4.299)%
  --(6.424,4.302)--(6.430,4.304)--(6.435,4.307)--(6.441,4.309)--(6.446,4.311)--(6.451,4.314)%
  --(6.457,4.316)--(6.462,4.319)--(6.467,4.321)--(6.473,4.323)--(6.478,4.326)--(6.484,4.328)%
  --(6.489,4.331)--(6.494,4.333)--(6.500,4.336)--(6.505,4.338)--(6.511,4.340)--(6.516,4.343)%
  --(6.521,4.345)--(6.527,4.348)--(6.532,4.350)--(6.537,4.352)--(6.543,4.355)--(6.548,4.357)%
  --(6.554,4.360)--(6.559,4.362)--(6.564,4.364)--(6.570,4.367)--(6.575,4.369)--(6.581,4.372)%
  --(6.586,4.374)--(6.591,4.377)--(6.597,4.379)--(6.602,4.381)--(6.607,4.384)--(6.613,4.386)%
  --(6.618,4.389)--(6.624,4.391)--(6.629,4.393)--(6.634,4.396)--(6.640,4.398)--(6.645,4.401)%
  --(6.651,4.403)--(6.656,4.405)--(6.661,4.408)--(6.667,4.410)--(6.672,4.413)--(6.677,4.415)%
  --(6.683,4.417)--(6.688,4.420)--(6.694,4.422)--(6.699,4.425)--(6.704,4.427)--(6.710,4.430)%
  --(6.715,4.432)--(6.721,4.434)--(6.726,4.437)--(6.731,4.439)--(6.737,4.442)--(6.742,4.444)%
  --(6.748,4.446)--(6.753,4.449)--(6.758,4.451)--(6.764,4.454)--(6.769,4.456)--(6.774,4.458)%
  --(6.780,4.461)--(6.785,4.463)--(6.791,4.466)--(6.796,4.468)--(6.801,4.471)--(6.807,4.473)%
  --(6.812,4.475)--(6.818,4.478)--(6.823,4.480)--(6.828,4.483)--(6.834,4.485)--(6.839,4.487)%
  --(6.844,4.490)--(6.850,4.492)--(6.855,4.495)--(6.861,4.497)--(6.866,4.499)--(6.871,4.502)%
  --(6.877,4.504)--(6.882,4.507)--(6.888,4.509)--(6.893,4.511)--(6.898,4.514)--(6.904,4.516)%
  --(6.909,4.519)--(6.914,4.521)--(6.920,4.524)--(6.925,4.526)--(6.931,4.528)--(6.936,4.531)%
  --(6.941,4.533)--(6.947,4.536)--(6.952,4.538)--(6.958,4.540)--(6.963,4.543)--(6.968,4.545)%
  --(6.974,4.548)--(6.979,4.550)--(6.984,4.552)--(6.990,4.555)--(6.995,4.557)--(7.001,4.560)%
  --(7.006,4.562)--(7.011,4.565)--(7.017,4.567)--(7.022,4.569)--(7.028,4.572)--(7.033,4.574)%
  --(7.038,4.577)--(7.044,4.579)--(7.049,4.581)--(7.055,4.584)--(7.060,4.586)--(7.065,4.589)%
  --(7.071,4.591)--(7.076,4.593)--(7.081,4.596)--(7.087,4.598)--(7.092,4.601)--(7.098,4.603)%
  --(7.103,4.605)--(7.108,4.608)--(7.114,4.610)--(7.119,4.613)--(7.125,4.615)--(7.130,4.618)%
  --(7.135,4.620)--(7.141,4.622)--(7.146,4.625)--(7.151,4.627)--(7.157,4.630)--(7.162,4.632)%
  --(7.168,4.634)--(7.173,4.637)--(7.178,4.639)--(7.184,4.642)--(7.189,4.644)--(7.195,4.646)%
  --(7.200,4.649)--(7.205,4.651)--(7.211,4.654)--(7.216,4.656)--(7.221,4.659)--(7.227,4.661)%
  --(7.232,4.663)--(7.238,4.666)--(7.243,4.668)--(7.248,4.671)--(7.254,4.673)--(7.259,4.675)%
  --(7.265,4.678)--(7.270,4.680)--(7.275,4.683)--(7.281,4.685)--(7.286,4.687)--(7.291,4.690)%
  --(7.297,4.692)--(7.302,4.695)--(7.308,4.697)--(7.313,4.699)--(7.318,4.702)--(7.324,4.704)%
  --(7.329,4.707)--(7.335,4.709)--(7.340,4.712)--(7.345,4.714)--(7.351,4.716)--(7.356,4.719)%
  --(7.362,4.721)--(7.367,4.724)--(7.372,4.726)--(7.378,4.728)--(7.383,4.731)--(7.388,4.733)%
  --(7.394,4.736)--(7.399,4.738)--(7.405,4.740)--(7.410,4.743)--(7.415,4.745)--(7.421,4.748)%
  --(7.426,4.750)--(7.432,4.753)--(7.437,4.755)--(7.442,4.757)--(7.448,4.760)--(7.453,4.762)%
  --(7.458,4.765)--(7.464,4.767)--(7.469,4.769)--(7.475,4.772)--(7.480,4.774)--(7.485,4.777)%
  --(7.491,4.779)--(7.496,4.781)--(7.502,4.784)--(7.507,4.786)--(7.512,4.789)--(7.518,4.791)%
  --(7.523,4.793)--(7.528,4.796)--(7.534,4.798)--(7.539,4.801)--(7.545,4.803)--(7.550,4.806)%
  --(7.555,4.808)--(7.561,4.810)--(7.566,4.813)--(7.572,4.815)--(7.577,4.818)--(7.582,4.820)%
  --(7.588,4.822)--(7.593,4.825)--(7.599,4.827)--(7.604,4.830)--(7.609,4.832)--(7.615,4.834)%
  --(7.620,4.837)--(7.625,4.839)--(7.631,4.842)--(7.636,4.844)--(7.642,4.846)--(7.647,4.849)%
  --(7.652,4.851)--(7.658,4.854)--(7.663,4.856)--(7.669,4.859)--(7.674,4.861)--(7.679,4.863)%
  --(7.685,4.866)--(7.690,4.868)--(7.695,4.871)--(7.701,4.873)--(7.706,4.875)--(7.712,4.878)%
  --(7.717,4.880)--(7.722,4.883)--(7.728,4.885)--(7.733,4.887)--(7.739,4.890)--(7.744,4.892)%
  --(7.749,4.895)--(7.755,4.897)--(7.760,4.900)--(7.765,4.902)--(7.771,4.904)--(7.776,4.907)%
  --(7.782,4.909)--(7.787,4.912)--(7.792,4.914)--(7.798,4.916)--(7.803,4.919)--(7.809,4.921)%
  --(7.814,4.924)--(7.819,4.926)--(7.825,4.928)--(7.830,4.931)--(7.835,4.933)--(7.841,4.936)%
  --(7.846,4.938)--(7.852,4.940)--(7.857,4.943)--(7.862,4.945)--(7.868,4.948)--(7.873,4.950)%
  --(7.879,4.953)--(7.884,4.955)--(7.889,4.957)--(7.895,4.960)--(7.900,4.962)--(7.906,4.965)%
  --(7.911,4.967)--(7.916,4.969)--(7.922,4.972)--(7.927,4.974)--(7.932,4.977)--(7.938,4.979)%
  --(7.943,4.981)--(7.949,4.984)--(7.954,4.986)--(7.959,4.989)--(7.965,4.991)--(7.970,4.994)%
  --(7.976,4.996)--(7.981,4.998)--(7.986,5.001)--(7.992,5.003)--(7.997,5.006)--(8.002,5.008)%
  --(8.008,5.010)--(8.013,5.013)--(8.019,5.015)--(8.024,5.018)--(8.029,5.020)--(8.035,5.022)%
  --(8.040,5.025)--(8.046,5.027)--(8.051,5.030)--(8.056,5.032)--(8.062,5.034)--(8.067,5.037)%
  --(8.072,5.039)--(8.078,5.042)--(8.083,5.044)--(8.089,5.047)--(8.094,5.049)--(8.099,5.051)%
  --(8.105,5.054)--(8.110,5.056)--(8.116,5.059)--(8.121,5.061)--(8.126,5.063)--(8.132,5.066)%
  --(8.137,5.068)--(8.142,5.071)--(8.148,5.073)--(8.153,5.075)--(8.159,5.078)--(8.164,5.080)%
  --(8.169,5.083)--(8.175,5.085)--(8.180,5.088)--(8.186,5.090)--(8.191,5.092)--(8.196,5.095)%
  --(8.202,5.097)--(8.207,5.100)--(8.213,5.102)--(8.218,5.104)--(8.223,5.107)--(8.229,5.109)%
  --(8.234,5.112)--(8.239,5.114)--(8.245,5.116)--(8.250,5.119)--(8.256,5.121)--(8.261,5.124)%
  --(8.266,5.126)--(8.272,5.128)--(8.277,5.131)--(8.283,5.133)--(8.288,5.136)--(8.293,5.138)%
  --(8.299,5.141)--(8.304,5.143)--(8.309,5.145)--(8.315,5.148)--(8.320,5.150)--(8.326,5.153)%
  --(8.331,5.155)--(8.336,5.157)--(8.342,5.160)--(8.347,5.162)--(8.353,5.165)--(8.358,5.167)%
  --(8.363,5.169)--(8.369,5.172)--(8.374,5.174)--(8.379,5.177)--(8.385,5.179)--(8.390,5.182)%
  --(8.396,5.184)--(8.401,5.186)--(8.406,5.189)--(8.412,5.191)--(8.417,5.194)--(8.423,5.196)%
  --(8.428,5.198)--(8.433,5.201)--(8.439,5.203)--(8.444,5.206)--(8.450,5.208)--(8.455,5.210)%
  --(8.460,5.213)--(8.466,5.215)--(8.471,5.218)--(8.476,5.220)--(8.482,5.222)--(8.487,5.225)%
  --(8.493,5.227)--(8.498,5.230)--(8.503,5.232)--(8.509,5.235)--(8.514,5.237)--(8.520,5.239)%
  --(8.525,5.242)--(8.530,5.244)--(8.536,5.247)--(8.541,5.249)--(8.546,5.251)--(8.552,5.254)%
  --(8.557,5.256)--(8.563,5.259)--(8.568,5.261)--(8.573,5.263)--(8.579,5.266)--(8.584,5.268)%
  --(8.590,5.271)--(8.595,5.273)--(8.600,5.276)--(8.606,5.278)--(8.611,5.280)--(8.616,5.283)%
  --(8.622,5.285)--(8.627,5.288)--(8.633,5.290)--(8.638,5.292)--(8.643,5.295)--(8.649,5.297)%
  --(8.654,5.300)--(8.660,5.302)--(8.665,5.304)--(8.670,5.307)--(8.676,5.309)--(8.681,5.312)%
  --(8.686,5.314)--(8.692,5.316)--(8.697,5.319)--(8.703,5.321)--(8.708,5.324)--(8.713,5.326)%
  --(8.719,5.329)--(8.724,5.331)--(8.730,5.333)--(8.735,5.336)--(8.740,5.338)--(8.746,5.341)%
  --(8.751,5.343)--(8.757,5.345)--(8.762,5.348)--(8.767,5.350)--(8.773,5.353)--(8.778,5.355)%
  --(8.783,5.357)--(8.789,5.360)--(8.794,5.362)--(8.800,5.365)--(8.805,5.367)--(8.810,5.370)%
  --(8.816,5.372)--(8.821,5.374)--(8.827,5.377)--(8.832,5.379)--(8.837,5.382)--(8.843,5.384)%
  --(8.848,5.386)--(8.853,5.389)--(8.859,5.391)--(8.864,5.394)--(8.870,5.396)--(8.875,5.398)%
  --(8.880,5.401)--(8.886,5.403)--(8.891,5.406)--(8.897,5.408)--(8.902,5.410)--(8.907,5.413)%
  --(8.913,5.415)--(8.918,5.418)--(8.923,5.420)--(8.929,5.423)--(8.934,5.425)--(8.940,5.427)%
  --(8.945,5.430)--(8.950,5.432)--(8.956,5.435)--(8.961,5.437)--(8.967,5.439)--(8.972,5.442)%
  --(8.977,5.444)--(8.983,5.447)--(8.988,5.449)--(8.993,5.451)--(8.999,5.454)--(9.004,5.456)%
  --(9.010,5.459)--(9.015,5.461)--(9.020,5.464)--(9.026,5.466)--(9.031,5.468)--(9.037,5.471)%
  --(9.042,5.473)--(9.047,5.476)--(9.053,5.478)--(9.058,5.480)--(9.064,5.483)--(9.069,5.485)%
  --(9.074,5.488)--(9.080,5.490)--(9.085,5.492)--(9.090,5.495)--(9.096,5.497)--(9.101,5.500)%
  --(9.107,5.502)--(9.112,5.504)--(9.117,5.507)--(9.123,5.509)--(9.128,5.512)--(9.134,5.514)%
  --(9.139,5.517)--(9.144,5.519)--(9.150,5.521)--(9.155,5.524)--(9.160,5.526)--(9.166,5.529)%
  --(9.171,5.531)--(9.177,5.533)--(9.182,5.536)--(9.187,5.538)--(9.193,5.541)--(9.198,5.543)%
  --(9.204,5.545)--(9.209,5.548)--(9.214,5.550)--(9.220,5.553)--(9.225,5.555)--(9.230,5.558)%
  --(9.236,5.560)--(9.241,5.562)--(9.247,5.565)--(9.252,5.567)--(9.257,5.570)--(9.263,5.572)%
  --(9.268,5.574)--(9.274,5.577)--(9.279,5.579)--(9.284,5.582)--(9.290,5.584)--(9.295,5.586)%
  --(9.301,5.589)--(9.306,5.591)--(9.311,5.594)--(9.317,5.596)--(9.322,5.598)--(9.327,5.601)%
  --(9.333,5.603)--(9.338,5.606)--(9.344,5.608)--(9.349,5.611)--(9.354,5.613)--(9.360,5.615)%
  --(9.365,5.618)--(9.371,5.620)--(9.376,5.623)--(9.381,5.625)--(9.387,5.627)--(9.392,5.630)%
  --(9.397,5.632)--(9.403,5.635)--(9.408,5.637)--(9.414,5.639)--(9.419,5.642)--(9.424,5.644)%
  --(9.430,5.647)--(9.435,5.649)--(9.441,5.652)--(9.446,5.654)--(9.451,5.656)--(9.457,5.659)%
  --(9.462,5.661)--(9.467,5.664)--(9.473,5.666)--(9.478,5.668)--(9.484,5.671)--(9.489,5.673)%
  --(9.494,5.676)--(9.500,5.678)--(9.505,5.680)--(9.511,5.683)--(9.516,5.685)--(9.521,5.688)%
  --(9.527,5.690)--(9.532,5.692)--(9.537,5.695)--(9.543,5.697)--(9.548,5.700)--(9.554,5.702)%
  --(9.559,5.705)--(9.564,5.707)--(9.570,5.709)--(9.575,5.712)--(9.581,5.714)--(9.586,5.717)%
  --(9.591,5.719)--(9.597,5.721)--(9.602,5.724)--(9.608,5.726)--(9.613,5.729)--(9.618,5.731)%
  --(9.624,5.733)--(9.629,5.736)--(9.634,5.738)--(9.640,5.741)--(9.645,5.743)--(9.651,5.746)%
  --(9.656,5.748)--(9.661,5.750)--(9.667,5.753)--(9.672,5.755)--(9.678,5.758)--(9.683,5.760)%
  --(9.688,5.762)--(9.694,5.765)--(9.699,5.767)--(9.704,5.770)--(9.710,5.772)--(9.715,5.774)%
  --(9.721,5.777)--(9.726,5.779)--(9.731,5.782)--(9.737,5.784)--(9.742,5.786)--(9.748,5.789)%
  --(9.753,5.791)--(9.758,5.794)--(9.764,5.796)--(9.769,5.799)--(9.774,5.801)--(9.780,5.803)%
  --(9.785,5.806)--(9.791,5.808)--(9.796,5.811)--(9.801,5.813)--(9.807,5.815)--(9.812,5.818)%
  --(9.818,5.820)--(9.823,5.823)--(9.828,5.825)--(9.834,5.827)--(9.839,5.830)--(9.844,5.832)%
  --(9.850,5.835)--(9.855,5.837)--(9.861,5.840)--(9.866,5.842)--(9.871,5.844)--(9.877,5.847)%
  --(9.882,5.849)--(9.888,5.852)--(9.893,5.854)--(9.898,5.856)--(9.904,5.859)--(9.909,5.861)%
  --(9.915,5.864)--(9.920,5.866)--(9.925,5.868)--(9.931,5.871)--(9.936,5.873)--(9.941,5.876)%
  --(9.947,5.878)--(9.952,5.880)--(9.958,5.883)--(9.963,5.885)--(9.968,5.888)--(9.974,5.890)%
  --(9.979,5.893)--(9.985,5.895)--(9.990,5.897)--(9.995,5.900)--(10.001,5.902)--(10.006,5.905)%
  --(10.011,5.907)--(10.017,5.909)--(10.022,5.912)--(10.028,5.914)--(10.033,5.917)--(10.038,5.919)%
  --(10.044,5.921)--(10.049,5.924)--(10.055,5.926)--(10.060,5.929)--(10.065,5.931)--(10.071,5.934)%
  --(10.076,5.936)--(10.081,5.938)--(10.087,5.941)--(10.092,5.943)--(10.098,5.946)--(10.103,5.948)%
  --(10.108,5.950)--(10.114,5.953)--(10.119,5.955)--(10.125,5.958)--(10.130,5.960)--(10.135,5.962)%
  --(10.141,5.965)--(10.146,5.967)--(10.151,5.970)--(10.157,5.972)--(10.162,5.974)--(10.168,5.977)%
  --(10.173,5.979)--(10.178,5.982)--(10.184,5.984)--(10.189,5.987)--(10.195,5.989)--(10.200,5.991)%
  --(10.205,5.994)--(10.211,5.996)--(10.216,5.999)--(10.222,6.001)--(10.227,6.003)--(10.232,6.006)%
  --(10.238,6.008)--(10.243,6.011)--(10.248,6.013)--(10.254,6.015)--(10.259,6.018)--(10.265,6.020)%
  --(10.270,6.023)--(10.275,6.025)--(10.281,6.028)--(10.286,6.030)--(10.292,6.032)--(10.297,6.035)%
  --(10.302,6.037)--(10.308,6.040)--(10.313,6.042)--(10.318,6.044)--(10.324,6.047)--(10.329,6.049)%
  --(10.335,6.052)--(10.340,6.054)--(10.345,6.056)--(10.351,6.059)--(10.356,6.061)--(10.362,6.064)%
  --(10.367,6.066)--(10.372,6.068)--(10.378,6.071)--(10.383,6.073)--(10.388,6.076)--(10.394,6.078)%
  --(10.399,6.081)--(10.405,6.083)--(10.410,6.085)--(10.415,6.088)--(10.421,6.090)--(10.426,6.093)%
  --(10.432,6.095)--(10.437,6.097)--(10.442,6.100)--(10.448,6.102)--(10.453,6.105)--(10.459,6.107)%
  --(10.464,6.109)--(10.469,6.112)--(10.475,6.114)--(10.480,6.117)--(10.485,6.119)--(10.491,6.122)%
  --(10.496,6.124)--(10.502,6.126)--(10.507,6.129)--(10.512,6.131)--(10.518,6.134)--(10.523,6.136)%
  --(10.529,6.138)--(10.534,6.141)--(10.539,6.143)--(10.545,6.146)--(10.550,6.148)--(10.555,6.150)%
  --(10.561,6.153)--(10.566,6.155)--(10.572,6.158)--(10.577,6.160)--(10.582,6.162)--(10.588,6.165)%
  --(10.593,6.167)--(10.599,6.170)--(10.604,6.172)--(10.609,6.175)--(10.615,6.177)--(10.620,6.179)%
  --(10.625,6.182)--(10.631,6.184)--(10.636,6.187)--(10.642,6.189)--(10.647,6.191)--(10.652,6.194)%
  --(10.658,6.196)--(10.663,6.199)--(10.669,6.201)--(10.674,6.203)--(10.679,6.206)--(10.685,6.208)%
  --(10.690,6.211)--(10.695,6.213)--(10.701,6.216)--(10.706,6.218)--(10.712,6.220)--(10.717,6.223)%
  --(10.722,6.225)--(10.728,6.228)--(10.733,6.230)--(10.739,6.232)--(10.744,6.235)--(10.749,6.237)%
  --(10.755,6.240)--(10.760,6.242)--(10.766,6.244)--(10.771,6.247)--(10.776,6.249)--(10.782,6.252)%
  --(10.787,6.254)--(10.792,6.256)--(10.798,6.259)--(10.803,6.261)--(10.809,6.264)--(10.814,6.266)%
  --(10.819,6.269)--(10.825,6.271)--(10.830,6.273)--(10.836,6.276)--(10.841,6.278)--(10.846,6.281)%
  --(10.852,6.283)--(10.857,6.285)--(10.862,6.288)--(10.868,6.290)--(10.873,6.293)--(10.879,6.295)%
  --(10.884,6.297)--(10.889,6.300)--(10.895,6.302)--(10.900,6.305)--(10.906,6.307)--(10.911,6.310)%
  --(10.916,6.312)--(10.922,6.314)--(10.927,6.317)--(10.932,6.319)--(10.938,6.322)--(10.943,6.324)%
  --(10.949,6.326)--(10.954,6.329)--(10.959,6.331)--(10.965,6.334)--(10.970,6.336)--(10.976,6.338)%
  --(10.981,6.341)--(10.986,6.343)--(10.992,6.346)--(10.997,6.348)--(11.002,6.350)--(11.008,6.353)%
  --(11.013,6.355)--(11.019,6.358)--(11.024,6.360)--(11.029,6.363)--(11.035,6.365)--(11.040,6.367)%
  --(11.046,6.370)--(11.051,6.372)--(11.056,6.375)--(11.062,6.377)--(11.067,6.379)--(11.073,6.382)%
  --(11.078,6.384)--(11.083,6.387)--(11.089,6.389)--(11.094,6.391)--(11.099,6.394)--(11.105,6.396)%
  --(11.110,6.399)--(11.116,6.401)--(11.121,6.404)--(11.126,6.406)--(11.132,6.408)--(11.137,6.411)%
  --(11.143,6.413)--(11.148,6.416)--(11.153,6.418)--(11.159,6.420)--(11.164,6.423)--(11.169,6.425)%
  --(11.175,6.428)--(11.180,6.430)--(11.186,6.432)--(11.191,6.435)--(11.196,6.437)--(11.202,6.440)%
  --(11.207,6.442)--(11.213,6.444)--(11.218,6.447)--(11.223,6.449)--(11.229,6.452)--(11.234,6.454)%
  --(11.239,6.457)--(11.245,6.459)--(11.250,6.461)--(11.256,6.464)--(11.261,6.466)--(11.266,6.469)%
  --(11.272,6.471)--(11.277,6.473)--(11.283,6.476)--(11.288,6.478)--(11.293,6.481)--(11.299,6.483)%
  --(11.304,6.485)--(11.310,6.488)--(11.315,6.490)--(11.320,6.493)--(11.326,6.495)--(11.331,6.498)%
  --(11.336,6.500)--(11.342,6.502)--(11.347,6.505)--(11.353,6.507)--(11.358,6.510)--(11.363,6.512)%
  --(11.369,6.514)--(11.374,6.517)--(11.380,6.519)--(11.385,6.522)--(11.390,6.524)--(11.396,6.526)%
  --(11.401,6.529)--(11.406,6.531)--(11.412,6.534)--(11.417,6.536)--(11.423,6.538)--(11.428,6.541)%
  --(11.433,6.543)--(11.439,6.546)--(11.444,6.548)--(11.450,6.551)--(11.455,6.553)--(11.460,6.555)%
  --(11.466,6.558)--(11.471,6.560)--(11.476,6.563)--(11.482,6.565)--(11.487,6.567)--(11.493,6.570)%
  --(11.498,6.572)--(11.503,6.575)--(11.509,6.577)--(11.514,6.579)--(11.520,6.582)--(11.525,6.584)%
  --(11.530,6.587)--(11.536,6.589)--(11.541,6.592)--(11.546,6.594)--(11.552,6.596)--(11.557,6.599)%
  --(11.563,6.601)--(11.568,6.604)--(11.573,6.606)--(11.579,6.608)--(11.584,6.611)--(11.590,6.613)%
  --(11.595,6.616)--(11.600,6.618)--(11.606,6.620)--(11.611,6.623)--(11.617,6.625)--(11.622,6.628)%
  --(11.627,6.630)--(11.633,6.632)--(11.638,6.635)--(11.643,6.637)--(11.649,6.640)--(11.654,6.642)%
  --(11.660,6.645)--(11.665,6.647)--(11.670,6.649)--(11.676,6.652)--(11.681,6.654)--(11.687,6.657)%
  --(11.692,6.659)--(11.697,6.661)--(11.703,6.664)--(11.708,6.666)--(11.713,6.669)--(11.719,6.671)%
  --(11.724,6.673)--(11.730,6.676)--(11.735,6.678)--(11.740,6.681)--(11.746,6.683)--(11.751,6.686)%
  --(11.757,6.688)--(11.762,6.690)--(11.767,6.693)--(11.773,6.695)--(11.778,6.698)--(11.783,6.700)%
  --(11.789,6.702)--(11.794,6.705)--(11.800,6.707)--(11.805,6.710)--(11.810,6.712)--(11.816,6.714)%
  --(11.821,6.717)--(11.827,6.719)--(11.832,6.722)--(11.837,6.724)--(11.843,6.726)--(11.848,6.729)%
  --(11.853,6.731)--(11.859,6.734)--(11.864,6.736)--(11.870,6.739)--(11.875,6.741)--(11.880,6.743)%
  --(11.886,6.746)--(11.891,6.748)--(11.897,6.751)--(11.902,6.753)--(11.907,6.755)--(11.913,6.758)%
  --(11.918,6.760)--(11.924,6.763)--(11.929,6.765)--(11.934,6.767)--(11.940,6.770)--(11.945,6.772)%
  --(11.950,6.775)--(11.956,6.777)--(11.961,6.780)--(11.967,6.782)--(11.972,6.784)--(11.977,6.787)%
  --(11.983,6.789)--(11.988,6.792)--(11.994,6.794)--(11.999,6.796)--(12.004,6.799)--(12.010,6.801)%
  --(12.015,6.804)--(12.020,6.806)--(12.026,6.808)--(12.031,6.811)--(12.037,6.813)--(12.042,6.816)%
  --(12.047,6.818)--(12.053,6.820)--(12.058,6.823)--(12.064,6.825)--(12.069,6.828)--(12.074,6.830)%
  --(12.080,6.833)--(12.085,6.835)--(12.090,6.837)--(12.096,6.840)--(12.101,6.842)--(12.107,6.845)%
  --(12.112,6.847)--(12.117,6.849)--(12.123,6.852)--(12.128,6.854)--(12.134,6.857)--(12.139,6.859)%
  --(12.144,6.861)--(12.150,6.864)--(12.155,6.866)--(12.161,6.869)--(12.166,6.871)--(12.171,6.874)%
  --(12.177,6.876)--(12.182,6.878)--(12.187,6.881)--(12.193,6.883)--(12.198,6.886)--(12.204,6.888)%
  --(12.209,6.890)--(12.214,6.893)--(12.220,6.895)--(12.225,6.898)--(12.231,6.900)--(12.236,6.902)%
  --(12.241,6.905)--(12.247,6.907)--(12.252,6.910)--(12.257,6.912)--(12.263,6.914)--(12.268,6.917)%
  --(12.274,6.919)--(12.279,6.922)--(12.284,6.924)--(12.290,6.927)--(12.295,6.929)--(12.301,6.931)%
  --(12.306,6.934)--(12.311,6.936)--(12.317,6.939)--(12.322,6.941)--(12.327,6.943)--(12.333,6.946)%
  --(12.338,6.948)--(12.344,6.951)--(12.349,6.953)--(12.354,6.955)--(12.360,6.958)--(12.365,6.960)%
  --(12.371,6.963)--(12.376,6.965)--(12.381,6.967)--(12.387,6.970)--(12.392,6.972)--(12.397,6.975)%
  --(12.403,6.977)--(12.408,6.980)--(12.414,6.982)--(12.419,6.984)--(12.424,6.987)--(12.430,6.989)%
  --(12.435,6.992)--(12.441,6.994)--(12.446,6.996)--(12.451,6.999)--(12.457,7.001)--(12.462,7.004)%
  --(12.468,7.006)--(12.473,7.008)--(12.478,7.011)--(12.484,7.013)--(12.489,7.016)--(12.494,7.018)%
  --(12.500,7.021)--(12.505,7.023)--(12.511,7.025)--(12.516,7.028)--(12.521,7.030)--(12.527,7.033)%
  --(12.532,7.035)--(12.538,7.037)--(12.543,7.040)--(12.548,7.042)--(12.554,7.045)--(12.559,7.047)%
  --(12.564,7.049)--(12.570,7.052)--(12.575,7.054)--(12.581,7.057)--(12.586,7.059)--(12.591,7.061)%
  --(12.597,7.064)--(12.602,7.066)--(12.608,7.069)--(12.613,7.071)--(12.618,7.074)--(12.624,7.076)%
  --(12.629,7.078)--(12.634,7.081)--(12.640,7.083)--(12.645,7.086)--(12.651,7.088)--(12.656,7.090)%
  --(12.661,7.093)--(12.667,7.095)--(12.672,7.098)--(12.678,7.100);
\gpcolor{color=gp lt color border}
\gpsetlinewidth{1.00}
\draw[gp path] (1.136,8.631)--(1.136,0.985)--(13.447,0.985)--(13.447,8.631)--cycle;
%% coordinates of the plot area
\gpdefrectangularnode{gp plot 1}{\pgfpoint{1.136cm}{0.985cm}}{\pgfpoint{13.447cm}{8.631cm}}
\end{tikzpicture}
%% gnuplot variables

\end{figure*}


%%%%%%%%%%%%%%%%%%%%%%%%%%%%%%%%%%%%%
\section{Linearização}
%%%%%%%%%%%%%%%%%%%%%%%%%%%%%%%%%%%%%

É comum realizarmos um experimento e obtermos um conjunto de dados que não segue uma tendência linear. Em um experimento de queda livre, por exemplo, a distância percorrida pelo objeto que cai está ligada ao tempo através de
\begin{equation}
	\Delta y = v_0^y t + \frac{a_yt^2}{2}.
\end{equation}
%
Como em geral estamos interessados em extrair dos dados experimentais informações acerca de constantes físicas, podemos recorrer a um processo de regressão. No entanto, não podemos utilizar uma regressão linear neste caso, pois os dados claramente não seguirão uma tendência linear. Podemos recorrer a uma regressão quadrática ou realizar uma \emph{linearização}.

Realizar uma linearização nada mais é do que fazer uma mudança de variáveis. No entanto, nem todos os casos são passíveis de serem linearizados. No caso da queda livre, por exemplo, precisamos garantir que $v_0$ seja muito próxiimo de zero. Nesse caso, podemos escrever
\begin{equation}
	\Delta y = \frac{a_yt^2}{2}.
\end{equation}
%
Fazendo agora a mudança de variáveis $\tau = t^2$, obtemos
\begin{equation}
	\Delta y = \frac{a_y}{2}\tau.
\end{equation}

Comparando a equação acima com a equação da reta $y = A + Bx$, vefificamos as seguintes relações, se considerarmos que o tempo é a variável independente:\footnote{Veja que aqui estamos usando $y$ para duas coisas distintas: um eixo vertical através do qual calculamos o deslocamento $\Delta y$ de um corpo sujeito à gravidade, e o eixo da variável dependente de um gráfico (que é representado no sentido ``da parte inferior da página, para a parte superior'').}
\begin{align}
	y &= \Delta y \\
	A &= 0 \\
	B &= a/2 \\
	x &= \tau.
\end{align}
%
Podemos, portanto, utilizar o processo de linearização associado ao processo de regressão linear para obtermos informações a respeito de constantes físicas mesmo que o conjunto de dados experimentais obtidos não siga uma tendência linear.

%%%%%%%%%%%%%%%%%%%%%%%%%%%%%%%%%%%%%
\section{Linearização e teste de hipóteses}
%%%%%%%%%%%%%%%%%%%%%%%%%%%%%%%%%%%%%

Nos casos em que um fenômeno não segue uma tendência linear, se tivermos um modelo do fenômeno físico considerado, podemos empregar o processo da linearização com o intuito de transformar o conjunto de dados de forma que ele passe a representar uma reta. O sucesso desse procedimento, no entanto, depende de a teoria usada como referência estar correta. Muitas vezes ocorre que não temos conhecimento acerca da teoria que descreve um fenômeno ou temos mais de uma teoria, sem ter certeza de qual é a mais adequada. Podemos então utilizar uma regressão linear como um método de teste, validando uma ou outra teoria com base nos valores obtidos para o coeficiente de dispersão linear $r^2$.

\begin{margintable}
\centering
\begin{tabular}{cc}
\toprule
$\rho$ & $\xi$     \\
\midrule
0.00    & 0.5909   \\ 
0.20    & 0.8310   \\
0.40    & 0.5301   \\
0.60    & 1.0066   \\
0.80    & 1.0345   \\
1.00    & 1.6364   \\
1.20    & 1.5547   \\
1.40    & 2.0911   \\
1.60    & 2.7331   \\
1.80    & 2.9338   \\
2.00    & 3.4851   \\
2.20    & 4.5948   \\
2.40    & 5.3332   \\
2.60    & 6.6722   \\
2.80    & 8.2156   \\
3.00    & 9.8453   \\
3.20    & 12.1483  \\
3.40    & 14.8377  \\
3.60    & 18.2303  \\
\bottomrule
\end{tabular}
\vspace{1mm}
\caption{Dados medidos para $\xi$ em função e $\rho$. \label{TabelaMedidasIniciais}}
\end{margintable}

Digamos que ao fazer uma série de medidas de uma grandeza $\xi$ em função de outra grandeza $\rho$ (Tabela~\ref{TabelaMedidasIniciais}), tenhamos um conjunto que segue uma forma que não é uma reta (Figura~\ref{GraficoMedidas}). Além disso, temos duas teorias, cada uma com as seguintes previsões:
\begin{equation}
	\xi =\begin{cases} \alpha e^{\beta\rho}, \qquad\textrm{Teoria 1} \\
	\alpha\rho^2 + \beta, \qquad\textrm{Teoria 2.} \end{cases}
\end{equation}

\noindent{}Podemos realizar a regressão linear em ambos os casos e comparar os coeficientes $r$ para verificar qual das duas teorias descreve melhor os dados obtidos. Tomando o logaritmo no primeiro caso, temos
\begin{align}
	\ln(\xi) &= \ln(\alpha e^{\beta\rho}) \\
	\ln(\xi) &= \ln(\alpha) + \beta\rho
\end{align}

\begin{figure}
\centering
\caption{Dados medidos de $\xi$ em função de $\rho$. Visivelmente o comportamento é não linear.}
\label{GraficoMedidas}
\begin{tikzpicture}[gnuplot]
%% generated with GNUPLOT 5.0p6 (Lua 5.3; terminal rev. 99, script rev. 100)
%% seg 30 jul 2018 14:19:34 -03
\path (0.000,0.000) rectangle (10.000,7.000);
\gpcolor{color=gp lt color border}
\gpsetlinetype{gp lt border}
\gpsetdashtype{gp dt solid}
\gpsetlinewidth{1.00}
\draw[gp path] (1.136,1.254)--(1.316,1.254);
\draw[gp path] (9.447,1.254)--(9.267,1.254);
\node[gp node right] at (0.952,1.254) {$0$};
\draw[gp path] (1.136,2.329)--(1.316,2.329);
\draw[gp path] (9.447,2.329)--(9.267,2.329);
\node[gp node right] at (0.952,2.329) {$4$};
\draw[gp path] (1.136,3.405)--(1.316,3.405);
\draw[gp path] (9.447,3.405)--(9.267,3.405);
\node[gp node right] at (0.952,3.405) {$8$};
\draw[gp path] (1.136,4.480)--(1.316,4.480);
\draw[gp path] (9.447,4.480)--(9.267,4.480);
\node[gp node right] at (0.952,4.480) {$12$};
\draw[gp path] (1.136,5.556)--(1.316,5.556);
\draw[gp path] (9.447,5.556)--(9.267,5.556);
\node[gp node right] at (0.952,5.556) {$16$};
\draw[gp path] (1.136,6.631)--(1.316,6.631);
\draw[gp path] (9.447,6.631)--(9.267,6.631);
\node[gp node right] at (0.952,6.631) {$20$};
\draw[gp path] (1.532,0.985)--(1.532,1.165);
\draw[gp path] (1.532,6.631)--(1.532,6.451);
\node[gp node center] at (1.532,0.677) {$0$};
\draw[gp path] (3.511,0.985)--(3.511,1.165);
\draw[gp path] (3.511,6.631)--(3.511,6.451);
\node[gp node center] at (3.511,0.677) {$1$};
\draw[gp path] (5.489,0.985)--(5.489,1.165);
\draw[gp path] (5.489,6.631)--(5.489,6.451);
\node[gp node center] at (5.489,0.677) {$2$};
\draw[gp path] (7.468,0.985)--(7.468,1.165);
\draw[gp path] (7.468,6.631)--(7.468,6.451);
\node[gp node center] at (7.468,0.677) {$3$};
\draw[gp path] (9.447,0.985)--(9.447,1.165);
\draw[gp path] (9.447,6.631)--(9.447,6.451);
\node[gp node center] at (9.447,0.677) {$4$};
\draw[gp path] (1.136,6.631)--(1.136,0.985)--(9.447,0.985)--(9.447,6.631)--cycle;
\node[gp node center,rotate=-270] at (0.246,3.808) {$\xi$};
\node[gp node center] at (5.291,0.215) {$\rho$};
\gpcolor{rgb color={0.000,0.000,0.000}}
\gpsetlinewidth{2.00}
\gpsetpointsize{4.00}
\gppoint{gp mark 5}{(1.532,1.413)}
\gppoint{gp mark 5}{(1.928,1.477)}
\gppoint{gp mark 5}{(2.323,1.396)}
\gppoint{gp mark 5}{(2.719,1.525)}
\gppoint{gp mark 5}{(3.115,1.532)}
\gppoint{gp mark 5}{(3.511,1.694)}
\gppoint{gp mark 5}{(3.906,1.672)}
\gppoint{gp mark 5}{(4.302,1.816)}
\gppoint{gp mark 5}{(4.698,1.989)}
\gppoint{gp mark 5}{(5.094,2.043)}
\gppoint{gp mark 5}{(5.489,2.191)}
\gppoint{gp mark 5}{(5.885,2.489)}
\gppoint{gp mark 5}{(6.281,2.688)}
\gppoint{gp mark 5}{(6.677,3.048)}
\gppoint{gp mark 5}{(7.072,3.463)}
\gppoint{gp mark 5}{(7.468,3.901)}
\gppoint{gp mark 5}{(7.864,4.520)}
\gppoint{gp mark 5}{(8.260,5.243)}
\gppoint{gp mark 5}{(8.655,6.155)}
\gpcolor{color=gp lt color border}
\gpsetlinewidth{1.00}
\draw[gp path] (1.136,6.631)--(1.136,0.985)--(9.447,0.985)--(9.447,6.631)--cycle;
%% coordinates of the plot area
\gpdefrectangularnode{gp plot 1}{\pgfpoint{1.136cm}{0.985cm}}{\pgfpoint{9.447cm}{6.631cm}}
\end{tikzpicture}
%% gnuplot variables

\end{figure}

\noindent{}Logo,
\begin{align}
	y &= \ln(\xi) \\
	x &= \rho \\
	A &= \beta \\
	B &= \ln(\alpha)
\end{align}

No segundo caso, temos uma identificação mais simples:
\begin{align}
	y &= \xi \\
	x &= \rho^2 \\
	A &= \alpha \\
	B &= \beta.
\end{align}

Podemos então usar os resultados dessas linearizações para transformar a tabela inicial em duas outras tabelas, cada uma considerando as previsões de cada uma das Teorias e com a esperança de que alguma delas siga o comportamento linear, indicando que a Teoria correspondente tem fundamento.

Baseando-se nas medidas e nos valores transformados dados na Tabela~\ref{TabelaMedidas}, podemos realizar as regressões lineares. Os resultados podem ser vistos nas Figuras~\ref{GraficoTeoria1} e~\ref{GraficoTeoria2}. Concluímos, portanto, que a Teoria 1 descreve melhor os resultados medidos.

\begin{table}
\centering
\begin{tabular}{ccccc}
\toprule
\multicolumn{2}{c}{Teoria 1} && \multicolumn{2}{c}{Teoria 2} \\
\cmidrule(lr){1-2} \cmidrule(lr){4-5}
$\rho$  & $\ln(\xi)$    && $\rho^2$   & $\xi$ \\
\midrule
0.00    & -0.5260      &&   0.00     & 0.5909  \\ 
0.20    & -0.1850      &&   0.04     & 0.8310  \\
0.40    & -0.6346      &&   0.16     & 0.5301  \\
0.60    & 0.0065783    &&   0.36     & 1.0066  \\
0.80    & 0.033918     &&   0.64     & 1.0345  \\
1.00    & 0.49249      &&   1.00     & 1.6364  \\
1.20    & 0.44133      &&   1.44     & 1.5547  \\
1.40    & 0.73770      &&   1.96     & 2.0911  \\
1.60    & 1.0055       &&   2.56     & 2.7331  \\
1.80    & 1.0763       &&   3.24     & 2.9338  \\
2.00    & 1.2485       &&   4.00     & 3.4851  \\
2.20    & 1.5249       &&   4.84     & 4.5948  \\
2.40    & 1.6740       &&   5.76     & 5.3332  \\
2.60    & 1.8980       &&   6.76     & 6.6722  \\
2.80    & 2.1060       &&   7.84     & 8.2156  \\
3.00    & 2.2870       &&   9.00     & 9.8453  \\
3.20    & 2.49719      &&   10.24    & 12.1483 \\
3.40    & 2.69717      &&   11.56    & 14.8377 \\
3.60    & 2.90308      &&   12.96    & 18.2303 \\
\bottomrule
\end{tabular}
\vspace{1mm}
\caption{Dados medidos para $\xi$ em função e $\rho$ e os resultados das transformações baseadas nas Teorias 1 e 2. \label{TabelaMedidas}}
\end{table}

%%% Gráficos

\begin{figure}
\centering
\begin{tikzpicture}[gnuplot]
%% generated with GNUPLOT 5.0p6 (Lua 5.3; terminal rev. 99, script rev. 100)
%% seg 30 jul 2018 14:19:34 -03
\path (0.000,0.000) rectangle (10.000,7.000);
\gpcolor{color=gp lt color border}
\gpsetlinetype{gp lt border}
\gpsetdashtype{gp dt solid}
\gpsetlinewidth{1.00}
\draw[gp path] (1.136,0.985)--(1.316,0.985);
\draw[gp path] (9.447,0.985)--(9.267,0.985);
\node[gp node right] at (0.952,0.985) {$-1$};
\draw[gp path] (1.136,2.114)--(1.316,2.114);
\draw[gp path] (9.447,2.114)--(9.267,2.114);
\node[gp node right] at (0.952,2.114) {$0$};
\draw[gp path] (1.136,3.243)--(1.316,3.243);
\draw[gp path] (9.447,3.243)--(9.267,3.243);
\node[gp node right] at (0.952,3.243) {$1$};
\draw[gp path] (1.136,4.373)--(1.316,4.373);
\draw[gp path] (9.447,4.373)--(9.267,4.373);
\node[gp node right] at (0.952,4.373) {$2$};
\draw[gp path] (1.136,5.502)--(1.316,5.502);
\draw[gp path] (9.447,5.502)--(9.267,5.502);
\node[gp node right] at (0.952,5.502) {$3$};
\draw[gp path] (1.136,6.631)--(1.316,6.631);
\draw[gp path] (9.447,6.631)--(9.267,6.631);
\node[gp node right] at (0.952,6.631) {$4$};
\draw[gp path] (1.532,0.985)--(1.532,1.165);
\draw[gp path] (1.532,6.631)--(1.532,6.451);
\node[gp node center] at (1.532,0.677) {$0$};
\draw[gp path] (3.511,0.985)--(3.511,1.165);
\draw[gp path] (3.511,6.631)--(3.511,6.451);
\node[gp node center] at (3.511,0.677) {$1$};
\draw[gp path] (5.489,0.985)--(5.489,1.165);
\draw[gp path] (5.489,6.631)--(5.489,6.451);
\node[gp node center] at (5.489,0.677) {$2$};
\draw[gp path] (7.468,0.985)--(7.468,1.165);
\draw[gp path] (7.468,6.631)--(7.468,6.451);
\node[gp node center] at (7.468,0.677) {$3$};
\draw[gp path] (9.447,0.985)--(9.447,1.165);
\draw[gp path] (9.447,6.631)--(9.447,6.451);
\node[gp node center] at (9.447,0.677) {$4$};
\draw[gp path] (1.136,6.631)--(1.136,0.985)--(9.447,0.985)--(9.447,6.631)--cycle;
\node[gp node center,rotate=-270] at (0.246,3.808) {$\ln(\xi)$};
\node[gp node center] at (5.291,0.215) {$\rho$};
\gpcolor{rgb color={0.000,0.000,0.000}}
\gpsetlinewidth{2.00}
\gpsetpointsize{4.00}
\gppoint{gp mark 5}{(1.532,1.520)}
\gppoint{gp mark 5}{(1.928,1.905)}
\gppoint{gp mark 5}{(2.323,1.398)}
\gppoint{gp mark 5}{(2.719,2.122)}
\gppoint{gp mark 5}{(3.115,2.153)}
\gppoint{gp mark 5}{(3.511,2.670)}
\gppoint{gp mark 5}{(3.906,2.613)}
\gppoint{gp mark 5}{(4.302,2.947)}
\gppoint{gp mark 5}{(4.698,3.250)}
\gppoint{gp mark 5}{(5.094,3.330)}
\gppoint{gp mark 5}{(5.489,3.524)}
\gppoint{gp mark 5}{(5.885,3.836)}
\gppoint{gp mark 5}{(6.281,4.004)}
\gppoint{gp mark 5}{(6.677,4.257)}
\gppoint{gp mark 5}{(7.072,4.492)}
\gppoint{gp mark 5}{(7.468,4.697)}
\gppoint{gp mark 5}{(7.864,4.934)}
\gppoint{gp mark 5}{(8.260,5.160)}
\gppoint{gp mark 5}{(8.655,5.392)}
\gpcolor{color=gp lt color border}
\node[gp node left] at (2.604,6.297) {$y = \np{0.9738} x - \np{0.6327}$, $r^2 = \np{0.986694}$};
\gpcolor{rgb color={0.000,0.000,0.000}}
\draw[gp path] (1.504,6.297)--(2.420,6.297);
\draw[gp path] (1.388,1.320)--(1.472,1.366)--(1.556,1.413)--(1.640,1.460)--(1.724,1.506)%
  --(1.808,1.553)--(1.892,1.600)--(1.975,1.646)--(2.059,1.693)--(2.143,1.740)--(2.227,1.786)%
  --(2.311,1.833)--(2.395,1.880)--(2.479,1.926)--(2.563,1.973)--(2.647,2.020)--(2.731,2.066)%
  --(2.815,2.113)--(2.899,2.159)--(2.983,2.206)--(3.067,2.253)--(3.151,2.299)--(3.235,2.346)%
  --(3.319,2.393)--(3.403,2.439)--(3.487,2.486)--(3.571,2.533)--(3.654,2.579)--(3.738,2.626)%
  --(3.822,2.673)--(3.906,2.719)--(3.990,2.766)--(4.074,2.813)--(4.158,2.859)--(4.242,2.906)%
  --(4.326,2.953)--(4.410,2.999)--(4.494,3.046)--(4.578,3.093)--(4.662,3.139)--(4.746,3.186)%
  --(4.830,3.233)--(4.914,3.279)--(4.998,3.326)--(5.082,3.372)--(5.166,3.419)--(5.250,3.466)%
  --(5.333,3.512)--(5.417,3.559)--(5.501,3.606)--(5.585,3.652)--(5.669,3.699)--(5.753,3.746)%
  --(5.837,3.792)--(5.921,3.839)--(6.005,3.886)--(6.089,3.932)--(6.173,3.979)--(6.257,4.026)%
  --(6.341,4.072)--(6.425,4.119)--(6.509,4.166)--(6.593,4.212)--(6.677,4.259)--(6.761,4.306)%
  --(6.845,4.352)--(6.929,4.399)--(7.012,4.446)--(7.096,4.492)--(7.180,4.539)--(7.264,4.585)%
  --(7.348,4.632)--(7.432,4.679)--(7.516,4.725)--(7.600,4.772)--(7.684,4.819)--(7.768,4.865)%
  --(7.852,4.912)--(7.936,4.959)--(8.020,5.005)--(8.104,5.052)--(8.188,5.099)--(8.272,5.145)%
  --(8.356,5.192)--(8.440,5.239)--(8.524,5.285)--(8.608,5.332)--(8.691,5.379)--(8.775,5.425)%
  --(8.859,5.472)--(8.943,5.519)--(9.027,5.565)--(9.111,5.612)--(9.195,5.659);
\gpcolor{color=gp lt color border}
\gpsetlinewidth{1.00}
\draw[gp path] (1.136,6.631)--(1.136,0.985)--(9.447,0.985)--(9.447,6.631)--cycle;
%% coordinates of the plot area
\gpdefrectangularnode{gp plot 1}{\pgfpoint{1.136cm}{0.985cm}}{\pgfpoint{9.447cm}{6.631cm}}
\end{tikzpicture}
%% gnuplot variables

\caption{Dados ajustados segundo as previsões da Teoria 1.\label{GraficoTeoria1}}
\end{figure}

\begin{figure}\forcerectofloat
\centering
\begin{tikzpicture}[gnuplot]
%% generated with GNUPLOT 5.0p0 (Lua 5.3; terminal rev. 99, script rev. 100)
%% 2015-05-18T22:59:35 BRT
\path (0.000,0.000) rectangle (14.000,9.000);
\gpcolor{color=gp lt color border}
\gpsetlinetype{gp lt border}
\gpsetdashtype{gp dt solid}
\gpsetlinewidth{1.00}
\draw[gp path] (1.136,1.349)--(1.316,1.349);
\draw[gp path] (13.447,1.349)--(13.267,1.349);
\node[gp node right] at (0.952,1.349) {$0$};
\draw[gp path] (1.136,2.805)--(1.316,2.805);
\draw[gp path] (13.447,2.805)--(13.267,2.805);
\node[gp node right] at (0.952,2.805) {$4$};
\draw[gp path] (1.136,4.262)--(1.316,4.262);
\draw[gp path] (13.447,4.262)--(13.267,4.262);
\node[gp node right] at (0.952,4.262) {$8$};
\draw[gp path] (1.136,5.718)--(1.316,5.718);
\draw[gp path] (13.447,5.718)--(13.267,5.718);
\node[gp node right] at (0.952,5.718) {$12$};
\draw[gp path] (1.136,7.175)--(1.316,7.175);
\draw[gp path] (13.447,7.175)--(13.267,7.175);
\node[gp node right] at (0.952,7.175) {$16$};
\draw[gp path] (1.136,8.631)--(1.316,8.631);
\draw[gp path] (13.447,8.631)--(13.267,8.631);
\node[gp node right] at (0.952,8.631) {$20$};
\draw[gp path] (1.561,0.985)--(1.561,1.165);
\draw[gp path] (1.561,8.631)--(1.561,8.451);
\node[gp node center] at (1.561,0.677) {$0$};
\draw[gp path] (3.259,0.985)--(3.259,1.165);
\draw[gp path] (3.259,8.631)--(3.259,8.451);
\node[gp node center] at (3.259,0.677) {$2$};
\draw[gp path] (4.957,0.985)--(4.957,1.165);
\draw[gp path] (4.957,8.631)--(4.957,8.451);
\node[gp node center] at (4.957,0.677) {$4$};
\draw[gp path] (6.655,0.985)--(6.655,1.165);
\draw[gp path] (6.655,8.631)--(6.655,8.451);
\node[gp node center] at (6.655,0.677) {$6$};
\draw[gp path] (8.353,0.985)--(8.353,1.165);
\draw[gp path] (8.353,8.631)--(8.353,8.451);
\node[gp node center] at (8.353,0.677) {$8$};
\draw[gp path] (10.051,0.985)--(10.051,1.165);
\draw[gp path] (10.051,8.631)--(10.051,8.451);
\node[gp node center] at (10.051,0.677) {$10$};
\draw[gp path] (11.749,0.985)--(11.749,1.165);
\draw[gp path] (11.749,8.631)--(11.749,8.451);
\node[gp node center] at (11.749,0.677) {$12$};
\draw[gp path] (13.447,0.985)--(13.447,1.165);
\draw[gp path] (13.447,8.631)--(13.447,8.451);
\node[gp node center] at (13.447,0.677) {$14$};
\draw[gp path] (1.136,8.631)--(1.136,0.985)--(13.447,0.985)--(13.447,8.631)--cycle;
\node[gp node center,rotate=-270] at (0.246,4.808) {$\xi$};
\node[gp node center] at (7.291,0.215) {$\rho^2$};
\gpcolor{rgb color={0.000,0.000,0.000}}
\gpsetlinewidth{2.00}
\gpsetpointsize{4.00}
\gppoint{gp mark 5}{(1.561,1.564)}
\gppoint{gp mark 5}{(1.594,1.652)}
\gppoint{gp mark 5}{(1.696,1.542)}
\gppoint{gp mark 5}{(1.866,1.716)}
\gppoint{gp mark 5}{(2.104,1.726)}
\gppoint{gp mark 5}{(2.410,1.945)}
\gppoint{gp mark 5}{(2.783,1.915)}
\gppoint{gp mark 5}{(3.225,2.110)}
\gppoint{gp mark 5}{(3.734,2.344)}
\gppoint{gp mark 5}{(4.311,2.417)}
\gppoint{gp mark 5}{(4.957,2.618)}
\gppoint{gp mark 5}{(5.670,3.022)}
\gppoint{gp mark 5}{(6.451,3.291)}
\gppoint{gp mark 5}{(7.300,3.778)}
\gppoint{gp mark 5}{(8.217,4.340)}
\gppoint{gp mark 5}{(9.202,4.934)}
\gppoint{gp mark 5}{(10.255,5.772)}
\gppoint{gp mark 5}{(11.375,6.751)}
\gppoint{gp mark 5}{(12.564,7.987)}
\gpcolor{color=gp lt color border}
\node[gp node left] at (2.604,8.297) {$y = 1.2180 x - 0.2341$, $r^2=0.9583298359$};
\gpcolor{rgb color={0.000,0.000,0.000}}
\draw[gp path] (1.504,8.297)--(2.420,8.297);
\draw[gp path] (1.509,1.237)--(1.633,1.302)--(1.758,1.367)--(1.882,1.432)--(2.006,1.497)%
  --(2.131,1.562)--(2.255,1.627)--(2.380,1.692)--(2.504,1.757)--(2.628,1.822)--(2.753,1.887)%
  --(2.877,1.951)--(3.001,2.016)--(3.126,2.081)--(3.250,2.146)--(3.374,2.211)--(3.499,2.276)%
  --(3.623,2.341)--(3.747,2.406)--(3.872,2.471)--(3.996,2.536)--(4.120,2.601)--(4.245,2.666)%
  --(4.369,2.731)--(4.494,2.796)--(4.618,2.861)--(4.742,2.926)--(4.867,2.991)--(4.991,3.056)%
  --(5.115,3.121)--(5.240,3.186)--(5.364,3.251)--(5.488,3.316)--(5.613,3.380)--(5.737,3.445)%
  --(5.861,3.510)--(5.986,3.575)--(6.110,3.640)--(6.234,3.705)--(6.359,3.770)--(6.483,3.835)%
  --(6.608,3.900)--(6.732,3.965)--(6.856,4.030)--(6.981,4.095)--(7.105,4.160)--(7.229,4.225)%
  --(7.354,4.290)--(7.478,4.355)--(7.602,4.420)--(7.727,4.485)--(7.851,4.550)--(7.975,4.615)%
  --(8.100,4.680)--(8.224,4.745)--(8.349,4.809)--(8.473,4.874)--(8.597,4.939)--(8.722,5.004)%
  --(8.846,5.069)--(8.970,5.134)--(9.095,5.199)--(9.219,5.264)--(9.343,5.329)--(9.468,5.394)%
  --(9.592,5.459)--(9.716,5.524)--(9.841,5.589)--(9.965,5.654)--(10.089,5.719)--(10.214,5.784)%
  --(10.338,5.849)--(10.463,5.914)--(10.587,5.979)--(10.711,6.044)--(10.836,6.109)--(10.960,6.174)%
  --(11.084,6.238)--(11.209,6.303)--(11.333,6.368)--(11.457,6.433)--(11.582,6.498)--(11.706,6.563)%
  --(11.830,6.628)--(11.955,6.693)--(12.079,6.758)--(12.203,6.823)--(12.328,6.888)--(12.452,6.953)%
  --(12.577,7.018)--(12.701,7.083)--(12.825,7.148)--(12.950,7.213)--(13.074,7.278)--(13.198,7.343);
\gpcolor{color=gp lt color border}
\gpsetlinewidth{1.00}
\draw[gp path] (1.136,8.631)--(1.136,0.985)--(13.447,0.985)--(13.447,8.631)--cycle;
%% coordinates of the plot area
\gpdefrectangularnode{gp plot 1}{\pgfpoint{1.136cm}{0.985cm}}{\pgfpoint{13.447cm}{8.631cm}}
\end{tikzpicture}
%% gnuplot variables

\caption{Dados ajustados segundo a Teoria 2.\label{GraficoTeoria2}}
\end{figure}


%%%%%%%%%%%%%%%%%%%%%%%%%%%%%%%%%%%%%%%%%%%%%%%%%%%%%%
\section{Erros nos parâmetros de uma regressão linear}
%%%%%%%%%%%%%%%%%%%%%%%%%%%%%%%%%%%%%%%%%%%%%%%%%%%%%%
\label{Chap:ErrosCoefAB}

Diferentemente da parte teórica, onde assumimos o conhecimento de constantes como a aceleração da gravidade $g$ e calculamos o deslocamento de uma partícula com o passar do tempo, no laboratório podemos medir o tempo e o deslocamento facilmente. O valor da gravidade, por outro lado, é mais difícil de ser determinado. Utilizando uma regressão linear, podemos identificar a relação entre o coeficiente $B$ da equação da reta e $g$. Sabemos, no entanto, que a toda medida temos um erro associado. Qual seria, nesse caso, o erro associado ao valor de $g$? 

Utilizando algumas considerações\cite{Taylor} acerca da distribuição das medidas $y_i$ em torno de seus valores ideais correspondentes, temos que o erro associado aos parâmetros $A$ e $B$ são dados por
\begin{align}
	\delta A &= \xi_y\sqrt{\frac{\sum x_i^2}{\Delta}} \label{Eq:ErrosCoeficientesRegLinear1}\\
	\delta B &= \xi_y\sqrt{\frac{N}{\Delta}}
\end{align}
%
onde $N$ é o número de pontos experimentais e a soma se dá sobre todos os pontos experimentais. Além disso,
\begin{align}
	\xi_y &= \sqrt{\frac{1}{N-2}\sum(y_i-A-B\,x_i)^2} \\
	\Delta &= N\sum x_i^2 - \left(\sum x_i\right)^2. \label{Eq:ErrosCoeficientesRegLinear4}
\end{align}

Podemos então calcular não só o melhor valor associado a uma constante física através dos coeficientes $A$ e $B$ da melhor reta, mas também determinar qual o erro associado a cada uma dessas constantes. Nem sempre temos uma correspondência simples entre uma constante física e um dos parâmetros da equação da reta, nesses casos, ao calcular o valor da constante física, precisamos calcular o erro propagado através da fórmula geral discutida no Capítulo~\ref{Chap:Erros}.

%%%%%%%%%%%%%%%%%%%%%%%%%%%%%%%%%%%%%%%%%%%%%%%%%%%%%%%%%%%%%%%%%%%
\paragraph{Exemplo: Cálculo dos erros dos coeficientes $A$ e $B$}
%%%%%%%%%%%%%%%%%%%%%%%%%%%%%%%%%%%%%%%%%%%%%%%%%%%%%%%%%%%%%%%%%%%

\begin{margintable}
\centering
\begin{tabular}{cccc}
\toprule
$i$	& $x_i$	& $y_{1,i}$	& $(\Xi_i)^2$ \\
\midrule
1	&	\np{0.714}	&	\np{14.577}	&	\np{0.0934} \\ 
2	&	\np{2.693}	&	\np{20.696}	&	\np{0.2687} \\ 
3	&	\np{4.389}	&	\np{25.226}	&	\np{0.0002} \\ 
4	&	\np{4.960}	&	\np{27.449}	&	\np{0.2555} \\ 
5	&	\np{6.245}	&	\np{30.242}	&	\np{0.2879} \\ 
6	&	\np{7.277}	&	\np{33.378}	&	\np{0.2309} \\ 
7	&	\np{7.579}	&	\np{34.195}	&	\np{0.3190} \\ 
8	&	\np{7.719}	&	\np{35.715}	&	\np{0.2887} \\ 
9	&	\np{7.912}	&	\np{35.011}	&	\np{0.5516} \\ 
10	&	\np{8.280}	&	\np{37.529}	&	\np{0.4584} \\ 
11	&	\np{9.034}	&	\np{40.590}	&	\np{2.2134} \\ 
12	&	\np{9.442}	&	\np{39.156}	&	\np{1.3547} \\ 
13	&	\np{10.306}	&	\np{43.238}	&	\np{0.1152} \\ 
14	&	\np{10.572}	&	\np{42.406}	&	\np{1.6548} \\ 
15	&	\np{11.177}	&	\np{44.796}	&	\np{0.4928} \\ 
16	&	\np{15.335}	&	\np{57.611}	&	\np{0.0879} \\ 
17	&	\np{17.023}	&	\np{63.832}	&	\np{0.7863} \\ 
18	&	\np{18.926}	&	\np{68.063}	&	\np{0.3156} \\ 
19	&	\np{20.608}	&	\np{74.408}	&	\np{0.5827} \\ 
20	&	\np{20.876}	&	\np{75.083}	&	\np{0.4077} \\ 
21	&	\np{21.095}	&	\np{75.248}	&	\np{0.0225} \\ 
22	&	\np{22.225}	&	\np{77.243}	&	\np{1.5069} \\ 
23	&	\np{22.407}	&	\np{81.058}	&	\np{4.1791} \\ 
24	&	\np{22.469}	&	\np{78.821}	&	\np{0.1427} \\ 
25	&	\np{23.077}	&	\np{80.714}	&	\np{0.0896} \\ 
26	&	\np{26.421}	&	\np{91.433}	&	\np{0.1932} \\ 
27	&	\np{26.863}	&	\np{91.777}	&	\np{0.2868} \\ 
28	&	\np{27.360}	&	\np{93.291}	&	\np{0.2549} \\
\midrule
\multicolumn{3}{r}{$\sum_i (y_{1,i} - A - B x_i)^2$:} & \np{17.4411} \\ 
\bottomrule
\end{tabular}
\vspace{1mm}
\caption{Tabela com os valores de $(\Xi_i)^2 = (y_{1,i} - A - B x_i)^2$ e da soma $\sum_i (y_{1,i} - A - B x_i)^2$. \label{Tab:DadosDeterminacaoErrosCoeficientesAeB}}
\end{margintable}

Vamos determinar os erros associados aos coeficientes da regressão linear apresentada no exemplo da Seção~\ref{Sec:RegressaoLinear}. Note que os termos $\sum x_i$ e $\sum x_i^2$ que aparecem nas Equações~\eqref{Eq:ErrosCoeficientesRegLinear1} e~\eqref{Eq:ErrosCoeficientesRegLinear4} estão disponíveis na Tabela~\eqref{Tab:TabelaExemploCalculoRegressaoLinear}, porém na prática utilizamos uma calculadora ou um programa de computador para realizar o cálculo dos coeficientes da regressão linear, e por isso não temos de antemão tais valores\footnote{Algumas calculadoras disponibilizam os valores de $\sum x_i$ e $\sum x_i^2$ juntamente com os valores de $A$, $B$, e $r$.}. Nesse caso, será necessário determinar os valores de $\sum x_i$ e de $\sum x_i^2$, além do valor de $\sum (y_{1,i} - A - Bx_i)^2$ ---~novamente, a elaboração de uma tabela ajuda na determinação de tal valor, como mostrado na Tabela~\ref{Tab:DadosDeterminacaoErrosCoeficientesAeB}. Uma vez determinados tais valores, determinar os erros $\delta A$ e $\delta B$ é relativamente simples:
\begin{align}
	\xi_y &= \sqrt{\frac{1}{N-2}\sum(y_i-A-B\,x_i)^2} \\
	&= \sqrt{\frac{\np{17.4411}}{28-2}} \\
	&= \np{0.05814}, \\
	\Delta &= N\sum x_i^2 - \left(\sum x_i\right)^2 \\
	&= 28\cdot\np{7295,757} - (\np{392,984})^2 \\
	&= \np{49844.766}, \\
	\delta A &= \xi_y\sqrt{\frac{\sum x_i^2}{\Delta}} \\
	&= \np{0.05814} \cdot \sqrt{\frac{\np{7295,757}}{\np{49844.766}}} \\
	&= \np{0.022244}, \\
	\delta B &= \xi_y\sqrt{\frac{N}{\Delta}} \\
	&= \np{0.05814} \cdot \sqrt{\frac{28}{\np{49844.766}}} \\
	&= \np{0.001378}.
\end{align}

Assim, podemos denotar os coeficientes linear e angular com seus devidos erros:
\begin{align}
	A &= (\np{12.140462642} \pm \np{0.022244})~{\rm{Un}}_y\\
	B &= (\np{2.984480401} \pm \np{0.001378})~{\rm{Un}}_{\nicefrac{x}{y}}.
\end{align}
%
Note que, como discutido anteriormente, a unidade do coeficiente angular é a mesma que a da variável dependente, enquanto a unidade do coeficiente angular é dada pela razão entre as unidades da variável dependente e da variável independente. Acima denotamos conceitualmente tais unidades como ${\rm{Un}}_{y}$ e ${\rm{Un}}_{\nicefrac{x}{y}}$, respectivamente. Resta ainda notar que não devemos carregar dígitos cuja ordem de grandeza é menor que a do erro, assim obtemos finalmente
\begin{align}
	A &= (\np{12.14} \pm \np{0.02})~{\rm{Un}}_y\\
	B &= (\np{2.984} \pm \np{0.0014})~{\rm{Un}}_{\nicefrac{x}{y}}.
\end{align}

