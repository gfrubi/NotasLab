\chapter*{Introdução}

Este livro/apostila tem como objetivo facilitar o desenvolvimento das atividades de laboratório. Para isso, todo o conteúdo de análise de dados e os próprios experimentos foram reunidos em um conjunto único. 

Na primeira parte, as técnicas de análise de dados são introduzidas e discutidas. Nenhuma técnica muito avançada é apresentada, somente o básico, já que o texto é de nível introdutório. Os assuntos cobertos são medidas, erros, erros propagados, gráficos, regressão linear, e um pouco sobre desvio padrão e erros aleatórios.

Na segunda, os experimentos são analisados de um ponto de vista teórico, seguidos dos procedimentos para coleta de dados. Após isso, as equipes de alunos devem responder um questionário. Os experimentos utilizam diferentes técnicas de análise de dados, sendo que à medida que o semestre avança, o número e a complexidade das técnicas envolvidas aumenta. Isso não é completamente linear, no entanto, já que algumas experiências são mais complexas e mais ricas que outras de um ponto de vista de análise dos dados.

Fiz um esforço grande visando obter um texto de fácil compreensão e bem ilustrado, mas sempre restam pontos que podem ficar confusos ---~ou posso ter cometido um erro!~---. Se isso acontecer, entre em contato comigo. Sugestões são sempre bem vindas.

\vspace{1cm}
\begin{flushright}
Clebson Abati Graeff,\\
Pato Branco, \monthyear.
\end{flushright}

