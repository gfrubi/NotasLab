%%%%%%%%%%%%%%%%%%%%%%%%%%%%%%%%%%%%%%%%%%%%%%%%%%%%%%%%%%%%%%%%%%%%%%%%%%%%%%%
\chapter{Medidas}
\label{Chap:ExpMedidas}
%%%%%%%%%%%%%%%%%%%%%%%%%%%%%%%%%%%%%%%%%%%%%%%%%%%%%%%%%%%%%%%%%%%%%%%%%%%%%%%

\begin{fullwidth}\it
	Realizaremos um experimento visando determinar a densidade de alguns sólidos. Para isso, revisaremos o conceito de densidade e o cálculo do volume de sólidos geométricos. Aplicaremos os conceitos sobre medidas diretas e indiretas realizadas com equipamentos analógicos e não-analógicos (equipamentos digitais e equipamentos dotados de escala auxiliar), observando ao obter os dados o número adequado de algarismos significativos. (Referências aos capítulos dos assuntos mencionados.) \comment{medidas, alg. sig.}
\end{fullwidth}

%%%%%%%%%%%%%%%%%%%%%%%%%%%%%%%%%%%%%%%%%%%%%%%%%%%%%%%%%%%%%%%%%%%%%%%%%%%%%%%
\section{Densidade}
%%%%%%%%%%%%%%%%%%%%%%%%%%%%%%%%%%%%%%%%%%%%%%%%%%%%%%%%%%%%%%%%%%%%%%%%%%%%%%%

Para explorar os conceitos mencionados acima, vamos calcular a densidade de alguns sólidos. Verificaremos as dimensões dos corpos utilizando réguas e paquímetros e utilizaremos esses resultados para calcular o volume, nos preocupando com o número de algarismos significativos adequado. Após isso, vamos calcular a densidade dos corpos utilizando o valor obtido para a massa com o auxílio de uma balança.

Intro antiga, fazer uma discussão de densidade, volume de sólidos, etc. Alterar o resto da prática para incluir cálculo do volume de esferas, cálculo do volume através de métodos indiretos (proveta). Discutir escalas auxiliares de maneira genérica na teoria, mas dando exemplo do paquímetro. Aqui falar no paquímetro detalhadamente, com um exemplo de medida e tudo mais.

%%%%%%%%%%%%%%%%%%%%%%%%%%%%%%%%%%%%%%%%%%
\subsection{Volume de sólidos geométricos}
%%%%%%%%%%%%%%%%%%%%%%%%%%%%%%%%%%%%%%%%%%

%%%%%%%%%%%%%%%%%%%%%%
\subsection{Densidade}
%%%%%%%%%%%%%%%%%%%%%%


%%%%%%%%%%%%%%%%%%%%%%%%%%%%%%%%%%%%%%%%%%%%%%%%%%%%%%%%%%%%%%%%%%%%%%%%%%%%%%%
\section{Experimento}
%%%%%%%%%%%%%%%%%%%%%%%%%%%%%%%%%%%%%%%%%%%%%%%%%%%%%%%%%%%%%%%%%%%%%%%%%%%%%%%

%%%%%%%%%%%%%%%%%%%%%%
\subsection{Objetivos}
%%%%%%%%%%%%%%%%%%%%%%

\begin{itemize}
     \item Determinar as dimensões de vários sólidos utilizando réguas e paquímetros;
     \item Determinar o número de algarismos significativos às medidas;
     \item Calcular o volume dos sólidos;
     \item Verificar com auxílio de uma balança a massa dos corpos;
     \item Utilizar os dados obtidos para calcular as densidades dos sólidos.
\end{itemize}

%%%%%%%%%%%%%%%%%%%%%%%%%%%%%%%%%%%%%%%%%%%%%%%%%%%%%%%%%%%%%%%%%%%%%%%%%%%%%%%
\section{Material Necessário}
%%%%%%%%%%%%%%%%%%%%%%%%%%%%%%%%%%%%%%%%%%%%%%%%%%%%%%%%%%%%%%%%%%%%%%%%%%%%%%%

\begin{itemize}
	\item Três paralelepípedos de tamanhos diferentes;
	\item Dois cilindros de tamanhos diferentes;
	\item Régua;
	\item Paquímetro;
	\item Balança.
\end{itemize}

%%%%%%%%%%%%%%%%%%%%%%%%%%%%%%%%%%%%%%%%%%%%%%%%%%%%%%%%%%%%%%%%%%%%%%%%%%%%%%%
\section{Procedimento Experimental}
%%%%%%%%%%%%%%%%%%%%%%%%%%%%%%%%%%%%%%%%%%%%%%%%%%%%%%%%%%%%%%%%%%%%%%%%%%%%%%%

Tome três paralelepípedos e dois cilindros e os utilize no decorrer do experimento. Anote os resultados obtidos nas próximas seções nas tabelas fornecidas, sempre observando o número de algarismos significativos adequados.

%%%%%%%%%%%%%%%%%%%%%%%%%%%%%%%%%%%%%%%%%%%%%%%%%%%%%%%%%%
\subsection{Determinação das medidas utilizando uma régua}
%%%%%%%%%%%%%%%%%%%%%%%%%%%%%%%%%%%%%%%%%%%%%%%%%%%%%%%%%%

\textbf{Atenção: as medidas realizadas com a régua permitem a estimativa de um algarismo significativo após a casa dos milimetros. \emph{Efetue esta estimativa.}}

\begin{enumerate}
\item Utilizando uma régua, determine as medidas dos sólidos (altura, largura e comprimento ou comprimento e diâmetro). 
\item Determine o volume de cada sólido observando o número de algarismos significativos. \emph{Explicite seus cálculos no verso}.
\end{enumerate}

%%%%%%%%%%%%%%%%%%%%%%%%%%%%%%%%%%%%%%%%%%%%%%%%%%%%%%%%%%%%%%
\subsection{Determinação das medidas utilizando um paquímetro}
%%%%%%%%%%%%%%%%%%%%%%%%%%%%%%%%%%%%%%%%%%%%%%%%%%%%%%%%%%%%%%

\textbf{Atenção: as medidas realizadas com o paquímetro consistem de duas partes: a verificação do valor na escala principal e a leitura do valor excedente em relação ao valor principal na escala auxiliar (nônio).}

\begin{enumerate}
\item Utilizando o paquímetro, determine as medidas dos sólidos (altura, largura e comprimento ou diâmetro e comprimento). 
\item Determine o volume de cada sólido observando o número de algarismos significativos. \emph{Explicite seus cálculos no verso}.
\end{enumerate}

%%%%%%%%%%%%%%%%%%%%%%%%%%%%%%%%%%%%%%%%%%%%%%%%%%%%%%%%%%%%%
\subsection{Determinação das massas e densidades dos sólidos}
%%%%%%%%%%%%%%%%%%%%%%%%%%%%%%%%%%%%%%%%%%%%%%%%%%%%%%%%%%%%%

\begin{enumerate}
     \item Utilize a balança para determinar a massa de cada sólido com o número de algarismos significativos adequado;
     \item Determine a densidade de cada sólido observando o número de algarismos significativos. Utilize as medidas de volume obtidas através das medidas realizadas com o paquímetro.\emph{Explicite seus cálculos no verso}.
\end{enumerate}

%%%%%%%%%%%%%%%%%%%%%%%%%%%%%%%%%%%%%%%%%%%%%%%%%%%%%%%%%%%%%%%%%%%%%%%%%%%%%%%
%%%%%%%%%%%%%%%%%%%%%%%%%%%%%%%%%%%%%%%%%%%%%%%%%%%%%%%%%%%%%%%%%%%%%%%%%%%%%%%
%%%%%%%%%%%%%%%%%%%%%%%%%%%%%%%%%%%%%%%%%%%%%%%%%%%%%%%%%%%%%%%%%%%%%%%%%%%%%%%
%%%%%%%%%%%%%%%%%%%%%%%%%%%%%%%%%%%%%%%%%%%%%%%%%%%%%%%%%%%%%%%%%%%%%%%%%%%%%%%
\cleardoublepage

\noindent{}{\huge\textit{Medidas}}

\vspace{15mm}

\begin{fullwidth}
\noindent{}\makebox[0.6\linewidth]{Turma:\enspace\hrulefill}\makebox[0.4\textwidth]{  Data:\enspace\hrulefill}
\vspace{5mm}

\noindent{}\makebox[0.6\linewidth]{Aluno(a):\enspace\hrulefill}\makebox[0.4\textwidth]{  Matrícula:\enspace\hrulefill}

\noindent{}\makebox[0.6\linewidth]{Aluno(a):\enspace\hrulefill}\makebox[0.4\textwidth]{  Matrícula:\enspace\hrulefill}

\noindent{}\makebox[0.6\linewidth]{Aluno(a):\enspace\hrulefill}\makebox[0.4\textwidth]{  Matrícula:\enspace\hrulefill}

\noindent{}\makebox[0.6\linewidth]{Aluno(a):\enspace\hrulefill}\makebox[0.4\textwidth]{  Matrícula:\enspace\hrulefill}

\noindent{}\makebox[0.6\linewidth]{Aluno(a):\enspace\hrulefill}\makebox[0.4\textwidth]{  Matrícula:\enspace\hrulefill}
\end{fullwidth}

\vspace{5mm}

%%%%%%%%%%%%%%%%%%%%%%%%%%%%%%%%%%%%%%%%%%%%%%%%%%%%%%%%%%%%%%%%%%%%%%%%%%%%%%%
\section{Questionário}
%%%%%%%%%%%%%%%%%%%%%%%%%%%%%%%%%%%%%%%%%%%%%%%%%%%%%%%%%%%%%%%%%%%%%%%%%%%%%%%
\emph{Nas questões seguintes, apresente os cálculos requisitados de maneira clara e sucinta, para que o professor possa acompanhar o raciocínio desenvolvido.}
\vspace{5mm}

\begin{question}[type={exam}]{2}
Lorem ipsum dolor sit amet, consectetuer adi-
piscing elit. Ut purus elit, vestibulum ut, placerat ac, adipiscing vitae,
felis. Curabitur dictum gravida mauris. Nam arcu libero, nonummy
eget, consectetuer id, vulputate a, magna. Donec vehicula augue
eu neque. Pellentesque habitant morbi tristique senectus et netus
et malesuada fames ac turpis egestas. Mauris ut leo. Cras viverra
metus rhoncus sem. Nulla et lectus vestibulum urna fringilla ultrices.
\end{question}



\vfill
%%%%%%%%%%%%%%%%%%%%%%%%%%%%%%%%%%%%%%%%%%%%%%%%%%%%%%%%%%%%%%%%%%%%%%%%%%%%%%%
\pagebreak
\section{Tabelas}
%%%%%%%%%%%%%%%%%%%%%%%%%%%%%%%%%%%%%%%%%%%%%%%%%%%%%%%%%%%%%%%%%%%%%%%%%%%%%%%

\begin{table*}[!htpb]
	\caption{Resultados obtidos para os paralelepípedos utilizando uma régua.}
	\label{TabelaDadosRegua}
	\begin{tabular}{ccccc}
		\toprule
		{\centering Paralelepípedo} & Altura & Largura & Comprimento & Volume  \\
		\midrule
		\rowcolor[gray]{0.9} 1 & \phantom{xxxxxxxxxxxxxxx} & \phantom{xxxxxxxxxxxxxxx} & \phantom{xxxxxxxxxxxxxxx} & \phantom{xxxxxxxxxxxxxxx} \\
		2 &  &  & & \\
		\rowcolor[gray]{0.9} 3 &  &  & & \\
		\bottomrule
	\end{tabular}
\end{table*}

\begin{table*}[!htpb]
	\caption{Resultados obtidos para os cilindros utilizando uma régua.}
	\label{TabelaDadosReguaCil}
	\begin{tabular}{cccc}
		\toprule
		{\centering Cilindro} & diâmetro & Comprimento & Volume  \\
		\midrule
		\rowcolor[gray]{0.9} 1 & \phantom{xxxxxxxxxxxxxxx} & \phantom{xxxxxxxxxxxxxxx} & \phantom{xxxxxxxxxxxxxxx} \\
		2 &  &  & \\
		\bottomrule
	\end{tabular}
\end{table*}

\begin{table*}[!htpb]
	\caption{Resultados obtidos para os paralelepípedos utilizando um paquímetro.}
	\label{TabelaDadosPaquimetro}
	\begin{tabular}{ccccc}
		\toprule
		{\centering Paralelepípedos} 				& Altura & Largura & Comprimento & Volume  \\
		\midrule
		\rowcolor[gray]{0.9} 1 & \phantom{xxxxxxxxxxxxxxx} & \phantom{xxxxxxxxxxxxxxx} & \phantom{xxxxxxxxxxxxxxx} & \phantom{xxxxxxxxxxxxxxx} \\
		2 &  &  & & \\
		\rowcolor[gray]{0.9} 3 &  &  & & \\
		\bottomrule
	\end{tabular}
\end{table*}

\begin{table*}[!htpb]
	\caption{Resultados obtidos para os cilindros utilizando um paquímetro.}
	\label{TabelaDadosPaquimetroCil}
	\begin{tabular}{cccc}
		\toprule
		{\centering Cilindro} 				& diâmetro & Comprimento & Volume  \\
		\midrule
		\rowcolor[gray]{0.9} 1 & \phantom{xxxxxxxxxxxxxxx} & \phantom{xxxxxxxxxxxxxxx} & \phantom{xxxxxxxxxxxxxxx} \\
		2 &  &  & \\
		\bottomrule
	\end{tabular}
\end{table*}

\begin{table*}[!htpb]
	\caption{Resultados obtidos para a massa e para a densidade dos paralelepípedos.}
	\label{TabelaDadosMassa}
	\begin{tabular}{cccc}
		\toprule
		{\centering Paralelepípedos} 				& Massa & Volume & Densidade \\
		\midrule
		\rowcolor[gray]{0.9} 1 & \phantom{xxxxxxxxxxxxxxx} & \phantom{xxxxxxxxxxxxxxx} & \phantom{xxxxxxxxxxxxxxx} \\
		2 &  &  &  \\
		\rowcolor[gray]{0.9} 3 &  &  &  \\
		\bottomrule
	\end{tabular}
\end{table*}

\begin{table*}[!htpb]
	\caption{Resultados obtidos para a massa e para a densidade dos cilindros.}
	\label{TabelaDadosMassaCil}
	\begin{tabular}{cccc}
		\toprule
		{\centering Cilindros} 				& Massa & Volume & Densidade \\
		\midrule
		\rowcolor[gray]{0.9} 1 & \phantom{xxxxxxxxxxxxxxx} & \phantom{xxxxxxxxxxxxxxx} & \phantom{xxxxxxxxxxxxxxx} \\
		2 &  &  &  \\
		\bottomrule
	\end{tabular}
\end{table*}
