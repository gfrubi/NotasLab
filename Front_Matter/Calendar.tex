\thispagestyle{plain}
\begin{fullwidth}
\begin{center}
{\noindent\LARGE\textsc{Cronograma}} \\
\end{center}
\end{fullwidth}

\vspace{1cm}
\begin{fullwidth}
\it
As aulas seguirão o planejamento abaixo. No calendário ao lado, estão circuladas as datas das provas.
\end{fullwidth}

%%%
% Engenharia elétrica
%%%

%\begin{marginfigure}[5cm]
%\centering
%Agosto\\
%\begin{tikzpicture}
%\calendar (mycal) [dates=2018-08-01 to 2018-08-last,week list] if (Saturday,Sunday) [gray];
%%\draw[dashed] (mycal-2018-03-13) circle (6pt);
%\end{tikzpicture}
%\end{marginfigure}
%\begin{marginfigure}
%\centering
%Setembro\\
%\begin{tikzpicture}
%\calendar (mycal) [dates=2018-09-01 to 2018-09-last,week list] if (Saturday,Sunday) [gray];
%\end{tikzpicture}
%\end{marginfigure}
%\begin{marginfigure}
%\centering
%Outubro\\
%\begin{tikzpicture}
%\calendar (mycal) [dates=2018-10-01 to 2018-10-last,week list] if (Saturday,Sunday,equals=2018-10-09) [gray];
%\end{tikzpicture}
%\end{marginfigure}
%\begin{marginfigure}
%\centering
%Novembro\\
%\begin{tikzpicture}
%\calendar (mycal) [dates=2018-11-01 to 2018-11-last,week list] if (Saturday,Sunday) [gray];
%\end{tikzpicture}
%\end{marginfigure}
%\begin{marginfigure}
%\centering
%Dezembro\\
%\begin{tikzpicture}
%\calendar (mycal) [dates=2018-12-01 to 2018-12-last,week list] if (Saturday,Sunday) [gray];
%\draw (mycal-2018-12-04) circle (6pt);
%\end{tikzpicture}
%\end{marginfigure}
%
%\vspace{1cm}
%\begin{center}
%\Large\textsc{Engenharia Elétrica}
%\end{center}
%
%As aulas seguirão o planejamento abaixo.
%\begin{center}
%\begin{longtable}{ccp{70mm}}
%\toprule
%Aula & Data & Conteúdo \\
%\midrule
%\endhead
%\bottomrule
%\endfoot
% 1 & 07/08 & Turmas A e B: Apresentação da disciplina. \\
% 2 & 14/08 & Turma A: Exp. 1, Elasticidade. \\
% 3 & 21/08 & Turma B: Exp. 1, Elasticidade. \\
%-- & 28/08 & Semana acadêmica. \\ 
% 4 & 04/09 & Turma A: Exp. 2, Empuxo. \\
% 5 & 11/09 & Turma B: Exp. 2, Empuxo. \\
% 6 & 18/09 & Turma A: Exp. 3, Oscilações. \\
% 7 & 25/09 & Turma B: Exp. 3, Oscilações. \\
% 8 & 02/10 & Turma A: Exp. 4, Ondas estacionárias. \\
%-- & 09/10 & Recesso. \\
% 9 & 16/10 & Turma B: Exp. 4, Ondas estacionárias. \\
%10 & 23/10 & Turma A: Exp. 5, Dilatação linear e lei de resfriamento de Newton. \\
%11 & 30/10 & Turma B: Exp. 5, Dilatação linear e lei de resfriamento de Newton. \\
%12 & 06/11 & Turma A: Exp. 6, Calor específico de sólidos. \\
%13 & 13/11 & Turma B: Exp. 6, Calor específico de sólidos. \\
%14 & 20/11 & Turma A: Exp. 7, Zero absoluto. \\
%15 & 27/11 & Turma B: Exp. 7, Zero absoluto. \\
%16 & 04/12 & Turmas A e B: Prova. \\
%17 & 11/12 & Turmas A e B: Apresentação das notas finais de laboratório.
%\end{longtable}
%\end{center}
%
%\cleardoublepage

%%%
% Química
%%%

\begin{marginfigure}[5cm]
\centering
Março\\
\begin{tikzpicture}
\calendar (mycal) [dates=2019-03-01 to 2019-03-last,week list] if (Saturday,Sunday) [gray];
%\draw[dashed] (mycal-2018-03-13) circle (6pt);
\end{tikzpicture}
\end{marginfigure}
\begin{marginfigure}
\centering
Abril\\
\begin{tikzpicture}
\calendar (mycal) [dates=2019-04-01 to 2019-04-last,week list] if (Saturday,Sunday,equals=2019-04-19) [gray];
\end{tikzpicture}
\end{marginfigure}
\begin{marginfigure}
\centering
Maio\\
\begin{tikzpicture}
\calendar (mycal) [dates=2019-05-01 to 2019-05-last,week list] if (Saturday,Sunday,equals=2019-05-01,equals=2019-05-13, equals=2019-05-14, equals=2019-05-15, equals=2019-05-16, equals=2019-05-17) [gray];
\end{tikzpicture}
\end{marginfigure}
\begin{marginfigure}
\centering
Junho\\
\begin{tikzpicture}
\calendar (mycal) [dates=2019-06-01 to 2019-06-last,week list] if (Saturday,Sunday,equals=2019-06-20,equals=2019-06-21) [gray];
\end{tikzpicture}
\end{marginfigure}
\begin{marginfigure}
\centering
Julho\\
\begin{tikzpicture}
\calendar (mycal) [dates=2019-07-01 to 2019-07-last,week list] if (Saturday,Sunday) [gray];
\draw (mycal-2019-07-02) circle (6pt);
\end{tikzpicture}
\end{marginfigure}

\vspace{1cm}
\begin{center}
\Large\textsc{Química}
\end{center}

As aulas seguirão o planejamento abaixo.
\begin{center}
\begin{longtable}{ccp{70mm}}
\toprule
Aula & Data & Conteúdo \\
\midrule
\endhead
\bottomrule
\endfoot
 1 & 12/03 & Turmas A e B: Apresentação da disciplina. \\
 2 & 19/03 & Turma A: Exp. 1, Elasticidade. \\
 3 & 26/03 & Turma B: Exp. 1, Elasticidade. \\
 4 & 02/04 & Turma A: Exp. 2, Empuxo. \\ 
 5 & 09/04 & Turma B: Exp. 2, Empuxo. \\
 6 & 16/04 & Turma A: Exp. 3, Oscilações. \\
 7 & 23/04 & Turma B: Exp. 3, Oscilações. \\
 8 & 30/04 & Turma A: Exp. 4, Ondas estacionárias. \\
 9 & 07/05 & Turma B: Exp. 4, Ondas estacionárias. \\
-- & 14/05 & \emph{Recesso.}\\
10 & 21/05 & Turma A: Exp. 5, Dilatação linear e lei de resfriamento de Newton. \\
11 & 28/05 & Turma B: Exp. 5, Dilatação linear e lei de resfriamento de Newton. \\
12 & 04/06 & Turma A: Exp. 6, Calor específico de sólidos. \\
13 & 11/06 & Turma B: Exp. 6, Calor específico de sólidos. \\
14 & 18/06 & Turma A: Exp. 7, Zero absoluto. \\
15 & 25/06 & Turma B: Exp. 7, Zero absoluto. \\
16 & 02/07 & Turmas A e B: Prova. \\
17 & 09/07 & Turmas A e B: Apresentação das notas finais de laboratório. \\
18 & 16/07 & --- \\
\end{longtable}
\end{center}


\cleardoublepage
